\documentclass[11pt]{scrreprt}

\usepackage[utf8]{inputenc}
\usepackage[top=3cm,bottom=4cm,left=3cm,right=3cm]{geometry}
\setlength{\parskip}{6pt}
\setlength{\parindent}{0pt}

\title{Die Sieben Gezeichneten}
\subtitle{Rückkehr der Finsternis}
\author{Die Herren von Klammsbrück}
\date{2011 - 2016}

\begin{document}
\maketitle

\begin{abstract}
Die gesammelten Tagebücher der Helden der Dritten Dämonenschlacht, die ihr Leben gaben, um dem Dämonenmeister, dem Schänder der Sphären, dem Alveraniar des Verbotenen Wissens Einhalt zu gebieten.
\end{abstract}

\chapter{Euch zum Geleit}
Liebste, hochverehrte Chaostruppe, 

die tatsächlich, trotz aller Widrigkeiten, den Dämonenmeister in die Knie gezwungen hat.\par\medskip

Meine Kommentare entstehen gerade lange vor eurem Sieg, in einer Stillen Stunde in Darmstadt. Seit dem ersten Spielabend sammle und formatiere ich eure Tagebücher in ein großes, allumfassendes Dokument eurer Taten. Ich musste streichen, einiges kürzen, von vielen der doppelten Tagebücher habe ich nur eine Version genommen, damit das ganze nicht zu lang wird, ich hoffe ihr verzeiht mir, dass nicht alle eure strahlenden literarischen Werke aufgenommen wurden.\par\medskip

Ja, für die die mich jetzt fragend anblicken, dass war (fast) immer einer der großen Gründe, warum ich so penetrant auf Tagebücher bestanden habe. Wer so viele Jahre ein und dasselbe Ziel verfolgt, soll auch eine Widmung haben, die diesem würdig ist. Und ja, die Druckkosten habe ich von euren Geburtstagsgeldern abgezogen (freches Grinsen). Ich habe mir außerdem die Freiheit genommen und Kommentare geschrieben, um euch einen kleinen Einblick in den Entstehungsprozess einiger Spielabende zu geben, und einen Anhang vorbereitet, mit den wichtigsten und schönsten Handouts.\par\medskip

Ich hoffe, es hat euch so viel Spaß gemacht, wie mir. Ich bedanke mich von ganzem Herzen bei jedem, der das zusammen mit der Gruppe durchgestanden hat. Über viele Jahre fallen auch strenge oder stechende Worte, die meisten davon bereue ich (einige nicht, der Würfelwurf bei Borbarads Verkörperung war vollkommen angemessen und hat die Stimmung nicht noch mehr ruiniert :-P).\par\medskip
...

\chapter{Ein Brief an einen Freund}


Lieber Waldemar,
ich will dir sehr herzlich zu deiner Ernennung zum Magister Magnus gratulieren, du hast es dir redlich verdient. Ich kann mich noch gut daran erinnern, wie du mit großen Augen in Temyrs erster Vorlesung sahst und dir vor dem Anblick eines echten Elementars Angst und bang geworden ist. Wenn ich den Titel deiner Habilitation so anschaue, hast du diese Furcht ja nun verloren.\par\medskip
Du hast mich einmal gefragt, was ich nun tun würde, in den Ruinen der Trollpforte, nach der schrecklichen Schlacht. Nachdem ich mit dem König Cuanu, dem Schwert der Schwerter, und der weinenden Aria die Leichen der Sieben Gezeichneten geborgen hatte, konnte ich dir nicht antworten. Ich habe die Akademie in Arias und deine treuen Hände gelegt und bin geflohen, zu schwer lasteten die Ereignisse auf meinem Gemüt. Ich habe mich auf den Rhodenstein zurückgezogen und mich dem Studium der Schriften gewidmet, die ich aus den Schlachten der Zeitalterwende retten konnte. Einige Berichte waren bruchstückhaft, viele Zeitzeugen waren gestorben und mein eigener Schmerz über den Verlust meiner alten Freunde lag noch frisch auf meiner Seele. Zusammen mit dem Ritter des RHodensteins, Bernfried von Kiesfurten, an den du dich vielleicht noch erinnern wirst, habe ich die Ereignisse aufgearbeitet, Mythen und Legenden von der Wahrheit getrennt und hoffe, nun den ersten vollständigen Bericht über die Taten der Sieben Gezeichneten abliefern zu können.\par\medskip
Ihre Erhabenheit von Schattengrund hat diesen Bericht schon gesiegelt und ich darf dir stolz verkünden, dass dies die hochoffizielle Version ist, die in das Register der Rondragefälligen Helden aufgenommen wird.\par\medskip

Dies sind die Taten\par\smallskip
Seiner Spektabilität Calhadril Ignisfulgur zu Klammsbrück, Boron möge seiner Seele Gnädig sein,\par
Seiner seligen Eminenz Firnen Wulfgrimm, verbannt, verstoßen, zu Recht und zu Unrecht verurteilt, möge Hesinde seiner Seele gnädig sein,\par\smallskip
Seiner heiligen Exzellenz und Eminenz Toran Ostik von Klammsbrück, Erwählter der Peraine, möge seine Seele den Verlorenen in den Schwarzen Landen den rechten Weg weisen,\par\smallskip
Seiner seligen Exzellenz Baron Ragnos vom Svellt zu Grunewaldt, schnellster Schütze aller Zeiten, dem selbst der Rote Pfeil  Respekt zollte,\par\smallskip
Seiner Durchlaucht Rezzanjin von Klammsbrück, Kriegsfürst von Maraskan, möge seine Seele den Weg zur Wiedergeburt finden,\par\smallskip
Seiner seligen Magnifiziens Archomagus Spektabilität Temyr ibn Sahid, Erwählter des Nandussohnes, möge seine Weisheit die Suchenden erleuchten,\par\smallskip
Seiner seligen Exzellenz Irian Stahlschwinge von Rabemund, verstoßen und verbannt, Erwählter des Graufang, möge seine Seele mit der Wilden Jagd dem Wintervater folgen,\par\smallskip
Seiner seligen Eminenz Oleg Sicut Tonitrus Sjepsen von Klammsbrück, Heermeister der Rondrakirche in der schwersten Zeit.\par\smallskip
Rondra, Hesinde, Firun und Peraine urteilten über diese Seelen und empfahlen sie uns als Vorbild und strahlenste Helden an, auf das wir ihren Wegen folgen mögen.\par\medskip

Wir gedenken auch den anderen strahlenden Recken und großen Männern und Frauen, die die Gezeichneten begleitet haben:
Erwen Lowanger, der Finsternis anheim gefallen und gerettet\par
Frenglion Krennelsieks\par
Cusimo Villar, der Finsternis anheim gefallen\par
Baron Arngrimm von Ehrenstein der Jüngere, in der Schlacht um den Tuzaker Fürstenpalast gefallen, als er den Gezeichneten das Leben rettete\par
Boronos te Partholon, in der Wüste Gor gefallen\par
Die Götter mögen ihren Seelen gnädig sein.\par\medskip

Diese aber, die ersten Abschriften dieses Berichtes will ich an jene geben, die die Gezeichneten am besten gekannt und mit ihnen gelebt und gelitten haben.\par\medskip

Ich bin nun alt, aber noch nicht gebrochen. Ich will mich dem Hochmeister Tarlisin von Borbra unterstellen und im Kampf gegen die verbleibenden Diener der Finsternis mein Ende finden, auf dass ich meine Freunde endlich wieder erblicken kann, in Alveran.\par\medskip

Dein alter Lehrmeister,\par
Spektabilitus Emeritus Iliricon Tannhaus, Chronist des Sacer Ordo Draconis



\chapter{Albtraum ohne Ende}

\section{Geleitwort}

Albtraum ohne Ende... das ist inzwischen so lange her, dass meine Erinnerungen langsam im Dunste Satinavs verschwinden. Ich hätte das Kommentieren früher anfangen sollen.\par
Ein wunderschöner Vorlesetext, der nur wegen seiner unglaublich packenden Geschichte nicht zu langweiligem Railroading führt, das ist meine heutige Einstellung zum Abenteuer. Ich kann mich noch daran erinnern, dass ich vor dem ersten Spielabend Albträume hatte, dass ihr innerhalb eines Jahres durch die Kampagne rennen würdet, ohne ihre Schönheit und Größe zu begreifen. Tja, da war ich wohl nicht ordentlich vorbereitet auf das, was noch kommen sollte. Inzwischen habe ich eher Albträume, das ich als Rentner die Dämonenschlacht spielen werde.\par
Alles in allem staune ich immer wieder, wie schnell wir das Abenteuer durchgezogen haben. Das epische In-die-Länge-ziehen fing erst während Pforten des Grauens an und erreichte mit dem Tobrienkrieg seinen glanzvollen Höhepunkt. Nur vier Spielabende haben wir von Klammsbrück bis nach Dragenfeldt gebraucht, ein absoluter Rekord!
Ich bin euch außerdem dankbar, dass ihr nach Markalms vorzeitigem Ableben seinen Brief immer als Mysterium behandelt habt und ihm einen epischen und weitreichenden Hintergrund unterstellt habt.\par
Philip, das geht vor allem an dich: Ich war dir so dankbar, dass du Markalm gegrillt hast! Das mag dich jetzt vielleicht enttäuschen, aber sein Grund euch nach Baliho zu rufen, war einfach fast nicht vorhanden, ich hatte nur keine andere Idee, wie ich euch aus dem friedlichen Klammsbrück hinein ins Abenteuer hätte Schubsen sollen. Der grobe Umriss dafür sah folgendermaßen aus:\par
Markalm kennt die Prophezeiungen und will sie sich zu nutzen machen, um Borbarads Rückkehr zu vereinfachen. Er glaubt, dass er das Schicksal so steuern kann, dass er die Sieben Gezeichneten erwählt, die er dann nutzt, um Borbarad zum Sieg zu verhelfen.\par
Vielleicht war die Begründung gut, vielleicht auch genauso fadenscheinig, wie sie mir heute vorkommt. So jedenfalls konnte der älteste oder vielleicht zweitälteste aller Running Gags daraus werden. Und nein, ich hatte niemals einen richtigen Monolog vorbereitet.\par
Ansonsten habe ich gute Erinnerungen an das Abenteuer, auch wenn es sicher geteilte Meinungen über die in mehr oder weniger guter Qualität aufgenommenen Albträume gab. Der Grund warum diese nicht wiederkehrten, war übrigens die saumäßige Arbeit für das Abmischen und Aufnehmen, die in jedem einzelnen steckte. Das war mir deutlich zu viel, um das selber zu machen, und auch noch neue Texte zu schreiben. Nach der Katastrophe von "Kanäle von Grangor“ in dem mein Rahja-Traum nicht nur vollkommen unverständlich, sondern wegen meines Versuchs, meine Stimme auf eine Frauenoktave zu heben auch noch sehr lächerlich war, wollte ich mich eurem Spott nicht mehr preisgeben.\par
Stimmungsvoll war es auf jeden Fall, der Horrortrip nach Dragenfeldt, vielleicht für den ein oder anderen zu spannend ;-).\par

Die mir am besten in Erinnerung gebliebenen Szenen sind:
\begin{itemize}
\item das Rendevous des Ragnos, nachdem ich drei Outgame-Jahre brauchte, bis ich seinen Erzfeind endlich wieder eingeführt hatte
\item Calhadrils Ignissphäro im passendsten aller Momente
\item Calhadrils unsterbliche Aussage "Ich gehe auf die Turmspitze, im Keller findet immer das finstere Ritual statt, das will ich jetzt nicht sehen“
\item "Who wants to live forever“, eine Idee, für die ich leider keinen Lob verdiene, weil sie geklaut war.
\end{itemize}


\begin{flushright}
Claas Völcker, Darmstadt, den 2.2.2015
\end{flushright}


\section{Die Tagebücher}


\subsection{Die Ereignisse in Baliho nach Rezzanjin al-Ahjan}

\paragraph{25.Peraine 1015 n. Bosparans Fall:} Am Morgen übte ich mich ein wenig im Schwertkampf und am Nachmittag war ich auf einer kleinen Pirsch mit Ragnos. Immerhin einen Hasen haben wir erlegt. Im Jagen bin ich trotz der Übungsstunden mit Ragnos noch nicht wirklich gut. Liegt wohl auch daran, dass man mit einem Diskus normalerweise nicht jagt.

\paragraph{1. Ingerimm:} Die letzten Tage war eine Delegation von uns mal wieder beim Baron, um ihm einen Besuch abzustatten. Ich war natürlich dabei. Aber es war nur das übliche Gerede, nichts besonderes.

\paragraph{3. Ingerimm:} Heute waren endlich mal wieder alle auf der Burg. Iliricon bat uns fünf in ein Turmzimmer, da er dort einen Brief verlesen wollte. Der Inhalt war höchst merkwürdig, gleich einer Prophezeiung, und wir konnten nur wenig eindeutig deuten. Im Inhalt war auch vom Schatten der Tobimora die Rede, eine merkwürdige Gestalt, die wir einmal gejagt hatten. Sie war uns leider entkommen, also immer noch auf freiem Fuß. Auch der Absender war höchst seltsam. Anstatt eines ordentlichen Namens fanden wir nur irgendwelche Buchstaben und Kommata. Der Briefschreiber, den wir angeblich kennen sollen, da er mit uns, oder einem Teil von uns durch die Lande gezogen sei, lud uns in eine Stadt mit zwei Rädern ein, doch auch darauf konnten wir uns keinen Reim machen. Wir fragten schließlich Oleg, ob er wisse was dies bedeuten konnte. Er wies uns darauf hin das Baliho, eine Stadt in Weiden, ein Wappen mit zwei Rädern drauf habe. Der Brief wies auch auf viele Rinderweiden hin, so passte Baliho ins Bild. Nach kurzer Besprechung war klar, dass auch die Magier für kurze Zeit mitkonnten und so zogen wir noch heute gen Baliho und ließen Oleg, Iliricon und auch den nervigen, weil immer Vorträge haltenden Hesindegeweihten Erwen in Klammsbrück zurück. Nach langer Zeit zogen wir fünf, Thoran, Temyr, Ragnos, Calhadril, und ich, die wir Liscom von Fasar besiegt hatten, wieder gemeinsam aus.

\paragraph{7. Ingerimm:}
Die letzten Tage ritten wir entlang der Tobimora und heute sind wir endlich auf den Passweg über die Sichel gekommen. Beim Abendessen in der Taverne trafen wir auf Flammenzunge, anscheinend einen ehemaligen Wegbegleiter von Temyr, Ragnos und Thoran. So wurde der Abend mit viel erzählten Geschichten und Met verbracht.

\paragraph{16.Ingerimm:}
Nach den letzten doch recht nicht sehr ereignisreichen Tagen, war heute auf der Straße viel los. Sehr viele Praioten waren in Richtung Anderath unterwegs, gar eine goldverzierte prachtvolle Kutsche mit viel Begleitung überholte uns. In Anderath selbst, wo wir nächtigten, waren sie dann auch sehr stark vertreten.

\paragraph{21.Ingerimm:}
Heute kamen wir endlich in der prachtvollen Stadt Baliho an. In der Stadt wuselte es nur so von Gestalten, die morgen das Volksfest besuchen würden. Händler bauten schon ihre Stände auf und die Stadt putzte sich heraus. Wir nahmen uns selbstverständlich Zimmer im besten Hotel der Stadt.

\paragraph{22.Ingerimm:}
Wir waren früh auf den Beinen, um bei verschiedenen Wettbewerben, sowohl zuzuschauen, als auch teilzunehmen. Zuerst meldete sich Calhadril beim Kuhfladensetzen an und kaufte zwei Felder. Als nächstes Stand der Bogenschusswettbewerb an, bei dem Ragnos teilnahm. In einem spannenden Duell mit einer Elfe unterlag er knapp,(Bei 150 Schritt ins Schwarze zu treffen, ist auch schwer) doch die Elfe schenkte ihm immerhin den Siegespreis, ein Pfeil von Tenobal Totenamsel, da sie der Meinung war, er habe sich diesen verdient.\par
Als nächstes war der Saufwettbewerb dran. Temyr nahm dran Teil, gewann aber nichts mehr als einen Kater(Nachdem Thoran ihn von sämtlichen Alkohol "befreit“ hatte). Nach dem Wettbewerb lud uns die Traviageweihte, die den Wettbewerb geleitet hatte, ein am Abend in den Nordstern zu kommen, denn dort würde eine sehr rahjagefällige Tänzerin auftreten. Wir bejahten selbstverständlich.\par
Calhadril gewann beim Kuhfladensetzen nichts und so schlenderten wir ein wenig herum, bis ein Ausrufer tapfere Helden dazu aufforderte, sich im Kaiserstolz zu melden, um einen wichtigen Auftrag zu erledigen, bzw. sich dazu zu bewerben. Da wir ja sowieso die heldenhaftesten Helden sind und im Moment nichts zu tun haben, gingen wir hin. Direkt in der Taverne trafen die Söldnergruppe "die formidablen Sechs“, die zum Teil auch an den Wettbewerben teilgenommen hatte. Als sie uns sahen machten sie sich direkt über uns lustig. Sie kamen gerade aus dem Zimmer, in das wir wollten. Calhadril schickte den arroganten Söldnern noch einen Zauber hinterher und dann gingen wir ins Zimmer. Dort saß Delian von Wiedbrück und schaute uns an. Nach einer kurzen Unterredung waren wir als undiszipliniert und unerfahren beschimpft wieder hinausgeworfen worden. Dieser Kerl hatte ja keine Ahnung, wen er da vor sich hatte. Auch meinte er, dass die formidablen Sechs sehr viel etablierter und erfahrener seien als wir. Dieser Typ hat doch keine blassen Schimmer, was er da faselt! Ragnos hat immerhin eine von denen in Grund und Boden geschossen. Aber nun gut, beide werden bald hoffentlich bald eines besseren belehrt. \par

Wir ruhten uns danach im Hotel aus. Am Abend gingen wir in den Nordstern, eine Spielhalle, die in einem alten Efferdtempel liegt. Wahrlich ein prächtiges Haus. Schnell fanden die Traviageweihte Linai, mit der wir uns verabredet hatten, zusammen mit zwei Begleiterinnen an einem Tisch sitzen. Der Abend ging schnell herum und die Hauptattraktion wurde präsentiert. Die wahrlich schöne Novaditänzerin begann ihre Aufführung. Sie war einfach genial. Diese Bewegungen. Fast wie in einem Traum. Doch gen Ende der Vorstellung bekam die Traviageweihte plötzlich heftige Fieberschübe und Krämpfe und brach zusammen. Erst am Ende der Vorstellung endeten diese. Körperlich fehlte ihr laut Thoran wohl nichts, aber sie hatte anscheinend eine ziemlich heftige Vision gehabt, wie sie uns erzählte. Mittlerweile war auch Ragnos zurück, der bei seinem Rendezvous mit einer beeindruckten "Zuschauerin“ wohl einen Dolchstoß abbekommen hatte. Leider konnte diese "Zuschauerin“(war vermutlich ein neidischer Teilnehmer, der Ragnos den Zweikampf mit der Elfe nicht gönnte) entfliehen. \par

Dank der Vision, die sie wirklich mitnahm, wollte die Traviageweihte jetzt nach Dragenfeldt , dem Ort der Vision gehen, da sie dort wohl eine Tsageweihte hatte verbrennen sehen, die sie kannte. Wir konnten sie nicht von ihrem Wunsch nach Dragenfeldt zu gehen abbringen und so boten wir an mit ihr zu ziehen und sie auf ihrer Reise zu beschützten. Sie nahm zögerlich an. Wir brachten sie noch zum Tempel und machten uns dann auch zum Hotel auf.

\paragraph{23. Ingerimm:}
Heute morgen lief uns direkt eine der Begleiterinnen der Traviageweihten über den Weg. Als wir ihr berichteten, dass wir vorhaben, die Geweihte zu begleiten, wollte sie, dass wir ohne die Geweihte gehen, da diese zu angeschlagen sei, um die Reise zu wagen. Sie meinte es wäre im Sinne der Geweihten, wenn wir für sie nach dem Rechten schauen. Selbstverständlich nahmen wir an. Wenn die Gläubigen der Helfer von Rur was nicht erledigen können, dann müssen es wohl Gläubige Rurs tun. Sie gab uns eine Erlaubnis, dass wir im Namen der Geweihten frei handeln können. Und so zogen wir los gen Dragenfeldt ...


\subsection{Von Baliho nach Braunsklamm nach Calhadril Ignisfulgur}
"Dunkles geschah dieser Tage in Dragenfeldt  und so zogen wir aus, alles wieder in rechte Bahnen zu lenken und das Mysterium um das dortige Geschehen zu ergründen." 
Calhadril, kurz vor seine Tode zu den Chronisten

\paragraph{23.Ingerimm}

Wir haben bei einem Alten in Baliho eine Karte gekauft, auf der Dragenfeldt  verzeichnet ist. Der Kartenmacher riet uns von einer Abkürzung des Weges im Norden ab, da dort kein Durchkommen sei, schon gar nicht zu Pferde. Wir haben dafür lediglich 5 Silbertaler gezahlt und Thoran hat ihn mit Hilfe der Göttin von seinem chronischen Husten befreit. Lange noch klang uns die erste Strophe der Orkentodballade hinterher, die der Alte aus Freude über seine Wieder erlangte Gesundheit nach unserem Abschied anstimmte - Laut und Falsch!!!\par

In Anderath hielt uns ein junger Diener des Praios an und fragte nach unserem Reiseziel. Wir nannten ihm Dragenfeldt, worauf hin er uns bat, ihm zu seinem Obersten zu folgen, der an jeglicher Information über Dragenfeldt  interessiert sei. Anderath ist ein hübsches kleines Dorf. Vielleicht etwas zu klein, aber ganz nett. Nur laufen hier irgendwie zu viele von diesen Praioten herum, seit der Spaltung ihrer Kirche trifft man sie überall, sogar Sonnenlegionäre sah man, und ,was mich erstaunte, auch einige Weißmagier. Man brachte uns zum Hochinquisitor, Amando Lacondo da Vanja. Das Gespräch, bzw. die Befragung, führte allerdings der "Erwählte" (Oberster Bannstrahler) mit Thoran. -> für diesen arroganten ... scheinen nur Praioten und im Notfall auch andere Geweihte der Zwölf zu existieren. (fast schon ein monotheistischer Praiosglaube) Er berichtete von einem Traum, den schon mehrere Geweihte gehabt hätten: Ein Greifen-Ei liegt auf dem höchsten Gipfel in der "Roten Sichel", aus ihm schlüpft eine Eidechse, die sich in eine schwarze Schlange verwandelt und ihre Greifen-Eltern tötet. Wir / Thoran berichteten ihm von unserem Auftrag im Namen der Perainkirche von Baliho. \par

Nun reisen wir auch noch im Auftrag der Praioskirche. Man gab uns Zeit bis zum 15.Rahja das Rätsel um Dragenfeldt  zu lösen und unseren Erfolg in Anderath zu melden, ansonsten würde die Praioskirche mit der Durchsuchung und Überprüfung der Tsakirchen im Umfeld auf verräterische Umtriebe beginnen. (Immer diese übereifrigen Bannstrahler [auf redaktionellen Wunsch geändert]) Gegen Abend trafen wir auf einen kleinen Jungen der von uns als "Räuberbaron" einen Zoll von fünf Dukaten und einer schönen Maid verlangte. Geistesgegenwärtig konnte ich R. (Ich kann mir die Schreibweise des Maraskaners nie merken) von einem übereilten Streich gegen den gutgläubigen Jungen abhalten und schenkte diesem 5 Kreuzer. Der Kleine war hoch erfreut und lud uns ein, die Nacht im Hause seines Vaters zu verbringen. Nun liege ich hier im Stroh, während die anderen in Betten schlafen und schreibe diese Zeilen nieder. Im Laufe des Abends kam Ayla von Schattengrund, eine Rondrageweihte und häufige Besucherin von Oleg an und begehrte ebenfalls eine Bleibe für die Nacht. Da die Bauern nun ein Bett zu wenig hatten, erinnerte ich mich meiner guten Schule und erklärte mich bereit, ihr mein Bett zu überlassen und in der Scheune zu schlafen. Erstaunlich sauber ist es hier allerdings: Nur eine Spinne viel von einem der oberen Balken nach meiner magischen schnell Reinigung.

\paragraph{24. Ingerimm}

Man hat mich viel zu früh aus dem Schlaf gerissen. Nun sitze ich hier noch etwas schlapp beim Morgenmal und schreibe dies auf, ob der erstaunlichen und beunruhigenden Ereignissen in den frühesten Morgenstunden. Man holte mich zum Zimmer in dem auch Ayla schlief. Unruhig wälzte sie sich in ihrem Bett hin und her, ihr Laken war von Blute getränkt, von ihrem Blute, hatte sie doch im Schlafe versucht sich den Bauch auf zu reißen. Wir weckten sie und nachdem Thoran sie mit Hilfe der Göttin Peraine geheilt, und sie sich ausreißend beruhigt hatte, war sie in der Lage uns von dem schrecklichen Traum zu berichten, der sie schon seit einiger Zeit verfolge und sie auch diese Nacht wieder geplagt habe. Sie sei umgeben von schwarzen Nebelschleiern, die sie, gleich Fesseln, zu umschlingen versuchen. Sie müsse diese aus einander reißen um sich ihrer zu erwehren. Noch zur frühen Stunde reiste sie ab, um so schnell wie möglich den Schutz ihres Ordens zu erreichen. Für den Bauern als Dank und Entschädigung ließ sie uns zwei Dukaten da, wir revanchierten uns mit nur vier Dukaten gesamt und werden unseren Weg heute hoffentlich bis Braunsklamm fortsetzen .
Wir kamen nur bis vor den Beginn der der engen, langen und gefährlich zu durchschreitenden Schlucht an den Braunwassern und schlugen dort unser Nachtlager auf. Rezzanjin (komisch jetzt fällt mir die Schreibweise wieder ein) übernahm die erste Nachtwache, ich befinde mich soeben bei der zweiten und die dritte wird meines Wissens Ragnos übernehmen.

\paragraph{25.Ingerimm}

Ich habe schon lange nicht mehr so gut geschlafen, ganz im Gegensatz zu Rezzanjin, der einen ziemlich lustlosen, erschöpften und missmutigen Eindruck macht, habe ihn doch die ganze Nacht ein Alptraum gequält: Er sei ein allmächtiger König und könne alles, alle würden ihm gehorchen und ganze Städte würden von Bergen verschluckt nur durch eine Fingerschnippen von ihm, oder so ähnlich. Allmählich mache ich mir ernsthaft Sorgen wegen der Träume. Werden sie uns noch alle heimsuchen?, Was werden wir uns oder anderen noch alles im Schlafe antun, wenn das so weitergeht!? Wir werden heute nun die Braunsklamm durchqueren und wahrscheinlich die Nacht im gleichnamigen Ort verbringen.
Ein seltsamer Tag und definitiv nicht Rezzanjins Phex-Tag. Es begann schon damit, dass er sein Pferd an einer besonders glitschigen Stelle zu führen versuchte, dabei trotz aller Vorsicht und redlichem Bemühen, stürzte er, immer hin zur vom Klammsfluss abgewandten Seite in eine abschüssige Felsgrotte. Als er daraus hervor kam brachte er einen seltsamen fremdmagischen Gegenstand mit: Eine Holzscheibe mit einer Nadel, die auf ein bestimmtes Ziel hinzuweisen scheint. Als wir schließlich in Braunsklamm ankamen erwartete uns dort gleich das nächste Mysterium. Das halbe Dorf war in Aufruhr und der Müller beklagte sich, das etwas das Mühlrad blockiere und das ganz Mühlgetriebe zerstört würde, unternähme man nicht bald etwas um das Mühlrad wieder frei zu bekommen.

\subsection{Ereignisse auf der Reise nach Dragenfeldt (bis Salthel) nach Rezzanjin al-Ahjan}

\paragraph{25. Ingerimm:}
Heute Nacht hatte ich einen richtig schlimmen Alptraum. Ich träumte Herrscher über alles zu sein und alles mit einem Fingerschnippen möglich machen zu können. So versenkte ich ganze Städte unter Bergen. Auf jeden Fall war ich ziemlich gerädert und fühlte mich total unwohl und schlapp. Wir ritten trotzdem weiter und erreichten die enge Schlucht. Der Boden war glitschig und so stieg ich ab, um nicht vom eventuell ausrutschenden Pferd zu fallen. Es half nichts, ich rutschte trotzdem aus und fiel in eine der höhlen am Rande des Wegs, die etwas tiefer lagen. Dort machte ich eine grausige Entdeckung: Hier lag ein Skelett. Außerdem hatte es etwas komisches in der Hand. Einen Holzpflock, auf dem eine Nadel montiert war, die in eine bestimmte Richtung zeigte. Ich nahm ihn mit und kletterte am Seil, das Ragnos mir zur Hilfe runter geworfen hatte, wieder ans Tageslicht. Dort begutachtete ich noch einmal den Pflock und drehte ihn.\par

Und siehe da, die Nadel zeigte immer noch in die selbe Richtung. Calhadril fand heraus, dass der Pflock mit irgendeiner fremdartigen Magie belegt war, aber nicht wie er funktionierte. Nach einem Ritt kamen wir dann in Braunenklamm an. Die Stadt war vollständig in den Fels gemeißelt und in einem Talkessel der Schlucht erbaut worden. Langsam ritten wir hindurch, als wir plötzlich auf einige Hilferufe aufmerksam wurden. Sie stammten, wie wir herausfanden, von einem Zwerg, der sich um seine Mühle sorgte, denn an dieser war das Hauptmühlrad stehengeblieben. Er vermutete, dass irgendwas am Grund des Mühlrades es blockierte. An der recht neuen Mühle konnte man den Wasserzufluss noch nicht umleiten, sodass jemand hinunter tauchen muss um die Blockade zu lösen. Da der Zwerg nicht schwimmen, geschweige denn tauchen konnte, musste einer von uns diese Aufgabe übernehmen. So übernahm ich diese Aufgabe bereitwillig, denn auch wenn ich nicht ein sonderlich guter Schwimmer bin, so bin ich immer noch der beste Schwimmer unserer Gruppe. \par

Also stellte ich mich dem kühlen Nass. Doch ich hatte die Strömung unterschätzt. Sie schleuderte mich direkt gegen einen Felsen und zog mich dann unters Mühlrad. Dort entdeckte ich ein komisches Wesen. Es sah aus, wie ein Mensch, nur hatte es leuchtend gelbe Augen und es schien das Mühlrad zu blockieren. Plötzlich griff es nach mir und hielt mich fest. Nach einem kleinen Gerangel konnte ich mich lösen und auch das Wesen wurde von der Strömung weg gezerrt. So wurden wir beide auf den kleinen Wasserfall des Braunwasser zugetrieben und stürzten letztendlich auch drüber. Glücklicher Weise wurde ich zusammen mit dem Wesen in eine kleine Bucht getrieben bei der auch schon die anderen warteten. Als wir das Wesen begutachteten stellte sich heraus, das dies ein toter Mensch war. Nach einigen arkarnen und heilkunde-technischen Analysen stellte sich heraus, dass diese Person schon recht lange tot war und anscheinend mal als Untoter durch die Gegend gelaufen ist. Außerdem fanden wir einen seltsamen Nagel in seiner Stirn. Wir beschlossen ihn zu bestatten, allerdings nicht auf herkömmliche Weise. Wir bauten ein Floß legten den Körper drauf und ließen in die Mitte der Bucht treiben. Dann schleuderte Calhadril einen Feuerball drauf, sodass das Floß und der Körper in den Fluten versanken. Doch es war schon zu spät zum weiterreiten geworden und der Besitzer der Mühle war bereit uns zum Dank für die Hilfe bei ihm übernachten zu lassen, sodass wir in Braunsklamm nächtigten.

\paragraph{26. Ingerimm:}
Auch diese Nacht suchte mich wieder ein schlimmer Alptraum heim. Ich träumte davon, in einer endlosen roten Ebene zu stehen, neben mir ein endlos hoher roter Turm. Irgendetwas verfolgte mich und ich rannte weg. Ich rannte immer weiter, hunderte von Meilen. Dann erreichte ich einen grünen Ort, durchschritt ein grünes Tor und wägte mich für kurze Zeit in Sicherheit, doch dann erwischte es mich und ich wachte auf. Anscheinend war ich nicht der einzige, der diese Nacht schlecht geschlafen hatte. Auch der Zwerg und Temyr waren nicht so gut drauf. Wir frühstückten und setzten unsere Reise fort. Bald trennte sich unser Weg von der Braunwasser und führte uns in eine Seitenschlucht. Als wir jedoch auf das seltsame Holzstück schauten, sahen wir, dass es uns weiter am Fluss entlang führen wollte. Wir entschieden uns jedoch weiter in Richtung Dragenfeldt  zu reiten. Nach einer Weile sahen wir drei Reiter aus einem kleinen Waldstück herausreiten. Sie stellten sich uns in den Weg und es stellte sich heraus, dass dies der hiesige Baron war, der gar nicht wie ein Baron aussah. Er forderte Wegzoll für seinen Abschnitt der Straße was durchaus verständlich war, allerdings war dieser viel zu hoch. Acht Silber von jedem wollte er haben. Nach einer kleinen Diskussion einigten wir uns auf vier Silber pro Person. Doch Calhadril wollte immer noch nicht zahlen. Doch anscheinend hatte der Baron außer seinen zwei Begleitern noch ein paar Untergebene, die im Busch drumherum saßen. Als Calhadril diese bemerkte sah auch er sich gezwungen zu zahlen. Mit erleichterten Geldbeuteln ritten wir weiter. Bei unserem Mittagsmahl in einem Gasthaus stellte sich heraus, was wir alle schon vermutet hatten: Der Baron war der berüchtigte Räuberbaron der Gegend gewesen. Die weitere Reise verlief ungestört, sodass wir am Abend in Salthel ankamen und uns in einem Gasthaus in die Arme von Bruder Boron begaben.

\subsection{Der dritte Teil nach Rezzanjin al’Ahjan}

\paragraph{27. Ingerimm:}
Auch diese Nacht blieben wir nicht von Alpträumen verschont. Diesmal hatte Calhadril einen. Er war anscheinend ein großer Krieger, der Feinde besiegen musste. Doch irgendwann waren alle Feinde tot und er richtete seine Wut gegen sich selber und riss sich das Herz aus der Brust. Das war natürlich begleitet von Schreien seinerseits, sodass wir wach wurden und ihn aufwecken konnten. Nach dem Frühstück machten wir uns schnell auf zu unseren Pferden. Wir ritten gerade über den Marktplatz, als wir in einer Seitengasse eine Menschenansammlung bemerkten. Wir bahnten uns einen Weg durch die Bauern hindurch und mussten den Anblick einer toten alten Frau, die ihr Herz in der Hand hielt und in ihren eigenen Blut lag, ertragen. Sofort tönte es aus der Menge: "Das war die Hexe“. Jemand anderes: "Sie ist bestimmt auch für die Alpträume verantwortlich, die ich nachts habe. Auf verbrennen wir sie!“ \par

Schon bewaffneten sich die Bauern und zogen zur Hexe. Wir ahnten sofort, dass wir nicht die einzigen von Alpträumen geplagten Menschen in der Umgebung waren. So beschlossen wir die Hexe zu beschützten, da sie wohl auf keinen Fall für die Alpträume verantwortlich war, die wir schon in meilenweit entfernten Städten hatten. Die Menge war schon drauf und dran die Hexe zu verbrennen, doch mit gemeinsamen Kräften konnten wir sie von einem fairen Prozess für die Hexe zu überreden. Thoran überredete die Dörfler den Vorsitz für das Gericht übernehmen zu dürfen und sprach die Hexe frei. Später riet ich ihr zur Flucht, da die Dörfler sie sicher für die mit Sicherheit noch folgenden Alpträume verantwortlich machen würden. Daraufhin gab sie mir ein Holzamulett, das wie Temyr herausfand anscheinend mit antimagischen Sprüchen belegt war.\par

Nach dieser Verzögerung machten wir uns auf nach Sichelweg. In dieser Ansammlung von Gehöften gab es leider kein Gasthaus und die Bewohner hatten gerade genug Mittagsmahl für sich selbst, sodass wir ihnen ein paar alte Äpfel abkauften. Wir erfuhren, dass der Schrecken der Tobimora anscheinend vor einem Tag hier gesehen worden war. Mit den Äpfeln im Gepäck ritten wir weiter. Nach gut einer Stunde sahen wir am Wegesrand zwei Gestalten, die sich über einen anscheinend toten Gefährten beugten. Als wir näherkamen riefen sie um Hilfe. Wir ritten zu ihnen heran. Sie erzählten uns, dass sie vom Schrecken der Tobimora überfallen worden sind. Eine Gefährtin von ihnen sei entführt, ihr Anführer, der Tote, getötet worden. Die Geschichte klang plausibel, aber sie verheimlichten uns etwas. Als wir sie aufforderten ihre Handgelenke zu zeigen, weil sie diese hinter ihren Rücken versteckten rannte einer weg, den anderen hinderten wir daran. An seinem Handgelenk war ein Eisenring. Er erzählte uns, dass sie Räuber seien und dem Schrecken der Tobimora aufgelauert waren. Allerdings hatte dieser sie vernichtend geschlagen. Das ganze war ungefähr einen Tag her. Der Schrecken der Tobimora hatte also immer noch einen Tag Vorsprung. Wir ließen die Räuber alleine und ritten weiter. Am Nachmittag erreichten wir Runhag, ein von vier Meter hohen Palisaden geschütztes Dorf. Angesichts der Räuberbanden schien diese Palisade auch nötig zu sein. Wir erfuhren hier, dass der Schrecken der Tobimora uns immer noch ungefähr einen Tag voraus ist. Da Runhag die letzte Siedlung vor Dragenfeldt  war, beschlossen wir hier zu nächtigen. Ich ging mit Ragnos in ein Zimmer. Die anderen drei zusammen in eines.

\paragraph{28. Ingerimm:}
Auch diesmal gab es kein Entkommen von den Alpträumen. Dieses Mal wurden Ragnos und Thoran heimgesucht. Wie sie erzählten, fielen sie unendlich lang und schlugen irgendwann auf den Boden auf. Wir deckten uns noch mit Vorräten ein und ritten dann los. Nach einer Weile bemerkten wir, dass die Gegend ungewöhnlich ruhig und trostlos war. Kaum ein Vogel zwitscherte und es lagen schon braune Blätter auf den Boden, obwohl wir erst Frühling hatten. Nach einer Weile kamen wir zu einer kleinen Klamm. Über sie war eine kleine Brücke gespannt, aber diese war zerstört und stand kaum noch. Temyr beschloss einen Dschinn zu beschwören. Das dauerte eine Stunde und währenddessen ging Thoran Kräuter sammeln.\par

Als das Hexagramm gezeichnet war, beschwor Temyr den Dschinn. Halb Mensch halb Stein baute er die Brücke wieder zusammen und verschwand danach. Sie war richtig kunstvoll geworden. Mittlerweile war Thoran war mit ein paar Heilkräutern in der Hand zurückgekommen. So ritten wir weiter. Wir ritten noch ein gutes Stück, als wir plötzlich Armbrustschüsse hörten und zwei Bolzen vor unseren Füßen einschlugen. Auf einmal schleppte sich eine schwer verwundete Gestalt über den Hügel. Zwei weitere Bolzen landeten neben ihr. Sofort ritten Calhadril und ich über den Hügel, doch dort entdeckten wir niemanden mehr. Als wir zurückkamen behandelte Thoran den Verletzten schon. Es war Delian von Wiedbrück und er war nur knapp dem Tode entronnen. Er wirkte überrascht als er uns erblickte. Auch meinte er, dass er sich wohl in uns getäuscht hatte. Als Thoran ihn verbunden und mithilfe von Schwester Peraine geheilt hatte, schlief er ein. Da wir ihn nicht alleine lassen konnten, nahmen wir ihn mit. Calhadril legte ihn vor sich auf sein Pferd. ar

Nach der Hälfte einer Stunde war auf einer Hügelgruppe plötzlich ein schwarz gekleideter Mann mit einen ebenso schwarzen Pferd aufgetaucht. Sofort sprang ich von meinen Pferd ab. Mit seiner beängstigender Stimme forderte er uns auf, ihm Delian zu überlassen. Wir verneinten und auf einmal hatten uns drei Untote umkreist. Nur Momente später hatte eine gewaltige Flammenlanze Korobars Pferd getroffen und ich bedrängte den Untoten, der vor uns stand mit einer genial geführten Finte, die er leider parierte. Wir wurden zusätzlich von einem Armbrustschützen attackiert. Dann setzte ich einen Ausfall an und nach zwei Streichen lag mein Gegner am Boden. Dann eilte ich rüber zu den anderen, wo Temyr schon arg Probleme bekommen hatte. Mit zwei weiteren Streichen setzte ich auch dem Dasein dieses Untoten ein Ende. Calhadril hatte währenddessen dem Magier mit Zaubern so arg zugesetzt, dass dieser sich wegteleportierte. Jetzt blieb nur noch der Armbrustschütze. Mit großen Schritten rannte ich auf den Schützen zu, wich im Rennen einem Bolzen aus und streckte ihn mit einem gewaltigen Schlag nieder. \par

Schon wieder war der Schrecken der Tobimora uns entwischt. Als wir die Untoten untersuchten entdeckten wir, dass sie die selben Nägel im Kopf hatten wie die Leiche am Mühlrad. Delian von Wiedbrück war jetzt wieder erwacht. Er erklärte uns, dass die Untoten Korobars mal seine Begleiter waren, also die formidablen Sechs. Wir gaben ihm ein wenig zu Essen mit, als er zurück nach Runhag ging. Wir ritten selbstverständlich weiter. Gen Abend wurde die Gegend immer trostloser. Der Boden war mit einer dünnen Schicht aus grauem Staub bedeckt. Die Bäume hatten mitten im Frühsommer kaum noch Blätter und die Äste der Bäume hingen Richtung Südwesten, als ob sie in diese Richtung fliehen wollten. Es war unheimlich still. Statt der für einen Wald typischen Geräusche hörte man nur noch das Schlurfen der eigenen Füße im Staub. Am Abend schlugen wir unser Lager in der Nähe der Herzogenstraße auf.

\paragraph{29. Ingerimm:}
Heute Nacht hatten Calhadril und ich einen besonders heftigen Alptraum. Ich war in einer ewig weiten Sandwüste und ging auf eine Düne. Von dort sah ich nur eine ewig lange trockene Ebene. Ich ging die Ebene entlang. Nach dem ich gefühlte Wochen gewandert war sah ich auf einen Hügel eine schwarze Gestalt. Sie hatte eine Maske auf und ich bekämpfte sie. Sie war genau gleich stark wie ich. Mittlerweile waren wir beide zu Giganten mutiert. Irgendwann konnte ich der Gestalt die Maske herunterreißen und schaute in mein eigenes Gesicht. Wir kämpften ewig so weiter. Jedes Mal wenn ich ihn traf, spürte ich die Schmerzen selber, die ich ihm zufügte. Bei ihm war es genauso. Irgendwann kam Temyr dazu und ich konnte ihn überreden mir zu helfen. Er wurde zum Giganten und gemeinsam besiegten wir den Gegner. Temyr hatte also ein in Liscoms Turm gefundenes Artefakt dazu genutzt in meinen, bzw. Calhadrils Traum einzudringen. Als wir uns gegenseitig ansahen, wirkten wir seltsam verändert: Der Bart sprießte als ob er einige Tage ohne Rasur geblieben wäre. Die Zeit schien schneller zu vergehen als normal. \par

Ein wenig beängstigt und frisch rasiert ritten wir weiter. Mittlerweile war jede Bewegung anstrengend geworden und wir waren ziemlich erschöpft. Die Gegend war mittlerweile noch seltsamer geworden. Wir sahen teilweise seltsam verstümmelte und verkrüppelte Tierkinder herumliegen. Der Boden war jetzt von einer fingerdicken Schicht Staub bedeckt. Die Vegetation war kaum noch vorhanden und alles sah seltsam grau aus. Wir machten kurz halt, um eine kleine Mahlzeit zu uns zu nehmen und bemerkten, dass auch das Brot und die Würste gealtert waren. Die Würste schmeckten als wären sie drei Wochen alt und das Brot war steinhart. Ich bemerkte plötzlich, dass mein Schwert nicht mehr so glänzte und mein Hartholzharnisch machte beunruhigende Geräusche. Das Übel breitet sich also auch auf unsere Waffen aus. Am Nachmittag kamen wir zu einer Stelle, an der ein riesiges Massaker stattgefunden hatte. In der Mitte einer Lichtung war anscheinend ein Kreis mit Planwagen errichtet worden, die jetzt alle abgefackelt waren. Überall lagen tote Menschen, teilweise auf grausamste Weise hingerichtet, herum. Als wir sahen, dass die gut ausgebaute Straße hier endete, schwante uns übles. Hier war anscheinend der Bautrupp der Herzogenstraße auf der Lichtung verteilt. Wir beschlossen die vielen Arbeiter zu begraben. Es dauerte zwei Stunden, um das Grab auszuheben, die Arbeiter hineinzulegen und das Grab wieder zuzuschaufeln. Nach diesem Kraftakt suchten wir uns in der Nähe ein Lager und nächtigten.

\paragraph{30. Ingerimm:}
Auch in dieser Nacht waren die Alpträume wieder präsent. Wer ihn über was hatte weiß ich nicht mehr. An diesen Morgen waren unsere Bärte noch länger und ich beschloss meinen erst mal nicht zu rasieren. Nachdem wir eine kurze Zeit ritten und feststellen mussten, dass wir noch viel fertiger waren, als am Tag zuvor, hörten wir ein paar Kleinkinder schreien. Die Schreie kamen aus Nordosten und so folgten wir ihnen. Es stellte sich heraus, dass ein Lager von extrem alten Menschen ein wenig abseits des Weges lag. Es wurde von Wachen bewacht und wir warteten vor dem Lager, bis der Anführer zu uns kam. Ein Mann um die sechzig grüßte uns. In einem Gespräch stellte sich heraus, das dies die ehemaligen Bewohner von Dragenfeldt  waren, nur um Jahrzehnte gealtert. Die Jüngsten, eigentlich noch kleine Kinder, waren 25 Jahre alt. Der älteste war gerade 31. Sie gaben zu, dass sie die Tsageweihte verbrannt hatten, aber die Probleme begannen schon früher. Die Geweihte brachte den Feldsegen in einer komischen Sprache dar. Anfangs war alles wunderbar. Auf den Feldern wucherte es regelrecht, doch dann schmeckte die Ernte fade und die Rinder hatten Fehlgeburten. Die Geweihte wurde verantwortlich gemacht. Ohne die Geweihte lief es aber nicht besser, sodass die Bewohner gealtert beschlossen den Ort zu verlassen. Drei Tage waren sie schon unterwegs, aber erst eine Tagesreise von Dragenfeldt entfernt. Das war die gute Nachricht für uns. \par

Schon am Abend könnten wir Dragenfeldt  erreichen. So wie sich die Dragenfelder sich zum Aufbruch bereitmachten, ritten wir los. Lange hielten die Pferde das aber nicht durch. Wir töteten eines, um das Fleisch als Proviant zu haben, die anderen ließen wir frei. Wie der Wind sausten sie los gen Südwesten. Mittlerweile erinnerte nichts mehr an die einst vermutlich grüne Landschaft. Der Staub lag spanndick über dem Boden und die gesamte Landschaft war grau in grau. Wir speisten am Mittag etwas, um bei Kräften zu bleiben, doch an jenen mangelte es uns schon lange. Ein wenig erinnerte mich die ganze Situation an einen der Alpträume, die ich hatte. Die Landschaft war genauso grau, doch hier gab es kein Grün, keine erkennbare Hoffnung. Wir schleppten uns so gut wir konnten durch die Landschaft, doch wir waren so gut wie am Ende. Doch da erklommen wir den letzten Hügel vor Dragenfeldt  und schauten auf den Ort hinunter. Er war praktisch nichts mehr übrig. Die meisten Häuser waren zerfallen oder es standen nur noch ein paar Holzbalken. Doch da erblickten wir den Tsatempel. Er stand noch und auch die ihn umgebende Wiese war noch vorhanden. Wir rannten auf den Tempel zu und erklommen erschöpft die letzten Stufen.  \par

Als wir hineingingen erstrahlte er in einer wahren Farbenpracht. Alle Regenbogenfarben waren vertreten und wir waren förmlich geblendet von gleißenden Licht. In Wahrheit war sogar der Tempel etwas zerfallen, aber im Vergleich zu dem, was wir gesehen hatten war er wunderschön. Wir wollten zuerst die Küche und etwas zu essen finden, stolperten aber über das Tagebuch der Geweihten, welches als wir es lasen fast alles erklärten. Die 200 Seiten legten uns die Geschehnisse offen. Die Geweihte war vor fünf Jahren damit beauftragt worden den Tsatempel in Dragenfeldt  zu übernehmen. Am Anfang war sie froh darüber, doch dann wurde ihr die Arbeit zu eintönig und sie bekam leichte Zweifel an ihrer Göttin. Sie hatte sich auf der Suche nach der wahren Gestalt Tsas sich mit einem seltsamen Fremden angefreundet, der sie alles über den früheren echsischen Tsakult lehrte. Diesen Frühling sprach sie in Abwesenheit des Fremden den echsischen Erntesegen und stürzte das Dorf ins Unglück. Der Fremde war ein seltsam anmutender Tulamide. Immer mit viel Duftwasser umnebelt, die Kapuze tief ins Gesicht geschoben und am gesamten Körper bekleidet schien er wahrlich seltsam zu sein. Er wohnte im Drachenturm und er ist es den wir finden und vernichten müssen, um das Übel zu beenden, für das er laut der Tsageweihten verantwortlich ist. Denn er hatte sie zum Sprechen der Liturgie ermutigt und er ist vermutlich der einzige der sie aufheben kann.

\subsection{Der Weg nach Dragenfeldt aus der Sicht des Ragnos}

Wir verlassen Runhag um weiter gen Dragenfeldt  zu reiten, doch schon bald kommen wir an eine Kluft 8 mal 8 Schritte so schätze ich, allerdings war ich nie besonders gut im Schätzen. Jedenfalls ist die Steinbrücke, die die Klamm einmal überspannte zerstört worden und ich werde es nach meinem Albtraum im Leben nicht wagen sie ohne eine anständige Brücke zu überqueren. Zum meinem Glück habe wir ja Magier unter uns, wie Temyr, der es schafft nach einiger Zeit, während Thoran Pflanzen sammeln ging, einen Erzdschinn zu rufen, welcher uns mit Freuden eine gigantische
eindrucksvolle Brücke aus dem Fels wachsen lässt. Ich muss dabei hinzufügen, dass diese Brücke wohl eher eine Prachtbaute der Natur ist, da sie unmöglich von Menschenhand geschaffen werden könnte. Nach diesem Hindernis reiten wir weiter nachdem Toran mit ein paar Kräuter, welche, so kann ich selbst mit meinen Augen von hier erkennen, nicht sonderlich spektakulär sind, aber wahrscheinlich kann der Geweihte sie irgendwie für Heilungen oder derlei Dinge gebrauchen.\par
Schon bald erreichen wir einen Hügel auf dessen Bergkuppe ein Mann erscheint, der auf uns zu rennt, doch vor unseren Augen zusammenbricht. Es ist Delian von Wiedbrück. Der Peraingeweihte macht sich sofort an die aufwendige Heilung des KGIA Agenten, der von mehreren Eisenbolzen schwer getroffen ist. Als er wieder außer Lebensgefahr ist legt Calhadril ihn sich quer über sein Pferd und wir reiten Vorsichtig den Hügel hinauf. Ein fauler Gestank schlägt uns entgegen und wir erblicken eine hünenhafte tiefschwarz gewandte Gestalt thronend auf einem Pferd, mit glühend roten Augen, verächtlich den Kopf nach uns drehend. Doch Calhadril hat seine Zeit genutzt und schleudert aus seinem Stab einen gigantischen Feuerball auf das Pferd, welches binnen Sekunden in Flammen aufgeht.\par
In dem Rauch erscheint nun aufrecht stehend grausam lächelnd der Schrecken der Tobimora, eine Hand auf uns deutend. Da bemerke ich dass wir umgeben sind von mehreren Gestalten, Untoten.\par
Ich erlege einige von ihnen mit grausamen Schüssen in ihre Köpfe in den etwas silbern schimmert.\par
Ich bekomme nicht viel mit von dem Kampf meiner Gefährten, versuche ihnen aber bestmöglich zu Helfen, als mir ein Schwert die Seite aufreißt und eine tiefe Wunde hinterlässt. Doch ich beherrsche mich und erlege einen um den anderen, doch als der letzte fällt und wir den Schrecken der Tobimora angreifen wollen, ergreift dieser feige die Flucht mit den Worten "Transversalis“.\par
Thoran verbindet die Verletzten von uns bestmöglich und der KGIA Agent kommt auch wieder zu sich er erzählt uns, dass er Korobar schon länger verfolge und er etwas von einem großen Ritual gehört habe, wo er viel Opfer bräuchte. Das würde gut passen zu den Entführungen.\par
Wir lassen Delian von Wiedbrück zurück, der wieder stark genug ist um alleine nach Runhag zurückzugehen.
Auf dem Weg weiter Richtung Dragenfeldt  kommt uns alles kraftloser vor wir, die Bäume, die Tiere, die Pflanzen, einfach alles.Es ist sehr seltsam an diesem Ort.Als wir um eine weitere Biegung der Straße reiten erblicken wir oder vielmehr riechen wir die niedergebrannten Zelte von den Arbeitern, welche zuvor wohl die Straße ausgebaut haben. Alles liegt voller Leichen zerstückelt, verbrannt, aufgeschlitzt oder von Bolzen durchlöchert. An diesem Schauplatz können wir nicht einfach vorüber reiten, wir beginnen damit ein großes Grab auszuheben und die Leichen und das was davon übrig ist zu bestatten. Nach einigen Worten des Perainegeweihten ziehen wir weiter nun auf einem Pfad.\par
Die Natur wird nun zunehmend seltsamer, die Bäume scheinen sich zu bewegen und fliehen zu wollen, aber wovor ? Es wird immer dunkler und ich mache mich mit meinen Gefährten auf die Suche nach einem sicheren Platz für ein Nachtlager. Doch meine Gefährten sind nicht sehr überzeugt von meinem Vorschlag sich vor einer Höhle zu Ruhe zu begeben aus der so meinen sie ein Knurren und röhren kämme, aber die haben ja auch keine Ahnung, naja ich lasse mich dann doch überreden woanders zu nächtigen. Wir teilen Wachen ein, ich übernehme die letzte Wache in der Morgendämmerung, meine Lieblingstageszeit, in der das Wild einem praktisch vor den Bogen springt. Da bemerke ich wie sich Calhadril und Rezzanjin am Boden wälzen, sich gegenseitig würgend. Ihre Augen aber sind noch geschlossen als würden sie träumen. Ich stürze mich auf sie und versuche sie auseinanderzuzerren, dabei wecke ich meine andern Gefährten mit einem Ruf aus dem Schlaf. Als ich es schaffe die Kämpfenden zu trennen, kippen sie zu Seite und bleiben reglos liegen, etwas rotes sickert aus ihnen hervor, doch der Peraingeweihte kann keine Wunden erkennen.\par
Sie scheinen zu sterben, da fällt Temyr der Traumring ein und streift ihn über. Auch er kippt zur Seite. Der Geweihte und ich können nichts tun, als alle Drei plötzlich wieder aufwachen und Calhadril als auch Rezzanjin, nachdem sie sich etwas erholt haben berichten, dass sie geträumt hätten sie würden gegen sich selbst kämpfen und kein Ende schien in Sicht, da sie sich nicht selber besiegen konnten, bis Temyr erschien und den Kampf beendete.\par
Als es heller wird brechen wir auf und bemerken in den ersten Sonnenstrahlen unseren ungewöhnlich starken Bartwuchs, welchen ich mit meinem Jagdmesser etwas zurechtstutze. Die Umgebung wird zunehmend älter so wirkt es und alles scheint sich von dem vor uns Liegenden zu fliehen, nicht nur die Tiere, selbst die einzigen jungen Knospen eines alten fast verrottenden Baumes weisen in die andere Richtung. Nach einigen Stunden bricht eines der Pferde zusammen sie
sind zu erschöpft um weiter zu gehen, wir beschließen sie zurückzulassen und eines als Nahrung zu schlachten. Ich bekomme die grausame Tat zu Teil ein Pferd zu schlachten. Als ich auf das am Boden liegende Pferd zu schreite, sehe ich das Leid in dessen Augen und ich erlöse es mit einem schnellen Schnitt. Ich will euch die Beschreibung des Zerlegens an dieser Stelle ersparen, doch so viel kann ich euch sagen schön ist was anders. Bevor wir weiter schreiten gehe ich zu meinem treuen Pferd, Lim'alagos, und befreie es von seinen Lasten und den Zügeln und mit den Worten
" northa na milbar sedyr meldir .“ und es galoppiert davon, in die Richtung aus der wir kamen.\par
Wir holen die Karte hervor um zu sehen wo wir sind, da bemerken wir etwas seltsames, die Karte scheint Augenscheinliche Jahrzehnte alt und auch mein Messer sieht rostig und etwas Stumpf aus.\par
Trotzdem ziehen wir weiter und müssen schon bald ein Lager aufschlagen, als wir endlich einen geeigneten Platz finden fühle ich mich etwas schwach und ausgelaugt, doch Thoran sieht man seine Kraftlosigkeit am ganzen Körper an und unsere Bärte sehen aus als ob sie heute morgen gar nicht gestutzt worden wären. Mir wird dieser Ort immer unheimlicher, etwas stimmt hier nicht.\par
Als wir am nächsten Morgen Aufwachen erzählen uns Temyr und Thoran von einem neuem Albtraum mit einer Wüste, die genauso grenzenlose war wie ihr Hass. In diesem Moment hören wir Kinderschreie nicht weit von uns und wir packen hastig zusammen, um zu sehen wer hier an diesem verlassenem Ort noch lebt. Als wir eine Hügelkuppe erreichen bietet sich uns ein bizarrer Anblick, ein Lager voller Menschen, doch es sind nur alte Menschen und die von mittleren Alter
krabbeln auf dem Boden herum und verursachen die von uns gehörten Kinderschreie. Ein 80 jähriger Man so schätze ich kommt auf uns zu, als er uns bemerkt. Anscheinend hat er das Kommando über diesen Haufen von alten und sich merkwürdig benehmenden Leuten. Er erklärt uns, nachdem wir uns vorgestellt haben, dass sie die letzten Bewohner von Dragenfeldt   seien und sie es waren ,die die Tsageweihte verbrannten, in der Hoffnung dieser böse Zauber hier habe ein Ende, da sie dachten die Geweihte habe das durch seltsame Rituale bewirkt. Er erklärt uns auch, dass hier seit die Ernteweihe von der Tsageweihte gesprochen wurden, alles sehr schnell altere, so seien die augenscheinlich 30 Jährigen noch Jung-geborene und er eigentlich gerade mal 35 anstatt 80. Diese Gegend ist verflucht mit schwarzer Magie.In dem Lager erblicken wir auch einig aufeinander gestapelte Leichen, denen die Eingeweide herausgerissen wurden, als wir den Mann darauf ansprechen berichtet er von den Träumen die bei ihnen die Anzahl, der überlebenden immer weiter verringert. Und das obwohl bei ihnen sowieso schon immer wieder Leute verschwinden wie seine Tochter. Wir bedanken uns für die Auskünfte und ziehen weiter, dar wir den Menschen nicht helfen können. Endlich erreichen wir Dragenfeldt  in der Stadtmitte das verkohlte Kreuz, wie ein Mahnmal aufragend. Alles ist verfallen als habe hier schon seit Ionen nichts mehr gelebt. Da erblicken wir den Tsatempel, der wie eine Oase in der Stadt aufragt. In diesem Moment huscht etwas in meinem Augenwinkel an uns vorbei und ist verschwunden, da bricht plötzlich der Boden vor uns auf und rotes Blut sickert aus dem Boden hervor. Wir flüchten und mit letzter Kraft in den geweihten Tempel, wo der Perainegeweihte, der sich nicht mehr auf den Beinen halten kann im Gebet an die Götter vor dem Altar zusammenbricht. Wir schleppen uns mit den letzten Kräften hoch in die Gemächer der nunmehr verbrannten Geweihten und finden ihr Tagebuch welches Temyr vorzulesen beginnt. Daraus wird uns einiges ersichtlich. Doch nun will ich mich zur Ruhe legen und meinen müden Körper eine Pause gönnen.\par

\subsection{Die Entstehung der Weidener Wüstenei nach Temyr ibn Sahid}

\paragraph{30. Ingerimm}, sofern ich mich nicht verrechnet habe – so lange schon ist mir jegliches Gefühl für den Fluss der Zeit vergangen.\par

Nach der Lektüre des Tagebuchs von Schwester Laniare saßen wir alle benommen in der kleinen Küche des Tempels, getroffen von der Wucht der Worte und Begebenheiten, von denen dieser schmale Band Zeugnis tut. Gleichwohl ich mich nicht als gläubigen Mann bezeichnen möchte, so erscheint mir der Frevel doch unvorstellbar – ich wage nicht zu denken, was Thoran in diesem Moment empfand. Mit tauben Gliedern verschafften wir uns Zugang zum Schlafzimmer – beinahe hätte mich der zweite Schlag innerhalb weniger Augenblicke getroffen: Achtlos auf die Pritsche geworfen lag tatsächlich eine Ausgabe des Liber Zhammoricam per Satinav, eines der kostbarsten, seltensten und gefährlichsten Bücher überhaupt. So unscheinbar der in Echsenleder gebundene Folio auch wirken mag, der Inhalt ist doch von zeitloser Brisanz. Aus meiner Studienzeit ist mir noch einiges über dieses verschollen geglaubte Werk bekannt: Es umfasst eine Sammlung altechsischer Riten und Bräuche und gibt einen Überblick des saurischen Pantheons. Weiter wollte man damals nicht ausführen – vielleicht liegt es daran, dass die von Laniare beschriebenen Echsengötzen in Name und Wirken unseren Zwölfgöttern so stark ähneln. Doch in der Schlafkammer wurden wir noch eines weiteren Buches gewahr – des Codex Sauris. Dieses Werk ist nicht weniger sagenumwoben als der LZS und die Wahrscheinlichkeit, beide gleichzeitig in einem Raum zu finden, ist verschwindend gering. Die bereits vergilbten Seiten waren von zahllosen, handschriftlichen Kommentaren übersät; vielleicht als ergänzende Informationen für die Schwester. Des weiteren fanden wir auf dem Kissen einige Manuskripte bezüglich der Tempelabgaben, dito seitenweise Register für Formulare.\par

In einem schulterhohen Holzverschlag entdeckten wir schließlich drei Ampullen mit grünlich schimmerndem Inhalt, welche in einem Halter aus Eibenholz ruhten. Ich unterzog die Flüssigkeit einer oberflächlichen alchimistischen Analyse, doch meine Mühen gaben mir wenig Aufschluss über die Wirkung des Trunks. Tatsächlich kam ich zu überhaupt keinem Ergebnis. Thoran hingegen spürte im Umfeld der Reagenzgläser eine zwölfgöttliche Präsenz, vielleicht sind es heilige Tränke der Göttin Tsa. Schweren Herzens zogen wir uns in die Kapelle zurück, denn die Gewissheit, diesen Ort in Bälde verlassen zu müssen, schlug uns aufs Gemüt. Thoran, den es mit am schlechtesten erwischt hatte, sank verzweifelt vor dem Altar zu Boden und schrie ein Gebet zu den Zwölfen in die umfassende Stille hinaus. Da geschah etwas ganz und gar erstaunliches: Ein Lichtstrahl durchbrach die Bleiglasfenster und tauchte den Raum in warmes, goldenes Licht. Für einen Augenblick verflog die Angst in unseren Herzen und der Schmerz aus unseren Gliedern, als das Licht über uns hereinbrach wie Wasser. Als die Feuerlanze langsam verblasste, war es, wie aus einer Trance zu erwachen: Die Götter hatten uns nicht verlassen…

\paragraph{1. Rahja} Wie jedes Jahr pflegt man heute das Fest der Freuden zu feiern, doch der Gedanke an Fest und Tanz erfreut mich ganz und gar nicht.\par

An den Gesichtern meiner Freunde sah ich, dass auch auf ihnen der Schatten der kommenden Ereignisse lastete. Die Hoffnung des Vorabends war im Laufe der Nacht dem zerstörten Boden des Diesseits gewichen. Mürrisch und schweigsam verzehrten wir unser karges Mahl, als Calhadril einen weiteren Bogen Papier entdeckte, der zwischen den einzelnen Pergamenten unserem müden Blick entgangen war. Allerdings waren die Zeilen nicht gerade dazu angetan, unsere Stimmung zu heben, hielten wir doch, es kann kein Zweifel bestehen, eine Abschrift der Orakelsprüche von Fasar in den Händen, welche unter anderem von der Rückkehr einer urbösen und mächtigen Entität künden.\par

Der Umstand, dass man uns im restlichen Mittelreich bereits für die Einsicht in den LZS und den CS bei lebendigem Leib verbrannt hätte, ringt der ganzen Situation eine tragische Ironie ab: Ich brauche wohl nicht zu erwähnen, dass auch die Orakelsprüche von Fasar auf dem Index Librorum Prohibitorum zu finden sind, höchstwahrscheinlich auf den oberen Rängen. Allerdings fehlt es mir an Weisheit, um Schlüsse aus den Prophezeiungen ziehen zu können. Der Text berichtet von Legionen des Blutgottes, welche ins Herz des Greifen stoßen würden, von einem Toten, der den Toten beschwört – trotz meiner mangelnden Kenntnisse auf diesem Gebiet scheint mir ein solcher Zauber schlechterdings unmöglich – vom Ersten der Sieben Boten und von Seiner Macht. Diese Worte erfüllten mich mit kalter Angst, zu sehr hatte ich noch Mutter Linai in Erinnerung, wie sie die Vision vom Ende Laniares empfangend, vor Schmerz schreiend auf ihrem Stuhl zusammensank.\par

Im Morgengrauen brachen wir schließlich zur Festung Dragentod auf. Schweigend marschierten wir durch die desolate Landschaft, über der sich drohend der schwarze Turm ben Seyschabns erhob. Sobald wir den schützenden, geheiligten Boden des Tempels verlassen hatten, zehrte abermals der böse Zauber des Rituals an unseren Kräften; Schritt für Schritt entzog er uns unsere Energien. Gleichwohl es nur ein paar Stunden dauerte, bis wir vor den drohenden Mauern der alten Feste standen, kam es mir doch so vor, als wären es Jahre gewesen. \par

Gegen Ende stolperten wir mehr unserem Ziel entgegen, während die Nacht über uns hereinbrach. Das Praiosgestirn war schon fast hinter den zerklüfteten Felsen der Sichel verschwunden, als wir endlich vor einem verrotteten Torbogen standen, der in einen verwinkelten Innenhof führte. Das mächtige Bollwerk hatte den Zorn des Verfalls am stärksten zu spüren bekommen: Sämtliche Steine waren brüchig und von rostroten Wurzeln aufgebrochen, das Metall der Türangeln war stumpf und weich geworden. Von dem einstigen Tor war nur noch kalter Staub zurück geblieben. Vorsichtig näherten wir uns dem verfallenen Portal, denn wir hatten leise Geräusche vernommen.\par

Als wir noch näher traten, sahen wir auch ganz deutlich, was die Ursache derselben war. In vielleicht zwanzig Schritt Entfernung drehte ein bleiches, vermodertes Gerippe seine Runden. Stofffetzen hingen von seinen blanken Knochen, vereinzelt waren Teile einer Ledergarnitur zu erkennen und in der Rechten hielt es ein schartiges Schwert. Es hatte uns noch nicht bemerkt, da wir durch eine vorschießende Wand verborgen waren, allerdings konnten wir so auch nur einen kleinen Teil des Hofes einsehen. Der Zwang zu einer schnellen Entscheidung lag schmerzhaft auf uns, wir mussten eine Wahl treffen, und zwar so schnell wie möglich. \par

Ohne die Gewissheit um weitere Gegner gingen wir stumm zum Angriff über und Rezzanjin stürzte sich mit gezogener Klinge auf den wandelnden Leichnam. Doch er wurde in seiner Angriffsbewegung jäh unterbrochen, als wir eines weiteren Skelettes gewahr wurden, welches einen kompletten Eisenharnisch trug und mit einem Zweihänder bewaffnet war. Wir hatten gerade den Fokus auf diesen mächtigen Gegner gelegt, als das Portal des Turmes mit unglaublicher Wucht aufflog. Zwei weitere Skelette stürzten heraus, begleitet von einer ganz in schwarz gehüllten Gestalt: dem Schrecken der Tobimora. Die Gestalt führte beide Hände zum Kopf und zog mit einem Ruck die Kapuze aus dem Gesicht. Eisiges Grauen befiel uns, als wir in eine grausige Fratze eines alten Bekannten blickten. Es war Markalm, der sinistere Schwarzmagier aus dem Turm von Liscom  von Fasar, welchen wir seit langem für tot gehalten hatten. Tatsächlich ähnelte sein Antlitz mehr einem Toten als dem eines Menschen: In blutigen Fetzen hing die Gesichtshaut von seinen hervortretenden Knochen herab; seine roten stierenden Augen funkelten und er lachte schaurig und anhaltend. Der Kampf dauerte gewiss eine eine beinahe eine Stunde, wobei sowohl Markalm als auch seine Diener ihr gerechtes Ende fanden. Unser Blutzoll aber war auch beachtlich, trugen wir doch zahlreiche schwere Wunden davon, und Calhadril und ich mussten unsere gesamte magische Kraft aufwenden. \par

Reichlich geschwächt betraten wir schließlich den Turm des Grauens. Überall an den kalten Steinwänden fanden sich schwarz-rote Banner mit dem Siegel der Thargunitoth, der Präzeptorin der heulenden Finsternis und Herrin der Nekromantie. Schaudernd erkundeten wir die oberen Etagen, wobei wir stets acht auf Fallen geben mussten. In einer Art Schrein befanden sich zwei Miniaturen aus schwarzem Stein, eine stellte offenbar eine Echse dar, während die andere das Bildnis eines Drachen mit dreizehn Hörnern war – Satinav, der Wächter der Zeit. Mit wild pochendem Herzen begaben wir uns anschließend in die unteren Geschosse des alten Turms. Dort entdeckten wir eine kupferne Falltür, die durch ein Pentagramm gesichert war. Das Siegel nannte nur einen Namen in Zhayad: Sordul, den säuretriefenden Wächter. Da erinnerte sich Ragnos an den Entschwörungsring aus dem Turm des Liskom von Fasar, und so umgingen wir auch dieses Hindernis.\par

Unter der Falltür erstreckte sich ein düsteres Kellergewölbe. Der ganze Raum wurde beherrscht von einer tiefroten Kuppel, unter der endloser grauer Nebel waberte. In weiter Entfernung sahen wir schwach die Umrisse eines Tridekagramms aus hellem Metall. Calhadril und ich wechselten bedeutungsreiche Blicke, wussten wir doch beide, dass unter uns ein Durchbruch in die Welt zwischen den Sphären lag: Der Limbus erstreckte sich unendlich weit zu unseren Füßen, nur begrenzt durch einen magischen Schutzzauber. Durch den Einsatz unserer Zauberstäbe gelang es uns, den astralen Schild zu durchbrechen und wir tauchten ein in den Limbus. Doch wir spürten keinen festen Grund unter unseren Füßen, nur ein ewiges Gefühl des Fallens, während wir doch nicht fielen. Die Leere war vollkommen, von allen Seiten umgab sie uns, hüllte uns in ihre grauen Wogen. Kein einziges Geräusch war zu hören, nur fortlaufend die Stille zwischen den Welten. Wie wir uns dem Beschwörungszirkel nähern sollten, war mir schleierhaft, denn es gab keinen Widerstand, an den ich mich hätte klammern können. Umso überraschter war ich dann als wir den Ritualplatz immer näher kommen sahen und uns schließlich direkt im Zentrum befanden – ich hatte nie das Gefühl gehabt, mich überhaupt zu bewegen. Silbriges Licht erhellte das Tridekagramm, an dessen Enden sich dreizehn Menschen in unvorstellbaren Qualen wanden, ihre Münder weit aufgerissen und in ihren Augen der Ausdruck von überirdischem Schmerz. Doch in der Mitte des Beschwörungskreises schwebte zwei Schritt über dem Boden ein weiterer alter Bekannter, den wir ebenso totgeglaubt hatten wie Markalm: Liskom von Fasar. Gekennzeichnet vom Tod durch das Drachenfeuer war er nicht mehr als eine Ansammlung von Knochen, die durch Sehnen und seine Kutte zusammengehalten wurden. In seinen Händen hielte er einen rot blitzenden Kristall. \par

Das Siegel der Thargunitoth schmückte diesen Turm nicht umsonst. Liskom hatte sich über seinen eigenen Tod hinaus überlebt – und von diesem erholte er sich erschreckend schnell: Während wir noch staunend das gesamte Ausmaß des Rituals zu erfassen suchten, gewann Liscoms Gestalt an Substanz, Muskeln sprossen aus seinem Leib, die Hautfetzen strafften sich, die Züge wurden fester und definierter. Sein gesamter Körper wurde von Sekunde zu Sekunde kräftiger und jünger, doch zu welchem Preis! Die gefesselten Opfer begannen, sich vor unseren Augen aufzulösen, je gesünder der Schwarzmagier wurde, desto mehr zerfielen sie. \par

Mit ungezügelter Wut fielen wir über Liskom her, durchbohrten ihn, schlugen ihn, sprachen Zauber über ihn, löschten seine beleidigende Existenz aus, bis wir all seine Lebenskraft endgültig ausgemerzt hatten. Da hielten wir in unserer Raserei inne: Das Ritualgefäß war durch unser Handeln zerstört worden, doch das Ritual selbst war noch nicht gestoppt. Es war eine Ahnung von Hass, der die Kontinuität des Ritus ermöglichte, ein kalter, endgültiger, reiner Hass. Uns blieb nichts anderes übrig, als den letzten Schritt zu tun und die Opfer von ihrem Leid zu erlösen. Rezzanjin tat es mit seinem Schwert, und nur durch einen Zauber konnte ich ihn davon abhalten, alle Dreizehn niederzuschlachten – immerhin zwei konnten wir mit den Tränken aus dem Tsatempel retten, die anderen bettete der Tod zur Ruhe. Kaum war der letzte von seiner Pein befreit, da durchströmte uns ein neues Gefühl, kein Hass mehr, sondern Macht. Grenzenlose Macht. Macht, so gewaltig, dass sie die Festen der Welt erschüttert – und unser Ende bringt. Ich wusste es, wir wussten es alle, denn unser Leben zog an unseren Augen vorbei, in allen Einzelheiten, die guten wie die schlechten Tage. Seltsamerweise erfüllte uns der nahe Tod nicht mit Angst, sondern mit Befriedigung. Es war das wohlige, triumphierende Gefühl, etwas zum Ende gebracht zu haben. Dann war alles schwarz, denn Boron breitete seinen Mantel gnädig über uns aus…

\paragraph{Irgendwann im Rahja}
Als ich schließlich erwachte, wurde ich vom strahlenden Licht eines neuen Tages begrüßt. Ich blickte auf. Kein Zweifel – dass hier war der Tsatempel von Dragenfeldt . Friedliche Stille umgab uns, so als hätte sich seit unserem Aufbruch nichts verändert. Noch immer hatte ich dieses Gefühl der Schwere in meinen Gliedern, immer noch dieses Gefühl des Alterns. Allerdings waren wir nicht allein. Auf den Altar gelehnt stand Delian von Wiedbrück, der KGIA-Agent, vor uns und betrachtete uns mit großer Sorge. Offenbar waren wir seit unserem letzten Zusammentreffen um Jahre gealtert – daher dieses Gefühl. Delian berichtete uns, wie er nach seiner Genesung unserer Spur gefolgt war und uns schlafend im Tempel gefunden hatte. Seinerseits erzählten wir von unseren Erlebnissen, die er mit ungläubigem Staunen quittierte. Nachdem wir unsere Sachen gepackt und den Verlust unserer Habe verschmerzt hatten (wir alle bis auf Rezzanjin) machten wir uns auf den Rückweg nach Anderath, um dem Inquisitor Bericht zu erstatten.\par

Lange haben sie über diese schrecklichen Ereignisse geschwiegen, sie waren sich selbst nicht sicher, was sie bedeuten sollten. Erst durch lange Gespräche mit Meister Ignisfulgur konnte ich verstehen, was damals geschehen war. Diese Ereignisse prägten ihn und seine Gefährten maßgeblich. Wichtig ist vor allem, dass du verstehst, wieso Borbarad zurückkehren konnte und welche Auswirkungen auch nur sein Geist auf sein Umfeld hatte.\par



\chapter{Grenzenlose Macht}

\section{Geleitwort}
Grenzenlose Macht, größte Chronologische Katastrophe für meine weitere Planung in der Kampagn\par
Wie ihr sicher wisst, ist dies der dritte Teil der Kampagne, der bei uns an zweite Stelle gerückt war. Welche Auswirkungen das ganze haben sollte, war mir bis dahin noch nicht bewusst, aber sie waren gigantisch. Inneraventurisch war das ganze gar kein Problem, Grenzenlose Macht war damit einfach der erste Versuch Borbarads an einen Körper zu kommen, irdisch hat es mich dennoch manchmal in Verzweiflung gestürzt. Noch bis La Tzoumaz gab es Fragen über die Zeichenreihenfolge, denn der Schlangenreif war ja bis zum Ende nur ein Übergangszeichen, welches dann auch noch an falscher Stelle kam.\par
Grenzenlose Macht war und ist immer noch eines meiner allerliebsten Abenteuer der Kampagne. Schön und stimmig, lokal sehr begrenzt, weswegen man einige NPC'S toll entfalten konnte. Schade ist eigentlich nur, dass der Detektivplot  nicht aufgelöst werden sollte, denn ansonsten wäre das Abenteuer ja vor dem epischen Finale vorbei. Aber das haben die Abenteuer der Kampagne ja leider so an sich.\par
Zweitens war dieses Abenteuer wohl Schuld an einer Entscheidung, mit der ich immer wieder gehadert habe, nämlich die, Victor in die Gruppe aufzunehmen. Keine einfache Entscheidung, die mir tatsächlich nicht aus den bekannten Gründen schwer im Magen lag, sondern aus einem ganz anderen. Wir haben hiernach die Schwelle einer manage-baren Gruppengröße überschritten, was für viele Lehrläufe in den späteren Abenteuern mit verantwortlich ist. Solltet ihr jemals den Meisterposten übernehmen: Nehmt nie mehr als 4 Spieler in eure Gruppe auf, nur im allergrößten Notfall fünf. Sechs ist die Urkatastrophe\par
Trotzdem, ohne diese Entscheidung, die damals glaube ich sehr autokratisch von Philip und mir getroffen wurde, wäre unsere Runde nicht das was sie heute ist, und unsere WG wäre auf jeden Fall ein ganzes Stück langweiliger geworden. Ich bereue die Entscheidung auf jeden Fall nicht, auch wenn ich sie wahrscheinlich im Nachhinein anders getroffen hätte (zwei G7 parallel, man hat ja sonst keine Hobbys :-).\par
Zurück zum Abenteuer: \par
Viele Namen aus diesem Abenteuer, welches wir ja nur mit der Hälfte der späteren Kerngruppe gespielt hatten, waren mir komplett entfallen. Ich weiß noch, dass ich den Namen von Robins Charakter glaube ich erst in der dritten Runde beherrschte, Victors Tagebücher stellen das erheiternd nach. Aber auch der Horasier bereitete mir Bauchschmerzen, denn nach nur drei Tagen Beschäftigung mit dem Hintergrund wusste Victor mehr über das Horasreich, als ich jemals wissen würde. Das hätte für mich eine rote Flagge sein sollen, aber ich dachte mir, diese Hintergrundbesessenheit legt sich bestimmt wieder. Hmm.... Außerdem war das Horasreich für die Kampagne ja auch nicht sonderlich relevant, ich hatte mich dort nur teilweise eingelesen. Als Arngrimm von Ehrenstein mir dann aber große Lücken in meinem Wissen um das tobrische Herrscherhaus offenbarte, da fing ich an mich zu fürchten, denn immerhin kannte ich mich dort ja, meiner Meinung nach, zu genüge aus.\par
Die Spielabende sind mir trotz, oder gerade wegen alle dem als sehr kurzweilig und schön in Erinnerung geblieben. Außerdem startete ich meinen ersten und bisher erfolgreichsten Versuch in pawlowscher Konditionierung. Sollte Philip euch jemals auf die Nerven gehen, spielt "Epilogue – Leaving Grasse“ aus dem OST zu \emph{Das Parfüm} ab. Bei Bedarf kann ich dieses Stück gerne weitergeben.\par

Meine liebsten Momente (sonderbarerweise ist mir wenig so direkt in Erinnerung geblieben):\par

\begin{itemize}
\item "SCHWEIG, Maraskaner“ Ucurian Jagos Catchphrase
\item Victor arbeitet einen gigantischen Hintergrund zu seinem Charakter aus. AM ersten Spielabend mit Arngrimm war alles, was zu seinem Charakter  gesagt wurde: "Arn“
\item Ein toller Abschlusskampf gegen den Quitslinga auf der Turmspitze
\item Offene Münder bei der Beschreibung des Tals der Elemente
\end{itemize}

\begin{flushright}
Claas Völcker, Darmstadt, 2.2.2015
\end{flushright}



\section{Die Tagebücher}


\subsection{Aus dem Tagebuch des Cavalliere Cusimo Villars}

\paragraph{Ende Praios / Anfang Rondra 1016}
Ich war schon einem Monat im diesem Kaff ,was die Einwohner Stadt nennen gefangen, keine Ahnung was Timor hier will, jedenfalls befahl er mir hier zu bleiben (Was hat er für einen Narren am Norden gefressen, Al'Anfa war viel amüsanter).
Ich stand wie immer spät auf, da man hier nichts Gescheites machen kann.\par
Zur Zeit der 2.Praiosstund ging ich ins Gasthaus “Drei Kronen“. Dort saßen schon ein paar der Greifenfurter und ein nostrianischer Weißmagier (Sein Name entfällt mir dauernd). Kurz nach mir kamen vier Personen in die Stube: Der Rondrianer Oleg ,der Perainegeweihte Thoran ,der Jäger Ragnos und der tulamidische Magier Temyr. Erst bestellten sie dann ich, eine unverschämte Handlung gegenüber eines Adeligen (Die sind hier anscheinend nicht gut auf Horasier zu sprechen, wahrscheinlich wegen der Krönung Amenes zum Horas). Ich bekam eine trübe Brühe (gleicher Grund s.o.) und natürlich beschwerte ich mich lautstark über dies, es ist ja schließlich nicht auszuhalten, was die hier einem zu muten, zu Hause wäre der Wirt bestraft worden. Der Weißmagier und die anderen redeten auf mich ein, bis ich mich zu Frieden gab, als meiner Ehre Genüge getan war. Der Wirt verlangte von wir eine Dukate, von den anderen aber nur 2 Sibertaler.\par
Ich überhörte ein Gespräch von ihnen, sie sprachen über einen Brief und den Auftrag sich beim hiesigen Custos Lumini Anselm Horniger zu melden, um das Kloster der Hüter Arras de Mott zu beschützen ( Endlich etwas sinnvolles zu tun). Ich sprach sie darauf an. Wegen meines vorherigen Verhaltens wollten sie erst mich nicht mit nehmen, es gelang aber mit Gelöbnis auf Besserung (AH!AH!AH!), dass sie mich mitnahmen. Wir sprachen beim Custos Lumini vor und dieser sagte uns, das die Hüter in den nächsten Tagen einträfen und wir hätten solange zu warten.\par
Danach trennten wir uns Temyr und der Nostrianer (Hesinde erleuchte mein Gedächtnis) redeten über Magie, Oleg tanzt einen Klingentanz,Thoran eilte in die Siechenhäuser, um als Perainegeweihter zu helfen, Ragnos ging in den Wald und erlegt einen Hirsch, den er später abgekommen bekam, wie ich am nächsten Tag hörte, auch trafen die sich später wieder im "Drei Kronen“ und tranken sich gegenseitig unter den Tisch. Ich verbrachte den Nachmittag, Abend und die Nacht in der Fuchshöhle (Freudenhaus) mit einer schönen Tulamidin (Rahjane wird damit kein Problem haben, denn sie wird wahrscheinlich Ähnliches in Neetha tun). Am nächsten Tag traf ich mich mit den anderen als die Hüter in Greifenfurt einzogen, wir folgten ihnen und einige versuchten schon mal mit ihnen zu sprechen. Im Praiostempel besprachen wir uns mit den Hüter und es wurde beschlossen, dass man am nächsten Tag in zwölfgöttlicher Frühe aufbräche. Verabschiedete mich von den anderen und ging wieder in die Fuchshöhle und teilte diesmal mit einer bornlandischen Schönheit das Lager, auch die anderen legten sich früh schlafen.\par
Am nächsten morgen brachen mir endlich nach Arras de Mott auf, mit uns kamen noch vier Handwerker/innen und ein verrückter Zwerg namens Arthak, den Temyr an offensichtlich schon kannte. Wir brachen zu Fuß auf und Temyr versuchte die Handwerkerin zu betören.\par
Am Abend nach eine landen Marsch und einen herzhaften Abendessen saßen wir am Feuer und als wir Geräusche in einem Busch uns gegenüber hörten. Neun Orks sprangen uns entgegen, ich griffen den ersten Ork an, konnte ihm aber nur mäßig Schaden zu fügen, Oleg zerteilte den zweiten Ork mit einem gewaltigen Hammerschlag, der Dritte wurde an zwei Pfeilen von Ragnos aufgespießt, Temyr flambierte den Vierten mit Ignifaxius und der Nostrianer tötete den fünften Ork mit einem mir nicht geläufigem Zauber. Die nächsten Vier starben auf gleiche Weise, der erste Ork dagegen besiegte mich und ohne Torans Eingreifen wäre ich sicherlich gestorben, Oleg erschlug auch noch den ersten Ork.\par

\subsection{Aus den Tagebüchern des Frenglion Krennelsieks}

Nachdem die Orks besiegt waren, schlugen wir die Zelte auf. Wir beschlossen auch noch Wachen aufzustellen, wobei ich die erste übernahm. Ich schlummerte gegen Ende der Wache jedoch ein, aber Oleg wachte zu Glück auf, weckte mich und begann dann seine Schicht. Ich dagegen ging ins Zelt und begrüßte Boron das zweite Mal die Nacht.\par
Am nächsten Tag zogen wir dann Richtung Norden weiter. Während dem Gehen erzählte Emmeran, dass Nicola de Mott, der "hohe Lehrmeister“, anfangs dagegen war Wachen einzustellen, aber als er selbst angegriffen wurde, entschloss er sich dazu doch welche einzustellen. Dennoch tat er es mit Widerwillen und würde uns vermutlich nicht allzu freundlich behandeln und Gründe suchen, um uns wider wegzuschicken. Auf die Frage, ob man nicht Bannstrahler oder andere Diener Praios als Wachen anheuern könne, antwortete Emmeran, dass diese zwar angefordert wurden, aber nicht bewilligt worden sind. \par

Nach drei weitestgehend ereignislosen Tagen kamen wir in einem etwa zwei Meilen breiten Talkessel an. Das ganze Tal war bewaldet mit Nadelbäumen. Nur das Kloster, eine Ruine, vermutlich die des Dorfes, welches im Orkensturm vernichtet wurde, wie uns Emmeran erzählte, und ein schwarzer Hügel waren unbewaldet. Der Weg führte am rechten Rand entlang, das Kloster lag etwa an der Nordseite. Kurz vor Arras de Mott gingen die Zwerge und Arbeiter in das hier, etwas außerhalb, gelegene Handwerkerlager. Wir begleiteten den Tross der Praioten weiter in die Klosterruine. Hier drinnen sah es aus wie in einer Werkstatt. Überall lagen Werkzeuge und Baumaterialien herum und das Kloster wirkte mit seinen löchrigen Mauern und kaputten Gebäuden wie ein unfertiges Gesellenwerk. Nur der Klosterbau hob sich aus dem ganzen Bild, denn dieser war schon fast fertig. Es würde wohl noch etwas dauern die Kuppel zu schließen und hier und da die Feinarbeiten zu machen, aber diese Halle des Betens würde wohl am ehesten fertig werden. Vor eben dieser erwartete uns auch schon Nicola de Mott. \par

Mit abwertenden Gesichtsausdruck erklärte er uns die Regeln, die größtenteils aus etlichen Verboten und Verpflichtungen zum Beten bestanden. Natürlich gab es ein striktes Magieverbot und leider durften wir die Bibliothek, in der die Bücher des Ordens der Hüter untergebracht waren, auch nicht betreten. Zum Schluss stellte er einen uns kaum weniger verachtenden Novizen mit Namen Serkis für uns ab, der uns das Kloster zeigen sollte. Dies tat er auch und wir bekamen Schlafplätze im Schlaftrakt der Praioten zugeteilt. Nachdem wir wussten wo unser Nachtlager zu finden war, ließen wir uns einen Plan des Klosters bringen und begannen unsere Wachschichten zu planen. \par

Doch auch ein weiteres Problem erschloss sich uns. Wir konnten wohl kaum alle Lücken in der Mauer bewachen, wenn wir, wie wir vorhatten drei Schichten mit je einer Person am Kloster und einer am Handwerkerlager schieben würden. Also fragten wir nach Handwerkern, die uns helfen die Mauer zu verschließen. Doch der Bauleiter, ein nerviger Zwerg, verwehrte uns auch nur einen Handwerker. "Sie würden alle am Kloster gebracht werden“, lautete die Begründung. Jedoch durften wir uns Werkzeuge aus dem Materialzelt holen. So fanden wir einige Äxte und Sägen und machten uns gleich an die Arbeit. Eine kleine Lücke konnten wir schließen und bekamen gleich Ärger mit Bormund, einen der oberen Geweihten, der peinlich genau auf die Ordnung achtete, weil wir keine Wachen aufgestellt hatten. Nach dem Abendgebet, welches wir alle besuchten, gab es das Nachtmahl und wir teilten die Wachen auf. Die Wache im Kloster, beschlossen wir, sollte sich hauptsächlich an der Kuppel und an zwei der Löcher in der Mauer aufhalten.\par
Die Kuppel war deshalb so wichtig, da dort bereits ein Arbeiter unter mysteriösen Umständen umgekommen war. Doch glücklicherweise passierte diese Nacht nichts.\par
Am nächsten Tag machte ich mich mit Oleg auf zum schwarzen Hügel, bei dem es, wie uns gesagt wurde, spucken soll. Auf einer vormaligen Lichtung waren die toten Orks aus dem Orkensturm verbrannt worden. Deshalb ist das jetzt wohl ein kleiner Hügel, auf dem verständlicherweise nichts wachsen wollte. Es lag ein Brand- und Modergeruch in der Luft und wir vermeinten gar eine blasse Gestalt zu sehen, waren uns beim in dem Moment herrschenden Nebel nicht sicher, ob es ein Geist war oder nur ein Nebelschleier. Daraufhin ritten wir zurück und mussten sogleich erfahren, dass alle sieben Hühner, die das Kloster besaß, gestorben waren. Wie unsere Freunde während unserer Abwesenheit herausgefunden hatten, wohl nicht auf natürlichen Weg, sondern durch ein Gift. Beim Mittagsgebet hielt der "hohe Lehrmeister“ eine Hasspredigt auf Magie, wobei ich demonstrativ herausgegangen bin.\par

Dank der toten Hühner war das folgende Mittagessen nicht sehr reichlich. Dann fanden wir heraus, dass die Hühner das Gift, vermutlich Rattengift, wohl erst nach dem morgendlichen Füttern bekommen hatten. Irgendwer mochte wohl kein Huhn zum Mittag. Dann machten wir uns wieder ans Holzfällen und füllten weitere Lücken in der Mauer, zum Spott der vorbeilaufenden Handwerker, die unsere Arbeit nur belächelten. Auf einmal kam ein Handwerker und berichtete uns von einem Pferd, das im Handwerkerlager sein Unwesen trieb. Wir eilten sofort dorthin und tatsächlich vergriff sich ein Pferd am Handwerkerlager. Der Perainegeweihte konnte es beruhigen und da es anscheinend keinem der anwesenden Personen gehörte schauten wir in den Satteltaschen, ob wir anhand der darin vorhandenen Gegenstände, den Besitzer erkennen konnten. In der Tasche befand sich eine Ausgabe der "Enzyklopaediea magicae“ und eine Ausgabe von "Wunderbare Heilung ohne Wunder“. Des weiteren eine Phiole mit klarer Flüssigkeit und Onyx-Stein Amulett. \par

Das Pferd gehörte offensichtlich einen Magier. Vermutlich Emmerich von Falkenstein, wie ein dazugekommener Praiot meinte. Dieser war vor einiger Zeit nach Lowangen geritten und schien, wie die Spuren der Pferdehufe andeuteten auch aus dieser Richtung gekommen zu sein. Nur hatte sein Pferd ihn abgeworfen oder er war sonst irgendwie zu Fall gekommen. Ein paar von uns suchten den Weg nach Norden nach den Spuren der Pferdehufe ab und kamen nach der Suche zum Ergebnis, dass das Pferd mindestens eine Stunde ohne Reiter geritten war, da wir den Reiter nicht ausfindig machen konnten. Ziemlich mysteriös. Temyr untersuchte unterdessen das Amulett auf Magie und stellte fest, das es höchstens vor längerer Zeit mit ebenjener in Kontakt kam.\par

Kurz vor dem Abendgebet war dann eine weitere spannende Sache. Der Zwerg Arthak schimpfte auch dem Hof in Rogolan. Wie einer der Zwerge meinte, beleidigte er die Praiospriester als Götzenanbeter. Er konnte beruhigt werden und einer der anderen Zwerge sagte uns, dass der vorherige Zwerg gelogen hatte und, dass Arthak in Wahrheit: "Gefräßiges rotes Mondauge“ "Wirbel des Regenbogens“ und "Wagen der schwarze am Himmel des Wahns“ gesagt hatte. Wir konnten uns keinen Reim auf die Worte des seltsamen Zwerges machen. Nach dem Abendgebet und dem Nachtmahl teilten wir Wachen ein. Ich übernahm die erste am Handwerkerlager, in der Hoffnung, dass es ruhig bleiben würde. Tatsächlich bemerkte ich leider etwas.\par

Ins Materialzelt schienen ein paar Gestalten eingedrungen zu sein. Nur Augenblicke später fing es Feuer. Sofort schrie ich um Hilfe und die Handwerker wurden wach und fingen an Wasser vom Kloster herunter zu tragen, während ich vergeblich versuchte den Caldofrigo zu wirken. Nach langer Zeit war das Feuer dann gelöscht und die Aufregung legte sich. Ich erinnerte mich, dass die Gestalten vom Westen her, also vom Geisterhügel her kamen. Ragnos fand noch zwei Orkspuren in der Nähe und wir fragten uns ob die Orks tatsächlich dazu fähig waren ein Feuer zu legen. Nachdem wir auch die Schichten für die Wachen weiter organisiert hatten, legte ich mich schlafen. Am nächsten Morgen wurde ich geweckt und mir wurde mitgeteilt, dass es kein Morgengebet geben würde, da ein Priester tot im Wohngebäude aufgefunden wurde. Wir eilten hin und sahen ihn, wie er in einer Badewanne lag, völlig ausgeblutet. Thoran untersuchte ihn und stellte fest, dass beide Pulsadern offen waren und, dass er eine Platzwunde am Hinterkopf hatte, an der er vermutlich gestorben war. Magie war keine im Spiel, das hatte ich überprüft. Anscheinend war hier ein Mord verübt worden. Und der Mörder hätte sich einzig und allein zur Zeit des Feuers ins Kloster begeben können. Es schien also genau geplant gewesen. \par

Wir erfuhren, dass er sehr erfahren in Sternkunde war. Des weiteren fanden wir in seiner Hand einen Zettel mit einer seltsamen Zahl. M-S4-17. Wir zeigten es einem Praioten und dieser meinte, dass dies eine Ordnungszahl für die Bücher der Bibliothek war, in die wir nicht hinein durften. Der Priester war aber so nett und holte uns eben jenes Buch. In diesem fanden wir einen Zettel. Er war voll mit düsteren Prophezeiungen, allesamt abgeleitet von der momentanen Sternenkonstellation. Es deutete alles auf Unheil hin und auf einen Feind, der von innen kommt.\par

Ich wollte dies überprüfen, doch die Sternentafel, die der tote Bruder immer in seinen Zimmer hatte war weg. Am Vormittag klopfte noch ein alter anscheinend verletzter Mann, mit Namen Orbrand von Havena, ein Wundarzt. Ich brachte ihn zu Thoran, der ihn heilte. Doch die Wunde war ein Rätsel auf. Angeblich von einem Dorn herbeigeführt, sah sie doch nicht so aus und war ungewöhnlich lang. Es musste schon ein komischer Busch gewesen sein, der so eine Wunde verursacht. Wir blieben skeptisch. Der alte Mann teilte uns mit, dass er vorhatte noch bis zum Mittagessen zu bleiben. \par

Doch die Geschehnisse hörten nicht auf. So kam es kurz vor Mittag zu einer Schlägerei beim Kuppelbau. Es brauchte einen Harmoniesegen um die daran Beteiligten, fast alle anwesenden Handwerker zu trennen. Orbrand war auch dabei. Anscheinend wurde das ganze durch eine Lappalie ausgelöst. Ein Mensch war wohl über die Mauer eines Zwerges gesprungen. Es heißt ja, über eine hüfthohe Mauer zu springen bringe Unglück. Nur sah der Mensch die Mauer als kniehoch an, was den Zwerg erzürnte. Danach gab es das Mittagsmahl. Als wir am Nachmittag noch auf der Baustelle waren, wurden wir urplötzlich von einem Schwarm Sturmkrähen angegriffen und wehrten uns mit Schwert, Stock und Magie …

\subsection{Niedergelegt von Ritter Flammenzunge, nach einem Bericht seines Lehrmeisters}

Nachdem die Praiosscheibe hinter dem Horizont versunken war, kehrte Ruhe ins Kloster Arras de Mott ein. die Handwerker entzündeten ein Feuer in ihrem Lager und tranken und redeten noch eine Weile vor sich hin. Ragnos der Waidsmann gesellte sich zu ihnen, während Thoran die Wache im Kloster übernahm. Der Rest der Gefährten sattelte die Pferde, um den unheimlichen Hügel im Tal noch einmal zu begutachten.\par
Einer der Zwerge, Kagrim, ging noch einmal zum Kloster hinauf und Ragnos begleitete ihn pflichtbewusst. Einen weiteren Mord wollte er nicht zulassen und ein einsamer Zwerg war ein verlockendes Opfer. So gingen die Beiden zum Kloster hinauf und besorgten beim Braumeister noch ein Fässchen Klosterbräu. Verwunderlicherweise bat Kagrim auch um ein Stückchen Kohle und auf Ragnos Frage, was das zu bedeuten habe, antwortete er: "Ich habe heute beim Arbeiten unten in der Krypta einige merkwürdige Meißelungen in einem Stein gesehen. ich würde sie gerne kopieren." Nach einigem hin und her nahm Ragnos das Fässchen und ging hinunter zum Lagerfeuer, während Kagrim in die Krypta hinabstieg. Nach einiger Zeit kehrte er auch wieder zurück, leise auf Rogolan vor sich hinmurmelnd, und zeigte Ragnos einen Zettel mit mysteriösen Symbolen, auf die sich keiner der beiden einen Reim machen konnte.\par
Die restlichen Gefährten hatten indes den Hügel erreicht und standen in völliger Dunkelheit an seinem Fuße. Da meinten sie Schnüffeln und Knurren zu hören und zwischen den Bäumen bewegten sich finstre Schemen. Vorsichtig stiegen die Vier den Hügel hinauf um den Überblick zu behalten. Unten schlichen etwa zehn gigantische Wölfe und fletschten ihre Zähne. Sie trauten sich nicht den Hügel zu betreten und so harrten die Helden auf dem selbigen aus. Die Wölfe machten jedoch nicht den Anschein als ob sie sich verziehen wollten und zogen weiter ihre Kreise. Oleg wurde das ganze zu bunt und beschloss auf sein Pferd zu steigen und mit Temyr und Cusimo einen Ausbruch zu wagen. Mit einem lauten: "Für Rondra!" preschte er vor und schwang sein mächtiges Schwert, trampelte einen Wolf nieder und verpasste einem anderen einen langen Schnitt am Rücken. Die anderem folgten ihm, doch die Wölfe griffen die Vorderläufe der Reittiere an und die beiden landeten unsanft im Unterholz. Nach einem kurzen, aber heftigen Gefecht, zogen die Wölfe sich zurück, aber weder Reiter noch Ross sahen sich im Stande, noch weite Strecken zurück zu legen.\par
Frenglion,der nostrianische Magier, war der einzige, dessen Pferd das Scharmützel überstanden hatte. So schnell er konnte ritt er zum Lager zurück um den Perainegeweihten zu holen. So ritt Thoran wieder zurück zum Hügel, wo Oleg inzwischen mit Rondras Hilfe sein Pferd kuriert hatte. Thoran versorgte auch die anderen Verletzten und so ritt die ganze Truppe wieder zurück, ohne neue Erkenntnisse, aber mit vielen neuen Narben. Die restliche Nacht verlief ereignislos, jedenfalls im Kloster. Was in der Nacht im Arbeiterlager passierte, wird wohl auf ewig ein Rätsel bleiben, denn beide Wachhabenden Thoran und Cusimo schliefen bei ihrer Wache ein.\par
Deshalb war es ein besonderer Schock, als am nächsten Morgen die Zwerge unter Jandrims Leitung ins Kloster gestürmt kamen und laut: "Mord" brüllten. In der Nacht war Kagrim erdrosselt worden. Thoran, der im Lager verblieben war, untersuchte das Zelt und die Leiche, während der Rest noch oben diskutierte, fand aber außer dem am Vorabend gezeichneten Zettel nichts. Diesen steckte er ein, als auch schon die Zwerge mit dem Rest der Helden im Schlepptau kam. Jandrim hielt die gaffenden Arbeiter zurück, während die Helden über die Interpretation der mysteriösen Zeichen berieten. Temyr identifizierte sie als alchemistische Symbole für die Mondphasen und die sechs Elemente. Einen Reim konnte sich jedoch keiner der Helden darauf machen. Die Debatte, wie die Zeichen zu deuten seien, wurde jäh von Hüter Bormund unterbrochen, der die Arbeiter mit einigen barschen Worten hoch zum Kloster schickte. Danach verlangte er von der Gruppe eine Erklärung, die ihm jedoch keiner liefern konnte. Wütend ging auch er zum Frühstück.\par
Die Zwerge nahmen kurz darauf den Leichnam mit, um die Begräbnisriten zu vollziehen. Als Oleg fragte, ob er sie begleiten könne, wiesen sie ihn schroff zurück, dies sei eine rein zwergische Angelegenheit.\par
Die Helden gingen auch zum Frühstück. Die Stimmung war gedrückt und es versprach ein schwüler, heißer Tag zu werden. Insgesamt hing ein Sturm in der Luft, der im Begriff war auszubrechen. Nach dem Frühstück scheuchte Jandrim die Arbeiter zum Klosterbau und Bruder Emmeran bat die Helden in einen Seitengang. Er wollte sie selber zu den Ereignissen befragen und versicherte ihnen, dass er sie unterstützen würde, soweit es in seiner Macht stünde.\par
Nach dem Essen gingen die Helden in die Krypta, um den Stein zu betrachten, den Kagrim abgepaust hatte. Emmeran händigte ihnen den Schlüssel zur Krypta aus, nicht ohne den Gefährten erst einzubläuen, Hüter Bormund nichts davon zu sagen. Der Stein fand sich direkt vor dem eisernen Tor zur Krypta. Dieses ging ohne Probleme auf und die Helden betraten die altehrwürdige Grabkammer Arras de Motts. Die Gräber waren alle unbeschädigt geblieben, ein Wunder bei der langen Belagerung und Besetzung durch die Orks. Das Grab war sehr kunstvoll behauen, ein Greifenkopf mit aufgerissenem Schnabel zierte den Sarkophag und eine steinerne Tafel war in ihn eingelassen, mit der Aufschrift:\par

Hier ruht Arras de Mott\newline
sein Name sei des Rechtschaffenen Erbauung\newline
Des Unwissenden Ermahnung\newline
Und der Gezeichneten strahlender Fingerzeig\newline
Wenn dereinst\newline
das Licht der Finsternis weicht.\par
Thoran fühlte sich sofort an die Prophezeiung erinnert, die er im Tsatempel zu Dragenfeldt gesehen hatte. Auch dort war von Gezeichneten die Rede und der erste Gezeichnete war Calhadril, einer seiner Freunde und Weggefährten. Sollten mit den Gezeichneten etwa sie gemeint sein?\par
Nach all diesen verwirrenden Rätseln verließen die Helden die Krypta wieder und beschlossen wenigsten zu versuchen zu klären, wer der Mörder Kagrims sein könnte. Die Helden befragten deshalb die Zwerge, was diese gesehen hätten. Jandrim meinte, nachdem Kagrim aus dem Kloster gekommen war, habe dieser ihn bei Seite genommen und habe ihn gebeten, dass Kloster mitsamt aller Zwerge zu verlassen. Jandrim habe natürlich abgelehnt. Trotzdem wurde Oleg das Gefühl nicht los, dass der Bauleiter etwas verschwieg.\par
Ballasch, der Bruder Kuwims und stellvertretender Bauleiter, konnte den Helden nicht weiterhelfen und auch der verrückte Arthak stellte nur rätselhafte Gegenfragen als die Helden ihn befragten. Kuwim war nicht auf zu finde, hatte aber wahrscheinlich eine Unterredung mit dem Hohen Lehrmeister. Die anderen Zwerge und Arbeiter waren an der Baustelle beschäftigt, weshalb die Helden beschlossen, sie nicht zu stören.\par
Fasziniert hörte Oleg den Ausführungen Torans zu den Orakelsprüchen zu. Um sowohl das Mysterium dieser, als auch das des Sarges in der Krypta zu klären, besuchte er Hüter Quanion in seinem Refugium in der Bibliothek. Hüter Bormund war, Rondra sei Dank, nicht zu gegen. Quanion war erst ein wenig abgeneigt, so mir nichts ,dir nichts, die Geheimnisse des Klosters aus zu plaudern, bekam es allerdings mit der Angst zu tun, als Oleg ihm eröffnete, dass die Prophezeiungen im Begriff seien, sich zu erfüllen. Er vertraute dem Rondrageweihten ein altes abgerissenes Stück Papier an, dass wohl schon bessere Tage gesehen hatte. Doch darauf verzeichnet waren Sehersprüche, ähnlich denen aus Fasar. Auch sie sprachen von Zeichen und Boten. Ganz aufgeregt fragte Oleg weiter nach, ob Quanion ihm noch etwas über den Gründer des Klosters geben könne. Dieser gab ihm dann tatsächlich noch ein kleines Büchlein, in dem unter anderem eine Interpretation des sonderbaren Namens des Hochheiligen stand. \par

Dankbar wandte Oleg sich zum gehen, als Quanion ihm noch einschärfte, alles was er ihm gegeben hatte vor Bormund geheim zu halten. Oleg stimmte natürlich zu. Während Thoran gerade dabei war, das Gelände zu patrouillieren, wurde er vom alten Regiardon überrascht, der ihm aufgeregt bedeutete mitzukommen. Er habe etwas seltsames entdeckt, meinte er und zeigte auf seinen Kräutergarten. Dort waren tatsächlich alle Pflanzen um mehrere Halbfinger gewachsen und das seit gestern. Thoran konnte weder göttlichen, noch dämonischen Einfluss finden, nahm aber, unter heftigen Protesten Regiardons, einige Proben mit, um sie Temyr zur Analyse zu geben. Dieser fand jedoch keine Spuren von Astraler Beeinflussung. Trotzdem wurde er das ungute Gefühl nicht los, dass die astralen Kräfte der Umgebung sich aufbäumen. Er und sein Kollege Frenglion hatten eine viel zu schnelle Regeneratio ihrer freien Astralenergie gespürt. Temyr fühlte sich ungut an das Ritual zu Dragenfeldt erinnert.\par

Kurz vor dem Abendessen meldete sich ein etwas verschämter Ballasch bei den Helden. Er meinte, er sei nicht ganz aufrichtig gewesen, als sie ihn nach seinem Wissen gefragt hatten. Dann zeigte er ihnen einen Draht, den er zwischen seinen Kleidungsstücken gefunden hatte. Thoran stellte nach kurzer, fachmännischer Untersuchung fest, dass dieser Draht sehr gut zu den Würgespuren an Kagrims Hals passe. Ballasch wusste jedoch nicht woher der Draht kam. Die Helden beschlossen, ihm zu glauben.\par
Am Abend ließ sich Emmeram alles was am Tag geschehen war von den Helden berichten. Er wirkte nachdenklich und lobte die Helden für ihre gute Arbeit. Trotzdem riet er ihnen vorsichtig zu bleiben. Bormund suche nur nach einem Vorwand, sie allesamt der Inquisition zu überreichen.\par
Die Nacht war genauso heiß, wie der vorangegangene Tag und die drückende Hitze legte sich schwer auf das Gemüt der Arbeiter. Als Oleg gegen Mitternacht an der Kapelle vorbei kam, hörte er auf einmal sonderbare Geräusche, wie von einem schlurfenden Wesen. Er zückte seinen Dolch und ging vorsichtig zur Kapellentür um hinein zu spähen. Drinnen sah er den Novizen Efferdin, der sich mit einem Topf roter Farbe, man will gar nicht wissen woraus er sich die zusammen gepanscht hat, an der Tempelwand zu schaffen machte. Als Oleg von hinten auf ihn zukam, bemerkte er es erst gar nicht. Er war wie in einem Traum gefangen. Als Oleg ihn jedoch an tippte, drehte der Novize sich blitzschnell um, und schlug im seinen Pinsel ins Gesicht. Er versuchte zu fliehen, hatte aber die schnellen Reflexe eines Rondrageweihten nicht bedacht. Mit einem schnellen Schlag auf die Schläfe schickte dieser ihn ins Traumland Borons. Schnell weckte Oleg die anderen und einer ging, um Emmeran zu holen. Zusammen befragten sie den Novizen, der sich an nichts mehr erinnern konnte. Erst eine magische Visitation durch Temyr enthüllte, dass ein komplexer Beherrschungszauber druidischen Ursprungs für das seltsame Verhalten des Jungen zuständig war.\par
Doch die Nacht hielt noch weitere Schrecken bereit. Kurz vor Morgengrauen schlug Cusimo Alarm, den einige Orks versuchten über die provisorischen Barrikaden zu springen und das Kloster zu plündern. Im Zuge eines heftigen Gefechts wurde ein großer Teil erschlagen und zwei durch vorschnelles Handeln von Temyr versteinert. Leider fiel den Plünderern das Dormitorium zum Opfer, dass angezündet wurde, um Verwirrung zu stiften. Durch die Hilfe der Arbeiter, die herbeigeeilt waren, löschten die Helden den Brand schnell, das Gebäude konnte jedoch nicht gerettet werden. Die beiden gefangenen Orks, die sich inzwischen wieder regten wurde bis auf weiteres in die Kerkerzellen des Klosters verfrachtet. 

\subsection{Die letzten Nächte im Kloster, nach Toran von Klammsbrück}

Nach dem nächtlichen Orkangriff mussten die Brüder im Spital übernachten. Schlafen konnten sie jedoch vermutlich nicht sehr viel, da für die nächsten paar Stunden ein heftiges Gewitter über das ganze Tal hereinbrach. Als die Praiosscheibe endlich am Horizont erschien hatte es sich zum Glück wieder verzogen und beim Morgengebet wurden alle von Bruder Emmeran aufgefordert offen und ehrlich alles zu melden was mit den mysteriösen Vorfällen der letzten Tage zu tun hat.\par
Später am Tag erreichte ein einsamer Pilger das Kloster. Er überreichte uns einen Beutel mit dreizehn Dukaten, die er angeblich von einem Wegelagerer bekommen hatte. Auf nähere Nachfrage hin verließ er das Kloster jedoch fluchtartig. Als wir Hüter Emmeran auf diesen merkwürdigen Fall hinwiesen empfahl er uns, aufgrund der unheiligen Anzahl der Münzen; möglichst schnell zwölf davon an die Praioskirche zu spenden, was wir auch umgehend taten.
Trotz dieser schnellen Maßnahmen wurden kurz danach alle außer den Geweihten (das heißt alle außer mir und Oleg) zu einer Anhörung ins Skriptorium beordert. Es stellt sich heraus, dass ein Anklage wegen Verstößen gegen die Klostergesetze und Vernachlässigung der Wachpflichten gegen sie erhoben wurde. Zwar konnten konkrete Bestrafungen erst einmal abgewendet werden, aber Oleg und Temyr mussten wegen der mangelnden Kommunikation zwischen Wachen und Bruderschaft alle bisherigen Ereignisse einem Novizen diktieren.\par
Während dem Mittagsgebet tauchte im Westen des Klosters ein hohe Rauchfahne am Horizont auf. Wir entschieden uns zum besseren Schutz des Klosters nur Cusimo auszusenden um nach dem Ursprung diese merkwürdigen Phänomens zu suchen. Im Nachhinein scheint das ein Fehler gewesen zu sein. Ich weiß nicht genau was passiert ist, doch als er wieder kam waren seine Aussagen eher unzusammenhängend. Von einer Gruppe Goblins war da die Rede, von einer Herde Bergziegen und zu allem Überfluss hatte er auch noch einen halbtoten Mann dabei der unter dem Schmutz eines offenbar sehr langen Waldaufenthalts immer noch ein Hütergewand trug. In seiner Tasche trug Cusimo auch noch ein paar Bücher wovon er eins sehr eifrig las. Er schien allgemein sehr merkwürdig.\par
Sein Verhalten wurde auch nicht besser nachdem wir aufs Klostergelände kamen. Nahe dem Baugerüst versuchte er, plötzlich vom Buch aufschauend, selbiges umzustürzen. Er wurde darauf hin in eine der Klosterzellen gebracht wo wir zwischen seine Sachen das Buch fanden, das er gelesen hatte. Es stellte sich als eines der schrecklichen Namenlosen Beherrschungswerke heraus und Oleg war gezwungen es mit heiligen Flamen zu zerstören... 

\subsection{Auszug aus den Tagebüchern des Oleg Sjepsen}

Cusimo war dem Namenlosen verfallen, bzw. dessen Buch. Ein schwarzer Tag für die Götter, vor allem Praios, in dessen Kloster dies geschah. Doch mit der geballten Macht der Zwölfe konnte der 13. nicht widerstehen und seine Schrift gab unter den starken Hieben von "sicut Tonitrus" nach. Cusimo aber war nicht mehr zu retten. Auf Bewilligung von Nicolas de Mott exekutierten wir ihn außerhalb des Klosters und begruben ihn in geweihter Erde. Mögen die Zwölfe sich seiner erbarmen.\par
Es kamen Pilger von Norden, drei an der Zahl und sie brachten uns, den Wachen, dreizehn Dukaten dar, von einem Alten im Wald. - Den Lohn eines Verräters, an die Wachen- Halbwegs geschickt entledigten wir uns ihrer in einer Spende von vierzehn Dukaten an die Kirche des hl. Praios.\par
Später kamen noch zwei Händler von Norden, denen ihr Führer, ein Elf, an etwa der Stelle des Alten entschwunden war. Bei ihnen war auch ein stämmiger Krieger, Arn. Er schloss sich uns, der Wache, an und ich denke, dass er ein brauchbarer Ersatz für Cusimo abgeben wird. Hoffentlich vor allem wacher.\par
Später am Nachmittag, kam es plötzlich zu einem heftigen Sturm, der das Gerüst der Kuppel zu zerstören drohe. Doch drei der Hüter wirkten diesem magischen Angriff entgegen und bald zeigte sich Praios Antlitz zwischen den Wolken. Da wir hinter dem Sturm wieder unsern alten "Freund" den Druiden vermuteten, zogen wir in den Wald gen Norden, doch auch Ragnos konnte keine Spuren ausmachen. Enttäuscht kehrten wir zurück ins Kloster.\par
Während meiner Schicht begegnete ich Quanion, der mir ein Schreiben überreichte. Er habe eine schreckliche Entdeckung gemacht, die er uns nach der nächsten Morgenandacht genauer erklären würde. Die Nacht überlebte Quanion nicht. Wir fanden seine Leiche am nächsten Morgen. Was immer er gewusst hatte, es war zu viel und richtig.\par
Es war unerhört. Ein Hüter war gestorben. Eine Schande. Alle waren schockiert.\par
und die Schuld gab man den Wachen, die es nicht verhindert hatten können. Die unter uns, die nicht einem zwölfgöttlichen Orden unterstanden, wurden vor ein Hütertribunal bestellt. Außer Arn, der mit den bisherigen Geschehnissen nichts zu tun hätte haben können. Bormund verkündete, das die Inquisition hinzugeholt werden würde unter der Leitung von Ucurian Jago ( der nach den Ereignissen in Dragenfeldt, wohl nicht mehr zu halten sein würde, meine Freunde und Gefährten zu verbrennen.) Die beschuldigten wurden bis zum Eintreffen der Inquisition unter Arrest gestellt. Wir drei übrigen begannen die Wache unter uns aufzuteilen.\par
Arn begab sich auf die nunmehr fertig gestellte Kuppel. Da erschien ihm, wie er berichtete, das Abbild eines Magiers, der behauptete im "goldenen Berg" von einem Druiden gefangen gehalten zu werden. Er bat um dringende Hilfe.
In der folgenden Nacht verließen wir das Kloster. Da die Inquisition von Greifenfurt her unterwegs war, zogen wir nach Norden.\par
An jener Stelle im Wald, an der wir so vergeblich nach dem Alten gesucht hatten, fanden wir Goblinspuren, und später auch die eines Menschen und die, eines Elfen. Wir folgten den Spuren und im Morgengrauen fanden wir das Goblinlager. Wir rasteten etwa eine bis zwei Stunden, anschließend begaben sich Frenglion, Temyr und Thoran zu den Goblins um etwas über die wahrscheinlich Gefangenen zu erfahren. Wir Krieger und Ragnos hielten uns bereit einzugreifen, sollten die Worte versagen, doch sie genügten. Die Goblins führten uns auf einen Berg in dem ein Magier wohne, der sie drangsaliere. Wir betraten eine Riesige Höhle, an deren Ende sich ein Magisches Tor nach Art der Geoden befand - ein einfaches Elementarrätsel. danach folgten Hallen zu jedem Element mit Gegnern als Prüfung, nur der Obelisk des Humus war zerstört.\par
Im Tal bot sich ein erschrecken der Anblick. - kein Leben regte sich, alles war versteinert. Am größten Baum lag ein angeketteter Kaiserdrache-Stein.\par
Wir fanden bald eine Hütte mit dem zugerichteten Magier und dem Elf. anschließend fanden wir auch den Verantwortlichen Den berühmt-berüchtigten Druiden Archon Megalon, der hier eine seiner Wahnwitzigen Studien betrieb.\par
Wir halfen ihm einen Geodensteinkreis zu richten und gelangten in eine darunterliegende Elementarhöhle.\par
Temyr gelang es, zum großen Ärger des Druiden, den Halsreif von Eschen vom Quell, einem Geoden, an sich zu bringen, während der Halsreif in Form einer Schlange, Archon Megalon in die Hand biss. Der Halsreif scheint etwas mit Elementen etc. zu tun.\par
Als wir die Höhle wieder verließen war die Mächtige Blutulme in der Mitte des Tals zerstört, der Kaiserdrache lag tot davor. Eine Baumdryade erzählte von einem die Elemente schändendem Ritual, das wohl, wie wir kombinierten, in Arras de Mott geschehen würde.\par
Geschwind ritten wir zurück. Das Tor gab unter den Hufen der Kriegsrösser nach und wir sprengten auf den Hof des Klosters.\par
Am Himmel bildeten sich leichte Astralwirbel. Die Magier führten das auf viel freigesetzte Astralenergie zurück. Auf dem Bergfried stand Nicola de Mott mit den zwei zwergischen Baumeistern, eindeutig die Initiatoren des Geschehens. Temyr, Thoran und ich begaben uns so gleich mit dem Zepter in die Krypta. In den Sarkophag eingesetzt, öffnete das Zepter einen Gang in den Untergrund.\par
Unter der Krypta befand sich ein Raum. Auf dem Boden waren magische Sterne gemalt, in denen dunkle Kutten erschienen, in der Mitte des Raumes schwebte eine gigantische magisch strahlende Kugel, die ein Hauptfokus des Rituals zu seien schien.\par
Die schwarzen Kutten zögerten nicht lange und warfen sich uns mit dämonischer Macht uns entgegen. Ich bereitete mich auf mein letztes Gefecht vor. Die Gesänge von Thalionmel laut anstimmend begegnete ich den Kutten mit Göttlicher Hilfe. Temyr bahnte sich einen Weg zu der mysteriösen Kugel und vernichtete sie mit seinem Halsreif. Von einem Streiche arg verwundet zog er sich anschließend aus dem Kampf zurück. Thoran versorgte ihn, während ich mich mit letzter Kraft der Dämonen entledigte. Dort unten sollte ich noch nicht sterben, denn Rondra schenkte mir ein weiteres Leben.\par
Die andern erwehrten sich oben auf dem Turm eines Dämons, der schon länger die Gestalt Nicola de Motts übernommen hatte. Sie besiegten ihn mit Ragnos magischem Elfenpfeil und auch die beiden Zwerge, die sich als Borbaradianer entpuppt, konnten ihrem Schicksal nicht entkommen. \par


\chapter{Unsterbliche Gier}

\section{Geleitwort}
Unsterbliche Gier, ein Abenteuer, das aus der Gruppe (fast) die finale  Belegung machte und so viele Unsterbliche Momente enthält, das ich einige vergessen werde.\par
Ich liebe dieses Abenteuer und alle Erinnerungen daran. Zusammen mit dem Finale vom Tuzaker Fürstenpalast und der Schlacht um Klammsbrück sind hier die Szenen versammelt, die mir immer noch so ins Gedächtnis gebrannt sind, dass ich sie nicht loswerde.\par
Eine Auswahl:\par
Die Einführung mehrerer wundervoller NPCs:\par
Waldemar der Bär hatte am ersten Spielabend sein Debüt, zusammen mit Walpurga von Löwenhaupt und Dietrand von Ehrenstein, auch wenn sich die meisten von euch an dieses kleine Gastspiel nicht erinnern werden. Es folgte dicht darauf Vodk... äh Borgil, mit dem ich meinen Ruf für den miserabelsten russisch-arabisch-türkischen Crossoverakzent gefestigt habe. Sein Stelldichein hat glaube ich immer für Erheiterung gesorgt.\par
Dschelef ibn Jassafer, ob Erzmagier oder nicht ist ja jetzt auch völlig egal :-). Auch wenn sein Stern später von einer ganzen Riege anderer Magier übertroffen wurde, die deutlich öfter vorkamen, bleibt der gute, alte Dschelef doch ein Urgestein der Kampagne und mir ein wichtiges Werkzeug, um ein bisschen Hintergrundwissen oder einfach nur eine gute Idee an die Gruppe zu bringen.\par
Und natürlich Borbarad und Pardona, die Bösewichter schlechthin. Wenn der Auftritt Borbarads hier auch nur kurz war, verglichen mit den späteren Abenteuern, so legte dieses Abenteuer ihn dennoch fest, als größten und genialsten aller Schwarzmagier, als Gott so erfüllt von Hybris, dass er ein Bündnis mit dem Namenlosen selbst ausschlägt und seinen eigenen Weg geht. Und Pardona als Sinnbild für Verführung war wohl eine Sache, die den armen Calhadril in den Selbstmord geführt hat.\par
Der Bauernhof von Übermorgen den wir gestern besucht haben, einen Weiher, der ein Weiher war, etc.
Dieses Zitat aus dem Tagebuch von Philip, dass ich gerne verwende, um mich daran zu erinnern, warum ma\par Abhängigkeiten in Sätzen klar machen muss: "Nur eine Spinne fiel von einem der oberen Balken nach meiner magischen Schnell-Reinigung. War interessant sie kennengelernt zu haben, aber mir wäre sie physisch zu präsent.“, er meinte im zweiten Satz wieder Ayla.\par
Ein weiteres Glanzlicht war sicherlich der tragische (wenn auch abgesprochene) Tod von Calhadril, von dem wir, glaube ich, nichts erzählt hatten, was sicherlich einigen eine Überraschung bereitet hat. Am besten ist mir noch die lange Absprache mit Philip in Erinnerung geblieben, was nun aus seinem Charakter werden sollte und die Verlesung von Calhadrils Testament.\par
Der wundervolle vorletzte und letzte Spielabend, die Acheburg, Luzelins Tod, die Zeichnung des Zweiten Gezeichneten, der Ritt vom Blautann über den Rhodenstein zum Nachtschattenturm, der furiose Kampf gegen die Shakagra, Walmir von Riebeshoff, der Erzvampir, und ein kleiner Würfelwurf für den ich mich mein Leben lang schäme, all das wären sicherlich meine besten Erinnerungen, wenn da nicht eins wäre:\par\bigskip

Der Pakt...\par
Was soll man dazu noch schreiben. Auftakt war die meiner Meinung nach dümmste, genialste Idee, die ich je in meinem Leben gehört hatte, zusammen mit einer Hintergrundgeschichte, die nur danach schrie, ausgenutzt zu werden und das größte Missverständnis aller Zeiten: Philip, als ich dir sagte, du solltest dich an überderische Wesen richten, meinte ich die Götter, nicht die verdammten Erzdämonen! Aber als diese Idee einmal ausgesprochen war, wollte ich dich in keinem Fall mehr aus meinem Netz lassen. Blakharaz zu verkörpern hat mir unglaublich Spaß gemacht, ich hätte dich so gerne in einen richtigen Pakt gedrängt, aber da war der Glaube eines späteren Theologiestudenten wohl zu fest :-). Trotzdem, ich bin so dankbar für diese einmalige Chance und werde sie auf jeden Fall immer wieder am Schopf packen, wenn sie sich mir bietet.

\begin{flushright}
Claas Völcker, Darmstadt, 2.2.2015
\end{flushright}



 
 
\section{Die Tagebücher}


\subsection{Tagebuch von Rezzanjin:}

Zwei Tage nachdem ich auf Klammsbrück angekommen war, kamen auch die anderen, die dem Ruf der Praioskirche zum Kloster im Finsterkamm gefolgt waren. Doch es war kein gewöhnlicher Wach-Auftrag gewesen. Wie sie mir erzählten, fanden sie das Tal der Elemente und gerieten in einen Astralsturm, der das ganze Praioskloster zerstörte. Natürlich passierte noch viel mehr, aber ich kann noch nicht alles in Worte fassen. Auf jeden Fall bereute ich es langsam, dass ich nicht mit zu dem Kloster gekommen war. Auch ich hatte natürlich einiges zu erzählen, sodass es ein lustiger Abend war.\par

In den nächsten Tagen setzten die Herbststürme ein und wir zogen es vor am warmen Kamin der Burg zu sitzen. Gut zwei Wochen nach Monatsbeginn kam dann ein völlig durchnässter Beilunker Bote und überreichte einigen von uns, unter anderem auch mir, einen Brief. Es war ein Siegel drauf, welches wir als das Siegel von Waldemar dem Bären identifizieren konnten. Ich war überrascht, denn ich wusste zwar, dass Waldemar der Bär Herzog von Weiden war, ich hatte aber nicht gewusst, dass er von mir wusste, obwohl es sich dennoch herumgesprochen haben musste, dass ich ein meisterhafter Führer des Tuzakmessers bin. Noch mehr überraschte mich der Inhalt des Briefes. Waldemar der Bär, der Herzog von Weiden wollte, dass ich mit Temyr, Calhadril, Toran und Ragnos nach Weiden kommen sollte, damit wir ein Problem für ihn lösen. Sofort fingen wir alle an, unsere Sachen zusammenzusuchen, um am nächsten Morgen aufbrechen zu können. Iliricon beschwerte sich mal wieder, dass er nicht zu solchen Abenteuern eingeladen werde und das er schon wieder die Schüler der Akademie alleine unterrichten müsste. Temyr versuchte ihn zu besänftigen, indem er ihm erzählte, dass er für seine Schüler Aufgaben aufgeschrieben hätte, die sie alleine machen könnten. Daraufhin meinte Iliricon nur noch, dass wir gefälligst mehr Geld mitbringen sollen, als beim letzten Mal. Wir packten also unserer Sachen zusammen, vor allem die warmen Klamotten, da wir ahnten, dass es in Weiden wohl kalt werden würde.\par
Doch als wir nach vier Wochen ankamen, wir brauchten länger, da im Pass über die schwarze Sichel noch die Wüstenei war und wir diesen Weg deshalb nicht nehmen konnten, war Weiden komplett verschneit und wir froh, wenn wir am Abend eine warme Schänke fanden. Besonders Temyr und ich hatten zu kämpfen, da wir solche Kälte nicht gewohnt waren. So waren wir froh, als wir ungefähr vier Wochen nach unserer Abreise in Klammsbrück in Trallop ankamen. \par

In der Stadt war nicht wirklich viel los und so kamen wir schnell zu der Herzogsresidenz, wo wir, nachdem wir unsere Briefe vorgezeigt hatten, auch Einlass bekamen. Auf dem Hof übte gerade eine Schar Krieger mit dem Schwert und als zwei derjenigen uns sahen kamen sie zu uns. Es stellte sich heraus, dass der eine Arn war, ein Ritter, der auch zeitweise beim Praioskloster war und ein Teil der anderen daher kannte, der andere hingegen war Dietrand von Ehrenstein, den zufälligerweise auch ein paar von uns kannten. Gemeinsam gingen wir zur Burg. Man ließ uns vorstellen und so traten wir in den gewärmten Thronsaal ein und erblickten am Ende einer langen Tafel einen Mann der Statur eines Bären. \par

Es war Waldemar der Bär und er, machte seinem Namen alle Ehre. Stämmig, wie ein Bär saß er an der Kopfseite des Tisches. Er wartete, bis wir vorgestellt wurden und erklärte uns dann den Auftrag. In Weiden gab es anscheinend eine Serie von verschwundenen Leuten. Bisher konnte man noch nicht aufklären, warum und wer diese Leute umgebracht hatte. Doch diese Fälle waren anscheinend nicht um Trallop herum geschehen, sondern eher um Baliho herum. Des weiteren schien man in Trallop nicht viel zu wissen, denn der Herzog fragte uns auch, was es neues im Orkensturm gab, welcher seit gut zwei Jahren vorbei war. Er bot uns an in der warmen Burg Quartier zu beziehen und meinte, dass am nächsten Morgen alles für unsere Abreise bereit sein würde. Er lud uns ein, dass wir am Abend noch mit ihm speisen konnten und ließ und dann noch unsere Zimmer zeigen. Wir schlugen das Angebot nicht aus und so aßen wir mit Waldemar dem Bären zu Abend. An der Tafel waren noch zwei weitere Personen des Adels, deren Namen ich mir jedoch nicht merkte. Auch Waldemars Frau war zugegen.\par
Am nächsten Morgen bestellte uns der Herzog noch einmal in den Thronsaal. Er gab uns einen Siegelring und einen Brief, den wir vorzeigen sollten, wenn wir Probleme bekämen. Warme Kleidung bekamen wir auch von ihm. Auch bat er Arn, dass er uns begleiten möge. Wir hatten nichts dagegen, da wir Verstärkung immer gebrauchen konnten. Zu unserer Überraschung stellte er uns noch eine Kutsche bereit, die wir nehmen sollten.\par
Der Kutscher, ein Bornländer, begrüßte uns im Hof und half uns unser Gepäck in die Kutsche mit Kufen einzuladen. So konnte wir am Vormittag noch losfahren. Ich saß auf dem Kutschbock, da mir der Innenraum zu eng war.\par
Durch das Schneegestöber kamen wir nicht sehr gut voran, sodass wir am ersten Tag nicht sehr weit kamen. Am Abend des zweiten Tages kamen wir dann in Braunfurt an und bereiteten unser Nachtlager in einen Gasthaus. Ich teilte mir mit Temyr ein Zimmer. Am Morgen wachte ich auf und fühlte mich wie gerädert. Toran war über mich gebeugt und hatte mir anscheinend warme Suppe eingeflößt. Die anderen standen in dem Zimmer und blickten mich mit besorgten Mienen an. Auf einmal bemerkte ich die vielen Kratzspuren an der Zimmerwand. Doch nicht nur die Zimmerwand war betroffen. Auch ich hatte einen zerkratzten Körper. Toran verband mir die schlimmsten Kratzer und ich erklärte darauf, dass es mir einigermaßen gutgeht und wir gingen runter, um zu frühstücken. Als wir unser Gepäck zusammengepackt hatten und in die Kutsche gepackt hatten und ich mich auch in jene gesetzt hatte meinte Toran, mich in einen Heilschlaf versetzen zu müssen, da ich mich noch ziemlich schwach fühlte.\par
Ich schlief also in der Kutsche ein... 

\subsection{Aufzeichnungen von Temyr ibn Sahid}

\paragraph{Travia}
Regungslos standen wir um den Aschenhaufen jener grausigen Kreatur, welche einst ein beherzter Junge von vielleicht zehn Jahren gewesen war – vor langer Zeit, scheint es mir. Welche unheilige Macht auf Deren ist in der Lage, solche Wesenheiten zu gebären? Mir schaudert bei dem Gedanken, die Antwort auf diese Frage zu erfahren. In solch düsteren Gedanken versunken fand uns schließlich Rezzanjin, der nach einer Weile mit gezogener Waffe die Holzstiege hinauf polterte – offenbar war es auch ihm nicht wohl, das Haus zu erkunden. Nachdem wir Calhadril notdürftig verbunden und von dem hölzernen Pfeil in seiner Seite befreit hatten, dito einiger sehr unfeiner Ausdrücke aus halbelfischem Munde, kämpften wir uns zurück zu unserer Kutsche. \par

Schnee… In Khunchom gibt es so etwas nicht. Und wenn ich ehrlich sein soll, so kann ich auch getrost darauf verzichten. Vor dem zweispännigen Schlitten erwartete uns, den Kopf reuevoll zum Boden geneigt, Arngrim, dem sein Rauschkrautexzess noch deutlich anzusehen war. Wir tadelten ihn nicht – selbst Thoran hielt an sich – aber er nahm trotzdem bereitwillig auf dem Kutschbock neben Borgil Platz. \par

Es zeigte sich jedoch alsbald, welchen Nutzen ein kräftiger Mann im Weidener Winter haben kann, denn an der Böschung eines zu durchquerenden Waldes wurde unser Vorankommen von einer gewaltigen Schneewehe gestört. Ich bezweifle, dass Oleg sie hätte überblicken können. Borgil zügelte seine Rösser und Arn stürzte sich mit einem Spaten auf das Hindernis. Er war noch nicht weit gekommen, als mit einem Mal ein gewaltiger Schatten über unsere Köpfe hinweg rauschte, größer als alle Vögel, die mir bekannt sind. Wir hielten in der Kutsche den Atem an – und hörten schließlich das Geräusch eines mächtigen Aufpralls. Beklommen traten wir ins freie hinaus. Der Spur nachgehen und auf ein Wesen treffen, dessen Möglichkeiten wir nicht einschätzen konnten erschien uns wenig reizvoll, aber hatten wir eine Wahl?\par

Wir ließen Arn und Borgil im Schutze der Kutsche zurück und gingen in jene Richtung, aus der wir den Aufschlag gehört hatten, tiefer in den Wald hinein. Nicht nur das dichte Unterholz, sondern auch die dräuende Nacht erschwerten uns die Sicht. Nach etwa hundert Schritt blickten wir in eine kleine Senke hinab, in der die Bäume lichter standen. Plötzlich sahen wir auf der anderen Seite einen Schatten zwischen den Bäumen. Ohne zu zögern riss Rezzanjin sein Schwert aus der Scheide und stürmte vor – wir anderen folgten weit weniger ungestüm. Doch als wir zu ihm aufschlossen, war von dem Wesen keine Spur zu finden. Stattdessen wurden wir eines leisen Schluchzens hinter einer mächtigen Eiche gewahr – das Schluchzen einer Frau.\par
  
Wir umrundeten den Stamm und erstarrten: Im Schnee kauerte, entblößt und an allen Gliedern zitternd vor Kälte und Angst, eine Elfe. Sie hielt den Kopf auf ihren Knien und bemerkte uns erst, als wir langsam näher kamen. Niemals zuvor habe ich solche Furcht gesehen, dass es mich im tiefsten Herzen traf. Calhadril jedoch schien aufgelöst in Empathie beim Anblick dieses zarten Wesens. Nachdem wir ihr versichert hatten, dass sie nun keine Furcht mehr leiden müsse und der schwarze Schatten weit und breit nicht zu sehen sei, beruhigte sie sich zusehends und erzählte uns von ihrer misslichen Lage. Offenbar war sie schon seit längerem von dieser unbekannten Wesenheit verfolgt worden, seit sie ihre Heimat verlassen hatte. \par

Auf Ragnos und Thoran schien ihr Bericht allerdings eine wesentlich nachhaltigere Wirkung gehabt zu haben, jedenfalls schloss ich dies aus dem staunenden und misstrauischen Ausdruck in ihren Gesichtern. Ich sollte ihre Gedanken erst später erfahren. Auf dem Rückweg schmiegte sich Lysira, so hatte sie sich uns vorgestellt, eng an Calhadril. Dies mag teilweise aus Gründen der Kälte geschehen sein, aber ich kenne die tiefere Bedeutung solcher Gesten…\par

Arngrim war keineswegs müßig gewesen und so fanden wir den Weg freigeräumt bei unserem Antreffen bei der Kutsche. Zwar verzichteten Rezzanjin und Arn auf ihren Sitzplatz, doch trotzdem war es rechtschaffen eng darin, was Lysira für weitere Avancen nutzte. Die beiden anderen warfen mir bedeutungsvolle Blicke zu, offensichtlich hatten auch sie begriffen. Wir erreichten Wittenberge erst nach Einbruch der Dunkelheit und fuhren sofort den Gasthof an. Während wir aus der Kutsche herauskletterten, ließen sich Lysira und ihr Beschützer länger Zeit – die ganze Nacht, wie wir bald erfahren sollten. Zunächst jedoch erfreuten wir uns an einer warmen Speise und an Met. So langsam gewöhne ich mich an diesen gegorenen Most, aber ich vermisse den südländischen Wein. Wir hatten es damit kaum bis zur dritten Runde geschafft, als Borgil an unseren Tisch trat und sich schwer auf einem der Holzschemel niederließ. Wir entnahmen seinen Schilderungen, dass es den beiden Abwesenden nicht allzu schlecht ergehen konnte und wandten uns unserer Reiseplanung zu. Irgendwann, es muss nach der zehnten Stund gewesen sein, betraten zwei Bardinnen den Schankraum und erfüllten die Halle mit anmutigem Gesang. Ich beschloss, es Calhadril gleich zu tun und diese Nacht nicht allein zu verbringen, schließlich lebt man in unserer Zeit recht gefährlich…\par

Allerdings sollte die Nacht noch einige Überraschungen bereithalten, von denen ich am nächsten Morgen erfuhr. Offenbar war wieder in das Zimmer meiner Gefährten eingedrungen worden und abermals wurde Rezzanjin angegriffen. Diesmal jedoch schien ein Gefühl der Kälte die Schrammen auf seinem Körper zu ersetzen – ein hoffnungsloses, verzweifeltes Gefühl. Noch schlimmer hatte es jedoch Calhadril getroffen: Lysira war verschwunden, ebenso wie Calhadrils magischer Augenersatz, wie wir entsetzt feststellen mussten. Blut lief unter seiner Augenklappe hervor und ergoss sich über seine Brust – es war ein furchtbarer Anblick. Trotzdem mussten wir das gute Wetter für unsere Weiterfahrt nutzen. Ich hatte auf der langen Fahrt in der Kutsche reichlich Gelegenheit, Calhadril zu studieren: Er war ja schon immer etwas exzentrisch in seinem Habitus, vor allem nach der Sache in Dragenfeldt, aber jetzt… Er sitzt die Stunden apathisch und stumpf ab und scheint in tiefen, düsteren Gedanken zu verweilen. Doch wer ist in diesen Zeiten nicht schwermütig?\par

Unser wilder Ritt wurde allerdings jäh unterbrochen, als Borgil fernab der nächsten Ortschaft seine Pferde zum Stehen brachte. Als wir aus dem Fenster blicken, konnten wir auch den Grund für diese Handlung sehen: Auf dem gefrorenen, vom Schnee bedeckten Boden lag die reglose Gestalt eines Mannes. Nach einer Weile gespannten Schweigens verließen wir unser Gefährt und näherten uns zögerlich. Doch es war keine Gefahr in Verzug – denn der Mann war bereits tot. Wir drehten ihn auf den Rücken und blickten in eine Fratze des starren Entsetzens. Mit ungutem Gefühl untersuchten wir seinen blassen Hals, und fanden enge Bissspuren, wie wir sie in letzter Zeit viel zu häufig gesehen haben. Ratlos waren wir über das Fehlen seiner rechten Hand, doch wir wollten uns verständlicherweise nicht allzu lange damit befassen. \par

Da wir ihn nicht der Witterung und den Tieren ausgesetzt gottlos und unwürdig herumliegen lassen wollten, beschlossen wir, seine sterblichen Überreste bis nach Ifirnskapell überzuführen – der nächsten Ortschaft, wie uns Borgil unterrichtete. Hinter Rezzanjin und Arngrim auf das Dach der Kutsche geschnallt setzten wir unseren Weg fort. Die Praiosscheibe war bereits im Sinken begriffen, als wir die Ansammlung von Hütten und Gehöften erreichten. Wir müssen keinen besonders vertrauenerweckenden Eindruck bei unserer Einfahrt gemacht haben; jedenfalls wurden wir mit grimmigen Mienen und Mistgabeln erwartet. Doch wir hatten gut daran getan, den Leichnam mit uns zu führen, denn der Dorfvorsteher konnte ihn, wenn auch widerwillig, identifizieren: Ein Mann namens Geldor, vom gleichen Aussehen wie dieser Körper hier, habe am Vorabend das Dorf passiert und sei seitdem nicht mehr gesehen worden. Zwar zeigten sie sich nicht angetan von dem Vorschlag, ihn auf dem nahen Boronsanger göttergefällig zu bestatten, doch wir konnten uns schließlich durchsetzen.\par

Bei Speis und Trank im Traviatempel erfuhren wir, dass seit dem letzten Neumond mehrere Menschen vermisst wurden, während die Bauern über gerissenes Vieh klagten. Einer dieser Fälle schien sich ganz in der Nähe abgespielt zu haben: Ein kleiner Junge namens Fredo war von seinem Vater als verschwunden gemeldet worden und wir erinnerten uns alle des kleinen Jungen vom Bauernhof, den ein grässliches Schicksal ereilt hatte. Wir beschlossen, den armen Mann über den Vorfall auszufragen, den seine Familie ereilt hatte, doch auf unser wiederholtes Klopfen erhielten wir keinen Einlass. Das Gebot der Travia in Erinnerung warteten wir einige Zeit vor der hölzernen Türe, doch als die Winde an uns zerrten, begaben wir uns zur hinteren Scheune mit dem Beschluss, notfalls gewaltsam um Eintritt zu ersuchen. Ich bot zwar meine magischen Künste zur Öffnung des Scheunentores an, welche mit bereits in Khunchom gute Dienste geleistet hatten, doch da hatte Arngrim die Pforte bereits "in Fetzen“ getreten. Körperliche Kraft schön und gut, aber ich bevorzuge doch subtilere Methoden. Innerhalb der Scheune vermochten wir ein leises Wimmern oder Schluchzen in der oberen Etage zu vernehmen. Mit blanker Klinge schlichen wir uns hinauf. \par

Im Halbdunkel entdeckten wir eine kleine Gestalt, die sich über einen reglosen Mann gebeugt hielt, in beständigem Weinen begriffen. Wir traten näher, die Gestalt fuhr herum – und wir blickten in hasserfüllte, entstellte Züge, Geifer und Galle verspritzend. Das Wesen richtete sich halb auf und zeigte uns seine langen, gewetzten Klauen, die mit scharfen Nägeln besetzt waren. Es war nicht schwer zu erraten, wozu es diese gebrauchen würde, wenn wir nicht schnell eine Entscheidung träfen. Calhadril und ich warfen uns schnelle Blicke zu; wir waren uns einig, ballten die Fäuste und versteinerten mit gemeinsamen Kräften das Untier. \par

Während Arngrim nach dem Traviageweihten lief, untersuchten wir die Leiche des Mannes. Er wies keine äußeren Verletzungen auf, nicht mal die bekannten Bisse konnten wir finden. Der Geweihte vergaß seinen Zorn über unser forsches Eindringen rasch, als er den versteinerten Corpus gewahr wurde: Seine Züge und sein Gesicht wurden aschfahl und er schlug eifrig schützende Handzeichen. Mit zitternder Stimme erklärte er, dass es sich bei dem Wesen um den vermissten Jungen Fredo handelte – der nächste Fall von Elternmord in wenigen Tagen. Wir beschlossen, den Unheiligen zu dem nahen Tempel zu begeben, damit eine weitere Gefahr für die ansässigen Menschen ausgeschlossen bliebe. Als der Geweihte den versteinerten Körper anfasste, um den Trägern behilflich zu sein, stieß er einen Schrei aus, denn seine Finger färbten sich urplötzlich tiefschwarz. Welche unheilige Kraft treibt diese Geschöpfe nur an, was verleiht ihnen diese Schnelligkeit, dieses Geschick? Mit diesen Gedanken verließ ich Ilmars Haus und wir begaben uns nach dem Tempel, doch als wir gerade die Schwelle geweihten Bodens zu überschreiten begriffen waren, zerfiel das versteinerte Wesen zu dunklem Staub. \par

Ich denke, in diesem Augenblick wurden wir der Gefahr bewusst, in der wir uns befanden. Ein Kampf gegen Kreaturen, derart von götterlästerlicher Kraft durchflossen, dass geheiligter Boden sie tötet – wir werden mächtige Verbündete brauchen, um das Gefecht heil zu überstehen. Dies riet uns auch der Geweihte der Travia: In der nächsten Stadt, Anderath, sollten wir die Diener des Praios um Beistand ersuchen. Zwar hatten wir bislang keine erfreulichen Erfahrungen mit diesem Menschenschlag gemacht, aber was blieb uns auch anderes übrig?\par

Die Fahrt verlief ereignislos, was durchaus von Vorteil war. Schließlich erreichten wir Anderath, das in den letzten Monaten schwer unter den Raubzügen und Plünderungen der Schwarzfelle gelitten hat. Der einstmals prächtige Tempel des Praios schwelt noch immer unter dem Rauch der Zerstörung und die Einwohner meiden diesen Ort. Wir schnappten weitere Gerüchte über Vermisste und neuerdings Gebissene auf und begaben uns anschließend nach dem provisorischen Sitz der Praioten. Dort zeigte man sich erschüttert über unseren Rapport und versprach uns Beistand: Ein alter Geweihter bot sich an, einen mächtigen Trank zu unserem Vorteil zu brauen, wenn wir ihm eine Menge an Bernstein brächten. Wir nahmen sein Angebot natürlich an, immerhin können wir Hilfe jeder Art gut gebrauchen. Des Abends im Gasthaus machten wir die Bekanntschaft eines ehemaligen Hochgeweihten des Praios, der sich inzwischen häretischen Äusserungen verpflichtet hatte. Von meinem Standpunkt aus ist seine Argumentation durchaus nachvollziehbar, aber Calhadril dürfte anderer Meinung sein. Das Gespräch nahm einen ungünstigen Verlauf, als Bruder Emmeran die Schenke betrat, Emmeran, der einzige erträgliche Praiosgeweihte, dem ich je begegnet bin. Unsere Zusammenkunft endete mit einer Prügelei zwischen Emmeran und unserem Gesprächspartner, die zu Ungunsten letzteren ausging, auch weil Arngrim intervenierte und ihn bewusstlos schlug.\par

Am nächsten Tag, nahmen wir die Weiterfahrt in Anlauf mit der Aussicht, Baliho noch am gleichen Tag zu erreichen. Tatsächlich fuhren wir am späten Nachmittag in der Grafenstadt ein und bezogen Quartier auf der Grafenburg. Dann stürzten wir uns in unsere Nachforschungen. Ich will mir die Beschreibung des ständigen Herumlaufens ersparen und lieber für später die Ergebnisse protokollieren, als da wären:

\begin{itemize}
\item Sämtliche Opfer starben in kurzem Zeitraum um Neumond herum.
\item Das Zeichen der Travia war ihnen auf den Bauch geritzt, darunter stand: Rache.
\item Vielen Opfern wurden Gliedmaßen abgetrennt.
\item Der Täter ist vermutlich Rechtshänder, der mit einer Säge arbeitet.
\item Treulinde, die zunächst verdächtigt wurde, ist unschuldig. Damit ist der potenzielle Verdächtigen Kreis erst einmal erschöpft
\end{itemize}


Am Abend gingen wir einer Spur der Kurtisane nach, welche uns in den Nordstern führte. Dieses Etablissement war mir natürlich gut bekannt, ebenso wie die Tatsache, dass dort bester Wein ausgeschenkt wird. Wein! Die entbehrungsreichen Stunden ohne diesen Trunk der Götter können in diesem kärglichen Land nicht gezählt werden. Verständlicherweise sprach ich dem vorzüglichen Wein ordentlich zu. Aber auch unsere Pflichten vernachlässigten wir nicht: Aus dem Gespräch mit dem Besitzer des Lokals haben wir nun zwei Hauptverdächtige: Den Pelzhändler und einen Schneider, welche beide am betrefflichen Abend gesehen wurden. Nun, es wird spät und der Tag morgen wird lang. Dies soll fürs erste genügen. \par

\subsection{Ein weiteres Tagebuch von Rezzanjin}

Den 28. Travia wollten wir für Ermittlungen nutzen. Nach ein paar kleinen Streitigkeiten beschlossen wir, dass wir das Haus der ermordeten Kurtisane noch einmal durchsuchen wollten. Leider mussten wir feststellen, dass keiner von uns wusste, wo dieses Haus war. Also fragten wir den Hausverweser. Etwas zögerlich gab uns dieser dann preis, wo das Haus der Kurtisane liegt. Anscheinend war er schon öfter in jenem Haus. \par

Wir gingen los, doch nicht alle kamen mit. Arn ging zum Waffentraining bei den Gardisten, während wir anderen uns auf die Suche mach dem Haus machten. Wir fanden es in der Oberstadt, nahe des Südtors und der Pelzhändlergasse. Das Haus bestand aus mehreren Apartments. Wir betraten jenes, in dem das Bordell untergebracht war und wurden direkt von einer etwas beleibteren Frau begrüßt. Nachdem wir ihr unser Anliegen vorgebracht hatten und erst einmal andere Dienste abgelehnt hatten, führte sie uns in ein einfach ausgestattetes Hinterzimmer. Die Frau stellte sich als Esmeranza vor und fing direkt an zu reden. \par

Sie wusste genau wer wir waren, was wir die letzten Jahre gemacht hatten und was wir wollten. Sie wusste Dinge, die sie unmöglich nur vom gewöhnlichen Klatsch kennen konnte. Uns wurde richtig unwohl, bei dem Gedanken andere könnten ähnlich viel wissen. Sie bot uns sogar noch mehr an. Sie würde uns die Informationen, die wir benötigen geben, wenn wir ihr im Gegenzug etwas anderes geben würden. Doch sie wollte kein Geld, sondern auch Informationen, die wir zu dem Zeitpunkt noch nicht wissen konnten. Informationen über den Grund des Übels. Informationen über das Rätsel, das wir noch nicht im Ansatz gelöst hatten. \par

Wir erklärten es ihr, woraufhin sie meinte, dass wir ihr einen Tropfen Blut von jemanden von uns übergeben sollten sollten, das wir zurück kriegen würden, wenn wir ihr die geforderten Informationen übergeben würden. Sie ließ uns kurz im Hinterzimmer allein und wir besprachen uns. Sie hatte vorher schon angedeutet, dass sie wohl nur das Blut eines Magiers akzeptieren würde. Nach dem die Magier es uns weniger gebildeten erklärten, verstanden wir warum und nach einiger Zeit ließ sich Calhadril breitschlagen einen Tropfen Blut zu spenden. Wir würden jedoch die nötigen Sicherheitsbedingungen treffen, damit sie den Tropfen nicht anderweitig verwenden könnte. So hatten wir vor die Frau mit einem Eidsegen zu belegen und die Flasche zu versiegeln und sie zusätzlich mit Zaubern gegen Missbrauch durch Diebstahl und ähnliches zu schützen. \par

Die Frau erklärte sich bereit und schwor den Eid. Calhadril spendete sein Blut und wir versiegelten die Flasche mit dem Siegel des Herzogs. Nun gab sie uns die Informationen: Esme, so hieß die Kurtisane, sei nicht, wie von der Stadtwache angegeben, erdrosselt worden. Sie schien schon vorher gestorben zu sein und zwar an einem Biss am Hals, der durch die roten Striemen der Erddrosselung überdeckt war. Doch es war kein Fleisch raus-gerissen worden, sondern es waren nur vier kleine Löcher zu erkennen gewesen, die in einem Rechteck angeordnet waren. Wir alle vermuteten sofort, dass es sich um einen Vampir handeln musste. Wir wussten nicht viel über diese sagenumwobenen Wesen und auch die Frau wollte uns keine Auskunft darüber geben. \par

Sie erzählte uns auch, wer ihr letzter Klient gewesen sei. Ein hochgewachsener Mann in einem ordentlichen Pelzmantel, den Namen wusste sie nicht. Des weiteren schien er nicht weit gegangen zu sein, da seine Stiefel wohl nicht besonders dreckig waren. Außerdem wusste sie, dass die Stadtwache anscheinend ein Überbleibsel von ersten Mord mitgenommen hatte. \par

Das war erst einmal mehr als genug Information für uns. Es war nun klar, dass die Mordserie durch einen Vampir verübt wurde, der in der Näher des Pelzhändlerviertels hauste. Warum er nur Huren umgebracht hatte, wussten wir noch nicht. Wir beschlossen in die Pelzhändlergasse zu gehen, um den Verdächtigen ausfindig zu machen. Und gleich das erste Haus in das wir gingen war ein Treffer. Es war das Haus des verschwundenen Pelzhändlers. Seine Witwe begrüßte uns und wir baten sie uns etwas über den Tag, an dem ihr Mann verschwunden war zu erzählen. Sie tat dies auch. Er war an dem Abend anscheinend im Nordstern gewesen und habe den Herzog von Mersing-Eberstamm getroffen. Danach wäre er nicht wieder zurück gekommen. Da dies nicht sehr aufschlussreich war und uns nur unwesentlich weiterhelfen würde, fragten wir sie ob ihr Mann irgendwelche Feinde hätte. Ihr fiel nur ein, dass der Tuchhändler Bertrand mit ihm im Streit gelegen habe.\par

Wir bedankten uns und machten und sofort auf zu jenem, der nur ein paar Straßen weiter wohnte. Doch wie sich herausstellte war dieser zum Zeitpunkt des Attentates in Altennorden gewesen, kam also als Täter nicht in Frage. Wir nutzten die Gelegenheit und fragten ihn, wo denn der nächste Feinschmied zu finden sei, denn wir hatten vor Bernstein einzukaufen und er erklärte uns den Weg. Einen Feinschmied gefunden, gingen wir nun zu diesem und kauften 20 Karat Bernstein, das wir für das Ritual des Praioten brauchten.\par

Nun wollten wir dem Hinweis nachgehen, dass die Stadtwache noch ein Beweisstück von ersten Mord hatte. Der von unseren vielen Besuchen angenervte Hauptmann der Wache bestätigte unsere Information, machte uns allerdings nicht viel Hoffnung das Beweisstück wiederzufinden. Wir konnten ihn davon überzeugen uns zu helfen, er führte uns ins Archiv, das, wie sich herausstellte, das komplette Chaos war. Beweisstücke waren achtlos übereinander gestapelt worden und vollkommen unsortiert. Der Wachhauptmann gab uns noch dem Tipp nach einem Messer zu suchen, da er sich daran erinnerte, dass ein solches beschlagnahmt wurde. Also suchten wir in einer halben Stunde alle Messer zusammen, die wir finden konnten. Wir fanden dann 15, die als Tatwaffe in Frage kamen. Dann kam auch der Hauptmann wieder und half uns beim aussortieren der Messer. \par

Wir konnten den Kreis der verdächtigen Messer auf vier reduzieren, wobei drei der Messer, wie uns der Hauptmann erzählte, von dem Zwergenschmied Ingramm geschmiedet wurden. Wir holten uns vom Hauptmann die Erlaubnis ein die Messer mitnehmen zu dürfen und machten uns auf zu besagtem Schmied. Der war etwas mürrisch und beschäftigt, konnte uns jedoch weiterhelfen, indem er ein Messer aussortierte, weil es zu verrostet war, um für den Mord in Frage zu kommen. Des weiteren konnte er uns sagen, dass eines der Messer ein Kürschnermesser war, das andere ein Fleischermesser. Da er nur an einen Kürschner verkaufte, nämlich Elbaran, machten wir uns gleich zu diesem auf, nachdem Ragnos sich noch ein neues Messer bestellt hatte. \par

So machten wir uns wieder auf und es war schon recht spät am Abend, als wir die Tür zur Kürschnerei öffneten. Die dort anwesende Assistentin meinte, dass Elbaran gerade nicht zu sprechen sei und wollte uns schon abwimmeln, doch wir schafften es sie zu überreden, dass wir Elbaran sprechen konnten. Sie ging die Treppe zu den Zimmern hoch und wir warteten ab, dass sie herunterkommen würde und uns Bescheid geben würde. Wir warteten die Hälfte einer Stunde, bis es uns zu blöd wurde und wir laut rufend nachfragten, was denn oben los sei. Als keine Antwort kam beschlich uns ein Verdacht und wir gingen hoch. Oben war es stockdüster und Temyr entfachte ein Licht. Calhadril war schon gegangen, da er am nächsten morgen früh los wollte und Kräuter kaufen gehen wollte. Aus einem Raum schien Licht und als wir darauf zugingen, stoppte uns eine düstere Stimme. Sie schien uns aufzuhalten wollen und betonte lautstark, dass wir verschwinden sollten. \par

Doch wir wollten nicht und nach einem riskanten Wortwechsel fanden wir heraus, dass er ein Vampir war und damit anscheinend gar nicht glücklich. Er fragte Toran um Hilfe und dieser bot an ihn mit einer Liturgie der Peraine zu heilen. Er ging in das Zimmre hinein, legte ihm die Hand zögerlich auf den Kopf und sprach ein Gebet. Nachdem er geendet hatte, schrie der Vampir auf, verdrehte Toran den Arm und brach auf dem Boden zusammen. Sterbend raunte er uns zu, dass, wenn wir noch etwas herausfinden wollten, wir nach Mensheim gehen müssten. Dann verging er zu einem Häufchen Asche... 

\subsection{Aus den Berichten des Arngrimm von Ehrenstein des Jüngeren}

\paragraph{28.Travia}
Nachdem Tod des Kürschners, einem Vampir, untersuchten Temyr, Thoran, Calhadril und Rezzanjin das verdunkelte Zimmer. Thoran untersuchte die Leiche und schiente seinen Arm. Sie berichteten dem Hauptmann anschließend davon und unterrichteten mich davon, da ich in der Stadtwache Leibesertüchtigungen nachgegangen bin. Den Rest des Tages verbrachten wir in der Grafenburg zu Baliho.

\paragraph{29.Travia}
Am Morgen des 29.Travia ritt Calhadriel frühmorgens den Kräuterhändler/Alchimist in Rudein, um Toniskraut zukaufen. Diesen fragte er auch, ob er Mittel gegen Vampire kennen. Laut der Volkssagen, helfe Sonnenlicht, Peraine Heiliges, heiliges Holz und die Al'Anfaner benüzten Silber, war seine Antwort. Währenddessen besuchten Rezzanjin und Temyr Mutter Linai im Traviatempel, um ihr die letzten Geschehnisse kundzutun. Ragnos holte seinen bestellten Zwergendolch ab und ich selbst stählte meinen Körper, nur Thoran schläft. Nach der Rückkehr der anderen und dem Austausch von Neuigkeiten ging ich Borgil holen, während die andere die Kutsche beluden. Zum Abschied gab uns der Vogt zu Baliho fünf Dukaten. Am Abend kamen wir in Altnord an, dort fragten wir über einen verschwundenen Pelzhändler nach. Man gab uns keine wirkliche Antwort.

\paragraph{1.Boron}
An diesem Morgen erfuhren wir, dass letzte Nacht ein toter Arbeiter gefunden wurde. Rezzanjin und ich ging die Boroni befragen. Ihr war nichts merkwürdiges aufgefallen. Beim Frühstück erzählte Thoran von seinem nächtlichen Traum: Er stand auf einem Feld und um ihn herum verging die Welt. Wir brachen auf, nach einer Stunde fahrt trafen wir die hexe Morena, die von Auen über Baliho nach Nordhag reisen wollte. Sie erzählte, dass sie Calhadriel kenne. Sie gehört dem Blautanner Hexenzirkel an. Dieser wurde von der Eulenkönigin kontaktiert. Diese hatte eine Prophezeiung, dass wenn die Elfen umher ziehen und sechsmal der Weltenbrand stattgefunden hat, etwas geschieht. Außerdem wollte der Zirkel etwas Calhadriel geben. Danach verabschiedete sie sich. Ein paar Stunden später zog ein gewaltiger Schneesturm auf. Wir schafften es gerade so noch zur Fallinger Linde. Die Perainepriester der Linde ließen uns im Gästehaus übernachten.

\paragraph{2.Boron}
Am nächsten Morgen fanden wir Thoran bewusstlos an der Fallinger Linde. Er hatte wieder eine Vision gehabt: das Land verging wieder vor seinen Augen, aber diesmal sah er auch sieben Gestalten, die ihn versuchten ein zu kreisen. Die Sieben wurden von Einen Bildnis zu sahen gehalten, und erst als dass Bildnis seinen Griff lockerte, konnte Thoran aus dem Kreis fliehen. Nach dem Frühstück beschlossen wir zu Fuß nach Mensheim uns zu begeben, Borgil sollte uns schnellst möglich mit der Kutsche folgen.\par
Wir stapften zwei Tage lang durch das winterliche Weiden, welches die Temperaturen der Grimmfrostöde hatte. Am Abend des 3.Boron quartierten wir uns in Mensheim in der Schenke "Zum Geldbeutel“ ein.

\paragraph{4.Boron}
Wir sprachen bei Baron von Mersing-Eberstamm zu Menzheim und er lud uns ein in seinem Gutshof zu nächtigen. Er verwies uns an den Haushofmeister, unsere erste Befragung des Haushofmeisters ergabt: Zwei Stallburschen waren verschwunden. Wir kehrten zur Schänke "Zum Geldbeutel“ zurück, um unsere Sachen zu holen. Dabei stellten wir fest, dass die Bevölkerung von Menzheim uns nichts erzählen wollte. Wir erfuhren trotzdem das es seit den namenlosen Tagen jeden Neumond Tote gab, außer während des Neumonds des Efferds. Wieder auf dem Gut besprachen wir unser weiteres Vorgehen ab.
Während ich herausfand, dass einer der verschwundenen Stallburschen des Haushofmeister Sohn war und der andere Verschwundene anscheinend eine Nacht mit der Baroness verbringen wollte. Dies passierte kurz vor dem Rondra Neumond. Zu dieser Zeit waren Baron und Baroness auf Reisen. Bei der zweiten Befragung erzählte uns der Haushofmeister, dass sein Sohn eines morgens in der Sonne einfach zu Staub zerfiel. Der Sohn des Haushofmeister war ein Vampir gewesen, es warf außerdem die Frage auf, ob die Baroness ein Vampir war. Wir beschlossen nachts einen Bannkreis um den Esstisch zu ziehen. Nach dem Abendessen beobachteten wir von unserem Zimmer aus durch einen Zauber wie der Baron und die Baroness heftig diskutierten. Ich wollte Wache halten, wurde aber kurz, nachdem sich die anderen niedergelegt hatten, niedergeschlagen.

\paragraph{5.Boron}
Früh morgens standen Thoran und Rezzanjin den Bannkreis um den Tisch. Wenig später wurden wir zu Tisch gebeten, interessanterweise waren weder der Baron noch die Baroness noch Calhadriel anwesend. Am Ende des Essen stand Temyr auf und sagte:“ Leute, wir haben Bannstaub gegessen.“ Dann kamen der Baron, die Baroness und Calhadriel herein, Calhadriel wurde befohlen uns zu töten. Er verwandelte sich in einen gleißenden Feuerball und das Haus in eine brennende Hölle.
Wir sprangen vom brennenden Teppich und Rezzanjin führte einen Ausfall gegen Calhadril. Fast gleichzeitig schoss Ragnos einen Pfeil in das linke Auge des Barons. Während Calhadriel starb, rief er etwas von einem Brief, deshalb stürmten Temyr und ich aus dem Zimmer um diesen Brief und um unsere Eigentum sicherzustellen. Nach dem Tod Calhadrils stürmten Rezzanjin gegen die Baroness und fällte sie nach zwei bis vier Schlägen. Währenddessen verschoss Ragnos Pfeile gegen den Baron, was aber keine Wirkung zeigte, und dieser fiel erst nachdem sich Rezzanjin mit ihm duelliert hatte.\par
Gerade so schafften wir es uns und unsere Sachen in Sicherheit zu bringen. 


\subsection{Bericht des Temyr ibn Sahid}

\paragraph{Am Morgen des 5. des Monats Boron}
Mit letzter Kraft rannten die Gefährten aus den brennenden Ruinen des Gutes Menzheim. Der bewusstlose Calhadril verging unter den Strahlen der Morgensonne und in Torans Armen blieb nur die Kleidung des alten Freundes und ein Häufchen Asche übrig. Hinter ihnen verging die Haupthalle in einem flammenden Inferno und begrub die Leichen der beiden Vampire unter sich. Keiner der 5 sprach, als sie sich umdrehten und ihr Werk betrachteten. Der Verrat des Calhadril, dessen, mit dem sie so lange gereist und so vielen Gefahren getrotzt hatten, lastete ihnen schwerer auf der Seele als Blei und schmerzte tiefer als die schweren Wunden, die sie im Kampf mit ihrem ehemaligen Gefährten erlitten hatten.\par
Immer noch wie gelähmt, konnten die Anderen nur stumpf zusehen wie der Magier Temyr ibn Sahid den Brief, den Calhadril kurz vor seinem Tode verfasste, an sich nahm und mit schwerer Stimme zu lesen begann. Fast schien es so, als würde die traurige Stimme des Verstorbenen den Brief verlesen und seinen Verrat erklären. Der Magier schrieb davon, dass er seine Seele für Informationen über die Plage verkauft habe und bat die Anderen darum, ihm noch ein Begräbnis zukommen zu lassen, wie er es als Mensch verdient hätte. Außerdem vermachte er all seinen weltlichen Besitz an die Kirche der Peraine, damit wenigstens ein Teil seiner Schuld gesühnt werden konnte.\par
Dem letzten Wunsch ihres Freundes nicht nachzukommen, wäre den Gefährten nie in den Sinn gekommen und so machten sich Ragnos und Rezzanjin daran, ein Boronsrad aus den Überresten des Gutes zu bauen, während Temyr und Thoran begannen, den gefrorenen Boden aufzuhacken und ein Grab auszuheben. Arngrimm erklärte derweil den aufgebrachten Bürgern von Menzheim, den Gaffern und Wachen, das die Gruppe im Auftrag des Herzogs handelte und scheuchte sie davon. Trotzdem legte man ihm nahe, das er mit den Seinen den Ort bald verlassen solle. Nachdem Calhadrils Grab fertig war, betteten sie seine Asche, seinen Stab und die Überreste seiner Kleidung in das kalte Loch und schlossen es. Die Gefährten standen noch eine Weile am Grab und nachdem Thoran sein Segensgebet beendet hatte und das Grab feierlich eingesegnet hatte, sprach jeder noch ein kurzes Wort des Abschieds. Dann versorgte der Priester der Peraine die Wunden seiner Kameraden und zusammen humpelten sie traurig von dannen.\par
Die Menzheimer wollten trotzdem nicht, dass die Gruppe noch lange in Menzheim bleibt, weshalb sie sich schon am nächsten Tag nach Norden aufmachte. Die neuen Schneeschuhe leisteten gute Dienste und am Mittag trafen sie auf Borgil und die Kaleschka. Zur allgemeinen Überraschung war Borgil jedoch nicht alleine angereist. Ein einsamer Wanderer, in einen Wolfspelz gehüllt saß auf dem Kutschbock. Trotz der schneidenden Kälte hatte er seine Oberarme frei, seine Unterarme und Handflächen waren mit einer seltsamen Kreuzung aus Armschiene und Handschuh verdeckt, wie man sie sonst nur bei Regimentsschützen sieht.

\paragraph{7. Boron}
Auf Anraten des Fremden, welcher sich als Firnen Wulfgrimm vorstellte, machten die Gefährten noch einen Abstecher ins Dorf Espen, da dort angeblich ein Vampir hausen sollte. Auf dem Weg zur Burg des "Vampirs" fanden sie lediglich einen alten Tulamiden, der halb erfroren im Schnee lag. Schnell verfrachtete man ihn in die Kutsche und auf dem Gutshof angekommen stellte sich auch heraus, dass der vermeintliche Vampir schon seit Wintereinbruch nicht mehr im Ort gewesen war. Thoran und die alte Haushälterin pflegten den Alten soweit gesund, dass er wenigstens in der Lage war, kurze Gespräche zu führen, aber noch nicht, länger wach zu bleiben oder gar zu laufen. Erstaunlicherweise stellte er sich als Kenner der Prophezeiungen heraus und gab seinen Rettern seine eigene Übersetzung eines der Sprüche. Schnell fanden sich Parallelen zur momentanen Situation und die Gruppe schloss darauf, dass bald der zweite Gezeichnete kommen sollte, was Verwirrung stiftete, da ja schon zwei Leute gezeichnet waren. Außerdem weihten sie Firnen in ihren Auftrag ein, da er sowieso schon zu viel mitbekommen hatte. Er fand zwar wenig Gefallen an der Aussicht, bald auf Vampire und andere unheilige Wesen zu stoßen, doch er versprach ihnen zu helfen.

\paragraph{8. Boron}
Nachdem sich die Fährte nach Espen als kalt herausgestellt hatte, wollten die Klammsbrücker ihre Reise nach Westen fortsetzten, um der enigmatischen Jägerin auf die Spur zu kommen. Der Tulamide Dschelef war jedoch in keinem Falle reisefähig. Am Morgen war er wieder in eine Ohnmacht gesunken und murmelte im Delirium tulamidische Wort- und Satzfetzen vor sich hin. Trotzdem beschloss man, ihn wenigstens bis Baliho zu transportieren und dort dem Vogt zur Genesung zu übergeben.

\paragraph{9. Boron}
Die Gruppe erreichte Baliho am späten Nachmittag. Schnell wollten sie zur Grafenburg fahren, um den kranken Dschelef zu versorgen. Doch auf dem Weg durch die Unterstadt stellte sich ihnen ein muskulöser Schrank von einem Mann entgegen. Er überreichte ihnen eine Vorladung in die Schenke Kaiserstolz \& Orkentod, die Rinderbarone wollten sie sehen. Die Klammsbrücker bezogen schnell Quartier auf der Burg und rüsteten sich, Arn sogar in voller Platte. Dann machten sie sich auf, den Baronen einen Besuch abzustatten.
Das Gespräch war kurz und fruchtlos. Die Barone verlangten Satisfaktion für den Handel mit Esmeralda, die sie wohl verachteten, und waren enttäuscht, dass Calhadril sein Leben gelassen hatte, da sie sein Kopfgeld kassieren wollten. Um dem Debakel ein Ende zu setzten, beschloss Firnen in einem unerklärlichen Akt der Selbstlosigkeit, sich selber an die Barone auszuliefern, da sein Kopfgeld weit höher war, als das des Calhadril, welches die Barone eintreiben wollten. Sein Plan war, heimlich zu entkommen (siehe Firnens Bericht der Ereignisse).
Er tauchte erst am frühen Morgen wieder auf, als seine Gefährten sich gerade rüsteten, um ihn zu suchen.

\paragraph{10. Boron}
In Anderath machten die Gefährten halt, um den Praioten den gewünschten Bernstein zu bringen. Diese beendeten daraufhin das Sonnenlichtelixier und überreichten es Thoran mit der Anweisung, es nur in aller größter Not zu verwenden.

\paragraph{11. Boron}
Schnell brachen sie in Richtung Efferd auf, um die Jägerin noch zu finden.

\paragraph{14. Boron}
Am Rhodenstein angekommen hörte die Gruppe Gerüchte über eine Geweihte der Rondra, die eine Novizen ermordet haben sollte und nach Norden geflohen sei. Als die Gefährten in den Blautann aufbrechen wollten um den Hexen von Calhadrils Tod zu berichten, trafen sie auf Morena. Die Schöne Hexe erzählte ihnen, dass ein Blutsaugerin den Hexen einen wichtigen Ritualgegenstand entwendet hat, der essentiell für das Vorhaben des Zirkels sei. Die Dieben war nach Norden geflohen, genau in Richtung Acheburg... \par


\subsection{Bericht der Verkörperung Borbarads von Rezzanjin al’Ahjan}
Nachdem uns die Hexe damit beauftragt hatte das Horn des schwarzen Einhorns von der Jägerin zurück zu holen, beschlossen wir zuerst nach Nordhag zu fahren, um uns von dort in Richtung Norden zu wenden, da wir Anzeichen dafür hatten, dass die Jägerin dort verweilen würde. In Nordhag selbst hörten wir in der Taverne Gerüchte, dass eine Frau am Tag Tage zuvor erschossen worden war. Wir fanden den Tatort, jedoch keine weiteren Spuren. Wir waren uns sicher, dass es die Jägerin war, da auch der Verlobte der Frau, die getötet worden war, bei der Tat anwesend war und danach verschwunden blieb. Wir reisten am nächsten Tag gen Norden und fanden am späten Nachmittag einen Weiler. Wir traten ein, nachdem keine Antwort auf unser Klopfen erfolgt war. Drinnen fanden wir eine komplette Familie ermordet vor. Zwei davon starben vermutlich am Pfeilen. Wir verbrannten ihre Leichen und reisten dann weiter. \par

Vor Einbruch der Dunkelheit erreichten wir keinen Weiler mehr und so schlugen wir ein Lager im Wald auf. Wir stellten auch Wachen auf, was sich gelohnt hatte, da Ragnos bei seiner Schicht ein Bande Orks bemerkte, die uns ausrauben wollte. Wir konnten sie in die Flucht schlagen und die die nicht vorher durch unsere Hand starben, schienen nicht sehr viel entwendet zu haben. Der Reisetag darauf verlief ereignislos, jedoch kamen wir in Wolfspfort an, einem kleinen Dorf. Jedoch erregten wir ganz schön Aufmerksamkeit, da wir am Dorfrand zwei Leichen gefunden hatten, die wir beim Boronsanger vergraben wollten. Als ein Dörfler uns hinführte entdeckte , dass eines der Gräber aufgewühlt war. Wir fanden keine Hinweise wieso und wohin die darin vergrabene Leiche hin sein könnte. In der Nacht kamen wir bei einem Bauern in der Scheune unter. Am Abend darauf kamen wir in Scheutzen an und konnten die Burgruine schon sehen. Dort soll der Herr der Acheburg hausen und dort würden wir wohl auch die Jägerin finden. Dieses Dorf hatte, im Gegensatz zu Wolfspfort eine Schänke, jedoch bestand dort nicht die Möglichkeit Zimmer zu mieten. Doch wir mussten nicht nach einem Zimmer Ausschau halten, da nach kurzer Zeit die Rittfrau der Stadt in die Taverne kam. Sie bot uns ein Zimmer bei sich an und wir nahmen an.
Am nächsten morgen brachen wir Richtung Acheburg auf. Nicht, ohne vorher etliche Male gewarnt zu werden, dass dies sehr gefährlich sei.\par
Nach einem Großteil der Strecke hörten wir auf einmal einen Pfeil zischen und instinktiv warfen wir uns zu Boden und versteckten uns. Dann hörten wir die Stimme der Jägerin und noch weitere Pfeile zischen.
Nachdem wir nach dem Verstecken und dem so-schnell-wie-möglich-durchs-Schussfeld-laufen im Turm angekommen waren, war es ein leichtes sie zu überwältigen und zu töten. Wir fanden das Horn bei ihr und machten uns, erleichtert doch nicht zur Acheburg gehen zu müssen, auf den Rückweg.\par
Die nächsten Tage beeilten wir uns, wieder rechtzeitig zurück nach Norburg zu kommen. Am Abend gingen wir zum Waldrand und wurden von einem Uhu, der seltsamerweise sprechen konnte, tief in den Wald geführt. Dort kamen wir zu Haus der Oberhexe, nachdem wir längere Zeit gewandert waren. Dort überschlugen sich die Ereignisse:
Toran wurde ein Zeichen zuteil. Er bekam den Auftrag Frieden zwischen den Völkern zu schließen und unnötigen Streit zu verhindern, um den Fokus einzig auf den Kampf gegen Borbarad zu richten.\par
Temyr braute mithilfe mehrerer uralter Elfen mächtige Heiltränke.\par
Toran bekam gleich eine Gelegenheit um Streit zu schlichten, da eine uralte Hexe, mit der einige aus der Gruppe anscheinend schon Bekanntschaft gemacht hatten, die Herrschaft über die Hexen an sich zu reißen versuchte. Toran schaffte es sogar, trug nur leider eine Wunde durch einen Dolch davon.\par
Wir waren bis in die Nacht beschäftigt und beschlossen dann am nächsten morgen aufzubrechen. Ein Problem hatten wir jedoch noch nicht gelöst.\par
Wir wussten nicht wo das Ritual stattfinden würde, geschweige denn wie viel Tagesreisen es entfernt war. Wir vermuteten, dass es auf einem Nodix stattfinden würde. Was das genau ist, war mir noch nicht klar. Doch anscheinend hatte Firnen durch das Auge die Gabe solche Dinge zu sehen. Und auch für den Transport hatten wir eine Lösung. Anscheinend waren meine Freunde noch in Besitz eine alten elfischen Artefaktes, mit dem man sich Firnponys rufen konnte, die einen Tag und Nacht unermüdlich folgen, bzw. transportieren würden würden. Wir beschlossen auch noch bei der Burg Rhodenstein vorbeizuschauen, um ein paar Rondrianer mitzunehmen, die uns beim Kampf behilflich sein würden. Nachdem wir dort zwei Rondrageweihte mitsamt Knappen für uns gewinnen konnten riefen wir weitere Firnponys für diese. Wir machten uns auf weiter Richtung Osten und kamen bald an einem See, in dessen Mitte ein kleiner Turm auf einer Insel stand. Firnen bestätigte, das hier das Ritual ablaufen würde. Über das dünne Eis schlichen wir uns in Richtung des Turms. \par

Schnell konnten wir sehen, dass um den Turm mehrere Gestalten herumrannten. Dies waren wohl die Vampire. Wir entschieden uns für einen Überraschungsangriff, doch wir wurden überrascht, dass sich nicht nur Vampire als unsere Gegner entpuppten, sondern auch Dunkelelfenkrieger. Letztendlich konnten wir den Kampf auch durch Torans Hilfe gewinnen(Vampire scheint der Heilsegen zu schaden), jedoch war es nicht einfach. In dem Turm wurden wir von einer Stimme in den Keller gelotst. Dort fanden den Herrn der Acheburg, der darum bat ihn aus einem Dreizehnstern gebildet aus Praiosheiligtümern zu befreien. Er versprach uns sogar, dass er uns hilft. Sobald wir jedoch die ersten Heiligtümer wegräumten verschwand er aus dem Turm in die Dunkelheit. Wir machten uns auf den Weg nach oben, denn dort erwartete uns das Ritual. Als wir oben ankamen sahen wir, wie Pardona ein Ritual um einen großen Kessel herum ausführte. Wir versuchten sie anzugreifen, doch sie hielt uns fest und wir mussten tatenlos zusehen, wie sie Borbarad erschuf, er sich nicht bei ihr bedankte, wie sie es gewollt hatte und dann verschwand Borbarad noch durch ein Loch in der Turmwand, das er vorher einfach dort hinein gesprengt hatte. Wir rannten wieder bewegungsfähig aus dem Turm hinaus und konnten noch Pardona beobachten, wie sie sich mit einem weißen Drachen in die Lüfte erhob. Die Rondrageweihten hatten tapfer gekämpft, waren jedoch beim Kampf umgekommen.\par
Wir machten uns zurück nach Trallop und wurden dort von Waldemar von Weiden empfangen. Er verlieh uns allen die Ritterwürde und gewährte jedem von uns eine Jahresrente von 100 Dukaten... 

\chapter{Schlussworte}
Dies beschließt den ersten Teil der Ereignisse um Borbarads Rückkehr. Hiernach trennten sich die Wege der Gruppe, jeder ging seines Weges, immer wieder versammelten sie sich auf Klammsbrück. Du kannst dich vielleicht noch an diese Zeit erinnern, es war der Anfang von Firnens Lehrtätigkeit in Klammsbrück, die Ernennung von Arngrimm zum Baron, was wegen der engen Freundschaft zwischen Spektabillität Temyr und dem Baron zu einem großen Wirtschaftlichen Aufschwung führte. In jene Zeit fällt auch jene furchtbare Nacht, in der Toran Firnens Packt brach. Wegen meiner Vergangenheit mit dem erzdämonischen Fürsten der Rache verbrachte ich diese Nacht betend in einem Schutzkreis, der von Erwen und Toran mit göttlicher Macht gespeist wurde. Die Berichte an jenes grauenhafte Ereignis sind verschollen, kaum einer kann sich noch daran erinnern. Ich weiß noch, wie Oleg tapfer dämonische Gestalten und wahnsinnige Geister abwehrte, Toran in tiefster Meditation entrückt den Erzheiligen Therbûn sprach, Firnen den Erzdämon selbst davon abhielt, seinen Geist in die Niederhöllen zu zerren. Ragnos strenge Gestalt, wie er Toran ein um das andere Mal das Leben rettete, Flammenzunge, der damals auf Klammsbrück weilte, den Weg zur Göttin fand und Temyr mächtige Zauber warb, während er Firnen die nötige Kraft gab, um gegen der Erzdämon zu bestehen.\par
Doch auch andere Ereignisse fallen in diese Zeit. Firnens lange Studienreise nach Fasar und Selem, wo er viel über die Geschichte Zulhamids und des Auges erfuhr, Temyrs und Arngrimms Reise nach Khunchom, Torans Treffen mit den Perainegeweihten Tobriens, bei dem sein Name das erste Mal einen glanzvollen Klang erhielt. Rezzanjins erste Rückkehr nach Maraskan fand auch in dieser Zeit statt, von seiner zweiten werde ich bald berichten. Dort half er den Priestern von Rur und Gror die Heiligen Rollen der Beni Rurech zu entschlüsseln, lernte einige der zerstrittenen Rebellengruppen des Eilandes kennen und machte sich bei den Besatzern nicht gerade beliebt. Was aus dieser Geschichte wurde, darf ich nicht schreiben, ich kenne auch die Einzelheiten nicht, aber wenn wir uns einmal persönlich wiedersehen, kann ich dir davon berichten.

\end{document}
