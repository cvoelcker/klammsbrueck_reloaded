\chapter{Euch zum Geleit}
Lang, lang ist es nun her dass die Gezeichneten gefallen sind. Während ich die Tagebücher der Rückkehr der Finsternis noch in Darmstadt zusammen gesammelt habe, ist es nun fast 10 Jahre her dass wir unsere Runde beendet haben. Naja, nicht ganz, aber 10 klingt dramatischer als 9.

Die Bücher der Borbaradkampagne sind inzwischen in Philips Besitz über gegangen, und ich glaube ich habe seit mehr als 5 Jahren kein DSA mehr gespielt. Es ist schwierig Leuten zu vermitteln dass sie erstmal 3 Bücher auswendig lernen müssen um dann vlt ab und an (wenn man es überhaupt organisiert bekommt) einen Spielabend besuchen zu dürfen :). Aber selbst in Kanada bin ich dem Hobby treu geblieben und habe, zusammen mit Freunden aus aller Welt, über einige Jahre eine erfolgreiche Kampagne an einem der größten KI Forschungsinstitute Nordamerikas organisiert. 


Diese Projekt habe ich wieder aufgegriffen als ich angefangen habe zu überlegen, was ich zu Philips Hochzeit vorbereiten könnte. Mein alter Laptop aus Darmstadt hatte den Zwölfen sei Dank noch eine komplette Sicherheitskopie des alten Forums! Und so ist dieses zweite Buch vor allem und natürlich euch, aber auch unseren Partnern und Partnerinnen gewidmet. Denn wir müssen uns wohl auch bei Feli dafür bedanken, dass ich es endlich geschafft habe meine Faulheit zu überwinden und das Projekt wieder aufleben zu lassen. Und bei Heiko, der mir geholfen hat die technischen Probleme von Festplatten- und HTML-Rettung zu lösen.

Amor vincit Omnia!

Aber genug über mein langweiliges Leben im Exil, wir wollen ja schliesslich in Erinnerungen schwelgen!

Der vorliegende Abschnitt enthält sowohl eines der besten, als auch zwei der schwächeren Abenteuer der Kampagne.
Ich muss zugeben dass ich mich an Schatten im Zwielicht nicht einmal mehr Schattenhaft erinnern kann. Aber mehr dazu in den einzelnen Abschnitten.
Derrisch fällt dieser Abschnitt in das Ende unserer Schulzeit, das erste Abenteuer im nächsten Abschnitt wurde dann schon in unserem legendären ersten Arosio-Urlaub gespielt!

\chapter{Bericht über die dreuenden Schatten}


Mein lieber Waldemar,

es tut mir Leid, dass du so lange auf den zweiten Teil dieses Berichtes warten musstest. 
Ich kann dir leider nicht verraten, was mich die letzten Jahre aufgehalten hat, aber ich habe weiterhin die Tagebücher der Gezeichneten studiert.
Hochmeister Tarlisin war fast schon erbost darüber, dass dieser Bericht einige weniger schmeichelhafte Abschnitte seines Lebens beschreibt, aber nach all meiner Hilfe (und der der Gezeichneten selbst) kann er sich wohl kaum beschweren.

Es war eine sonderbare Zeit, nach der Verkörperung, aber vor dem Angriff der dämonischen Horden.
Nur wenige wussten um das was uns bevorstehen würde, und ich gestehe, selbst ich war mir nicht sicher ob alles wahr was, was sie mir erzählten.
Nachdem die Rückkehr des Dämonenmeisters vor allem im Weidenschen stattfand, führten diese Ereignisse unsere Freunde in weit entfernte Lande.
Eine Zeit lang reisten sie alleine durch die Gegend, und versuchten Verbündete gegen die drohende Gefahr zu finden.
Verbündete fanden sie dabei leider vorerst nur in der Grauen Gilde und bei einigen hochrangingen Agenten der KGIA, aber auch diese waren kaum gewillt, große Vorbereitungen zu treffen.
Selbst die Ereignisse auf Maraskan wurden wegen der schwierigen politischen Situation mehr vertuscht als aufbearbeitet.
Das sollte sich spätestens nach dem Angriff auf Andalkan rächen.

Ab hier tritts auch du in die Geschichte ein, wenn auch zuerst nur im Hintergrund.
Nach der Verkörperung des Dämonenmeisters wurde die Akademie zu Klammsbrück gegründet und von der Grauen Gilde aufgenommen, und das obwohl Firnen sich kaum zurückhalten konnte, bei jeder Erwähnung des Namens ``Garlischgrötz'' nicht laut zu lachen. Was die arme Frau getan hat um diese Behandlung zu verdienen wissen nur die Götter, aber beim Gedanken daran kann ich nicht anders als zu schmunzeln.

Ob du es glaubst, oder nicht, die Akademie war ein schwer umstrittenes Projekt! Immerhin sollte nicht jeder dahergelaufene Magus sich anmaßen einfach seine eigene Schule zu eröffnen. Das sie durch die späteren Ereignisse schnell zu einem Ort werden würde, dessen Namen ähnlich mythisch würde wie Drakonia konnte natürlich niemand wissen.
Ich hoffe das Exil in Punin bekommt euch allen gut! Ich will mich bald auf die Reise machen und euch tatsächlich besuchen, aber meine Pflichten hier, und dieses Projekt, lassen mir kaum einen Tag Zeit.

Es tut mir Leid, dass du die volle Geschichte von Firnens unglücklichem Pakt und der furchtbaren Tortur seiner Seelenrettung erst durch meinen letzten Bericht erfahren hattest. Diese Geschehnisse musst vor den Eleven natürlich geheim gehlaten werden, und ich hoffe, du denkst nun nicht schlechter über deine alten Lehrmeister, nachdem du seine Geschichte kennst.
Und nach der Schlacht an der Trollpforte wäre es wohl kaum angemessen gewesen, über die alten Verfehlungen der Helden zu reden, so grauenvoll sie auch gewesen sein mochten.


Dein alter Lehrmeister,

Spektabilitus Emeritus Iliricon Tannhaus, Ordo Defensores Lecturia



\chapter{Pforten des Grauens}

\section{Geleitwort}

Die Kommentare werden nun wohl noch schwieriger werden, nachdem ich weitere 5 Jahre habe verstreichen lassen. Trotzdem, Pforten des Grauens hat mir persönlich wahrscheinlich fast am meisten Freude bereitet. Wobei das vielleicht etwas zu viel über meinen Hang zum Sadismus aussagt? Mir ist noch sehr deutlich in Erinnerung was für eine Folter dieses Abenteuer für die Helden war, und ich habe Stunden damit verbracht, mir die bescheuertesten Kreaturen auszudenken.

Reine Wildnisabentuere sind oft schwierig besonders interaktiv zu gestalten, weil die Wildnis einfach nicht viele Müglichkeiten für gute Spielentscheidungen liefert. Ich hoffe, es ist euch trotzdem eher als spannende Geschichte und weniger als kitschiger Horrorvorleseroman in Erinnerung geblieben. Nachdem ich vor kurzem einige Tage in Sngapore verbracht habe, habe ich inzwischen deutlich mehr Mitleid mit armen Abenteurern, die sich durch schwüle Hitze schlagen müssen.

Die mir am besten in Erinnerung gebliebenen Szenen sind:
\begin{itemize}
\item Jagdgraß, eine wirklich gruselige Vorstellung für einen Campingurlaub
\item keine Szene an sich, aber eine unserer längsten Spielabende an Victors Geburtstag bei ihm zu Hause, bei dem wir wirklich alle fast eingepennt sind
\item Victors nächster Charaktertod, wirklich schnell nach dem vorherigen
\end{itemize}


\begin{flushright}
Claas Völcker, Toronto, den 25.05.2025
\end{flushright}


\section{Die Tagebücher}

\subsection{Maraskan aus ``Blutrosen und Marasken''}
Informationen aus den Büchern des Hesindetempels:

Die maraskanische Hauptinsel hat eine Länge von rund 650 Meilen und mißt an der breitesten Stelle etwa 250 Meilen. Dem Südzipfel Maraskans sind (von Nord nach Süd) die Inseln Jandraskan, Etlaskan und Jilaskan vorgelagert, der Südostküste (ebenfalls von Nord nach Süd) die Inseln Buli, Saijana und Andalkan sowie dort etliche namenlose Eilande mit häufig wechselnden Küsten. Direkt unter der Südspitze Maraskans, nur wenige Meilen von Sinoda entfernt, liegt die ``Affeninsel'' Beskan.

Abgesehen von einem schmalen Küstenstreifen von 10 bis 20 Meilen Breite ist die Insel bergig bis gebirgig und mit dichten Urwäldern bedeckt. Wie ein steinernes Rückgrat zieht sich die Maraskankette von Norden nach Süden. Eine schmale Senke trennt sie vom zerklüfteten Amdeggynmassiv. Die höchsten Berge sind der Amran Gerbald (6.500 Schritt) im Norden der Insel und der Amran Thjemen (6.300), der ziemlich genau im Schwerpunkt des Dreiecks Jergan-Boran-Tuzak liegt. Dagegen wirken die höchsten Erhebungen des Amgeggyns -- Amran Thurak (3.400 Schritt) und Amran Thjalgyn (3.200) -- fast bescheiden. Die Insel besitzt vier größere Flüsse, die in etwa den vier Himmelsrichtungen folgen, und zwar den Hira im Norden, die Mendrina im Osten, den Roab im Süden und die Obogyn im Westen. Mehrere kleinere Flüsse münden in sie. Die Küstenlinie weist zahlreiche Buchten auf, die schon immer beliebte Piratenverstecke waren. Auch nach achthundert
Jahren Besiedlungen sind große Teile des maraskanischen Binnenlandes unbekannt.

Das Küstenklima ist angenehm, auch wenn es oft von Festlandsbewohnern gleicher Höhe als zu heiß eingeschätzt wird. Im Landesinnern hingegen herrscht ganzjährig eine unerträgliche, schwül-drükkende Hitze vor, die erst in den größten Höhen der Maraskankette, deren höchste Gipfel auch im Winter nur selten von Schnee bedeckt sind, gemäßigten Temperaturen Platz macht. Die Hauptregenzeit ist zwar im Phexmond, verschiebt sich jedoch gelegentlich um mehrere Wochen. Schwere Regenstürme überziehen dann die ganze Insel und lassen die Flüsse über die Ufer treten. Eine zweite, kleinere Regenzeit (von den Einheimischen nicht als solche bezeichnet) tritt im Frühherbst ein. Generell ist die Osthälfte der Insel wesentlich feuchter als
die Westhälfte.

Zu den vielen Eigenheiten der maraskanischen Pflanzenwelt gehört eine starke Rottönung des Blattwerks während des kurzen Winters, so daß das Rot der Wälder, unterbrochen vom Gelb der verdorrten Felder und Lichtungen, dem Ocker und Schwarz der Berge aus der Luft an den Rücken einer Maraske erinnert. Typisch für die Flora der Insel ist eine Neigung zum Riesenwuchs, was sich nicht nur in Baumriesen ausdrückt, sondern etwa auch in Farnen mit über drei Schritt hohen Wedeln.

Der maraskanische Dschungel gilt als einer der gefährlichsten Wälder ganz Aventuriens. Nicht wegen Großraubtieren, von den es nur wenige Arten gibt, sondern wegen seiner Schlangen, seines schier unendlichen Artenreichtums an kleinen und kleinsten, oft giftigen Lebewesen und nicht minder bedrohlichen giftigen und fleischfressenden Pflanzen, von denen sogar einige Arten -- wie etwa Jagd- und Fallgras -- selbstbeweglich sind. Das nachfolgende Zitat des Magisters Majian da Meran ist keineswegs eine Übertreibung: ``Die maraskanischen Wälder stelle man sich besten wie ein Land vor, um dessen Besitz viertausend Könige gleichzeitig erbittert kämpfen. Jeder von ihnen hat mächtige Heerscharen aufgeboten. Menschen gelten nicht als kriegsführende Partei. Sie sind nur für die Provianteure und Marketenderinnen von Interesse.''

Die einheimische Tierwelt -- wir nehmen hier die Tierarten aus, die seit der Besiedlung eingeführt wurden, wie etwa Phraischaf oder Rind -- kennt nur wenige Großtiere, erwähnt seien Perldrache, Maran, Parder, Maraskantarantel, der waschbärgroße Baumwürger, der Harnischgürtler oder das mittlerweile äußerst seltene maraskanische Wollnashorn. Eine größere Vielfalt zeigen kleinere Tiere wie Mungo, Stinkfrettchen, Baumschleimer, Biberratte, das giftige Stachelschwein und verschiedene Kleinaffenarten. Unter Berücksichtigung der Unerforschtheit großer Teile der Insel sind diese Aussagen mit einem gewissen Vorbehalt versehen: Es gilt als nicht sehr wahrscheinlich, daß der Tuzakwurm der einzige Vertreter seiner Art auf der Insel war, auch wenn seither kein ähnliches Tier mehr gesichtet wurde.

Maraskan ist reich an Bodenschätzen: Das Gestein der Maraskankette enthält Kupfer, Quecksilber, Zinn, Blei und Eisen, letzteres vor allem im Südwesten in gehaltvollem Erz, doch auch Edelmetalle sind in geringerem Maße vorhanden, hinzu kommen Alabaster und Graphit; berühmt ist die Insel wegen der einzigen zugänglichen Enduriummine Aventuriens. Doch auch die Wälder Maraskans sind eine Schatztruhe für Heilkundige, Gewürzsucher, Giftmischer und Alchimisten. Traditionelle Anbauprodukte der Insel sind Zuckerrohr, Reis, Pfeifenkraut, Tee, Shatak, Obst und Mandeln. Die bekanntesten Ausführgüter waren bisher Edelhölzer, die als Roabwolltuch bekannte Phraischafwolle, Gewürze wie Paprika, Pfeffer oder Ingrim, Heilpflanzen, Erze und Tuzaker Stahl.

Das maraskanische Volk zählt nach unterschiedlichen, stark divergierenden Schätzungen 80.000 bis 150.000 Köpfe. Bis vor rund dreißig Jahren galt: Ein Sechstel aller Maraskaner lebt in den vier Städten Jergan, Boran, Sinoda und Tuzak, ein weiteres Fünftel in etwa vierzig Ansiedlungen, manchmal so groß wie Kleinstädte, in ihrer nächsten Umgebung, der Rest in zahllosen kleinen Dörfern, Weilern, Gehöften und auf Plantagen, von denen die wenigsten mehr als fünfzig Meilen weit im Landesinneren lagen.


\subsection{Die Anreise nach Firnen Wulfgrimm}
\paragraph{20. Ingerimm 1018 BF}
Heute kam Arngrimm mit einer Boronia vorbei. Sie überbrachte uns den Aufruf uns in Punin vor dem hl. Raben einzufinden. Ich denke wir werden gleich morgen aufbrechen.

\paragraph{22. Rahja 1018 BF}
Sind vorhin in Punin angekommen. Ein ziemlich großes Gebäude dieser Borontempel. Die Boronkirche (sowohl Punier- als auch Al'anfaner-Ritus) sendet uns aus, illegal auf Maraskan Nachforschungen über eine Endurium- Lieferung anzustellen, die verschwunden ist (nicht einmal ein Zeichen auf dem Schwarzmarkt, deshalb wahrscheinlich nicht die Rebellen.) Zur Überfahrt wird uns in Kunchom eine Zedrakke erwarten.

\paragraph{1. Prajos 1019 BF}
Die Namenlosen Tage haben wir in einem Dorf, eine Stunde vor Zhammorah, verbracht. Im Anschluss habe ich mich an den Rand der uralten Ruinen begeben, sehr zur Freude Zulhamids (Zulhamid iben Zahier = das almadiene Auge), welcher mir eine eindrucksvolle Stadtbeschreibung gab. Außerdem offenbarte er mir den mächtigen astralen Nodix, dessen Kraftlinien seltsam zerrissen und zerfranst schienen. Die Stadt strahlt etwas dunkles aus, aber auch die Verlockung uralter Geheimnisse und Macht. Wer kann sagen, was man in Zhammorah alles zu finden vermag, und welche Schrecken vergangener Beschwörungen noch immer lauern, bereit jeder Zeit wieder zu erscheinen. Sehr bedeutend schien auch ein gewisser Bastrabun zu sein (ich glaube Erbauer der Stadt\footnote{Was für eine gescheiterte Geschichtswissen-Probe spricht.}). Zhulhamid erwähnte ihn häufig und sprach auch von den Fünf Obelisken von Bastrabuns Bann. Die Stadt zu betreten wagte ich nicht

\paragraph{4. Praios 1019 BF}
Haben Kunchom erreicht. Habe mit Kadil O'Karim gespeist und einen fünmal nutzbaren Abvenumstein erhalten, als Gegenleistung steht noch ein kleiner Gefallen aus.
Ich habe mir auch die astralen Strömungen in Kunchom angesehen, der Nodix-Astralis scheint gestört, und auch Temyr meinte etwas von einer spürbaren ``Schändung der Elemente''.
Habe außerdem in Reiseberichten von zwei Hesindegeweihten einiges über Maraskan geslesen, hauptsächlich zu ihrer Kultur. Wichtige Erkenntnis: Irgendwie sind die Maraskaner untereinander über ein paar Ecken alle verwandt. Dies führt zu einem leichten Paradoxon bei Familienfehden zwischen zwei Familien (oder wäre es nicht eigentlich dann innerhalb einer Familie / der Familie: ``Maraskan''?)

\paragraph{5. Prajos 1019 BF}
Sind heute ausgelaufen. Haben die Passierscheinkontrolle gut überstanden. Hängen jetzt kurz vor Maraskan in der ``Flaute vor dem Sturm''

\subsection{Der Sturm nach Ragnos zu Grunewaldt}

\paragraph{5. Praios}
Ein Windhauch streicht über meine Wange, als ich Richtung Maraskan schaue, welches nicht mehr weit entfernt sein sollte. In diesem Moment werde ich der tief schwarzen Wolken am Horizont gewahr, wie auch der Rest der Mannschaft an Bord. Es bricht Hektik aus, die Seeleute beginnen Ladung zu vertäuen und die Segel einzuholen, auch ich beginne unsere Ausrüstung in Bündel zu schnüren und zu sichern. Der Wind nimmt zu und ein Schaukeln ergreift das Schiff welches uns hin und her wirft, als auch die Wellen sich zu echten Wänden auftürmen. Ich sehe noch wie eine gewaltige Welle den Steuermann erfasst und in den Fluten verschwinden, Boron sei ihm gnädig. Firnen hangelt sich Richtung Ruder und versucht einen Kurs zu halten, auch wenn ich nicht weiß wir er diesen bestimmt. 

Plötzlich kehrt Ruhe ein und ein kurze Pause wird uns gewährt, während das Auge das Sturms über uns zieht. Toran heilt und versorgt die schwer verletzten Matrosen wo er nur kann doch bei einigen reichen auch die Segen der Perain nicht aus. Der Kapitän geht seiner grausamen Pflicht nach und überreicht die nicht mehr zu rettenden Boron und der ewigen See. Doch es bleibt kaum Zeit die Toten zu betrauern und um ihre ewige Ruhe zu bitten. 

Die übrigen Verletzten werden unter Deck vertäut und von Toran in einen heilenden Schlaf gelegt. Firnen hält das Ruder und bindet sich zusätzlich an der Reling fest um nicht wie der Steuermann über Bord zu gehen. Da bricht auch schon mit einem Schlag die schwarze wirbelnde Wand über uns herein und zerfetzt den Hauptmast, welcher nun droht das Schiff in die See zu ziehen. Zum Glück schaffen wir es noch die Seile zu kappen. Ein Ruck geht durch das Schiff und Schlag aus dem Bug lässt mich darauf schließen das sich die Ladung gelöst hat und dabei die Wand zerschlagen hat. 

Als kurz darauf die letzten beiden Masten abbrechen und dem Sturm zum Opfer fallen, greifen wir uns unsere wichtigsten Habseligkeiten und Ausrüstung um anschließend über Bord zu springen. Wasser ist überall, in meinen Augen, meinen Mund, um mir herum und wir schwimmen immer weiter weiter bis alles schwarz wird \dots

\paragraph{6. Praios}
Es ist hell mitten am Tag. Ich denke es der 6 doch ich bin mir nicht mehr sicher.
Als ich mich aufsetzt wird mir kurz schwarz vor Augen, dann sehe ich 300 Fuß entfernt eine grüne Wand, da wird mir bewusst, dass wir es geschafft haben, wir sind auf Maraskan, der grünen Hölle. Als ich meine Freunde gefunden habe und einige verbliebene der Schiffsbesatzung, bauen wir ein Lager am Rande des Dschungels auf. Toran vollbringt dort ein wahres Wunder, er ruft einen Krug welcher nicht auszutrinken ist und wenn ihr mir es nicht glaubt mögt, ich habe es selber versucht. Als wir 3 Stunden später ein neues Lager aufbauen in der nähe eines Flusses, verschwindet Toran im Dschungel, er sagte irgendwas von einer Pflanze und der Harmonie mit der Natur.

Am Abend legen wir uns erschöpft schlafen, leider scheint aber auch die Wache der Schlaf überkommen zu haben, da ich unsanft von einem Schlag in den Rücken geweckt werde und ich in ein wenig schmeichelhaftes Gesicht eines Rebellen aufsehe, wie ich vermutete. Nach einer kurzen Rangelei werden wir in den Dschungel geführt, ich verstehe kein Wort aber das haben diese verdammten in Maraskan so an sich. Nach einem längeren Gespräch mit dem Anführer erhalten wir von diesem Ausrüstung und Wasser und einen Führer, da er unseren Auftrag anscheinend für sinnvoll hält ich weiß es nicht.

\paragraph{7. Praios}
Nach einem wenig geruhsamen Schlaf bei den Gefangen, brechen wir auf mit unserem Führer in den Dschungel. Nachdem wir eine Marschordnung gefunden haben, finden wir uns schon bald tief im tiefem Grün wieder. Nach kleineren Hindernissen und einigen befremdlichen Tieren lässt uns unserer Führer eine Wasserquelle suchen, um darin etwas Übung zu haben bevor er uns am nächsten Tag verlassen wird. 

Nach kurzer Zeit finde ich mit meinen Kenntnissen der Natur einen Tümpel. Als Firnen jedoch Wasser schöpfen will greift ihn plötzlich ein Alligator an und in diesem Moment fallen mir weitere seltsame Schatten im Wasser auf und einige weiter Alligatoren kommen aus dem Tümpel. Nach einigen Ignifaxius Zaubern und einigen wohl platzierten Schüssen, sowie kräftigen Schwertstreichen, ziehen sich die Alligatoren zurück und lassen nur einen Toten Artgenossen zurück, welchen ich mit etwas Mühe frisches Fleisch entnehmen kann. 

Nachdem wir unsere Wasservorräte aufgefüllt und gereinigt haben, machen wir uns wieder auf in die alles verschlingende grüne Hölle, bei allen zwölften wie ich diese Insel jetzt schon hasse.

\subsection{Die ersten Tage auf Maraskan nach Temyr ibn Sahid}

\paragraph{Maraskanexpedition 3. Tag}
Nachdem wir die geschuppten gûl zurück in ihr brackiges Wasser gescheucht hatten, füllten wir unsere Taschen mit zähem, sehnigem Fleisch, derweil Thoran sich der Wunden unseres halbelfischen Freundes annahm. Ich fürchte, in diesem maldrûn Dschungel wird die Versorgung von Kranken ein schwieriges Unterfangen werden, zumal die Moskitos und die drückende Hitze -- gegen welche die khunchomer Wärme eine schmeichelnde Brise ist -- auch uns Gesunden arg zusetzt. 

Wir setzten den Gewaltmarsch fort, nicht ohne ein ungutes Gefühl meinerseits, welches ich jedoch als unbedenkliche Nichtigkeit abtat. Das Letzte woran ich mich deutlich erinnere sind Koliken, Fieberschütteln und Halluzinationen, zu denen sich die Beschwerden erwuchsen\dots Dieses Land haben selbst die seleyah -- der große Bastrabun möge meine Worte verzeihen -- nicht verdient, das einen Mann derart vertrocknen und verfaulen lässt. (Die nächsten Zeilen sind angefüllt mit wüsten Flüchen und Verwünschungen, die einen Al`Anfaner Kopfschlächter zum Erbleichen bringen könnten, glücklicherweise aber auf Urtulamidya verfasst sind.)

\paragraph{4. Tag}
Wie dem auch sei, müssen meine Gefährten um mein Wohl während der Umnachtung besorgt gewesen sein, denn den nächsten Morgen erwachte ich mit neu erwachsenen Kräften und ohne Anzeichen meines vorabendlichen Kollapses. Dies, so wird sich zeigen, sollte uns zum Vorteil in einer späteren Auseinandersetzung gereichen. Zuvor sahen wir uns jedoch einem natürlichen Hindernis ausgesetzt: Ein Gehölz aus scharfen Dornen und kräftigen Ranken erstreckte sich zu allen Seiten unseres Pfades, dem wir seit den frühen Morgenstunden gefolgt waren. Nach Erwägung der uns gegebenen Wegbeschreibung und dem Rat der Sterne entschlossen wir uns für ein Umgehen des Dickichts, was gerne den halben Praioslauf in Anspruch nahm. 

Mit dem Erreichen des ausgewiesenen Sumpfes setzte auch der beständige Regenguss wieder ein, der auf Maraskan wegen der besonderen klimatischen Verhältnisse alltäglich herabfährt. Vom nassen Element umgeben fuhren wir in unserem Marsch fort, bis auf einmal die dunstverhangene Luft von einer Vielzahl klebriger Fäden zerteilt wurde und Rezzanjins Beine umschlungen. Wir duckten uns sofort hinter die Pfahlwurzeln naher Bäume und versuchten, des Aggressors gewahr zu werden, doch durch den feinen Nebel war nichts auszumachen. 

Bedächtig tasteten wir uns in die Richtung des Schusses, bis Arngrimm wortlos auf eine Baumruine vor uns deutete. Dort, von den knorrigen Ästen hängend, sahen wir zwei weiße, fahläugige und für meinen Geschmack deutlich zu große Spinnen. Der Weidener löste seinen Bidenhänder in der Scheide und stürmte auf die Monstrositäten zu, während ich eines günstig hängenden Astes über ihnen ersichtig wurde. Mit eines einfachen Motoricus` Hilfe verstärkte ich seinen natürlichen Wunsch, Sumus` Griff nachzugeben und ließ ihn auf die Spinnen fallen. Eine wurde durch die Wucht des Treffers und des eigenen Aufschlages getötet, die andere fällte Arngrimm mit einem kühnen Schwertstreich. 

Nach diesem denkwürdigen Zwischenfall suchten wir einen Rastplatz für die Nacht, einen solchen erspähte Ragnos mit geübtem Auge. Wie betäubt fielen wir in tiefen, in meinem Fall traumlosen Schlaf.

\paragraph{5. Tag}
Der nächste Morgen graute unter wenig guten Voraussetzungen. Als wir erwachten, fanden wir Thoran in einem fürchterlichen Zustand vor: Aus seinen Armen und Beinen brachen aus schwarz geränderten Wunden Büschel von Gras, so als wollte eine Pflanze aus seinem Innern erwachsen. ``Jagdgras``'', meinte Rezzanjin, ``wir werden es ausbrennen müssen.``'' Nach seiner Anleitung verfuhren wir und entfernten das Unkraut aus Thorans Körper -- der Anblick war nicht erbaulich. Geradezu erfreulich stellte sich hingegen der Gesundheitszustand meines geschätzten Collegus Firnen heraus, der über Nacht komplett genesen zu sein schien. 

Wenig gestärkt brachen wir zu einem weiteren Teil unserer Durchquerung auf, der bis auf einen Moment vollkommener Stille im Wald dankbarerweise ereignislos verlief, bis wir an die ersten Ausläufer des maraskanischen Primärwaldes gelangten. Etwa um die Mittagsstunde gelangten wir nämlich auf eine Lichtung, welche einen grausamen Fund bereithielt: Auf blutgefärbtem Waldboden, das Herz aus dem geöffneten Brustkorb gerissen, lag dort die Leiche eines Echsenmannes. Seine Kleider waren überreich mit Musterungen und Applikationen versehen und in einer seiner Taschen fanden wir einige Edelsteine, weshalb Firnen und ich auf ein Mitglied der Kristallomantenkaste schlossen. 

Angsterfüllt schlugen wir einen Weg gen Osten ein, so wie es uns einer der Rebellen geraten hatte. Hier wucherten allerorts große Pilze mit schwarzgefleckten Fächern aus dem Boden, manche davon einen Schritt hoch. Zu gerne hätte ich eines dieser Gewächse näher untersucht, doch eine Probe wäre unnötiger Ballast auf unserem kräftezehrenden Marsch gewesen. 

Einige Zeit später hallten unvermittelt durchdringende Schreie durch den Baumstand, und als wir unsere Blicke zum Himmel wandten, sahen wir dort einen Schwarm großer Vögel mit scharfen Krummschnäbeln, die über einer vor uns liegenden Lichtung ihre Kreise zogen. Kaum hatte Rezzanjin seinen Fuß auf diese gesetzt, so stürzten sich die Flugbestien auf ihn und hackten mit ihren Krallen nach ihm. Wir entschlossen uns einstimmig, um die Lichtung herumzugehen, eine Entscheidung, die wir im Nachhinein noch bereuen sollten. Das Gebüsch, welches wir der fliegenden Gefahr vorzogen, war nämlich von weitaus gefährlicheren Übeln bewohnt. 

Abermals wurde Rezzanjin, welcher vorne mit Arngrimm lief, angefallen, doch mit welchen Folgen! Urplötzlich schoss, nur als weißer Schemen erkennbar, eine armdicke Schlange aus einem Strauch in der Nähe und verbiss sich in seine unbedeckte Haut. Noch im Sturz durchstieß Rezzanjin das Ungetüm und nagelte es in die weiche Erde, als Arngrimm vorsprang und der zweiten lauernden Viper glatt weg den Kopf abhieb. Thoran eilte dem Getroffenen zu Hilfe, der, all gesunder Lebensfarbe beraubt und am Leib zitternd, keuchend auf dem Boden lag. Verzweifelt kniete der Geweihte neben ihm und musste ansehen, wie die Kraft aus dem Körper unseres Gefährten rann. Doch die Göttin der Felder hatte Erbarmen mit Rezzanjin, und so sollte er weiterhin auf unseren Pfaden wandeln. Doch der Erfolg der Behandlung blieb für uns zunächst ungewiss\dots

\subsection{Die Hochzeit nach Arngrimm von Ehrenstein}

Wieder sind wir einer Bestie der grünen Hölle entkommen. Dank Thoran kann Rezzanjin sich wieder bewegen und so brechen wir wieder auf in die Richtung des Herzen dieser verderbten Insel Maraskan und ich will nicht wiesen was uns noch an tödlichen Gefahren erwartet. Ich sehe nur grün, grün und noch mehr grün und man weiß nie was einen nach ein paar Metern erwartet. 
Temyr sagt er höre menschliche Stimmen aus nördlicher Richtung, ich höre nur die Geräusche des Dschungels. Woher weiß er eigentlich bei diesem dämmrigen Grün aus welcher Richtung die Stimmen kommen?

Nach Stunden, so kommt es mir vor, erreichen wir einen Pfad, es ist erste Zeichen von Zivilisation auf diesem götterverlassenen Eiland. Es entbrennt eine Diskussion in welche Richtung wir gehen sollen, schlussendlich entscheiden wir uns Richtung Süden zu begeben. 

Endlich verlassen wir den Dschungel und ich wünschte mir müssten ihn nicht nochmal betreten, aber die Ohren der Götter sind taub. Der Pfad führt über einen Damm durch eine Sumpflandschaft und Ragnos schießt einen fremdländischen Vögel, der laut Rezzanjin eine Spezialität ist, aber bei manchen Rebellen als heilig gilt. Weiter dem Damm runter finden wir eine zwei Tage alte Lagerstätte, dort bereitet Ragnos den Vogel vor und wir braten den Vogel über den Flammen der Zauberstäbe. Wir lassen den allabendlichen Regen über uns ergehen, der anscheinend abnimmt. 

Wir raffen uns nochmal auf, um noch etwas weg zurückzulegen. Hinter einem Hügel erblicken wir einen Bauernhof. Es öffnet uns ein kleines Mädchen, das meint:``Ihr stinkt!`` Die Hausherrin lässt uns herein, davor müssen wir alle baden, wobei Firnen als Letzter dran ist und zweimal muss, da er in irgendwelchen Käfern geschlafen hat. Während wir ins Haus gehen, muss Firnen die Nacht in der Scheune verbringen. Es gibt Reis zu essen, die Gastfreundschaft und Küche dieses Landes ist exquisit. 

In dieser Nacht habe ich einen Traum, an den letzten kann ich mich nur noch schwach erinnern, es kommt mir vor, als wäre es in einem anderen Leben geschehen. Ich wache frühmorgens auf und warte bis die anderen aufstehen. Die alte Frau erklärt uns den weg nach Alrurdan und rät uns im Rur \& Gror-Tempel nachzufragen, wenn wir den Haran suchen.

Am selben Nachmittag brechen wir nach Alrurdan auf. Dort erfahren wir im Tempel, dass wir alle steckbrieflich gesucht werden. Firnen wir im Fort vom Kommandanten erkannt und kann gerade so in den Tempel fliehen. Eine Nacht verbringen wir in Alrurdan. Ein Priester will uns zum Haran führen. Um nicht erkannt zu werden, färben wir uns mit einen Kraut die Haare.

Schon wieder betreten den Dschungel, aber diesmal den inneren tödlicheren Teil. Ich verstehe nicht, warum das Mittelreich so viele Ressourcen für diese Insel verschwendet, außerdem sind mir auf Maraskan mittlerweile die Rebellen deutlich sympathischer als diese von der KGIA gesteuerten Gardisten.

Unser Pfad kreuzt sich einmal mit dem Weg einer Patrouille, die aber zu unserem Glück ihre Steckbriefe vergessen hat. Später stoßen wir auf einen Kampfplatz, wo der Diskus von Boran gegen den Dajin gekämpft hat. Abend kommen wir zu einem Dorf, indem gerade eine Hochzeit gefeiert wird. Rezzanjin und ich lassen uns die Haare noch einmal färben. Während die anderen feiern, bleibe ich nüchtern und ausgerechnet mir passiert es, dass die Braut sich an mich schmiegt und der Bräutigam mich schlagt. Ich beherrsche mich und schlage nicht zurück, weil es anscheinend ein maraskanischer Hochzeitbrauch ist. Mitten in die Feier platzt der Weibel der Patrouille mit elf Drachengardisten. Er hält die Feier für rebellische Umtriebe und will alle verhaften. Nach dem wir den Weibel, seine beiden Korporale und drei weitere Soldaten getötet haben, flieht der Rest. Derweil sinke ich halbtot und fast ohnmächtig zu Boden.

\subsection{Rebellen nach Rezzanjin al'Ahjan}

\paragraph{Mitte Praios im Lager des Haraniadad}
Nachdem wir den Hauptmann und die Weibel niedergerungen hatten, flohen die restlichen mittelreichischen Feiglinge. Doch wir und die Dorfbewohner waren nicht ohne Wunden davongekommen. Zwei Tote gab es auf Seiten des Dorfes zu beklagen. Zu ihnen zählte unserer Führer. Doch auch Arngrimm und mich hatte es erwischt. Ihn, als unbegabten Kämpfer, der auch noch mit der für ihn falschen Waffe kämpfte, hatte es natürlich noch schlimmer erwischt als mich. Toran hatte seine helle Freude an dem Heilen der Wunden. Doch auch diesmal verbanden seine Hände die Wunden als wären sie von Schwester Peraine persönlich geführt worden. Ich redete mit den Maraskani und einer von ihnen war so nett uns für die Nacht aufzunehmen. Vorher jedoch durchsuchte Firngrimm die Mittelreicher und fand das Dienstbuch des Hauptmannes. Es stellte sich heraus, dass der Trupp zu einem Fort, anscheinend nicht allzu weit von hier entfernt, von dem man anscheinend seit längerem keine Nachricht mehr erhalten hatte, gehen wollte. In diesen Fall waren das fünf Monde. Meine Freunde begruben die Mittelreicher und wir nahmen an dem üblichen maraskanischem Begrabungsritual für die Toten teil.

Nachdem wir am nächsten Morgen gefrühstückt hatten, erfuhren wir von den erfahreneren Mitgliedern der Dorfgemeinschaft, dass Delian von Wiedbrück anscheinend mit einem Begleiter vor circa fünf Monden bei dem Dorf vorbeigekommen ist und zum Fort 16 oder Fort Retogau, dem Fort zu dem auch der Trupp Mittelreicher gehen wollte, weitergegangen ist. Des Weiteren konnten wir in Erfahrung bringen, dass Hilbert von Puspereiken, der Echsenforscher, circa zwei Tagesreisen von hier in Richtung Norden sein Lager aufgeschlagen hat. Dieser, so die Brüder und Schwestern, wüsste vielleicht, wo das Haraniadad zu finden sei. Jedoch sei die Reise gefährlich, da das Gebiet links von dem Bach, der den größten Teil des Weges bildet, von der Rebellengruppe Diskus von Boran besetzt ist, die, wie wir wussten, mit dem Haraniadad verfeindet sind.

Wir entschieden uns jedoch zum Fort zu gehen, um zu schauen was dort los war. Der Waldweg der einst dort hingeführt hatte, war größtenteils überwuchert und sehr schwer wiederzufinden. Noch dazu kam, dass es in diesem Teil des Dschungels besonders schwül und heiß war. Nach einer Weile hörten wir die Geräusche einer Axt. Wir warteten ein wenig und schnell fanden wir uns mit einem mittelreichischen Soldaten konfrontiert. Dieser, namentlich Gefreiter Bernhelm, wirkte ein wenig verwirrt und suchte einer seiner Kameraden, der angeblich in den Dschungel gegangen seien soll. Er war uns nicht böse gesonnen und wir konnten ihn schnell überreden uns zum Fort zu begleiten. Auf den Weg dorthin kamen wir an einem Wachturm vorbei. Ich wollte hochklettern, um denjenigen der dort oben schlief zu wecken, jedoch krachte direkt die erste Stufe unter meinen Füßen zusammen. Wir weckten den oben Schlafenden, der direkt die Stufen hinunter steigen wollte, jedoch brach bei der ersten Stufe das Holz direkt weg und er stürzte in die Tiefe und blieb reglos liegen. Das wirkte doch alles recht seltsam, als ob er mehrere Monate schlafend auf den Turm verbracht hätte. 

Als wir in das Lager kamen trauten wir unseren Augen nicht. Der Kommandant scheuchte einige Soldaten mit voller Ausrüstung im Lager herum, an einer Destilliere lagen ein paar offensichtlich durch gepanschten Wein gestorbene herum und allgemein wirkte das Fort ziemlich planlos und verwittert. Selbst die Mittelreicher dürften wissen, dass man sich in einem Dschungel anders verhält. Wäre eine Rebellengruppe vorbeigekommen, würde sie innerhalb einer halben Stunde wohl alle Mittelreicher vernichtet haben. Wir fanden noch einen einigermaßen bei Sinnen gebliebenen Heiler, der uns ein wenig durch das Lager führte. Während wir noch in dessen Zelt weilten, war Arngrimm hinaus gegangen und hatte es doch tatsächlich geschafft, sich mit dem Hauptmann anzulegen und sich verhaften zu lassen. Wir konnten ihn befreien und ich konnte den Hauptmann davon überzeugen, dass Arngrimm eigentlich ein Baron sei, woraufhin dieser beschloss ein Festmahl für ihn auszurichten. Der Fraß war grausam und wir aßen allesamt nichts. Irgendwann beschloss der Hauptmann den noch übrigen Gefangenen, die wohl schon alle gestorben waren auch etwas zu essen zu geben. Alle anderen Soldaten gingen mit dem Hauptmann mit, bis auf einen. Dieser wirkte relativ normal, denn auch er rührte sein Essen nicht an. Auf Nachfrage bestätigte er, dass er erst seit kurzem im Lager weilt und, dass sogar der Medicus, den wir für normal gehalten haben, eigentlich ziemlich verrückt sei, da er das Skelett, das beim ihm Zelt liegt, eigenständig mit dem Mund von dem unnötigen Fleisch befreit hat. Wir versuchten die Nacht unbeschadet zu überstehen und kehrten am nächsten Tag wieder ins Dorf zurück, wo wir noch eine Nacht verbrachten. 

Am nächsten Tag deckten wir uns im Dorf mit Nahrungsvorräten ein und holten uns letzte Informationen über den Echsenforscher ein. Den nach Süden führenden Bach zur Quelle Richtung Norden folgend, machten wir uns auf und marschierten durch das Sumpfgebiet. Der Tag war brutal, wir kamen jedoch vergleichsweise schnell voran. Am Abend suchten wir, von unzähligen Mückenstichen überseht, ein Lager und fanden dies schnell. In dieser halbwegs ruhigen Nacht konnten alle außer Firnin recht gut schlafen.

\paragraph{Einschub aus dem Tagebuch Firnin Wulfgrimms} Der Traum im Dschungel:
Verfolge die Spur eines Pferdes, doch es ist kein Pferd, das die Spur
hinterlässt. Bei dem Wesen trottet ein Bluthund treu ins Verderben. Wen
ich ihn nicht finde, ist er verloren. Am Himmel im Efferd zeigt sich ein
sechs fingeriges Madamal.

\paragraph{Der folgende Tag}
Den Tag drauf folgten wir dem Bach weiter und trafen bald auf die Quelle. Eigentlich ein wirklich beschaulicher Ort, so aussehend wie der Dschungel klischeehaft immer beschrieben wird. Jedoch waren wir uns nicht mehr sicher, ob wir hier richtig waren. Nach kurzer Diskussion machten wir uns weiter auf Richtung Osten zur Ruine, die der Echsenforscher gerade erforscht. Schon bald gingen wir bergan und trafen auch schon bald auf ein Hindernis. Gerade so konnten wir vor einem jähen Abgrund stehen bleiben. Mittels mehr oder weniger magischen Mittel schafften wir es alle sicher hinunter. Nicht weit dahinter fanden wir einen wahrhaft imposanten Echsentempel, der von einigen Zelten umstellt war. Es handelte sich glücklicherweise um den von Puspereiken erforschten Tempel. Diesen fanden wir recht schnell und nachdem wir ihn ein wenig ausgefragt hatten, wurde offensichtlich, dass wir auf dem richtigen Weg waren. Um das Haraniadad zu finden empfahl er uns an seinen Zeugmeister weiter. Dieser bestätigte was Puspereiken schon angedeutet hatte. Die Forschertruppe unterhielt Beziehungen zu den Rebellen. Kurze Zeit später kamen auch schon zwei dieser vorbei und ließen sich überreden, uns ins Lager des Haraniadads mitzunehmen. Dann am Abend kamen wir ins Lager, was sich gar nicht großartig von anderen Rebellenlagern zu unterscheiden schien: Die Häuser waren unscheinbar in die Bäume gebaut worden. Wir wären wohl daran vorbei gelaufen, hätten unsere Führer nicht Halt gemacht. Schnell wurden wir zu einem kleinen Platz gebracht, wo schon mehrere Moha-Maraskani auf uns warteten. Ein großer muskulöser begrüßte uns und fragte uns erstmal aus. Erst nach einer Weile wurde mir bewusst, dass ich dem Haran gegenüber saß. Er willigte sogar ein, dass er uns die Mine zeigen würde, jedoch mussten wir ihm vorher mit einem Spinnenproblem helfen, was wohl ziemlich groß war, wenn man die Größe der Spinne betrachtet. Wir bekamen ein Nachtlager zugeteilt und waren froh dass wir uns nach den Strapazen des Dschungels endlich mal wieder auszuruhen zu können.

\paragraph{In der Nacht nach dem Traum:}
Doch die Nacht wurde unruhig. Ich hatte wieder eine dieser extrem unheimlichen, doch anscheinend bedeutsamen Träume. Wieder kämpfte ich, wieder besiegte ich, wie könnte es anders sein, alle meine Gegner. Doch der letzte, der stärkste, der Platzhalter, der, der dazu bestimmt war zu verlieren, stand mir noch im Weg. Eine Echse, doch alt, zu alt. Ich wusste, ich würde ihn besiegen.
An dieser Stelle wachte ich auf einmal auf und blickte auf. Wie schon bei meinem ersten Traum dieser Art im Dschungel blickten mich zwei gelbe Augen durch die Tür des uns zugeteilten Hauses an. Doch genauso schnell, wie ich mir überlegte aufzustehen, waren sie auch schon verschwunden.

\paragraph{Der Tag danach:}
Der Tag brach an, doch etwas schöneres als das Lager sollten wir an diesem Tag nicht erblicken. Nach dem Frühstück brachen wir mit dem Haran, fünf maraskanischen Kriegern und der Maraskani Alwidja auf um nach der Spinne zu suchen. Schon bald fanden wir erste Hinweise: Netze, die von Baum zu Baum gespannt waren. Doch schon bald konnte man nicht mehr durch zwei Bäume hindurchgehen, da man sonst in ein Spinnennetz geraten wäre. Der ganze Wald war ein einziges Spinnennetz. Wir schlugen uns durch und fanden nach einigem Suchen einen Gang der ins Erdreich führte. Da hier der Boden und die Bäume besonders dicht von Spinnennetzen bedeckt waren, vermuteten wir, dass die Spinne sich in der Höhle befinden musste. Nach einiger Zeit schafften wir es durch Ausräuchern sie aus der Höhle herauszuholen.Wir kämpften gut, doch die Spinne verteidigte ihr Leben aufopferungsvoll und sie ging , sich mit eine Übermacht konfrontiert sehend, vom Ausweichen ins Gegenhalten über. Das setzte uns ziemlich zu und Arngrimm brach kurz nachdem er die Spinne mit einem Streich besiegt hatte zusammen. Er scheint doch kein allzu schlechter Kämpfer zu sein. Toran versorgte ihn und wir machten uns auf den Rückweg\dots

\subsection{Fragmente gesammelt von Iliricon Tannhaus}

\paragraph{Mitte des Sonnenmondes, aus dem Tagebuch der Alwidja}
Die Fremdjis haben sich der schwarzen Brazurgah (AdV: die schwarze Spinne) angenommen. Gepriesen sei Rurs göttlicher Plan, dass dieser Stachel von unserem Rücken genommen wurde. Wie versprochen bringe ich die Fremdjis zur Amran-Anji-Mine. Ein junger Späher begleitet uns, hoffentlich ist das genug um Zusammenstöße mit dem Diskus zu vermeiden. Die Gruppe wird es nie bis zur Mine schaffen. Der Fremdji-Adlige (es ist unglaublich auffällig, dass keiner wirklich auf seinen Stand hört) ist schwer mitgenommen von seinem Kampf gegen die Spinne. Trotzdem hat er es geschafft ,mehrmals auf sein dummes Schwert hinzuweisen, Rur, warum hast du den Garethjis keinen Verstand gegeben. Seine Waffe wird ihm im Dschungel auch nicht viel nützen, wenn er sie nicht schwingen kann.

\paragraph{Aus dem Spionagebericht 26/b, Empfangene Depesche des Spitzels im Pass-Fort}
Die Gotongi berichten, dass die gesuchte Gruppe sich weiter durch den Dschungel kämpft. Ein Überfall durch maraskanische Tausendfüßler, Federn genannt, hat ihren Marsch kurzzeitig gestoppt, jedoch nicht aufgehalten. Ihre Rebellen-Freunde scheinen jedoch tot. Die Hexe Laaranya immer noch wütend wegen ihres Familiari. Einer der Flüchtlinge, der tobrische Magus, ist fast zu Boron gegangen, wurde jedoch gerettet. Im selben Zuge wurden alle daimonischen Spitzel vernichtet. Vorfall muss erst weiter überprüft werden.

\paragraph{Bericht ZZsl'ast, Petromant und Diener der H'sintoi}
Wir haben den Auftrag des Wächters von Ssel'Altach erfüllt. Das Szepter der H'Charyb'Yzz wird geborgen. Wir fanden den Warmblüter im Tempel der ZZah, und seine Gefährten scheinen den Namen gefunde zu haben. Die Tür bleibt jedoch verschlossen, das Siegel ungebrochen.
Ich weiß nicht ob dieser müde Warmblüter bereit für das N'Churr ist, aber wir stellen die Weißheit des Eivaters nicht in Frage. Ein weiterer H'Charyb'zak wurde ausgemacht. Irgendetwas braut sich am Meer zusammen, die tiefe Göttin hat ein Auge auf uns geworfen. Das ist nicht gut.
Einer unserer Späher ist zum Heiligtum aufgebrochen. Sollten die Warmblüter Erfolg haben, so wird er sie zum Tempel bringen.

\paragraph{Ende des Sommermondes, Aus dem Notizbuch Irasijad}
Wir haben eine Gruppe abgerissene Reisende aufgegriffen. Der ewige Wald scheint ihnen heftig zugesetzt zu haben, denn sie sind abgemagert und stinken. Doch, bei Rur, uns geht es auch nicht besser. Habe mit ihnen vereinbart, sie bis zur Mine zu führen, dort werden wir herausfinden, was mit Dajin (AdV: ihrem Gatten) geschehen ist. Die Reisenden sind eindeutig Fremdjis, aber scheinen von ihren eigenen Leuten verfolgt zu werden. Eine wirre Geschichte, der ich nicht ganz folgen konnte. Merechdjin, Alrandra und mein Sohn begleiten uns.

\paragraph{Einige Tage später:}
Wir haben die Mine erreicht. Alle tot, ich kann nicht mehr. Haben nur einen kleinen Teil unserer Gefährten gefunden, wo ist der Rest. Die Mine scheint verlassen, die Garethjis und unsere Mannen, sowie Paktierer der Bruderlosen, so behaupten die Magier der Fremdjis, alle tot. Die Garethjis kümmert nur das Erz, sie pressen uns weiter. Bruder Boron, wenigstens haben sie eingewilligt, uns zu helfen, die Toten zu verbrennen.
Sie murmeln unter sich, meinen wir hören nicht zu. Ein gewisser Praiot von Rallegau, so glauben sie, ist irgendwie für das Massaker verantwortlich. Bei den zwölf Geschwistern, bei Rur und Gror, ich werde diesen Kerl finden und ihn so lange prügeln, das er Tsas Geschenk gar nicht mehr empfangen will.


\subsection{Der Friedhof der Seeschlangen nach Firnen Wulfgrimm}
\paragraph{Vor wenigen Tagen war wohl Anfang Rondra:}
Wir haben eine mehrfach wieder Freigehauene ehemalige Straße entdeckt, die vom Friedhof der Seeschlangen zum Tempel, welcher der Ursprung der der dämonischen Pervertierung der elementaren Kraftlinien zu seien scheint, führt. ein stück weiter Richtung Tempel sind wir in ein Kreuzfeuer geraten, welches wir bloß alle überstanden haben, weil sich uns Perain ein weiteres mal gnädig erwies. Es waren der Bogenschützen zweie, ihre Pfeile verströhmten Niedrhöllische Kälte, und ein Schwertkämpfer der mächtig drein zuschlagen vermochte und mehrfach den ämon Kamoth erwähnte in seinen Redewendngen. Meinen ersten vermutungen nach, handelt es sich hier bei um zwei Nagrach- und einen Belhalharpaktierer. Für die Nacht haben wir uns hinter den Tempel begeben, weil dort werden sie gewiss nicht nach uns suchen.
Die See liegt gespenstisch dunkel vor der Steilküste, auf welcher der Tempel steht, und Toran der noch schwer verwundet darniederlag, wollte weg von diesem Ort, wo die Dämonen ihrere Präsenz so zur Schau stellen, das der Rechtgläubige die nähe seiner Götter nicht mehr zu spühren vermag, als mieden selbst sie diesen Ort.

\paragraph{Nächster Tag:}
Geschlafen habe ich wieder rcht schlecht lediglich meine Astralen kräfte fühlen sich stärker an, als ich nach nur einer Nacht vermutet hätte.Nach unserem kurzen Morgenmahl werden wir uns dem Tempel widmen, auf dass er uns sen Geheimnis, welches er birgt, offenbare, und wir herausfinden, wer die Paktierer als Wachen aufgestellt hatte.

\paragraph{Notizen zum Tempel:}
Wir haben eine Schmiede gefunden in einer ehemaligen Schlafstatte. Eine unvollendete Endurium-Plattenrüstung hing auf einem Ständer und drei Stein Endurium in Barren lage auf der Wegerkbank. Aus den Notizen, die in einer Mischung aus Zelemja, Hialding und Zayat verfasst waren, kann ich entnehmen, dass sie außerdem hier noch drei Schwerter aus Endurium geschmiedet haben, die sie in einem längeren Ritual drei Erzdämonen zu weihen gedenken.

Fast noch verstörender ist die achsische Schrift, im geradeaus führenden Gang des Tempels, die von einer Weiheanrufung des neun Armigen Sohnes der nachtblauen Tiefen, Sohn der dunklen Herrin, und Zorn des Meeres, verkündete. Wahrscheinlich auch ein Bestandteil der Schwertweihe.

In einer noch vor kurzem benutzten Schlaf- und Wohnstätte fanden wir soeben weitere, sehr interessante Informationen:
Einen Briefwechsel von Rallerau, der meist mit irgendwelchen Kontaktpersonen in Tuzak geführt wurde, und einen beutel mit (entwerteten) Effertsteinen.
Ebenso fanden wir einen gefesselten Magus, der bei lebendigem Leibe zu verrotten schien. Der Preis für den Nutzen niederhöllischer Macht ist oftmals der eigene Zerfall.
 Wir erlösten ihn durch einen schnellen Kopfschuss.

Am Ende des Haupteinganges befindet sich eine Kammer, in der ein gigantisches Portal konstruiert wurde. Es führt in irgendeiner Weise in die Tiefen, wo das Ritual von Statten geht. Es ist ein grausermer Anblick mächtiger, bösartiger Magie -- keine ganze Ottar bekommt mich dort durch (Mir graust es bei jedem Gedanken daran, was sich in den Tiefen alles befinden könnte.)

Nachtrag: Keine ganze Ottar? Nein, aber Arngrimm! Die anderen schienen nicht begriffen zu haben, was ich ihnen über mein Analyseergebnis das Portal betreffend mit zuteilen versucht hatte. Mit roher Gewalt buxierte Arngrimm mich hindurch, die anderen folgten uns einfach freiwillig.

Wir fanden uns in einem uralten Kellergang wieder, der schließlich in einer großen Grotte endete. Dort gewahrten wir dann das Ritual:
3 Pforten zu den Niederhöllen waren geöffnet worden, dem Herrn des verbotenen Wissens, der frostigen Jagd, und der Herrin der tiefblauen See zum Durchschreiten. 3 Dämonen (wohl jeder Domaine entsprungen) waren gebannt in ein großes Heptagramm in der Mitte zwischen den Pforten.
Rallerau befand sich auch unter den Ritualhelfern!

Wir schlugen sie vernichtend und schickten einen Großteil der Beteiligten in die Niederhöllen. Doch mussten wir schließlich fliehen, als ein Paktierer aus der Charyptoros Pforte den neunarmigen Sohn selber zu beschwören schien. Ein Kraken von gigantischem Ausmaß und Niederhöllischer Macht. Er zerschlug den mächtigen magischen Schutzschild, den ich während des Kampfes gegen die Dämonen und ihre dienenden Herrn gewebt hatte, mit nur einem Schlag einer seiner Tentakeln, fast vollständig. Aus dem Ritual retten konnten wir, neben unserem Leben, nur das Zepter der Charyptoros, das die Echsen von Rezzandjin begehrten. Selbiger aber kann froh sein dass er überhaupt entkam, denn aus unzähligen Wunden bluten, hatte er während des Kampfes danieder gelegen, doch Peraine gewährte Toran ein weiters Leben zu retten, und er schaffte tatsächlich Rezzandjin am Leben zu halten.
Die dämonisch ``geweihten'' Schwerter stürzten wohl mit in die Niederhöllen, auf dass sie dort verrosten mögen in Belhalhars kaputtem Reich!

Es grenzt an ein Wunder, das wir den Tempel überhaupt wieder verlassen haben und ich frage mich ernsthaft, wie meine Gefährten noch immer sich um des verräterischen Mamons wegen (Endurium) streiten können. Den Zusammenhalt unserer treuen Gefährtengrupe vermag er zu gefärden, und trotz solch göttlicher Wunder, wie sie uns fast schon zuofft wiederfahren sind, wollen sie entgegen Torans Weisung ihr Endurium nicht zu Ehren der Götter, denen sie ihr Leben verdanken, deren Kirchen opfern, sondern aus reiner Eitelkeit und purem Egoismus sich selbst daran, über ein gebotenes Maß hinaus, berreichern. Oh Hesinde, du Mutter aller Weißheit, Nandus du Herr der alles Wissen verschenkt und gütige Mutter Peraine, die du uns Menschen stets zugewand warst, in unseren Nöten, lasset Einsicht wachsen in meinen Gefährten, auf das sie nicht um des Mamons willen fehlen und die ``Gefährten im Kampf gegen Bobarad'' auseinander brechen und sich zerstreuen, ohne je etwas zu erreichen versucht zu haben.

\subsection{Der Dritte der Sieben Gezeichneten nach Temyr ibn Sahid}

\paragraph{Maraskanexpedition 21. Tag}
Schwarze Wolken bekränzten das Firmament, als der Tempel des Schreckens für alle Zeiten in den Fluten versank und die Ausgeburt der Niederhöllen mit in die brodelnde Tiefe zog -- ein wahrlich bittersüßer Moment für uns. Denn weit draußen wartete bereits die Dämonengaleere, Paktierer und Ritualfoki aufzunehmen. Der harte Kampf und all die Opfer waren vergebens, wir hatten gefehlt: Die Pforten des Grauens waren geöffnet worden. In sicherer Entfernung betrachteten wir das Schauspiel, die Finsternis in Herz und Hirn spürend, bis wir schließlich den Rücken kehrten und nicht mehr zurücksahen. Schweigend zogen wir an den gewaltigen Knochen der Seeschlangen vorbei und folgten der herabschießenden Steilküste, bis wir endlich wieder frei atmen konnten -- wir hatten die pervertierte Zone hinter uns gelassen. Doch den daimonischen Fängen waren wir noch längst nicht entkommen, zeigten die Verletzungen meiner Gefährten doch nun das volle Ausmaß ihrer üblen Kraft. Rezzanjin war wohl am härtesten getroffen und musste von Ragnos und Arngrimm mehr tot als lebendig getragen werden; er hatte es in den Katakomben mit drei Gegnern zugleich aufgenommen. Überhaupt schienen Thoran und ich als einzige von jeglichen Wunden verschont geblieben zu sein, trotzdem hatten wir einen hohen Blutzoll gezahlt. Mühsam schleppten wir uns denn zu einem Rund aus Baumstämmen und Findlingen, welches Ragnos als brauchbarsten Lagerplatz erspäht hatte. Während die Sonne blutrot im Perlenmeer versank, errichteten wir eine provisorische Bettstatt und behandelten die schwärenden Verletzungen des Gefechtes. Unter den heilsamen Händen des geschätzten Thoran versanken die Getroffenen alsbald in einen heilsamen Schlaf, in welchen ich mich in Anbetracht des beshar Essens ebenfalls gewünscht hätte. (Ragnos ist ein guter Mann und ich möchte seine Jagdkünste nicht missen, aber über die Zubereitung von Speisen weiß er herzlich wenig\dots)

Im Schein des Feuers studierte ich das Kartenmaterial, welches ich vorsorglich in dieses Tagebuch übertragen hatte, insbesondere jene Aufzeichnungen über Maraskan sowie die Berichte der Besatzungsmannschaften. Am Klügsten erschien es mir, zuvörderst den Weg nach Boran einzuschlagen, dann jedoch die Straße in Richtung kleinerer Küstendörfer zu verlassen. Mit Glück und Verstand ließe sich gewiss ein Schiff finden, welches uns durch die Blockaden schleusen könnte. Thoran stimmte meinen Überlegungen zu, wollte jedoch zunächst auf das Urteil Rezzanjins warten, welcher über profunde Kenntnisse der maraskanischen Geographie und der mittelreichischen Besatzung verfügte. Zudem gab er zu bedenken, dass wir zur Überfahrt nach Khunchom einerseits ein potentes Schiff benötigten, andererseits aber auch noch dem Rätsel um Delian von Wiedbrücks Verhalten nachgehen müssten. ``An Tuzak kommen wir also nicht vorbei'', meinte der Geweihte und widmete sich dem soeben erwachten Rezzanjin, der, ausreichend genesen, die Bewachung unseres Lagers übernehmen wollte. In Gedanken über das weitere Vorgehen schlief ich schließlich ein\dots

\paragraph{22. Tag}
Der nächste Tag dräute unter dem Stern des Verrats: Tatsächlich hatte Rezzanjin während seiner Nachtwache einigen Echsen den Zugang zu unserem Lager gewährt. Echsen! Es grenzt an ein Wunder, dass uns diese geschuppten gûl nicht Nächtens erdolchten, während wir hilflos schlafend danieder lagen! Als ob der Maraskaner nichts aus den letzten Tagen gelernt hätte, wo er doch unter dem Einfluss dieser niederträchtigen Kreaturen fast getötet worden wäre -- überhaupt, wann hätte man je Gutes von diesen Wesen gehört? All ihr Wissen bedient sich höchst zweifelhafter Praktiken, und ihres Hanges zur Beschwörung von Wesenheiten der siebten Sphäre sind wir ja alle ansichtig geworden\dots

Dennoch bin ich von ihnen beeindruckt. Sie strahlen eine Ruhe und Sicherheit aus, wie es wohl nur alten Kulturen vergönnt ist. Besonders faszinierend ist ihre Art der arkanen Wirkung. Die Zauber, welche ich sie wirken sah, erscheinen mir alle eine ähnliche Entsprechung in der Spruchmagie zu haben, jedoch überwiegt bei ihnen die rituelle Komponente: Zur Ausführung bediente sich der Echsenmagier einiger Edelsteine, wodurch er etwa die Kraft eines Balsam Salabunde potenzierte und die Wunden meiner Gefährten mit Leichtigkeit schloss. Vielleicht sollte ich mich in naher Zeit der Erforschung der Kristallomantie widmen, allein des uralten Wissens wegen.

Wie sich herausstellte, waren uns Gruppen von Echsen gefolgt, um die Erfüllung ihres ``Auftrags'', namentlich die Beschaffung des von uns geretteten Artefaktes aus dem Echsentempel, zu überwachen. Dementsprechend erregt waren sie, als wir ihnen das Zepter aushändigten. Ihr Anführer, der Kristallomant, dankte uns in der gemeinen Sprache für den Dienst, den wir seinem Volk erbracht hätten und forderte uns auf, ihn bei der Übergabe des Zepters zu begleiten: Abermals zog er edle Steine aus den Taschen seines bestickten Gewandes und vollführte eine Art magischen Tanz; minutenlang zischte er Beschwörungen in der Sprache der Echsen und rieb dabei die Diamanten in seinen Händen. Schließlich richtete er sich unvermittelt auf, nickte uns bedeutend zu und lief mit seinen Begleitern in den Dschungel hinein -- allerdings an den zweiten Teil so schnell wie ein kräftiger Mann zu rennen vermag. Sprachlos blickten wir ihnen hinterher, dann folgten wir den Echsenmännern. Diesen wilden Lauf durch den Urwald werde ich niemals vergessen: Mit ungeheurer Geschwindigkeit jagten wir durch das Unterholz, während links und rechts Pflanzen wie Tiere an uns vorbei schossen, ohne unseren Weg zu stören. Es war, als wichen alle Hindernisse, die unser eigenes Vorankommen so sehr verzögert hatten, bei unserem Erscheinen aus, stets fand sich eine Lücke im dichtesten Gestrüpp und kein einziges Mal wurden wir von Raubtieren behelligt. Dabei wurden wir des Laufens nicht müde, unerbittlich und fern jeglicher Anstrengung preschten wir voran. Firnen bestätigte mir schließlich, was ich bereits vermutet hatte: Der echsische Magus hatte die Kraft des Humus und des Windes auf uns herab geholt und so ein gefahrloses und schnelles Bewegen innerhalb der Wälder ermöglicht. Meine Achtung für die Echsen wuchs durch dieses Ritual ganz erheblich.

\paragraph{25. Tag}
Volle drei Tage rannten wir so durch den Dschungel und legten beträchtliche Entfernungen zurück. Jeden Morgen wiederholte der Kristallomant seinen Zauber und den restlichen Tag ging es in einem Ritt durch maraskanischen Wald. Am Abend jenes letzten Tages traten wir schließlich unvermittelt auf eine blau schimmernde Lichtung, die von einem gewaltigen Tempelbau beherrscht wurde. Rund um die Kultstätte drängten sich zerschlissene Zelte und Werkbänke, Tonamphoren und Steinscherben lagen wild verstreut auf dem zerstampften Boden -- augenscheinlich handelte es sich um das verlassene Lager des Hilbert von Puspereiken. Unsere echsischen Begleiter bedeuteten uns, vor dem Tempel zu warten und verschwanden für gewisse Zeit in der Stufenpyramide. Die Minuten verstrichen in angespannter Erwartung -- dann traten sie, gekleidet in kunstvoll ziselierte Brustpanzer, wieder auf den Vorplatz. Gleichzeitig erhoben sie die Stimmen und stießen einen kehligen, zischenden Schrei aus, der sanft in den Wäldern hinter uns nachhallte. Wie zur Antwort trat ein uralter, von Wunden gezeichneter Echsenkrieger aus dem Schatten des Tempelportals auf die Lichtung. Mit seinen drei Schritt Höhe überragte er Arngrimm mühelos um vier Haupteslängen, und die brachiale Kriegsaxt, welche er schulterte, maß gewiss die Länge eines Andergasters. Lange starrte er uns aus getrübten Augen an, dann wandte er sich in seiner gutturalen Sprache an die gerüsteten Echsen. ``Kniet nieder, ihr, die ihr solch einen großen Dienst an unserem Volk getan habt und neigt euer Haupt vor dem Wächter Akrabaals, des Letzten der Leviatanim'', wies uns der kleinere Echsenkrieger an. ``Ihr seid würdig, euch mit den größten Kriegern der Echsen zu messen. So stellt euch der verpflichtenden Ehre im Duell, als Beweis eurer Würde!''

Ohne eine Sekunde zu zögern, trat Rezzanjin entschlossen nach vorn: ``Ich werde für die Ehre der Menschen kämpfen!'' Ich kann nicht sagen, dass mich dies überraschte, hatte Rezzanjin doch auf unserer Reise die engsten Kontakte mit den Echsen geknüpft und zeichnete sich zudem durch seinen meisterlichen Umgang mit der Klinge aus. Die Echsen nickten ernst und forderten uns auf, ihnen in den Tempel zu folgen. Doch kaum hatte ich die Schwelle übertreten, regte sich die Schlange um meinen Hals und zuckte ekstatisch, so wie ich es nie zuvor gesehen hatte. ``Das zerrissene Weltengefüge!'' zischte sie angeregt in mein Ohr, ``Es setzt sich wieder zusammen -- doch nicht wie du es kennst!'' 

Verwirrt schloss ich wieder zu meinen Gefährten auf, die dem Gang unentwegt hinunter gefolgt waren. Der steinerne Korridor erweiterte sich bald zu einer geräumigen Halle, die angefüllt war mit den verschiedensten Waffen: An den Wänden hängend, in schmalen Ständern stehend oder verstreut auf ehernen Tischen lagen Exemplare jeder erdenklichen Waffe, sowohl bekannte menschliche, als auch fremdartige, von anderen Humanoiden gefertigte, versehen mit Zacken und Stacheln und oftmals gefertigt aus den seltensten Materialien. Ein alter Echsenmann in schimmernden Kleidern empfing uns: ``Der Streiter der Menschen ist eingetroffen. Wer sind eure Sekundanten?'' Rezzanjin deutete auf Firnen, Thoran und mich. ``Diese drei sollen den Wettstreit bezeugen.'' Doch es war keine menschliche Stimme, die da aus Rezzanjins Munde kam, sondern ein tiefes Zischen und Fauchen, welches wir zu verstehen jedoch keine Not litten. Denn auch wir waren keine Menschen mehr, sondern waren unbemerkt selbst Echsen geworden. Erstaunt blickten wir an unseren geschuppten Körpern herunter, während der Turniermeister die Regeln des Schiedes erklärte: ``Es wird gekämpft bis zum zweiten Blut, bis zum Fasttod. Es ist den Probanden untersagt, göttlichen oder arkanen Beistand zu fordern, sowie Maßnahmen, die im Widerspruch zum Ehrenkodex der Echsen stehen: Die vollkommene Gleichheit der Kämpfer. Der Glaube eurer Kriegsgöttin zählt in dieser Welt nicht. Wählet nun eure Waffe, Mensch!''

Rezzanjin zog sein Tuzakmesser wie zur Antwort aus der Rückenschnalle und hielt es beidhändig vor sein Gesicht: ``Ich habe die Wahl bereits getroffen.'' Der alte Echsenmann geleitete uns in einen anschließenden Gang, welcher zu einer Seite hin in ein enormes Portal überging, ausgefüllt von einer hölzernen Wand mit eisernen Beschlägen. Gedämpft drangen von der anderen Seite Geräusche an unsere Ohren: Stimmen, Gelächter, der Lärm einer großen Menge von Echsen. Auf ein Hornsignal hin wurde die Wand an Ketten nach oben gezogen und gab den Blick frei auf eine wahrlich gigantische Arena, ein sandgefülltes Rund von wenigstens 100 Schritt Durchmesser, umgürtet von hohen Mauern. Rings darauf waren Emporen mit Rundgängen und Sitzplätzen für die Zuschauer angebracht; die Reihen des Publikums erhoben sich scheinbar endlos in den gelben Himmel. Die Zahl der Anwesenden mag in die Zehntausende gegangen sein, ein Meer aus Schuppen und Reptilienaugen fieberte angespannt dem schicksalsträchtigen Ereignis entgegen. Manche der Echsen ähnelten eher uns, andere hingegen unterschieden sich in Form und Farbe ihrer Augen oder der Körpergröße: Es gab welche, die an Wesen des Meeres erinnerten, mit kurzen gedrungenen Gliedmaßen und Schwimmhäuten an Händen und Füßen, aber auch solche mit ledrigen Flügeln und krallenbewerten Greifern. Alle Echsen Aventuriens, vielleicht auch Deres, schienen diesem Ereignis beizuwohnen, dem Ruf des Kampfes und der Ehre gefolgt.

Auf der uns gegenüberliegenden Seite hatte der alte Echsenkrieger bereits seine Kampfstellung eingenommen -- jedoch war er von allen körperlichen Gebrechen geheilt, keine einzige Wunde bedeckte sein Schuppenkleid. Zusammen mit dem Turniermeister zogen Firnen, Thoran und ich uns zurück, während Rezzanjin in der Arena selbst zurückblieb. Auf Geheiß des Meisters wurde ein silberner Gong geschlagen, augenblicklich wurde es still in der Arena und der Kampf der Völker konnte beginnen.

Regungslos standen sich die beiden Kontrahenten zunächst an den Enden der Arena gegenüber, ein gespanntes Warten auf die erste Aktion des Gegners, ein Duell im Geiste, wer zuerst dem Druck nachgeben werde -- die Spannung war körperlich spürbar. Fast zeitgleich begannen die beiden, einander zu nähern, in weiten, langsam gezogenen Bahnen umkreisten sich Rezzanjin und der Echsenkrieger, als tanzten sie eine höfische Choreographie. Plötzlich sprang Rezzanjin vor, keine zehn Schritt mehr von seinem Gegner entfernt, und stürmte mit gestrecktem Schwert auf ihn zu. Dieser wartete regungslos, seine langstielige Axt zur Abwehr erhoben. Rezzanjin zog sein Schwert in einem diagonal geführten Streich gegen die Schulter seines Kontrahenten, doch jener blockte den Schlag mit dem flachen Axtbart ab und setzte zum Konter an. Mit dem stählernen Ende seiner Stabaxt stieß er nach Rezzanjins Kopf, nur um in letzten Augenblick abzubremsen und die Axtschneide rückwärtig auf Rezzanjins Beine zu führen. Der konnte seine Klinge gerade noch rechtzeitig herunterreißen und dem Schlag die Wucht nehmen, dennoch riss die Axt eine Fleischwunde in seinen Oberschenkel. Rezzanjin sprang zurück und attackierte seinen Widersacher erneut, diesmal mit einem Hieb gegen den Kopf. Doch der Echsenmann schlug das Schwert behände zur Seite, während er sich darunter hinweg duckte, täuschte abermals einen Streich zur Brust an und versenkte seine Waffe tief in unseres Freundes Seite. Blut schoss aus der Wunde und färbte den Sand des Kampfplatzes rot ein; gleichzeitig ging Rezzanjin auf die Knie nieder, das Gesicht schmerzverzerrt. Ohne die Worte des Turniermeisters abzuwarten sprangen wir hinunter zu Rezzanjin und versorgten die grässliche Verletzung mit Magie und Götterhilfe, derweil die Zuschauer ob des spektakulären Kampfes schrien und ihrem Streiter applaudierten. Dann erhob sich Rezzanjin wieder und abermals ertönte der Gong, die zweite Runde des Schieds anzukündigen. Dieses Mal blieb Rezzanjin wachsam, ohne sich durch den Punktverlust einschüchtern zu lassen. Ruhig und abwartend nahm er Verteidigungsstellung in der Mitte der Arena ein, die schlanke Klinge vor das Gesicht erhoben, jede Bewegung seines Gegners verfolgend. Durch seinen Sieg ermutigt wagte sich nun der Echsenmann als erster vor: Beidhändig schwang er seine Axt herunter, als wolle er den Schädel seines Kontrahenten spalten, zog sie jedoch vorbei und schlug auf die Brust Rezzanjins ein. Dieser freilich hatte ein solches Manöver erwartet und das Schwert nur halb zur Verteidigung geführt, so dass er flink zur Seite sprang und die Wucht des Axthiebes gebrauchend mit doppelter Härte gegen den Gegner schlug. Das Stadion erzitterte, als Stahl auf Stahl traf und die Axt der Echse zur Seite gedrängt wurde; sie versuchte noch, nach oben auf Rezzanjins Hüfte zu schlagen, doch im nächsten Augenblick trieb der die feindliche Waffe aus den Händen seines Besitzers. Die Stabaxt flog durch die Luft und landete scheppernd im Staub der Arena. Der Echsenkrieger wich ein paar Schritte zurück und klaubte seine Waffe vom Boden auf, während Rezzanjin ungerührt auf seine Rückkehr wartete. Nach diesem Zwischenfall wurde die Runde fortgesetzt: Mit einem wilden Vorhandschlag drosch Rezzanjin auf den Echsenmann ein, welchen dieser mit der Stabseite abwehrte und mit einem ebenso heftigen Hieb gegen die Schläfen erwiderte. Rezzanjin zog die gebogene Klinge in einem weiten Kreis durch die Luft, wobei er die nahende Axt parierte und sofort gegen die Beine seines Widersachers vorging. Der hatte Mühe, den tiefen Schlag abzufangen, und öffnete so einen Moment lang seine Deckung. Der Stahl biss in seine Seite und sprenkelte den Tuzaker Stahl mit schillerndem Echsenblut. Rezzanjin befreite seine gierige Klinge aus dem Körper der Echse und ging einige Schritte zurück -- diese Runde hatte er entschieden. Auf der Empore brachen wir in Zustimmung über diesen glänzenden Sieg aus, während die Echsen anerkennend murmelten. Rezzanjin jedoch blieb vollkommen ruhig und besonnen, so als existierte kein Stadion und kein johlendes Publikum, kein Kampf der Völker. Er befand sich im Zustand jener höchsten Konzentration, die alles andere vergessen macht, die unser Denken beschleunigt und den Körper mit analytischer Kälte erfüllt. Ich vermag es nicht anders auszudrücken, aber -- Rezzanjin lebte in diesem Moment nur noch für den Kampf. Als der Echsenkrieger versorgt war, erklang wieder der Gong, doch sein hallender Ton ging in den Zurufen der Zuschauer unter, die sich angesichts der entscheidenden letzen Runde von den Sitzen erhoben und das Stadion mit ungeheurem Lärm erfüllten. Weder Rezzanjin noch sein Gegner hielten sich nun mit langem Herantasten auf, mit gezückter Waffe stürmten sie aufeinander zu und offenbarten ihre wahre Kampfstärke in einem Schauspiel, wie man es zuvor noch nicht gesehen hatte: Axt und Schwert küssten sich mehrmals die Sekunde, so schnell, dass man kaum mit den Augen folgen konnte; es war, als führten die beiden eine einstudierte Choreographie der Kampfkunst aus. Jede Attacke und jeder Treffer des Echsenkriegers fanden eine entsprechende Antwort von Rezzanjin, jeder Schlag wurde gleichfalls erwidert. Es ist jener Tanz, den zwei vollkommen gleich starke Recken zu tanzen pflegen, wenn sie aufeinander treffen, ein Wirbel von Stahl und Holz, die ekstatisch miteinander ringen. Der Echsenkrieger schlug eine tiefe Wunde in Rezzanjins Schwertarm und erhielt sogleich einen Schnitt ins Bein, Rezzanjin zog einen Streich quer über die Brust seines Gegners und nahm dafür einen Schlag in die Seite hin. Beide waren blutüberströmt, doch hielten sie keinen Moment in ihrer Raserei inne und nahmen die Verletzungen hin. Niemand im gesamten Stadion hätte zu sagen vermocht, welcher der beiden den Sieg wohl davontragen würde. Schließlich holten beide Streiter zu einem vernichtenden Schlag aus und hieben ihre Waffen mit übermenschlicher Wucht gegeneinander. Es gab ein hässliches Klingen, als Echsenbeil und Schwert aufeinander krachten, die Axt aufgrund ihres höheren Gewichtes und der größeren Hebelwirkung Rezzanjins Klinge zurückdrängte und in hart auf der Brust traf. Die Härte des Aufpralls schleuderte ihn nach hinten, so dass er in den fahlen Staub der Arena fiel. Der Echsenkrieger hatte gewonnen.

Dennoch rührten sich keine Jubelschreie unter den Zuschauern, niemand sprang auf und spendete Beifall -- Stille lag über dem Stadion. Schwer atmend ließ der große Echsenkrieger seine Axt sinken. Sollte alles vorbei sein? ``Ihr habt uns einen großen Kampf geliefert'', hub endlich der Turniermeister an, ``Aber dieser Mensch hat euch einen harten Kampf geliefert -- einen Kampf auf Augenhöhe. Er ist würdig, dass N'Churr zu erhalten. Dann werden wir sehen, wer der größere Streiter ist!''

Und mit einem Mal erhob sich Rezzanjin aus dem blutigen Sand des Kampfplatzes, von allen Wunden ebenso wie sein Gegner geheilt, seine Waffe fest umschlossen. Der wahre Kampf hatte noch nicht einmal begonnen. Es fällt mir schwer, Worte zu finden, die den nun darauffolgenden Geschehnissen gerecht werden -- denn unsere Sprache hat die Worte vergessen, die dazu in der Lage wären, ebenso wie wir entsprechende Kämpfe vergessen haben. Im Rund der Arena kämpften keine Echsen oder Menschen, sondern Naturgewalten gegeneinander, so zäh und verbissen wurde die Schlacht geführt. Die beiden Kämpen lebten nicht nur den Kampf -- sie waren der Kampf. Jede ihrer Bewegungen verriet eine unirdische Präzision und Hingabe, so als wohne man einem Kampf zweier Halbgötter bei. Der Kampf schien mehrere Stunden zu dauern, vielleicht auch Tage; ich weiß es nicht. Fest steht nur, dass einer am Ende siegreich die Arena verließ: Rezzanjin. Kaum hatte er den schicksalhaften Streich geführt, der Stahl, Lederweste und Schuppen durchfuhr, fanden wir uns zu unser aller Erstaunen auf der Lichtung vor dem Echsentempel wieder, die Glieder von schmerzhaftem Ziehen erfüllt -- wir hatten unsere menschliche Gestalt wieder angenommen. Auf den Tempeltreppen lag der hünenhafte Echsenmann, vom Kampf schlimmer gezeichnet denn je. ``Ihr habt für euer Volk einen großen Sieg errungen, Mensch'', meldete sich der Kristallomant der Echsen zu Wort, nachdem er unseren tapferen Gefährten zu sich gerufen hatte. ``Auch habt ihr unserem Volk einen großen Dienst erwiesen, indem ihr das Zepter der Charypto an seinen angestammten Ort gebracht habt. Ihr habt gezeigt, dass ihr Ehre habt. Wahrlich ist niemand würdiger als ihr, der neue Wächter von Akra'baal zu werden. Ihr seid der Streiter der zwei Völker, Rezzanjin.''

Nach dieser bewegenden Ansprache war es für die Echsen Zeit, wieder ihren eigenen, verschlungenen Pfaden zu folgen, wussten sie doch ihre Stadt in sicheren Händen. Gleichwohl ich nicht behaupten möchte, Freundschaft mit diesen seleyah geschlossen zu haben -- der Abschied fiel mir schwer. Gerne hätte ich mehr über ihre Kultur und ihr Wissen erfahren, und ich werde in nächster Zeit gewiss in dieser Richtung forschen. Doch noch hielten uns dringende Fragen auf Maraskan: Wer hatte von Rallerau seine Befehle erteilt? Welche Rolle spielte Delian von Wiedbrück in dieser Sache? Nicht zu vergessen unsere Verpflichtung gegenüber der Boronkirche, die uns erst auf diese verfluchte Insel gebracht hatte. An Tuzak kamen wir also nicht vorbei, und so setzten wir unseren Plan zum Erreichen dieser Stadt in die Tat um. In den folgenden Tagen reisten wir also stets nach Süden, den letzten freien Häfen auf Maraskan entgegen. In der Nähe der großen Urwaldstraße stieg jedoch auch die Gefahr, patrouillierenden Gruppen mittelreichischer Soldaten zu begegnen, und es forderte all unser Geschick heraus, solche Zusammentreffen zu vermeiden. Oft genug bot uns der so gehasste Urwald Zuflucht und Schutz, insbesondere in der Nacht, wo wir zu ständiger Aufmerksamkeit genötigt waren. Nach drei Tagen Reise erreichten wir schließlich die Ostküste Maraskans -- schon von weitem konnten wir die Seeblockade vor Boran sehen. Dort schlugen wir den Weg zu einem der zahlreichen kleinen Fischerdorfe ein, die noch über einen ungehinderten Zugang zum Meer verfügten. Tatsächlich gelang es uns in jenem Dorf einige Fischer zu finden, die einwilligten, uns nach Sinoda zu bringen. Von dort würde man Wege nach Tuzak finden, bestenfalls direkt in den Borontempel hinein. Ich für meinen Fall bin gespannt, ob unserer Aktion Erfolg beschieden sein wird, zumindest sind die Fässer für unsere Überfahrt recht geräumig..

\subsection{Die Ankunft in Tuzak nach Temyr ibn Sahid}

\paragraph{Maraskanexpedition 32. Tag Anfang Rondra, der Leuin}
Eine gute Woche mag vergangen sein, seitdem ich die Fortführung dieses Berichtes in Anbetracht einer streng geheimen Seereise zu westlicheren Gefilden unterbrach -- und nun doch heil in Tuzak angekommen! Unter dem Gesichtspunkt einer ausgewachsenen Seeblockade ein durchaus ansehnlicher Erfolg, doch von diesem erreichte uns zu diesem Zeitpunkt in unserem meisterlichen Versteck -- enge modrige Fässer, zum Transport weitaus unempfindlicherer Güter gedacht -- keinerlei Gehör. So verließen wir also unser schwimmendes Gefährt ebenso, wie wir es in Sinoda betreten hatten, von Matrosen unsanft verschleppt und ungewiss, welches Schicksal uns ereilen werde. Nach einer Weile in den lebhaft lauten Straßen der fürstkomturischen Hauptstadt wurden unsere Fässer eine Art Stiege hinuntergelassen, um anschließend in einem muffigen, weitläufigen Raum aufgestellt zu werden. Die vernagelten Deckel unserer Gefängnisse wurden mit Brecheisen geöffnet und gaben den Blick auf ein finsteres, unterirdisches Gewölbe frei -- und auf zwei Gestalten im dunklen Gewand der Diener Borons. Diese bedeuteten uns schweigend, ihnen tiefer in das Gebäude zu folgen, während die Matrosen durch eine eherne Luke zum Tageslicht zurückkehrten. Die beiden Novizen führten uns -- immer noch schweigend -- durch eine Vielzahl fackelbeschienener schmuckloser Gänge, bis sich diese schließlich zu einer weiten Halle öffneten. An den Wänden hingen polierte mannsgroße Boronsräder aus glattem Dschungelholz, welche vom allgegenwärtigen Kerzenlicht nur schwach beleuchtet wurden. Die Luft war schwer vom Weihrauch, der in fein gearbeiteten Kassetten büschelweise verbrannt wurde und den Raum mit seinem beißenden Dunst erfüllte. In der Mitte des Raumes stand eine in schwarzen Samt gehüllte Frau, deren erhabener Habitus und Ornat sie als Donna des Tempels auswiesen. Haar und Augen waren von tiefstem Schwarz und schienen alles Licht in der Umgebung aufzunehmen; und obwohl ihre Züge Härte und Kälte ausstrahlten, sah sie uns doch keinesfalls unfreundlich an. Sie empfahl uns höflich aber bestimmt, die Spuren der Reise abzuwaschen und neue Gewänder anzulegen, bevor nähere Einzelheiten besprochen werden sollten. In dieser Weise wies sie die beiden Novizen an, die bislang respektvollen Abstand gewahrt hatten, und bereits eine Stunde später saßen wir zusammen mit der Donna um einen runden Tisch bei einem Glas schweren, schwarzen Weines zusammen.

``Euer Ruf eilt euch wahrlich voraus'', eröffnete sie uns mit leiser, weihevoller Stimme. ``In der ganzen Stadt wurden Steckbriefe auf eure Verhaftung verteilt und unter gutes Kopfgeld gesetzt. Auch streifen Wachen umher und beobachten jeden, der auch nur im Entferntesten mit euch in Kontakt stehen könnte -- und manchmal schleifen sie einen Unglücklichen in den Palast und er kommt nie mehr zurück. Ihr sollt euch mit Freischärlern eingelassen und daimonisches Gezücht bekämpft haben. Der Kladdj behauptet auch, ihr betriebet Umgang mit dem Echsenvolk\dots Ich weiß nicht, welche dieser Gerüchte wahr sind, aber ihr bereitet der ansässigen Herrschaft Unbehagen, die Situation hier in Tuzak steht kurz vor der Eskalation. Ich würde euch also bitten, mir zu erklären, was bei allen Göttern ihr in der letzten Zeit getan habt!''

So berichteten wir also in knappen Worten von unserer Reise zur Miene und dem dortigen Zusammentreffen mit dem Daimonen, von der Sicherung des Erzes und der unheilvollen Entdeckung der Soldatenleichen nahe der Küste, um schließlich die grauenhaften Ereignisse im Echsentempel zu repetieren: ``Rallerau hat seine Männer vergiftet und die Pforten des Grauens geöffnet'', führte Firnen am Ende unseres Berichtes aus, ``aber er handelte nur auf Befehl hin. Der eigentliche Drahtzieher muss aus den höchsten Kreisen der Tuzaker Militärregierung stammen, und somit ein Kommissar der mittelreichischen Besatzung sein. Alle unsere Beweise deuten auf Delian von Wiedbrück hin, doch das ergibt keinen Sinn: Zuerst bekämpft er im Auftrag der KGIA Borbaradianer, bittet in einem Schreiben um unser Kommen und stellt die Expedition von Puspereikens unter seinen persönlichen Schirm und Schutz, nur um anschließend seine Männer das Lager verwüsten zu lassen, uns strafrechtlich verfolgt und sich mit Paktierern einlässt? Das ist nicht der Delian von Wiedbrück, den ich kenne.''

Die Donna, welche während unserer Ausführungen lediglich nachdenklich geschwiegen hatte, meldete sich wieder zu Wort: ``Diese Widersprüche folgen in der Tat einer Reihe von Ungereimtheiten, die in der letzten Zeit aufgetreten sind. Seht, nach der faktischen Entmachtung Fürst Herdins im Zuge der Besatzung Maraskans wurde Delian von Wiedbrück an die Spitze der neuen Regierung gesetzt -- zur Zufriedenheit der meisten. Denn er übte seine Macht zum Wohle des Volkes aus, stets gerecht und umsichtig. Doch dann änderte sich alles: Es schlug ein neuer, despotischer Ton an, Unschuldige wurden verhaftet, Dörfer ausgelöscht. Maraskan wurde ein weiteres Mal von Gewalt verschlungen. Jene, die ihren Unmut laut zu äußern wagten, wurden als Erste entfernt, bis die Übergriffe immer willkürlicher, chaotischer wurden. Erst jüngst sind die Spektabilitäten der ansässigen Magierakademie verschwunden, die restlichen Magi stehen unter strenger Beobachtung. Tuzak wurde abgeriegelt -- gegen Fremdeinwirkung, hieß es. Heute wissen wir, was wirklich damit gemeint war. Auch wir kennen einen anderen Delian von Wiedbrück.''

``Also stellt sich doch die Frage'', übernahm meine Wenigkeit, ``inwieweit ist er selbst für diesen Sinneswandel verantwortlich? Ich wäre nicht verwundert, stünde er unter bösem Einfluss borbaradianischer Elemente, nach allem, was wir bereits erlebt haben. Aber gleich wie die Antwort lauten sollte, denke ich doch, dass die Ereignisse eine nähere Untersuchung rechtfertigen.''

``Wenn ihr mit euren Vermutungen Recht habt, so müsst ihr umgehend Nachricht an den Reichsbehüter und die Kirchen sowie die Magiervereinigungen senden.'', antwortete die Donna mit besorgter Miene, ``Einer solchen Bedrohung muss rasch beigekommen werden, sonst vermag sie leicht zu wachsen. Nun'', sprach sie, während sie sich von ihrem Sitz erhob, ``ich danke euch für eure Schilderung, und auch für euren Einsatz. Aber eure Anwesenheit bedeutet Gefahr für den Tempel und für die Bewohner Tuzaks. Sobald ihr also eure Nachforschungen abgeschlossen habt, werdet ihr diese Insel verlassen. Ihr sollt die nächste sichere Überfahrt erlangen, die wir arrangieren können.''

Mit diesen Worten wandte sie sich zum Gehen, blieb jedoch an der Tür ein weiteres Mal zurück: ``Eins noch: Wenn ihr die Mauern dieses Tempels verlasst, so seid ihr auf euch allein gestellt. Tragt die Roben unserer Novizen, während ihr in den Straßen wandert, und geht immer nur paarweise aus. Fragt nach ehemaligen Palastdienern, wenn ihr Informationen über Aufbau und Funktion der Machtzentrale erhalten wollt. Möge der Segen des Herrn Boron euch in euren Träumen beschützen.'' Dies gesagt, verließ sie die Kammer.

Nachdem die Donna die Tür hinter sich geschlossen hatte, ergingen wir uns in Plänen über unsere weiteren Investigationen in den nächsten Tagen: Rezzanjin wollte sich nach seinem Onkel, einem berühmten Schwertmeister mit angeblichen Kontakten zum Widerstand, erkundigen und zu diesem Zwecke dem Tempel der Geschwister Rur und Gror einen Besuch abstatten. Firnen und ich zogen es vor, in den Spelunken Tuzaks nach ausgeschiedenem Personal der Alabasternen Festung zu recherchieren, während sich Arngrimm erbot, unsere Ausrüstung zu bewachen.

\paragraph{33. Tag}
Doch dieser Teil des Plans schlug fehl -- selbst unter Verwendung unseres beträchtlichen Scharfsinnes gelang es uns nicht, einen suspendierten Diener ausfindig zu machen. Rezzanjin seinerseits hatte größeren Erfolg: Zwar schien sein Onkel das Eiland Marustan bereits vor geraumer Zeit verlassen zu haben, doch die Vorstehenden des Tempels versprachen, Untergrundorganisationen auf ganz Marustan zu informieren und möglicherweise zu mobilisieren. Zudem versprach er uns, das Haus seines Onkels in der Stadt nach hilfreicher Ausrüstung zu durchsuchen, sollte sich dazu die Gelegenheit bieten. An diesem Abend begaben wir uns alle zeitig in Borons Arme, um den nächsten Tag gebührend nutzen zu können.

\paragraph{34. Tag}
Am nächsten Morgen wurden wir allerdings noch vor dem Frühstück einer nächtlichen Aktivität ansichtig, welche unsere Pläne schlagartig um eine neue Komponente erweiterte: Jemand hatte einen beschriebenen Zettel auf dem Boden hinterlassen, auf welchem die ``Freunde Alwidjas'' uns nahelegten, in der ``Fidelen Zedrakke'' nach dem ``Habiharan'' zu Fragen. Ragnos und Arngrim nahmen sich dieser Spur an, während der Rest unserer Truppe im Tempel verblieb. Spät am Nachmittag kehrten die beiden mit weiteren Informationen in die Sakristei zurück: Ein Kontaktmann, zu welchem Ragnos und Arngrimm durch Nennung des Kennwortes Einlass ersucht hatten, hatte sie an zwei weitere Etablissements verwiesen, welche seiner Aussage nach die Aufenthaltssorte zweier ehemaliger Kammerdiener seien. Einer, eine junge Frau, habe eine neue Anstellung als Schankmaid im Teehaus ``Zur Jerganlilie'' gefunden, während der andere, ein älterer Mann, sich in der Gaststube ``Zur goldenen Gans'' beworben habe. Firnen und Rezzanjin beschlossen, dem Teehaus einen Besuch abzustatten, Ragnos und Arngrim verfolgten derweil die zweite Fährte.

Diesen Abend kamen wir zu einem außerordentlichen Treffen zusammen: Firnen und Rezzanjin hatten offenbar die Schankmaid ausfindig gemacht, wohingegen unser Waidmann und Ritter mit Schnitten und Prellungen aufwarteten. Auf ihrer Suche nach dem Kammerdiener waren sie von Wachen erkannt und in einen Kampf verwickelt worden, denen sie nur unter größten Anstrengungen entkommen konnten. Doch die Sichtung der Zofe erwies sich als Glücksgriff: Sie hatte einen Plan der Festungsanlagen zeichnen können, der uns bei unserer strategischen Arbeit sicherlich äußerst nützlich sein wird. Im Gegenzug verlangte sie allerdings eine Fluchtmöglichkeit vom Eiland, einen Gedanken, den ich nur zu gut teile.

``Ihr könnt auf demselben Schiff reisen, sobald wir eines gefunden haben.'', meinte die Donna nachdenklich, ``Doch zuvor sollten wir uns Gedanken über das weitere Vorgehen machen. Wenn ihr Truppen haben wollt, so müsst ihr sie erst einmal mobilisieren. Überlegt euch gut, an wen ihr welche Worte richten wollt, um Hilfe zu ersuchen. Ruft mich, wenn ihr die Nachricht geschrieben habt.''

Damit verließ sie, in Begleitung der jungen Frau, die Kammer. Nach langen Überlegungen kamen wir schließlich zu dem Schluss, Sendung an Dexter Nemrod, Amando Laconda da Vanya, Ayla von Schattengrund sowie, gegen Firnens Unmut, Saldor Foslarin aufzugeben. Im knappen Stil einer Eildepesche redigierten wir so folgende Zeilen:

``Borbaradianer unterwandern Maraskan;\\
Rallerau ist Paktierer, Position von Wiedbrücks unbekannt;\\
Dämon in Enduriummine, Pforten des Grauens geöffnet;\\
Brauchen größtmögliche Unterstützung in Tuzak;\\
Weiterleiten an Nemrod, da Vanya, von Schattengrund, Foslarin.''

Jetzt bleibt uns nur noch zu warten; zu warten und zu hoffen\dots

\subsection{Maraskan-Bericht des Inquisitionsrats Amando Laconda da Vanya zu Ragath an Großinquisitor Rapherian von Eslamshagen}

Wir erhalten die Nachricht, dass die Diskusstafette verboten worden ist, da anscheinend Rebellen die Stafette als Deckung benutzen wollen, um in den Palast einzudringen. Daraufhin kommt es zu Aufständen in Tuzak am 19. Rondra, die aber von der Stadtgarde niedergeschlagen werden. Desweiteren wird eine totale Nachrichtensperre von von Wiedbrück über Maraskan verhängt. Die Gezeichneten beschwören einen Feuerdschinn und bitten die Magierakademie von Tuzak um Hilfe, die Magier lehnen auf Grund eigener Probleme jegliche Hilfestellung ab. Die Rur \& Gror-Kirche erzählt den Gezeichneten von gefangenen Rebellen im Kerker des Palastes.

\paragraph{27. Rondra}
Am 27. Rondra erreicht die Seeadler von Beilunk endlich Tuzak . Unser Trupp besteht aus 64 Sonnenlegionären, 12 Bannstrahlern, 10 KGIA-Agenten, Hochgeweihten Rondrasil Löwenbrand von Arivor, Hochmeister Ucurian Jago und mir, Inquisitionsrat da vanya zu Ragath. Im Hafen von Tuzak werden wir von alten Bekannten, den Reichsverrätern, in Empfang genohmen. Wir werden zum Borontempel geführt, in dem uns die Vermutungen der Gezeichneten dargelegt werden. Die KGIA-Agenten versuchen unterdessen den Tuzaker Fürstenpalast zu infiltrieren. Am Abend werden zwei Köpfe der Agenten von einem offensichtlich besessenen Mann zurückgebracht. Wir sprechen die vollständige Begnadigung der Reichsverräter aus.

\paragraph{28.Rondra}
Am nächsten Morgen wird ein Gottesdienst im Namen des Herren Praios, der Herrin Rondra, des Herren Boron und der Herrin Peraine abgehalten. Rondrasil spricht das Gebet ``Segnung der Schlacht'', danach ziehen wir zum Fürstenpalast. Der Weg dorthin ist von Schaulustigen gesäumt und die beiden ersten Tore des Palasts werden ohne Vorkommnisse passiert. Der Trupp nimmt Aufstellung auf dem Platz vor dem Palast. Auf dem Dach steht der Obrist der Drachengarde und führt ein Gespräch mit Arngrimm von Ehrenstein, welches mit dem Austausch von Bolzen endet. Ein Trupp aus 8 Sonnenlegionären, 6 Bannstrahlern, den Gezeichneten, Jago, Löwenbrand und mir dringt Richtung Kerker vor. Auf dem Weg trifft mein Trupp auf Liliengardisten. Nachdem 10 besiegt worden sind, dringen die Gezeichneten und ich weiter vor, während der Rest die Gardisten beschäftigt. Vor dem Kerker besiegen wir mit Praios Hilfe ein Pandämonium und dämonisch pervertierte Hunde. Im Kerker finden wir den seelenlosen Gefährten der gezeichneten und knapp zehn wahnsinnige Rebellen, die die restlichen Hunde zerfleischen. Als wir den Kerker wieder verlassen, stelle ich fest, dass die Hälfte unserer Truppen tot ist, dafür aber der Palast zum großen Teil gesäubert ist, es wird nur noch um den Fürstentrakt und ein paar letzte Räume in den anderen Trakten gefochten. Wir unterstützen Löwenbrand gegen götterlose Paktierer, bevor wir uns in die Eingangshalle des Fürstentrakts begeben, wo Sonnenlegionäre gegen Soldaten, Hunde, Heshtotim, einen großen schwarzen Dämon und einen Schwarzmagier. Der Magier Firnen bannt durch einen Pentagramma mehrere Dämonen. Arngrimm von Ehrenstein erschlägt den Schwarzmagier. Die restlichen Dämonen werden von Praios gleißendem Licht niedergeworfen. Im Audienzsaal treffen wir auf den wahnsinnigen Fürst Herdin und auf Delian von Wiedbrück, der sich als Borbarad, der sechs Finger hat, entpuppt. Borbarad bietet von Ehrenstein an das Reich in Frieden zulassen, wenn er Maraskan dafür bekommt. Von Ehrenstein geht nicht auf sein Angebot ein, sondern erklärt, man habe Borbarad einmal besiegt, man werde Borbarad wieder besiegen. Daraufhin verlässt Borbarad uns auf einem Wagen, der von sieben Schlangendämonen gezogen wird. Durch Borbaradsstab entsteht eine Pforte des Grauens im Audienzsaal. Durch ein Wunder der gütigen Mutter Peraine heilen unsere Wunden. Die Gezeichneten versuchen den Stab zu erreichen. Auf dem Weg dorthin opfert sich Baron Arngrimm von Ehrenstein für den Maraskaner Rezzanjin al'Ahjan. Der tulamidische Magier Temyr ibn Sahid schafft es schlussendlich den Stab zu kontrollieren und bannt alle Dämonen.

Hochmeister Jago und ich überantworten den Stab Borbarads den Gezeichneten. Der Baron Arngrimm von Ehrenstein soll ein Ehrengrab in der Familiengruft der von Ehrenstein auf Burg Ehrenstein erhalten. Wir fragen sie, wer ihrer Meinung nach Delian von Wiedbrück auf Maraskan ersetzen soll. Wir kommen zu dem Schluss, dass dies nur Reichserzmarschall Graf Helme Haffax von Wehrheim kann.

\paragraph{29.Rondra}
Die Gezeichneten brechen mit dem Schiff Perlbeißer Richtung Khunchom auf, um von dort weiter zum Gebrochenen Rad von Punin zu reisen.

\subsection{Der Sturm des Fürstenpalasts nach Temyr ibn Sahid}

\paragraph{Maraskanexpedition 40. Tag Mitte Rondra, der Leuin}
Die Nachricht wurde erhalten! Nach fast einer geschlagenen Woche der Anspannung und Angst ist diesen Morgen endlich die lang ersehnte Meldung eingetroffen: Die Angehörigen der Geweihtenzirkel, die Mitglieder der Magierakademien und die Repräsentanten der Reichsordnung sind über die Zustände auf Marustan in Kenntnis gesetzt; nun müssen Taten und Truppen folgen. Die Nachricht wurde erhalten! Niemals vermochten vier Worte mir eine solche Linderung und Hoffnung zu geben wie diese. Doch wir dürfen uns nicht von Gefühlen leiten lassen -- noch nicht. Die Männer Fürst Herdins streifen weiterhin durch die Straßen Tuzaks und durchsuchen die Tempel nach uns. Magierkollegen stehen unter strengster Bewachung oder werden getötet, während ich diese Zeilen schreibe. Nein, noch ist nicht die Zeit des Aufatmens, da Finsternis über Marustan liegt. Aber es ist ein kleiner Erfolg, der in den Herzen der Menschen ein Feuer der Hoffnung entfacht.

\paragraph{43. Tag -- 19. Rondra}
In der Zeit, die wir bislang in Tuzak verbrachten, war die Verzweiflung und Angst der Menschen deutlich spürbar -- sie leerten die Straßen und Gassen der Stadt, bedrückten die Gemüter und legten sich einem Schatten gleich auf die Herzen. Doch nun wurden sie zu Wut und Zorn entflammt und gegen ihre Peiniger gerichtet. Heute erhob sich ganz Tuzak aus seiner Starre und füllte die Plätze mit pulsierendem Leben und brennendem Hass. In diesem Zorn geeint begehrten die Menschen gegen die Machthaber in der Alabasternen Festung auf -- und wurden blutig niedergeschlagen. Veränderung liegt in der Luft, und nur die Zwölfe wissen, was diese Veränderung bringen wird. Abermals wurden heute die Tempel auf der Suche nach zwei Magiern, zwei Kriegern, einem Geweihten der Perain und einem Jägersmann durchsucht -- doch diese waren nicht dort. Schon als die Aufstände begannen, nahmen wir unsere Ausrüstung zusammen und flohen zum Hause von Rezzanjins Onkel. Den Tag über schlossen wir uns in den Kellergewölben ein und trafen erste Vorbereitungen für die Erstürmung der Festung: So erneuerten wir die arkanen Ladungen der Artefakte; auch einen Dschinn des Feuers beschworen wir zu diesem Zwecke und bannten ihn in meinen Ring. Spät in der Nacht kehrten wir zum Tempel zurück. Die Donna erwartete uns bereits in unseren Gemächern. ``Ich habe Schmuggler ausfindig machen können, die euch von dieser Insel bringen werden -- so ihr dies wünscht.'' Verständlicherweise wünschten wir nicht\dots

\paragraph{45. Tag -- 21. Rondra}

Ragnos ist verschwunden! Diesen Morgen zum Hafen aufgebrochen, ist er von seinen Untersuchungen nicht zurückgekehrt -- ich fürchte, die haben ihn. Mein Herz krampft bei dem Gedanken, ihn in den Kerkern der Alabasternen Festung zu wissen, in den Händen von Borbardianern. Doch alleine können wir ihn nicht befreien, wir sind zum Warten und Nichtstun verdammt, das ist das Schlimmste. Zu mutlos zum Weiterschreiben.

\paragraph{52. Tag -- 28. Rondra}
In der gestrigen Nacht hatte ich einen bemerkenswerten Traum: Ich übersah eine leere Fläche, welche sich unendlich in alle Richtungen erstreckte und von blendender Finsternis erfüllt war. Über mir am Firmament dräute ein Madamal, dem, einer menschlichen Hand gleich, sechs lange, feingliedrige Finger entwuchsen. Die Finger waren in rastloser Bewegung, ballten sich zur Faust, entspannten sich, nur um wieder in wilde Zuckungen zu fallen. Doch all das konnte mich nicht schrecken, denn am Horizont sah ich ein Licht aufflammen, zuerst unendlich klein, ein Funke nur, aber rasch wachsend und näher kommend. Dabei wurde ich eines hellen Schemens ersichtig, der die Quelle dieses überirdischen Leuchtens zu sein schien: Mit sinkender Entfernung konnte ich vermehrt Einzelheiten des Wesens ausmachen, Arme und Beine, die aus einem mächtigen Körper sprossen, ein Hals, der sich in die Länge zog und dann zu einem Kopf mit Raubtierschnabel verdichtete. Goldbraune Federn zogen sich über den Leib des Wesens und mächtige Schwingen schossen aus den Flanken, die wie die Augen selbst von purem Gold und Licht erfüllt waren: Und ich erkannte die Gestalt und wusste, dass eine Entscheidung bevorstand. Denn es war ein gigantischer Greif, welcher der schrecklichen Hand gegenübertrat im entscheidenden Kampf.

Am Morgen wollte ich meinen Gefährten gleich von dieser prophetischen Erscheinung berichten, doch ihnen war dieselbe in eben jener Nacht widerfahren. Wie von einer frohen Ahnung getrieben, brachen Firnen und ich zum Hafen auf, der ob der frühen Stunde und der Seebarrikade in sanfter Stille da lag. Das Perlenmeer, für seinen rauen Wellengang gefürchtet, glich diesmal einer glatten spiegelnden Fläche, nur gelegentlich von sanftem Wind gekräuselt. Gedankenversunken sahen wir in die Ferne und genossen den Anblick der sich erhebenden Praiosscheibe, die den Himmel mit Feuerschein überzog. Da tauchte am fernen Horizont ein Segel auf, wie ein Funken, der dem Feuer entspringt, doch deutlich sichtbar. Rasch kam das Schiff im nun stärkeren Wind näher und durchpflügte die glitzernde blaue Fläche unter sich. Weitere Masten wurden sichtbar, Streichhölzer vor einer goldenen Leinwand, erst einer, dann zwei; insgesamt drei Masten, behangen mit weißen Segeln: Eine Fregatte schickte sich an, in den Hafen einzulaufen. Firnen und ich liefen den größten Pier entlang, während die Umrisse des Schiffes immer größer wurden, und endlich erkannten wir, dass auf die größte der Leinenwände ein Tier genäht war: Der güldene Greif der Seeadler von Beilunk. Majestätisch flog sie über das Wasser in einer Krone aus Schaum und Gischt, und der Greif wand sich im Wind, als sei er tatsächlich lebendig geworden. Firnen reckte die Faust in die Luft und ein roter Lichtblitz schoss hinauf in den Himmel. Die Mannschaft der Seeadler hatte offensichtlich verstanden, denn sie drehten bei und liefen nun unseren Pier an, welcher bereits von zahlreichen Schaulustigen bevölkert war. Die Offiziere der Fregatte riefen Befehle durch die Luft, Matrosen kletterten in die Wanten und refften die Segel, und gerade, als die Sonne vollends aus dem Meer gestiegen war, lief die Seeadler in den Hafen Tuzaks ein.

Als die Fregatte endlich am Steg befestigt war, fuhren Matrosen Planken aus und zwei mir wohlbekannte Männer, gefolgt von einem dritten, verließen das Deck: Der Großinquisitor der Praioskirche Amando Laconda da Vanya sowie der Hochmeister der Bannstrahler Ucurian Jago. Der Inquisitor schien bei meinem Anblick nur milde überrascht und meinte lächelnd: ``Wo auch immer sich große Ereignisse zutragen, pflegt eure Gruppe zu sein, scheint mir. Mit dem ehrenwerten Ucurian Jago seid ihr ja vertraut, doch vielleicht kennt ihr den dritten in unserem Bunde noch nicht?'' Der solcherart Angesprochene trat nach vorne. Er war groß kräftig und stützte sich auf einen edlen Biedenhänder, welcher ihn mitsamt seinem roten und weißen Ornat als Geweihten der Rondra auswies. ``Dies ist Rondrasil Löwenbrand, Ritter und Hochgeweihter der Leuin und ein ebenso furchtloser wie fähiger Streiter gegen die dunklen Mächte.'' Rondrasil zeigte sich ob dieses Lobes bescheiden und meinte nur, dass es ihm eine Ehre sei, an der Seite zweier so rechtschaffener Männer für seine Göttin zu kämpfen. Ucurian Jago war über den Umstand meiner Anwesenheit weitaus weniger erfreut und stellte meinen Gefährten und mir einen schmerzhaften Tod in Aussicht, sollten unsere Schilderungen über die vorläufige Situation nicht zutreffen: ``Stellt sich heraus, dass wir eine Kohorte Sonnenlegionäre und mehrere Bannstrahler aufgrund unhaltbarer Spekulationen mobilisiert haben, so werdet ihr brennen, Magier! Ihr und der Rest eurer Gemeinschaft.'' ``Wir werden sehen, Hochmeister.'', erwiderte da Vanya trocken. ``Vorerst sollten wir aber einen genauen Bericht dieser Herren abwarten und Unterkünfte für unsere Männer finden. Kennt ihr einen solchen Ort, Magister Temyr? Ich möchte mich ungern über Angelegenheiten der Sicherung der göttergefälligen Ordnung hier im Hafen unterhalten.'' So führten also Firnen und ich die drei Hochgeweihten, 64 Männer und Frauen der Sonnenlegion, 12 Bannstrahler sowie 10 verhüllte Agenten der KGIA durch die engen Straßen Tuzaks zum Borontempel.

An diesem Abend saßen denn also die Vertreter der Kirchen des Praios, der Rondra und des Boron mit einer Gruppe von Abenteurern in der Sakristei des Tuzaker Borontempels zusammen, um den Sturm der Alabasternen Festung zu planen. Einstweilig zeigte sich selbst Ucurian Jago von unseren Argumenten und Beweisen überzeugt, gleichwohl er in seiner üblichen Skepsis verharrte. Die in der Schlachtenführung erfahrenen Rondrasil und Arngrim arbeiteten auf der Grundlage des Grundrisses des Fürstenpalastes eine Strategie zur Eroberung desselben aus. ``Während wir mit dem Großteil unserer Streitkräfte den direkten Weg über das Hochplateau der weißen Residenz nehmen, werden die Agenten der KGIA versuchen, einen alternativen Weg ins Innere der Anlage zu finden.'', führte Rondrasil seine Überlegungen aus. ``In Anbetracht der Autorität des ehrenwerten Inquisitors und des Hochmeisters der Bannstrahler werden sich die einfachen Soldaten der Öffnung des ersten Tores nicht erwehren. Sollten wirklich korrumpierende Kräfte am Werk sein, so werden wir mit dem Zweiten größere Probleme haben, fürchte ich. Aber mit dem Beistand der Götter werden wir auch dieses Hindernis überwinden. Innerhalb des Palastes erscheint es günstig, die Attacken auf den Fürstentrakt zu fokussieren, welcher sicherlich auch am stärksten bewacht ist. Derweil kann eine kleinere Gruppe, '', dabei wies er auf uns, ``in die Kerkeranlagen vordringen und die dortigen Gefangenen befreien. Wir können jeden erdenklichen Kämpfer auf unseren Seiten gut gebrauchen, und festgesetzte Feinde der Tuzaker Machthaber werden gewiss für diese Zeit zu unseren Freunden zählen.'' Dies gesprochen, nahm er wieder an der Stirnseite der Tafel Platz. Der Großinquisitor erhob sich. ``Ich danke euch für eure Ausführungen, Hochgeweihter. Wir werden so vorgehen, wie ihr es vorgeschlagen habt. Wir alle hier werden an der Spitze unserer Truppen ins Gefecht ziehen als Zeichen unseres Mutes und unserer Rechtschaffenheit. Wenn es euch Recht ist, so werde ich mich an unsere Abenteurer hier wenden, sobald wir das Innere der Festung erreicht haben. Ich eigne mich weniger für den frontalen Kampf, als ihr oder Ucurian es tun. Lasst uns nun zusammen speisen und dann die Nacht über Kräfte für unsere Queste sammeln. Morgen früh werden wir in einem Götterdienst für den Beistand der Zwölfe und für das Gelingen unserer Unternehmung bitten.'' Nach diesen Worten wurde ein Nachtmahl aus Suppe, Brot und gebratenem Fleisch von Novizen aufgetischt. Während des Essens klopfte es an der Pforte und einer der wachhabenden Offiziere trat ein. ``Verzeiht die Störung, edle Herren, doch die Männer berichten von frevlerischen Machenschaften von der weißen Festung her. Wir haben dort\dots grausam zugerichtete Köpfe gefunden, die auf dämonische Rituale hinweisen. Eine gute Nachtruhe.'' ``Nun,'', meinte Ucurian Jago und trank einen Schluck Wein, ``jetzt werden wir euch wenigstens nicht hinrichten lassen -- in dieser Hinsicht habt ihr wahr gesprochen.''


\paragraph{53. Tag -- 29. Rondra}
Am folgenden Morgen fanden sich alle zum Götterdienst zusammen. Die Hallen des Borontempels waren mit Standarten des goldenen Greifen und der feuerroten Leuin ausgeschmückt worden und wurden vom einfallenden Licht der Sonne in strahlenden Glanz getaucht. Der Inquisitor da Vanya, der Hochmeister Jago und der Hochgeweihte Löwenbrand standen in glänzenden gewirkten Roben über ihrem Rüstzeug vor dem zentralen Altar und vollzogen die traditionellen Riten. In Lobliedern, Chorälen und Gebeten beschworen sie die Tapferkeit und den Mut aller Anwesenden im Angesicht der gesichtslosen Gefahr. Nun, ich bin beileibe kein Mann der Götter, doch die ernste und feierliche Atmosphäre berührte mich tatsächlich tief in meinem Innersten, und wie eine einzige Stimme erfüllte unser vielstimmiger Gesang den Raum. Als das Praiosunser gesprochen war, bat uns Rondrasil, nach vorne zu treten. ``Kniet nieder und empfangt den Segen Rondras, der Blitzeschleuderin und feurigen Leuin Alverans. Ihre Hitze möge euer Blut durchströmen, ihre Kraft eure Arme führen, und ihr Mut eure Herzen erfüllen. Ich segne diese Waffen und ihre Streiter, auf dass sie ein Leuchtfeuer der Hoffnung in diesen finsteren Tagen sein mögen und die Feinde der göttlichen Ordnung in Furcht erzittern lassen, denn rechtschaffen ist ihr Werk. Amen.'' Bei diesen Worten breitete er die Hände über uns und unseren blank gezogenen Waffen aus, und ein heißer Wind schien von ihm auszugehen, der jeden in der Halle schaudern ließ. Nachdem wir uns erhoben hatten, bat auch Toran um Aufmerksamkeit: ``Gebt mir eure Hände.'' Wir legten unsere Hände in die seinen und Toran sprach: `` Mutter Perain, halte deine heilende Hand schützend über diesen Männern, dass sie von deiner Weitsicht und Güte erfüllt seien. Lasse das Licht deiner Weisheit über ihnen erstrahlen und gib ihnen den Willen, dem Bösen zu trotzen und das lebendig Gute zu schützen. Möge ihr Geist die Saat deiner großartigen Schöpfung empfangen.'' Kaum hatte Toran geendet, so spürte ich eine Klarheit in meinen Gedanken, wie ich sie schon lange nicht mehr gespürt hatte. Endlich war mein Geist von allen Sorgen und Ängsten, allem Ballast und störenden Einflüssen befreit, die ansonsten meine Hand zittrig oder mein Handeln zögernd gemacht hätten. Gestärkt brachen wir zur Festung auf, gefolgt von 70 tapferen Männern und Frauen.

Die Stadt lag friedlich im Morgennebel vor uns, während wir durch das Gewirr der menschenleeren Gassen zur Alabasternen Festung des Herdin von Tuzak schritten. Sobald wir die ersten Ausläufer des äußersten Mauergürtels erreicht hatten, wurden die gusseisernen Tore von einem verborgenen Mechanismus geöffnet und die Wachmannschaften ergaben sich geschlossen in unsere Hände. So marschierte unsere Kohorte verschworener Krieger und Geistlicher ohne Gegenwehr auf dem Plateau zu Füßen des inneren Festungsbezirks in Formation auf. Der Großinquisitor trat vor und rief zu den besetzten Zinnen hinauf: ``Im Namen des zölfgöttlichen Zirkels und der Autorität des Reiches und seines Behüters Brin von Gareth! Ihr werdet beschuldigt des Vergehens wider weltliches Recht und des infamen Frevels an der göttergefälligen Ordnung, der ihr untersteht. Weiterhin wurden Beweise für diese gefunden und in den Augen dieses Inquisitionsrates für angemessen erfunden. Öffnet die Tore und legt die Klingen nieder, auf dass wir die Schuldigen von den Unbescholtenen trennen mögen, oder wir werden im Kampfe all die töten, die sich uns verweigern! Die Autorität des Diesseits und des Jenseits schaut auf euch herab!''

Da schallte ein ungeheures Geschrei von den Mauern hinab und auf die Brüstung trat ein Mann in schimmernder schwarzer Robe, die ringsum mit roten Schriftzeichen besetzt war. Höhnisch beugte er sich hinab: ``Gleich, auf welche eurer nichtswürdigen Autoritäten ihr euch berufen wollt, wir wählen den Kampf! Überschreitet nur diese Pforte und es wird euch schlecht ergehen, denn eure Götter werden euch nicht schützen!'' ``Sprecht nur ein Wort, Herr,'' , meldete sich einer der Offiziere zu Wort, ``und unsere Bogenschützen holen ihn von dort herunter. Seine lästerliche Rede soll er büßen.'' Der Inquisitor hatte während den Beleidigungen keinerlei Regung zeigen lassen, doch das Funkeln in seinen Augen verriet seine Wut. ``Gebt Befehl zum Angriff. Lichtet ihre Reihen mit Pfeilen, dann brechen wir das Tor auf.'' Die Sonnenlegionäre spannten ihre Bögen und legten Pfeile an, zielten und füllten den Himmel mit tödlichen Geschossen. Diesmal waren es Schmerzens- und Todesschreie, die von den Drachengardisten erklangen -- doch vom Schwarzmagier prallten alle Pfeile wirkungslos ab, und er lachte abermals laut und schallend. Dann drehte er sich um die Achse und verschwand im Wirbeln seines Umhanges. ``Dieser feige Rückzug wird ihn auch nicht retten!'', rief Ucurian Jago, ``Lasst uns die Tore stürmen.'' Daraufhin stellten sich der Inquisitor, Rondrasil und er selbst Schulter an Schulter zusammen und erhoben die Hände gen Himmel. Vom Firmament herab schossen goldene Strahlen und Blitze und zerschmetterten das schwere Portal samt der angrenzenden Mauer. ``Es hat begonnen!'', riefen sie und liefen in den Palast hinein, gefolgt von uns und den Truppen, den Schlachtruf ``Für die Zwölfe'' rufend. Hinter dem Torhaus lag eine große, mit Marmorfliesen ausgelegte Halle, von der zahlreiche Gänge und eine zentrale Treppe tiefer in die Anlage hinein führten. Kaum hatten wir den Raum betreten, da flogen die Türen am Ende des Aufstiegs auf und Drachengardisten drängten heraus. Ich legte rasch ein arkanes Fesselfeld auf das Portal, so dass die feindlichen Soldaten in ihrer Bewegung gehindert wurden und gegeneinander prallten. Sofort fielen unsere Männer mit blank gezogenem Schwert über sie her und tränkte den Boden mit ihrem Blut. Doch immer mehr Feinde liefen gegen die Tür an und schoben die dort gehaltenen trotz allem heraus, und bald hatte sich die Zahl der Kämpfenden verdreifacht. ``Wir sollten zum Kerker vorstoßen, sonst werden wir in der Halle zerrieben!'' , rief der Großinquisitor uns zu, während er mit seinem Zepter den Helm eines Drachengardisten zerschmetterte. ``Es geht durch diesen Gang dort!'' Somit beendeten wir unsere Scharmützel und rannten mit da Vanya in den angrenzenden Trakt hinein. Doch auf einmal brachen aus dem Boden Tentakel hervor, geifernde Mäuler schnappten nach unseren Füßen oder verspritzten ihr ätzendes Sekret. Beißender Schwefelgestank erfüllte die Luft; niederhöllische Kälte ergriff unsere Glieder. ``Ich kann diesen Schrecken exorzieren, doch währenddessen müsst ihr mich beschützen, das Ritual dauert seine Zeit.'' Mit diesen Worten ließ sich da Vanya auf die Knie fallen und begann, den Ritus zu intonieren. Kaum hatte er begonnen, da hörten wir laute Rufe und das Geräusch vieler Schritte, die sich in unsere Richtung bewegten: Am Ende des Ganges erschienen mehrere Drachengardisten sowie Hunde, die mit unnatürlicher Geschwindigkeit daher preschten und ein Spur von Geifer und schwarzem Dunst hinterließen. Ihre Augen waren feuerrot, ihr Fell pechschwarz, und die Laute, die sie ausstießen, waren nicht von dieser Welt. Arngrim und Rezzanjin sprangen sofort mit dem Schwert in der Hand nach vorn, um ihnen den Weg zu versperren, wohingegen Toran einen Schutzkreis um uns zog und Firnen sein astrales Schutzschild beschwor. Mit unvermittelter Geschwindigkeit sprangen die Hunde heran und prallten gegen göttlichen und magischen Schirm. Zwei Schwerter blitzen freudig im Fackellicht und rissen tiefe Wunden in die daimonischen Leiber. Dies hinderte freilich nicht die Gardisten, die nun unsere Gruppe in die Zange nahmen: Je zwei kamen auf Arngrim und Rezzanjin, die anderen beiden stürzten sich auf Firnen und mich. Es gelang mir glücklicherweise, den wild geführten Streichen meines Gegners zu entgehen und gewann so ausreichend Zeit, ihm mittels eines Zaubers hart gegen den Helm zu schlagen. Er sackte bewusstlos in sich zusammen und ich flankierte Firnens Widersacher. Dieser konnte zwar meinen Stockhieb parieren, wurde jedoch von Firnens Spruch gegen die Wand geschleudert und blieb tot liegen. Doch schon nahten weitere Mordbuben und droschen mit geschwungenen Klingen auf uns ein. Einer schlug Arngrim auf den Arm, welcher gleichzeitig noch mit einem Hund zu kämpfen hatte, doch der Schlag hinterließ dank der Rüstung nur einen kleinen Schnitt. Arngrim riss sein Schwert herum, durchbohrte den Gardisten in Brusthöhe und trieb den Stahl glatt durch den Hals des Hundes. Von unheiliger Macht erfüllt lief der Rumpf noch weiter, wurde jedoch von den anderen Bestien gierig verschlungen. Endlich richtete sich der Inquisitor keuchend auf und göttliches Licht flutete den Gang. Die Tentakel, Mäuler und Hunde zerfielen umgehend zu Staub, die Gardisten fielen geblendet den durstigen klingen zum Opfer. ``Ich bin sehr geschwächt.'', erklärte da Vanya mit dünner Stimme und lehnte sich gegen die Wand. ``Wir sollten rasch die Häftlinge befreien, bevor mehr dieser Abscheulichkeiten kommen.'' So spurteten wir den Gang entlang und erreichten schließlich eine verschlossene Stahltüre, die Rezzanjin aufbrach. Wir hasteten eine raue Steintreppe hinunter und fanden uns in einem dunklen, stinkenden Gefängnis wieder, das von Kohlefeuern nur unzureichend beleuchtet wurde. In den Zellen saßen zehn abgemagerte, von der Folter gezeichnete Maraskaner und sahen uns mit dumpfen Augen an. Ihre zerschlissene Kleidung zeigte noch Reste der bunten Farbe, wie wir sie so oft in den Straßen Tuzaks gesehen hatten. Doch das grausamste war der Anblick Ragnos‘ in der letzten Zelle. Obwohl er körperlich keinen Schaden gelitten zu haben schien, waren seine wirklichen Wunden deutlich sichtbar: Nicht eine einzige Regung ging von ihm aus, kein Wort verließ seinen Mund, kein Anzeichen deutete auf Empfindung hin. Er saß einfach nur da und starrte mit leeren Augen in die Ferne. Sofort eilte Toran hilfsbereit auf ihn zu, doch kaum hatte er ihn berührt, fuhr er erschrocken zurück. ``Sie haben ihn seiner Seele beraubt!'', stieß er schockiert hervor und sank auf dem Boden zusammen. Entsetzt starrten wir auf die traurige Hülle, die einst unser Jägersmann gewesen war. Doch vorerst konnten wir ihm nicht helfen, nein, vorerst mussten wir einen Auftrag erfüllen. Wir schlossen die Zellen der Maraskaner auf und diese stürmten umgehend die Treppe hinauf -- direkt in die Fänge der Hunde. Es folgte ein furchtbarer Schmerzensschrei, gefolgt von wildem Fauchen und dem Geräusch zerreißenden Fleisches. Als wir den Absatz erreichten, fanden wir die Überreste von vier Maraskanern, blutige Fetzen und Eingeweide, doch von den Hunden lebte kein einziger mehr. Die Männer hatten sie mit Nägeln und Zähnen zerfetzt. Als wir in die große Halle zurückkehrten, war diese gesäumt mit den Leichen von Drachengardisten und Sonnenlegionären und der kalte Geruch des Blutes war allgegenwärtig. Ucurian Jago erwartete uns bereits, auf sein Richtschwert gestützt. Blut lief aus einem tiefen Schnitt an seiner Seite, und sein Umhang hing in Fetzen von seiner Schulter. ``Wir haben sie zurückgeschlagen!'', knurrte er mit zusammengepressten Zähnen. ``Aber unser Blutzoll war hoch, wir haben fast die Hälfte der Männer verloren. Die anderen führt Rondrasil gegen den Fürstentrakt und diesen verruchten Schwarzmagier. Wir sollten ihnen helfen.'' Also eilten wir die Treppe hinauf und folgten der Spur aus toten Körpern zum Fürstentrakt. Plötzlich erschienen direkt vor uns mehrere verhüllte Wesen, die mit feurigen Peitschen und Schwertern bewaffnet waren: Heshtotim. ``Ihr werdet niemals siegen, Sterbliche!'', zischten sie und funkelten uns mit roten Augen boshaft an. ``Ihr seid schwach, aber er ist stark, ihr könnt ihn nicht bezwingen.'' Damit gingen sie zum Angriff über. Firnen zog aus seinem Gewand einen Teppich und warf ihn auf den Boden; darauf befand sich, aus feinen Silberfäden gewirkt, ein Pentagramm. Die Heshtotim zuckten zurück und schrien in ihrer grausamen Sprache Flüche und Verwünschungen. ``Pentagramma!'', rief Firnen mit donnernder Stimme und das Pentagramm begann zu glühen. Die Linien traten hervor, verschmolzen und öffneten sich zu einem Strudel magischer Gewalt, der die Wesen mit unwiderstehlicher Kraft in ihre schreckliche Welt herüber zerrte. Mit einem letzten Kreischen schloss sich der Durchbruch zur Sphäre und die Heshtotim waren verschwunden. Nach diesem Zwischenfall verloren wir keine Zeit und stürmten in den großen Audienzsaal Tuzaks hinein.

Die Schlacht war bereits in vollem Gange, als wir endlich eintraten und des fürchterlichen Chaos‘ gewahr wurden: Gut 20 verbliebene Sonnenlegionäre und Bannstrahler fochten gegen eine kleine Armee aus Heshtotim, Hunden und Drachengardisten, die unerbittlich die Reihen unserer Freunde lichteten. In der Mitte des Saales befand sich ein sechs Schritt großer, vollschwarzer Daimon, der mit seinen gewaltigen Klauen und Füßen alles und jeden zerquetschte, der das Pech hatte, in diese hineinzugeraten. Der Boden war derart bedeckt mit toten Körpern, dass man kaum einen Schritt tun konnte, ohne auf eine Leiche zu steigen. Der Gestank von Blut, verbranntem Fleisch und Sulfur hing in der Luft und vermischte sich mit dem niederhöllischen Gekreisch und dem Schreien der Männer zu einer Kakophonie des Todes und der Gewalt. Und im Zentrum des Grauens stand der Schwarzmagier und lachte aus vollem Halse. Ungeachtet der Gefahr stürzte sich Arngrim auf den Magus und drosch mit seinem Zweihänder auf ihn und seine Knechte ein. Diese rissen ihm schwere Wunden, doch er kämpfte ungeachtet dessen weiter. Wir anderen scherten uns um Toran zusammen, der abermals einen Schutzkreis gegen daimonische Kräfte zog und somit die Heshtotim auf Abstand hielt. Nicht weit genug jedoch, um sie mit Schwert und Zauber anzugreifen, was wir auch umgehend taten. Trotzdem mussten wir ansehen, wie ein Sonnenlegionär nach dem anderen zu Boden sank und nicht mehr aufstand oder von gierigen Kiefern zermalmt wurde. Derweil war es Arngrim gelungen, den Magus in arge Bedrängnis zu bringen und ihm zahlreiche Wunden zu reißen. Doch dieser drehte sich auf einmal auf der Stelle und tauchte am anderen Ende der Halle wieder auf -- dafür stand Arngrim nun dem großen Daimonen gegenüber. Wir sprangen ihm zu Hilfe und sofort fielen Heshtotim von allen Seiten über uns herein, doch sie konnten Firnens Schild nicht durchdringen. Hiebe nach links und rechts austeilend bahnten wir uns so unseren Weg durch die daimonische Masse, gerade als der Daimon Arngrim zertreten wollte. Im nächsten Moment waren wir bei Arngrim, der ob seiner Verletzungen zu Boden gesunken war, und der Fuß des Riesen prallte an der magischen Wand ab. Während Toran sich um unseren Ritter kümmerte, fochten wir wider das daimonische Ungetüm. Wieder und wieder schlugen wir Schwerter und Stäbe in das schwarze Fleisch, ohne ihn wirklich verletzen zu können. Doch dafür zeigten unsere Zauber größere Wirkung und der Gigant krümmte sich vor Schmerz. ``Kämpft euren Kampf allein weiter, Magier!'', rief er dem Beschwörer zu und verschwand ohne ein weiteres Wort. Der Kultist sah sein Ende nahen und floh zum Eingang der Halle, nur um dort Ucurian Jago und Amando Laconda da Vanya in die Arme zu laufen. Ein Schwertstreich später war auch diese Gefahr gebannt. Wieder vereinigt entledigten wir uns der restlichen Daimonen und trieben sie in ihre unheilige Welt zurück. Die Schlacht des Audienzsaales war gewonnen -- doch zu welchem Preis. Kein einziger unserer Soldaten weilte noch unter den Lebenden, zudem das ungewisse Schicksal der Truppe um Rondrasil Löwenbrand, die einen Weg am den Audienzsaal herum gesucht hatten. Nun, es blieb keine Zeit, über diesen Umstand nachzudenken, wir mussten uns einem größeren Übel stellen. Und so betraten wir durch eine große Flügeltür den kleinen Audienzsaal Tuzaks.

Der Raum war vom glühenden Licht der Abendsonne erfüllt, deren breite Strahlen durch Fenster zur Meerseite hin einfielen. An der Kopfseite der Halle saß einsam und zusammen gesunken Fürst Herdin auf seinem reich geschnitzten Thron und kicherte leise vor sich hin. Er bemerkte uns und sah zu uns herüber: Der schiere Wahnsinn funkelte in seinen leeren Augen und verzog sein Gesicht zu einer diabolischen Fratze. ``Seid still, Herdin!'', wies ihn eine Gestalt am Fenster mit gelangweilter Stimme zurecht, ohne dabei den Blick vom Perlenmeer zu wenden. Endlich wandte sich der Mann vom Fenster ab und blickte uns an: Es war Delian von Wiedbrück, den Mund zu einem spöttischen Lächeln verzogen, und die Hände, mit sechs langen Fingern besetzt, gelassen ineinander verschränkt, einen dünnen schwarzen Zauberstab haltend. ``Dieser Makel scheint mir jedes Mal anzuhängen, egal welche Gestalt ich annehme -- nun ja.'' Mit diesen Worten griff er sich ins Gesicht und zog das Fleisch einer Maske gleich ab. Darunter lag ein altes, aber vollkommenes Antlitz jenes Mannes, dessen Geburt wir vor Jahren im Nachtschattenturm miterlebt hatten. Borbarad. Wie er so da stand und die Hülle des Delian von Wiedbrück achtlos hinter sich warf, konnten wir keinen Muskel im Leib bewegen, festgehalten von seiner grenzenlosen Macht. ``Ich hatte euch bereits früher erwartet -- aber so seid ihr Menschen eben, unzuverlässig und schwach.'' ``Immerhin konnten wir eure Schergen beseitigen, Schurke!'', spuckte Arngrim vor ihm aus. ``Denkt ihr wirklich, dies müsse euren Fertigkeiten zugeschrieben werden? Glaubt ihr im Ernst, dass es euer eigener Verdienst war, der euch her brachte? Nein, ihr seid hier, weil ich es so wollte, und nicht weil euch einer eurer Götter die Kraft dazu gab. Eure Geweihten dort können nicht einmal sprechen, solange ich es ihnen nicht gestatte. Wo sind ihre Götter jetzt, statt ihnen beizustehen? Ich will es euch sagen: Die Götter helfen nicht, weil sie keine Macht haben, vielmehr noch, sie nehmen jenen Macht die sich ihnen unterwerfen. Stellt euch nur das ungeheure Potential der Wissenschaften vor, befreit von den Ketten eurer belanglosen Moral, losgelöst von Gesetzen und Ordnung, einzig und allein dem Willen des Suchenden verhaftet. Ich habe Dinge gesehen, gespürt, die keiner von euch sich je erträumen könnte, den Schlüssel zu grenzenloser Macht. Ihr hingegen führt sinnlose Kriege, zerstört das Wissen alter Kulturen und bindet euch Fesseln um, alles im Namen eurer Götter. Ihr Magi solltet das am besten verstehen. Wie leicht stößt man doch an die Grenzen der konventionellen arkanen Kunst, muss sich zähneknirschend einer höheren Instanz beugen, muss sich seine eigene Schwäche eingestehen. Doch je weniger ihr an Götter glaubt, desto weniger Einfluss haben diese, und ihr realisiert zum ersten Mal, dass ihr frei seid, frei zu tun und zu lassen, was ihr wollt. Will ich unendlichen Reichtum, so nehme ich ihn mir einfach. Will ich unendliches Wissen, so nehme ich es mir einfach.'' ``Und was wollt ihr diesmal?'', entgegnete Firnen zornig. ``Nun, vorerst gebe ich mit dieser Insel zufrieden. Überlasst mir Maraskan für meine Pläne, so wird euch und eurem Reich nichts geschehen. Wer von euch ist in dieser Angelegenheit befugt?'' ``Ihr werdet meine Heimat niemals kriegen!'', schrie Rezzanjin wutentbrannt auf. Borbarad blinzelte nur einmal kurz und der Maraskaner stieß einen gellenden Schmerzensschrei aus. ``Euch habe ich nicht um Antwort gefragt, Maraskaner, schweigt einfach. Aber ihr, Ritter von Ehrenstein, könnt sicherlich für die mittelreichischen Machthaber sprechen. Wie lautet eure Entscheidung?'' ``Ich schließe mich meinem Freund an. Ihr sollt Maraskan niemals rechtmäßig besitzen.'' ``Ich hatte erwartet, dass ihr ablehnt, dennoch überrascht mich immer wieder die Dummheit und Ignoranz der Menschen. Nun, ich habe noch einige meiner Pläne umzusetzen, deshalb wollt ihr mir sicher für mein Gehen verzeihen.'' Mit diesen Worten sprengte er die ganze Westfassade des Audienzsaales mitsamt Fürst Herdin zu Asche und trat durch die Trümmer nach draußen, wobei er den Zauberstab zurückließ. Im nächsten Augenblick erschien ein von sieben Schlangendaimonen gezogener Wagen, den Borbarad bestieg, und wir konnten uns wieder regen. Toran eilte umgehend zu Rezzanjin, um sich seiner Verletzungen anzunehmen, da begann der Zauberstab sich vom Boden zu lösen und senkrecht in der Luft zu verharren. Ein fahles Leuchten ging von ihm aus, ebenso wie dünne pulsierende Schwingungen, die jedoch rasch stärker und kräftiger wurden. Plötzlich riss die Welt hinter dem Zauberstab wie nasses Papier und die Temperatur sank um mehrere Grade. Ein eisiger Hauch fuhr durch den Saal, ein Hauch aus einer anderen, verbotenen Welt: den Niederhöllen. Und mit einem Mal sprangen drei Zantim aus dem Riss hervor, drei Schritt hohe Säbelzahntiger mit säuregetränkter Haut im aufrechten Gang, gefolgt von einem formlosen Knäuel von Tentakeln und Mäulern. ``Erscheine, Dschinn!'', rief ich, und die elementare Gewalt schoss aus dem Ring und stürzte sich auf das Untier. Hinter uns wurde die Tür aufgetreten und Rondrasil stand im Eingang gefolgt von zwei Bannstrahlern, alle übel zugerichtet, aber mit grimmigem Funkeln in den Augen und blank gezogenem Schwert. ``Ich hole mir den Stab!'', schrie ich durch das Getöse zu den anderen hinüber, ``Aber ich brauche Rückendeckung!'' Die anderen nickten zum Zeichen ihres Verständnisses und stürzten sich auf die Daimonen, während ich mich geschirmt von Firnen, Toran und Rezzanjin auf den Zauberstab Borbarads zuarbeitete. Es war, als kämpfte man gegen einen übergroßen Widerstand an, so langsam bewegte ich mich vorwärts, ungeachtet der Gefahren, die um mich tobten. Ich war der Hilfe meiner Freunde vollkommen ausgeliefert, da ich mit aller Kraft fortzukommen suchte. Nur verschwommen nahm ich meine Umgebung war, auch im Geiste stets fokussiert auf das Ziel, den Stab. Derweil herrschte ein vollkommenes Chaos im Saal: Ucurian Jago warf sich mit einem ``Für Praios'' auf den Lippen gegen einen Zant, der gerade im Sprung zu mir begriffen war; der Dschinn des Feuers riss dem unbeschreiblichen Wesen einen seiner Tentakel aus; die Rüstung eines Legionärs wurde von spritzender Säure verbrannt; Rondrasil tobte wie ein Derwisch unter den Daimonen und sein Schwert stach und schlitzte unermüdlich. Immer wieder droschen die Zantim auf Firnens Schutzkuppel ein und ließen den Boden erbeben, doch sogleich war ein Kämpfer zur Stelle und focht gegen das Untier. War jedoch ein Tiger besiegt, so erschienen an seiner Stelle drei neue, und in kurzer Zeit war die Zahl der Monster auf das Doppelte angestiegen. Ungeachtet dessen bewegten wir uns weiterhin auf die Quelle des Übels zu, die nun noch eine halbe Raumeslänge entfernt war. Aus den Augenwinkeln sah ich, wie der von mir beschworene Dschinn von Tentakeln zerrissen wurde und flammende Überreste auf die Kämpfenden herab regnete. Ein Zant packte einen Sonnenlegionär und schleuderte ihn quer durch die Halle. Und immer mehr der Untiere entstiegen dem Sphärenriss und stürzten sich auf die tapfere Schar Menschen. Auf einmal war das Tentakelwesen direkt über uns und holte mit seinen kraftvollen Armen zu einem verheerenden Schlag aus -- da warf sich Arngrim nach vorn und wurde frontal getroffen. Sein Körper fiel leblos zu Boden und sein Schwert rutschte über den Boden. Von grauenhafter Wut gepackt sprang Rondrasil nach vorn und trieb sein Schwert bis zum Heft ins Herz der Tentakel. Schwarzes Blut spritzte über seine Arme und Beine und das Monster sackte in sich zusammen. Durch die Trauer und den Zorn angespornt verdoppelten wir unsere Anstrengungen und standen nun endlich direkt vor dem Zauberstab. Eine unwirkliche, aber körperlich spürbare Macht ging von ihm aus und durchströmte unsere Körper. Ich versuchte den Stab zu greifen, doch es gelang mir nicht -- eine unsichtbare Barriere hielt mich zurück. Gemeinsam drückten Toran und ich immer wieder gegen den Schild, und endlich kamen wir näher, während die Schlacht mit unverminderter Härte tobte. Quälend langsam schoben sich unsere Hände auf den Stab zu, Stunden schienen zu verstreichen, dann packte ich endlich das Holz mit festem Griff. Warm pulsierte es in meiner Hand, als sei es von wärmendem Blut durchströmt. Und um mich herum fiel ein Schleier, der mich den Saal, die Kämpfenden, das ganze Chaos gedämpft sehen ließ. In dieser unendlichen Weite, in der ich schwebte, waren sie alle bewegungslos, in ihrer Aktion erstarrt, nur Spielfiguren in einem zeitlosen Spiel. Ich wusste, dass ich mit diesem Zauberstab alles erreichen könnte, solange ich es nur forderte, und sei es unsterbliches Leben. Mit einem einzigen leichten Wink bannte ich die gesamten Daimonen im Raum und vertrieb sie in ihre Sphäre zurück. Dann fiel ich in eine tiefe Ohnmacht.

Als ich erwachte, befand ich mich noch immer im zerstörten Audienzsaal Fürst Herdins, den Zauberstab Borbarads in der Hand. Am anderen Ende der Halle standen Firnen, Toran, da Vanya, Rondrasil und all die anderen Überlebenden in einem Kreis zusammen und blickten betrübt zu Boden. ``Mühsam richtete ich mich auf und humpelte auf die anderen zu. ``Was ist passiert?'' Statt zu antworten, traten sie nur schweigend beiseite und gaben den Blick auf Arngrims toten Leib frei, neben welchem Rezzanjin kniete.

\paragraph{54. Tag -- 30. Rondra}
Am nächsten Morgen brachen wir mit dem Schiff ``Perlbeißer'' nach Khunchom auf, Arngrims sterbliche Überreste sicher im Laderaum verstaut. Es schmerzt mich, einen weiteren Gefährten verloren zu haben; noch dazu einen, der so tapfer kämpfte. Wir werden ihn in einem Ehrengrab der Familiengruft der Ehrensteins beisetzen lassen, sobald wir wieder das Festland erreichen. Wir sind es ihm schuldig, sein Andenken zu bewahren und ihm die letzte Ehre zu erweisen. Maraskan wird an den Reichserzmarschall Helme Haffax übergeben werden; die Sendung wurde bereits aufgegeben. Er scheint der fähigste zu sein, diese strategisch wichtige Insel zu halten und gegen jenen Schrecken zu verteidigen, dem wir dort begegnet sind. Doch wir dürfen uns nun keine Ruhezeit gönnen, jetzt wo Borbarad wieder verschwunden ist. Wo auch immer er das nächste Mal auftaucht, wir sollten vorbereitet sein. Vielleicht kann uns dabei ja sein Zauberstab helfen, der sich ebenfalls in unserem Gepäck befindet. Nach einer oberflächlichen Analyse scheint er Ragnos' Seele gefangen zu halten, aber auch keine eigene astrale Macht zu besitzen -- zumindest keine, die wir erkennen konnten. Die Geschichte berichtet vom Zauberstab des Tharsonius von Bethana ``Seelenfresser'': Ein Stab, aus einer Blutulme geschnitzt, fähig, Seelen zu reißen und halten. Die Parallelen liegen auf der Hand. Jetzt gilt es, geschlossen und stark vorzugehen, denn das Schicksal des Kontinents liegt in unseren Händen. Wir dürfen nicht die Trauer über das Vergangene in unsere Herzen lassen, sondern müssen froh der Dinge ausharren, die da kommen mögen. Ein Glück, dass wir dies nicht alleine zu tun haben; wir können immer noch auf Freunde zählen. Diese Hoffnung darf niemals sterben.

\chapter{Bastrabuns Bann}

\section{Geleitwort}

Bastrabuns Bann ist leider wohl eines der schwächeren Abenteuer der Kampagne, auch wenn die Prämisse an sich nicht schlecht ist.
Archäologie und alte Geschichten in einer orientalischen Landschaft, mit Rivalen und dem vagen Versprechen, etwas über die Vergangenheit des Dämonenmeisters zu erfahren, all das sind ja eigentlich vielversprechende Elemente.
Aber leider bekommt man als Spielleiter sehr wenig Hilfe dabei, wie man diese Elemente aneinanderknüpfen soll.
Hinzu kommt, dass die Bannmauer selbst für den weiteren Verlauf der Kamapgne keine Rolle spielt, was natürlich eine vergebene Chance ist.

Die Vergangenheit und der Charakter Borbarads sind mit die größten Schwierigkeiten der ganzen Kampagne.
Der Dömonenmeister beginnt am Anfang als ein Gerücht, als ein vager Name ohne Form, EIgenschaften, oder Ziel, und dennoch muss er über den Verlauf der gesammten Erzählung konkret genug werden um am Ende als tatsächlicher Gegner auftreten zu können.
Die Kampagne hilft allerdings nur wenig dabei, diesen Bogen sauber aufzuspannen.
Borbarad selbst kann nicht (oft) auftreten, denn er ist zu allmächtig als dass die Helden es überleben sollten.
Insgesamt trifft die Gruppe ihn planmäßig nur 5 Mal: in den ersten 4 Abenteuern tatsächlich jedes Mal, allerdings während er noch seine Kraft sammelt.
Nach dem Tuzaker Fürstenpalast gibt es nur zwei weitere Begegnungen: das Duell mit Rohal nach dem Magierkonvent, und die Endschlacht an der Dämonenpforte.
Damit ist Bastrabuns Bann auch das erste Abenteuer dass sich sonderbar weit weg von der eigentlichen Geschichte wendet, denn der Bösewicht taucht gar nicht auf!
Und seine dämonischen Horden sind auch noch nicht dabei das Kaiserreich direkt anzugreifen.

Irgendwo zwischen diesen Begegnungen müssen die Helden verstehen dass Borbarad ein Wiedergänger ist, der in jedem vergangenen Zeitalter erscheint und dessen Taten mehr oder weniger direkt zum Untergang des selben führen.
Eigentlich ist diese Lektion der Sinn von Bastarbuns Bann, aber als das Abenteuer ursprünglich geschrieben wurde gab es diesen gesammten Plan noch nicht, weshalb die Rahmengeschichte, Bastrabuns Bann, etwas losgelöst von der Mythologie existiert.

Das hier so einmal auszuschreiben ist wahrscheinlich nur ein billiger Versuch einer Selbsttherapie, aber sollte jemals einer von euch wieder Finger an die Geschichte der sieben Gezeichneten legen (Philip?), dann schreibt dieses Abenteuer einfach komplett um.
Macht aus der ganzen Sache explizit einen Wettlauf gegen Borbaradianer die im Tulamidenland nach Artefakten ihres Meisters suchen, spannt die Gorische Wüste als Schaffungsort Borbarads ein, und bereitet den großen Plan vor.
Was der Plan genau ist, schreibe ich euch später wenn es ans Magierkonvent geht.

Doch genug der Beschwerden, das Abenteuer hatte auch viel Gutes zu liefern! Beim Durchgehen der Tagebücher wurde ich direkt daran erinnert, dass direkt am ersten Spielabend der wahrscheinlich bescheuerteste Einkauf aller Zeiten geschah: das Astrolabium! Komplett nutzlos, 8000 Dukaten, und ein eindeutiges Zeichen dass Geld aufgehört hatte eine sinnvolle Rolle zu spielen.
Später


\begin{flushright}
Claas Völcker, Toronto, den 25.05.2025
\end{flushright}


\section{Die Tagebücher}

\subsection{Auftakt nach Rezzanjin al'Ahjan}

\paragraph{Ende Rondra 1019 BF}
Tuzak schrumpfte in der Ferne immer mehr zusammen und verschwand schließlich aus unserer Sicht, während wir an Bord der Zedrake Perlbeißer meine schöne Heimat gen Kunchom verließen. Bald erinnerten nur noch die hoch aufragenden Berge an die Schönheit, die dieses Eiland bietet und an die Strapazen, Gefahren und Kämpfe, die wir dort erlebt hatten. Beinahe so ein Gefühl wie Erleichterung und Entspannung machte sich an Bord breit, einzig durch das Schicksal unserer beiden Freunde betrübt. Arngrimm war im Kampf gegen die von Stabe Borbarads beschworenen Dämonen gefallen und Ragnos, der nur noch körperlich anwesend ist, hatte seine Seele an den Stab verloren, doch waren wir guter Hoffnung, dass die Seele sich noch im Stab befinden würde und, dass man sich an der Magierakademie in Kunchom unseres Problems annehmen und es lösen würde. Gen Abend, als die Sonne sich schon anschickte hinter dem Horizont zu versinken, machte ich mich daran, mich auf die Kraft zu konzentrieren, die ich seit dem Ritual und dem Kampf mit dem Wächter spürte, aber bisher nie richtig fassen konnte. Doch diesmal schien es ein leichtes zu sein. Als ich die Kraft nach kurzem Ringen gefasst hatte und in mir spürte schlug ich die Augen auf und erblickte die Welt mit anderen Augen. Meine Gefährten schienen seltsam hervorgehoben aus der der sonst monotonen Umgebung, und auch die Flammen der Laternen strahlten heller, sie blendeten mich sogar, sodass ich meinen Blick abwenden musste. Als ich dann auf das Meer blickte, das wiederum irgendwie kalt erschien, dämmerte es mir: Ich konnte Wärme sehen. Ein weiterer Aspekt dieser geheimnisvollen Kraft, die ich seit den Tag bei den Echsen zu besitzen scheine. Ich bin gespannt, ob ich noch mehr Kräfte finden kann.

Kunchom ist noch nicht in Sicht, doch wir hoffen es morgen zu erreichen.

\paragraph{Anfang Effert, Kunchom}
Nach nur einem Tag Seereise liefen wir von der aufgehenden Sonne begleitet in den Hafen von Kunchom ein. Nach einer kurzen Diskussion war schnell klar, dass wir erst zur Magierakademie gehen würden, um den Stab zu untersuchen und um Ragnos Seele mit seinen Körper zu vereinen und danach Richtung Punin zur Boronkirche, um unsere verdiente Belohnung abzuholen und zumindest Teile des Enduriums abzuliefern. Schon im Hafen merkten wir, dass wir im Tulamidenland und nicht mehr auf Maraskan waren. Braungebrannte Händler in reich verzierter Gewandung scheuchten ihre Schiffsjungen durch die Gegend, mehrere Händler, die neu in den Hafen eingelaufenen Reisenden versuchten alles mögliche zu verkaufen und die prächtigen Schiffe der Tulamiden vervollständigten das Bild. 

Schon im Hafen war das Gedränge dicht, doch in der Stadt war es noch dichter. Temyr warnte uns noch einmal vor den vielen Straßenjungen, die nur darauf aus waren, einem dem Geldbeutel vom Gürtel zu stibitzen. Eine Warnung, die nicht neu für mich war, dennoch schaffte es ein flinker Junge mir meinen Geldbeutel zu entreißen. Zu schnell war er in der Menge verschwunden als das ich ihm hätte nachjagen können. Groß war der Verlust so wie so nicht. 

Mir fiel auf, dass die Leute uns mehr Aufmerksamkeit schenkten als sonst. Bald fiel mir auf, dass Ragnos es war, der die Aufmerksamkeit auf sich lenkte. Wie er geistesabwesend in der Gegend herumstolperte, seinen Blick dauerhaft in die Ferne gerichtet, das war fürs geschäftige Tulamidenvölkchen sicherlich ein erschreckender Anblick. Temyr fühlte sich in der bekannten Umgebung sichtlich wohler und führte uns zielsicher an den größten Gedrängen vorbei zum Platz vor der Magierakademie. 

Dort ließen wir es uns dann nicht nehmen noch ein wenig in der dort feilgebotenen Auslage zu stöbern. Oder genauer gesagt: Die Magier ließen sich das nicht nehmen. Das meiste erwies sich als nutzloser Kram, doch ein paar Objekte waren, so die Temyr und Firnen, sehr vielversprechend. Firnen zeigte gleich sein Gespür, das teuerste Objekt auf dem gesamten Markt zu finden. Ein plastisches Modell der Zugbahnen der bekannten Sterne für nur 8000 Dukaten. Es schien zwar gut gearbeitet, aber 8000 Dukaten waren wahrhaftig zu viel. Firnen konnte den Magier, der es feilbot zwar noch auf 6000 Dukaten herunterhandeln, dennoch überstieg es die gesamten Mittel der Akademie, wie mir schien. Temyr fiel die auch auf, sodass wir Firnen, der schon fast versprochen hatte es zu kaufen doch noch davon losreißen konnten. Als wir uns dann endlich durch das Gedränge am Markt gekämpft hatten, standen wir vor dem imposanten Tor der sogenannten Dracheneiakademie in Kunchom. Der standesgemäße Dschinn im Tor der Akademie machte keine allzu großen Probleme, sodass uns recht schnell Einlass gewährt wurde. Doch viel weiter kamen wir nicht. Wie es sich für arrogante und große Magier wohl gehört, wurden wir erstmal warten gelassen, obwohl unser Anliegen sehr dringend war und unsere Beziehungen dank Temyr auch recht groß. Nach einer nicht wirklich kurzweiligen Zeit wurden wir von einem Luftdschinn namens Wolkentanz, der uns zu seinem Herrn ``Khadil'' bringen würde, abgeholt. Natürlich verzichtete dieser nicht darauf uns auf dem Weg zum Gemach von Khadil Okharim die halbe Geschichte der Akademie und die der Magier, die einst hier verweilten näher zubringen. Nach kurzer Zeit lernte ich mehr über Magier, als ich in meinem gesamten Leben zuvor gehört hatte. Dann endlich kamen wir auch zu Khadils Empfangsraum, wo wir auch schon von ebenjenem Magier und zwei anderen Gesichtern, die ich schon mal auf Klammsbrück gesehen hatte, begrüßt wurden. Wie sich herausstellte waren das Rakorium Muntagonus, der uns schon in Verbindung mit dem Echsenforscher auf Maraskan bekannt und ein zweiter Magier Dschelef, den wir aus dem tobrischen Winter gerettet hatten. Als wir erwähnten, dass wir gerade von meiner schönen Heimatinsel kamen, machte sich der eindeutig paranoide Rakorium dran uns alle zu mustern. Doch alles weitere der Gespräche würde wohl warten müssen, da Khadil Okharim es wohl nicht mehr ertrug unseren ziemlich heruntergekommenen Anblick in seinem reich geschmückten und prächtig ausgestatteten Zimmer zu ertragen. Sein Angebot ein Bad zu nehmen und uns noch ein wenig auszuruhen kam gerade recht. Als wir nach zwei Stunden ausgeruht, rasiert und in neuen tulamidischen Kleidern zu ihm ins Zimmer kamen, erwartete uns gleich ein ``kleines'' Mahl, mit dem man wohl jeden Straßenjungen in Kuhnchom für einen Tag durchfüttern könnte. Hungrig wie wir waren, machten wir uns auch gleich ans Essen. Schon während dem Essen erzählten wir von unseren zurückliegenden Erlebnissen. Von dem Schiffsbruch im Sturm über das Treffen mit Puspereiken, die Begegnung mit dem Dämon in der Mine, die Ereignisse im Tempel auf dem Friedhof der Seeschlangen bis hin zur entscheidenden Schlacht um die alabasterne Residenz. Meine Zeichnung ließen wir außen vor, um Rakorium nicht vollständig zu Brodeln zu bringen und selbstverständlich erzählten wir auch nichts über das Haraniadad. Unsere Zuhörer lauschten der Geschichte gespannt und stellten ab und zu mal Zwischenfragen. Kurz nachdem wir fertig waren mit erzählen wandelte sich das Gespräch in eine Diskussion, woher Borbarad seine Kraft hätte und nach kurzer Zeit verlor ich den Gesprächsfaden komplett und verstand das Kladeradadj der Magier gar nicht mehr. Irgendwann wollten die Magier dann den Stab analysieren. Doch bevor diese das tun konnten, ließ Toran uns alle einen Eid auf die zwölf Brüder und Schwestern sprechen, dass wir den Stab nicht benutzen würden, um irgendwem zu schaden. Nachdem das vollzogen wurde machten sich die fünf anwesenden Magier daran den Stab mittels Magie zu analysieren. Die Prozedur gab in etwa das Bild ab, das man in tulamidischen Märchen von mächtigen Magiern beim Zaubern bekommt. In einem Kreis um das mächtige Artefakt stehend, murmelten die fünf in reich geschmückten tulamidische Gewändern gekleideten Magier gleichzeitig die Formel, die die Zauber, die in den Stab hinein gewebt waren, für sie zum Vorschein bringen würden und machten dabei geheimnisvolle Bewegungen mit den Händen, um es mystischer aussehen zu lassen. Das konnte man zumindest von Kadil Okharim behaupten. Doch nachdem sie die Formel gesprochen hatten, zeigten ihre Gesichter eine Mischung aus Enttäuschung und Erstaunen, anstatt des eigentlich erwarteten wissenden Lächelns. Wie sie kurz später erklärten sei der Stab von Zaubern geschützt, die kein weiteres Analysieren in dieser kleinen Runde zuließen. Statt der erwarteten Lösung also nur weitere Rätsel und Schwierigkeiten. Dann machte sich Toran daran den Stab zu untersuchen. Er, mit der Fähigkeit Seelentiere zu sehen erkannte recht schnell, dass ziemlich viele dieser nahezu regungslos im Stab verharrten, darunter auch das von Ragnos. Es war also klar, dass seine Seele noch im Stab gefangen war. Weitere Versuche Torans dem Stab weitere Informationen zu entlocken scheiterten an der Mächtigkeit desselben.

Doch wir waren nicht die einzigen, die etwas zu erzählen hatten. Auch die drei Magier, besonders Dschelef, schienen nicht untätig gewesen zu sein, denn sie erzählten uns von Nachforschungen, die sie gemacht hätten. Bastrabuns Bann, der wie ich wusste die Echsen aus den Tulamidenlanden fernhält, sei aus einem Netz arkaner oder irgendwie anders gearteter Artefakte aufgebaut und man könnte es gegebenenfalls Nutzen, um Borbarad fernzuhalten, wenn man es denn modifiziert. Mehr habe ich davon nicht mitbekommen, da die Magier wieder in ihr eigenwilliges Kladeradadj zurückfielen. Des Weiteren seien die von uns aufgestellten Theorien vielleicht richtig, aber man müsse das nachprüfen. So beauftragten sie uns ein paar Feldforschungen anzustellen. Wir nahmen an, jedoch unter der Bedingung erst nach Punin gehen und Ragnos heilen zu wollen, bevor wir uns ihren Studien widmen würden. Bei der Erwähnung, dass wir nach Punin gehen wollten, fragte und Khadil Okharim, ob wir denn einen Geweihten des Borons mitnehmen würden, der hier in Kunchom nur nErwen und die angehenden Magier in der Bibliothek nur von der Arbeit abhalten würde. Wir willigten ein ihm diesen Gefallen zu tun.

Aus Gründen, die ich immer noch nicht ganz nachvollziehen kann, wollten wir uns um Mitternacht an der Akademie treffen. Die Zeit davor nutzte ich um meine Verwandten im Maraskanviertel in Kunchom zu treffen. Es war eine schöne Zeit und ich war froh endlich wieder ein paar nähere Verwandte zu treffen. Mein Onkel und meine Tante waren von der Geschichte, bzw. den Neuigkeiten, die ich mitbrachte erstaunt und ich wusste: In drei Tagen würde es das gesamte Viertel wissen, was es neues von unserer Heimatinsel gab.

Danach wurde es komisch. Das letzte woran ich mich erinnern konnte war, dass ich in die Akademie in Kunchom ging, danach war ich mit meinen Freunden anscheinend schon in Punin in der dortigen Akademie. Zusammen mit Toran wurde ich zu den Magiern gebracht, die anscheinend getrennt von uns hergebracht wurden. Gereist sein konnten wir nicht sein, daran hätte ich mich erinnert und jede andere Form von Teleportation etwa durch Firnen wäre, nach allem was ich mitbekommen habe, unmöglich. Dennoch vermute ich, dass die Magier etwas damit zu tun haben. Immerhin sind wir in einer Magierakademie wieder aufgetaucht. Doch in diesem Falle sollten ich ihnen wohl das Geheimnis überlassen. Komisch war auch, dass der Geweihte des Borons nicht bei uns war, obwohl er eigentlich mit uns kommen sollte. 

Nachdem wir uns einigermaßen zurechtgefunden hatten, wollten wir zum Haupttempel von Bruder Boron gehen. Dort herrschte über unsere Ankunft helle Freude. Na ja sollte wohl herrschen. Doch wie immer waren die Brüder und Schwestern in Bruder Borons Diensten sehr schweigsam, als ob sie die Wunder der Welt gar nicht kennen würden, sondern sich nur im Dunkeln nach draußen wagen und dann nur mit verbundenen Augen, um auch gar nichts von der Welt mitzubekommen. Doch wie immer war der Tempel von Bruder Boron ein Erlebnis. Er war ziemlich abgedunkelt und man sah die Schreine an denen verschiedenen Geweihte von Bruder Boron im stillen Gebet versunken waren erst spät. Insgesamt hatte das Bauwerk seit unserem letzten Mal nichts an Imposantheit verloren. Wir wurden recht schnell durch die ganzen Tunnel zum Oberhaupt der Boronkirche weitergeleitet. Der Rabe, wie er sich nannte gab immer noch die selbe unheimliche und gleichzeitig geheimnisvolle Gestalt ab. Seltsamerweise schien der Raum um ihn herum noch dunkler und noch stiller zu sein, als er es in einer Boronkirche eh schon war. Wir sahen, dass auch die Vertreter der Al Anfaner Boronkirche anwesend waren. Als wir berichteten was auf Maraskan vorgefallen war und, dass wir das verlorene Endurium und sogar noch mehr dabeihatten, konnte man fast meinen, dass bei einigen der Boronis ein Lachen im Gesicht zu finden war. Leider war es mir nicht gelungen etwas von den Endurium für mich und die restlichen Maraskaner mitzunehmen, da ich eigentlich geplant hatte auf den Weg nach Kunchom über meine Verbindungen etwas mitgehen zu lassen. Wir bekamen natürlich Belohnungen in entsprechender Form: 200 Dukaten für jeden. Diese ganze Prozedur wurde immer wieder unterbrochen von Zwischenrufen der der Al Anfanerin, die klarstellen wollte, dass auch die Al Anfanische Boronkirche an der Belohnung und dem Dank beteiligt war. Uns wurde dann eine ganz besondere Belohnung versprochen und wir wurden durch weitere Kellergewölbe immer tiefer hinein in die Räume unter der Hauptkathedrale geführt. Auf einmal standen wir vor einer großen Holztür, die für uns geöffnet wurde. Der Raum in den wir hineingeführt wurden war riesig. Ganz aus Stein gehauen standen hier sieben beeindruckend große aus Stein gehauene Gräber. Die Tafeln auf denen die Namen derjenigen stehen würden, die einmal hier begraben werden würden waren leer. Die Geweihte von Bruder Boron erklärte uns, dass hier einmal die sieben Gezeichneten liegen würden, also wir. Man würde den Raum noch vergrößern wenn ein paar Gezeichnete sterben würden, was nicht unwahrscheinlich war, um auch sie hier bestatten zu können. Man würde den ganzen Kontinent absuchen um unsere Leichen zu finden, nur um uns hier begraben zu können. Ich nahm das ganze mit gemischten Gefühlen auf. Einerseits war es grausam so direkt die Möglichkeit des eigenen Todes aufgezeigt zu bekommen, andererseits war mir das neuerdings auf eine seltsame Weise weniger wichtig. Was jedoch noch verstörender war, war die Tatsache, dass diese Fremdjis doch tatsächlich glaubten mich hier in diese Gruft einsperren zu können wo die Schönheit dieser Welt weiter entfernt war, als damals in der Gor. Hier drinne würde mich Schwester Tsa doch nie finden können und ich würde bis zu dem Zeitpunkt hier verharren müssen, wenn Gror die Weltendiskus fängt, wenn nicht noch länger. Doch in diesem Augenblick hielt ich mich zurück: Die Fremdjis würden die Schönheit der Welt hier unten und mit ihrer einseitigen Sichtweise nie erkennen können, so sehr ich sie ihnen erklären wurde. Doch einen Lichtblick gab es. Die Kirche von Bruder Boron hatte die Gefahr Borbarads erkannt, sonst hätten sie diese Gräber nicht gebaut. Wir bedankten uns und wollten zur Magierakademie zurückkehren, um uns des zweiten Problems anzunehmen, als uns die Al Anfanische Geweihte des Boron aufhielt. Sie versprach uns eine Überraschung, wenn wir morgen früh zu ihr kommen würden, was nicht mehr viele Stunden hin sein würden. Wir kamen dann an der Magierakademie an und wurden erst einmal zu Zimmern gewiesen, da die leitende Spektabilität erstmal schlafen wollte. Es war mitten in der Nacht und so nahmen wir das Angebot an. Erst jetzt merkte ich meine Müdigkeit, die schon beträchtlich war und legte mich schlafen. Am Morgen wurden wir dann geweckt. Firnen und Temyr erklärten der Akademieleiterin, was sie vorhatten. Als diese hörte, was das für ein Stab war, fuhr sie aus ihrer Haut. Sie schimpfte wie Magier nur so etwas so unsicher aufbewahren konnten und noch ein paar Sachen mehr, doch hatte sie sich bald beruhigt. Firnen und Temyr wollten mehrere Zauberer sammeln und gemeinsam den Stab analysieren. Nachdem die Leiterin der Akademie dies hörte wollte sie noch ein wenig Zeit für haben, um Räume für das Ritual vorzubereiten. Ich wurde während dieser Zeit regelrecht eingesperrt und vermute es ging den anderen genauso. Als ich herausgelassen wurden, wurde ich in einen Raum geführt, in dem mehrere Pentagramme auf den Boden eingezeichnet waren. Ich wurde angewiesen mich in ein eher abseits befindliches Pentagramm zu stellen und sah, dass auch Toran dies tun musste. Die Magier, Es waren Rakorium, Temyr, Firnen, die Leiterin der Akademie Garlisch-Grötz und noch ein paar andere Magier, die ich nicht kannte. Das Ritual war im Vergleich zu dem in Kunchom eher langweilig und eintönig. Doch es schien mehr dabei herauszukommen. Die Zauberer verweilten noch etwas in der Unterredung in ihrem eigenen Kladderradatsch, bis sie den Wunsch äußerten den Stab noch zum Tempel von Bruder Boron zu bringen, um Ragnos Seele aus dem Stab zu bekommen, da sie bei der Analyse wohl keine Mölichkeit gefunden hatten es auf magische Weise zu tun. Die Leiterin der Akademie beschwerte sich dann als Firnin den Stab mit der bloßen Hand nehmen wollte und zeterte ein wenig herum. Mir wurde es zu viel und ich nahm mir den Stab und wollte davongehen, doch die Akademieleiterin hatte für mich außer dem lauten Aufschrei auch noch einen Versteinerungszauber parat. Ich konnte mich nicht mehr bewegen und mir wurde der Stab aus der Hand genommen. Das schlimmste bei einen solchen Zauber ist die Tatsache, dass man alles mitbekommt, was um einen herum geschieht, man kann aber nicht reagieren. Nach einer gefühlten Stunde kam ich frei und eilte meinen Freunden hinterher, die schon fast beim Tempel von Bruder Boron angelangt waren. Im Tempel ließen wir einen Geweihten holen und vertrauten ihm unser Problem an. Er leitete uns in einen Raum und kurze Zeit später kam eine Boroni rein, die uns erklärte, sie sei das Ordensoberhaupt der Golgariten und wolle den Stab untersuchen. Nachdem wir den beinahe leblosen Körper von Ragnos in die Mitte gelegt hatten fing sie an leise vor sich hin zu murmeln. Doch nach einer Stunde Gebet teilte sie uns mit nichts für unseren Freund tun zu können. Wir sollten warten. Sie ging fort und nach etwas längerer Zeit kam dann der Rabe, das Oberhaupt der Kirche von Bruder Boron und fing an im Gebet versunken leise, aber doch deutlich hörbar Bruder Boron um Gnade für diese Seelen zu bitten. Nach geschätzten zwei Stunden wandte er sich an uns und zum ersten mal hörten wir ihn sprechen: ``Selbst mit der Kraft Borons konnte ich nur zwei Seelen aus diesem Stab befreien. Die eures Freundes und noch eine andere. Wir werden denStab bei uns behalten bis alle Seelen aus ihm befreit sind. Ich muss mich jetzt ausruhen. Boron sei mit euch.'' Diese Worte schienen keinen Wiederspruch zuzulassen und so beugten wir uns ihnen und eilten, als er den Raum verlassen hatte, zu Ragnos. Dieser schien eine ganze Zeit lang keine Veränderung zu zeigen, doch dann konnten wir kleine Bewegungen erkennen. Er zuckte ein paar mal, doch seine Augen blieben geschlossen. Ein paar Novizen kamen herein und trugen ihn hinaus. Sie erklärten uns, dass unser Freund ein wenig Ruhe brauchte und das sie ihn in einen Raum für Kranke bringen würden. Unentschlossen, ob wir auf ihn warten sollten oder nicht wurden wir von der Al Anfanischen Geweihten von Bruder Boron unterbrochen und gefragt, ob wir denn kurz Zeit hätten. Wir bejahten und sie leitete uns nach draußen wo schon eine Kutsche wartete. Als wir alle in der Kutsche waren wurden wir ins Umland von Punin, dessen malerische Weinberge man aus der Stadt nur unzureichend sah, gebracht. Nach einer Weile bogen wir von der Hauptstraße ab und fuhren einen Hügel hoch. Als die Kutsche hielt, sah man aus dem kleinen Fenster ein stadtliches Haus idyllisch zwischen den Weinbergen gelegen. Wir wurden hineingeführt und standen in einer prächtigen Empfangshalle. Drei Leute, offensichtich Dienstleute warteten schon auf uns. Die Boroni eröfnete uns dann, dass das Haus ein Geschenk der Al Anfarischen Kirche von Bruder Boron war. Sie führte uns herum und zeigte uns alles. Auch wenn das Haus nichts von der praktischen Schönheit eines maraskanischen Hauses hatte, so war es doch auf seine Weise schön und gur eingerichtet. Zusätzlich zu dem Haus wurde jedem von uns ein prächtiges Pferd versprochen, das wir uns aussuchen dürften. Als ich auf dem Balkon im ersten Stock stand und das Treiben auf der Handelsstraße beobachtete hatte ich das Gefühl, dass das das erste mal seit langem war, dass ich richtig glücklich war und, dass Ragnos sich erholen würde\dots

\subsection{Aufbruch aus Khunchom nach Rezzanjin al'Ahjan}

\paragraph{Anfang Efferd in Kuhnchom, nach den Abendessen mit Khadil Okharim}
Nachdem wir die kurze Strecke von unserer Villa nach Punin zurückgelegt hatten, gingen wir zum Borontempel und holten unseren Freund Ragnos ab. Dieser saß ziemlich verwirrt bei einem Nebenaltar des Tempels und schien sich seit seiner Gefangennahme an nichts mehr zu erinnern. Wir erzählten ihm die Geschichte und berichteten auch vom Tod unseres Gefährten Arngrimm. Ragnos schien auch noch danach ziemlich verwirrt zu sein, dennoch beschlossen wir nach Kunchom aufzubrechen. Den Teil unserer Reise habe ich vergessen. So wie unsere Hinreise haben wir uns aber anscheinend hin und wieder zurück teleportiert, da anscheinend kaum ein Tag verging. Vielleicht lüftet sich das Geheimnis noch irgendwann. Die Magier scheinen bei dieser Geschichte mehr zu wissen als sie vorgeben.
In Kuhnchom wurden wir gleich bei Khadil Okharim zum Abendessen eingeladen. Eine Einladung die wir nicht ausschlagen konnten. Doch bis zum Abendessen war noch Zeit und so ging ich noch in das Maraskanerviertel und besuchte meinen Onkel. Wir tauschten den üblichen Tratsch aus und er konnte mir netterweise noch einen guten Lehrmeister empfehlen, den ich am nächsten Tag aufzusuchen beschloss. Am Abend saßen wir im prächtigen Saal der Magierakademie und speisten. Außer Okharim waren noch Rakorium, Dschelef und ein paar andere Magier anwesend, die wohl ausschließlich dazu da waren, um den ebenfalls an der Tafel sitzenden Borongeweihten vom Essen abzuhalten. Nach dem Essen erklärte Khadil Okharim noch einmal allen Anwesenden was den unser Auftrag sei. Da merkten wir, dass er uns den Borongeweihten unterjubeln wollte, was wir nach kurzer Diskussion nicht mehr verhindern konnten. Wir sollten im Tulamidenland nach Artefakten suchen, die einst den Bann von Bastrabun unterstützt haben sollten. Die Magier der Akademie hatten die Hoffnung, dank der gefundenen Artefakte den Bann wieder herstellen zu können. Suchen könnten wir am Mahanadi, und bei der Linie Kunchom, Anchopal, Fasar. Nach einer weiteren Einweihung durch andere Magier, ließ Khadil Okharim uns alleine, nur einen Geldbeutel für Besorgungen ließ er zurück. Wir planten über die nächsten paar Tage Besorgungen wie Pferde, Schaufeln, Ausgrabungswerkzeuge, Zelte und ähnliche Dinge zu tätigen und uns in der Stadt nach Informationen umzuhören. Mir fiel dabei der Teil zu, die Erzähler der Stadt nach mystische Geschichten, von alten Grabstädten und ähnlichem abzufragen, ein Teil den ich sehr gerne übernehme.

\paragraph{Eine Woche später, Ende erster Reisetag}
Nach dem wir die Woche Informationen gesammelt hatten, beschlossen wir erst über den Mahanadi nach Zhamorrah und dann nach Anchopal und Fasar zu reisen. Die erste Station würde Temphis sein. Danach würden wir einen Abstecher nach Al`Ahabad machen. Doch am letzten Tag vor der Abreise erreichte jeden von uns ein Brief vom Fürsten von Kunchom. Als wir am Vormittag in seinen Palast eintrafen, traten wir dann auch schon eine halbe Stunde später, mit den Gepflogenheiten vertraut, wie man sich denn gegenüber einem Kalifen zu verhalten habe, vor den Kalifen höchstselbst. Er lobte unsere Taten, von denen er schon viel Gehört habe und machte schnell klar, was er von uns wollte: Sein Sohn, ein kleiner, dicklicher Knabe von 14 Jahren, der neben dem Herrscher saß, habe die Gabe der Magie und solle in Mherwed unterrichtet werden. Da wir in diese Richtung reisen, könnten wir ihn mitnehmen. Bisher sei er immer nur im Palast gewesen und solle auf dem Weg ruhig mal ein paar Abenteuer erleben. Nichts Böses dachten wir uns am Anfang. Die Belohnung würde gut sein und es würde auch keinen Umweg bedeuten. So waren wir am nächsten Morgen früh mit gepackten Sachen vor dem Palast und warteten. Während Temyr und Firnen aus Langeweile eine Partie Rote und Weiße Kamele spielten und Temyr dabei haushoch gewann, hatten wir anderen nicht so viel Zeitvertreib. Nach zwei langen Stunden kam der kleine Kalif dann aus dem Palast und ließ sich sein großes Reisegepäck von ein paar Dienern oder Sklaven nach draußen stellen. Sofort forderte er uns auf, dass wir doch beginnen könnten, die Kutsche zu beladen. Nachdem wir ihm klargemacht hatten, dass er nicht mit einer Kutsche, sondern zu Pferd reisen würde und auch nicht sein Gepäck verpackt bekommen würde, war er ziemlich überrascht. Nach etwas längerer Diskussion, gingen dann Firnen, Toran und Boronos, der Borongeweihte noch ein Pferd kaufen, da der Prinz keines hatte. In der Zwischenzeit hatte er uns seinen Namen mit geteilt, was bei uns für Erheiterung sorgte. Er hieß Stippen. Als wir nach der Herkunft dieses Namen fragten, bekamen wir die Antwort, dass es der Name eines nebachotischen Ur -onkels sei. Daraufhin entglitt Toran die Frage, ob das eine Herkunft oder eine Krankheit sei und wir lagen lachend am Boden. Als Firnen irgendwann mit den Pferden wiederkam, weigerte der Junge sich immer noch. Als mir irgendwann der Einfall kam, seinen Vater Bescheid zu sagen, kam der Junge in die Gänge und packte seine Sachen. Doch als ich mich umdrehte riss mir die Naht meiner Hose am Hintern auf. Anscheinend hatte der Junge das mit seinen noch unreifen magischen Kräften geschafft. Ich beschwerte mich nicht weiter, sondern ging nochmal auf den maraskanischen Markt und kaufte mir zwei Ersatzexemplare. Dann endlich, es war schon fast Mittag brachen wir nach Temphis auf\dots

\subsection{Das Grab des Shamscherib nach Toran Ostik}

\dots Später am selbigen Tage verließen wir Kunchom. Mit uns reiste der Sohn der Fürsten von Kunchom ein gewisser Stippen. Unsere Queste sollte uns durchs Elmadital und nach Temphis führen. Doch schon nach einer halben Tagesreise stießen wir auf Probleme. Nachdem man uns freundlicherweise beim Dorfältesten eines kleinen Dorfes in dem wir die Mittagshitze überdauern wollten zum Essen einlud, mussten wir feststellen, dass unsere Pferde gestohlen worden waren. Dank Ragnos geschultem Auge konnten wir sie jedoch schnell an einer naheliegenden Quelle wiederfinden und schafften am selben Tag noch ein gutes Stück Strecke bevor wir in einem angenehmen Olivenhain Unterschlupf für die Nacht fanden. Die nächtliche Unterbrechung durch in unserem Proviant wühlende Hyänen bekam ich kaum mit, da Rezzanjin sich ihrer annahm.

Am nächsten Morgen ging es weiter bis Chefe wo wir unseren knapp gewordenen Proviant aufstocken konnten. An die unangenehmen Vorkommnisse des nächsten Reiseabschnitts kann ich mich schlecht erinnern. Es kam zu einem schlimmen Unwetter und eine Gruppe rauer Gesellen wollte wohl die Gelegenheit beim Schopf packen und sich an uns etwas bereichern. Der gute Temyr, dem die Hitze viel weniger zuzusetzen schien als uns anderen, war jedoch so schnell mit seinen Zaubern wie er es sonst mit seiner Zunge ist und wir ließen die Ganoven gefesselt zurück. Trotz des bereits sehr ereignisreichen Tages schafften wir es zum Mittag noch nach Challef. Firnen schien die Reise nicht sehr gut bekommen zu sein, er wirkte kränklich. Bis auf weiteres schätzte ich seinen Zustand jedoch nur als einen vorübergehenden Schwächeanfall durch zu wenig Schlaf ein. Als wir am Abend jedoch Jamin erreichten schien sich sein Zustand jedoch noch verschlimmert zu haben. Mit Prainors Hilfe erkannte ich, dass es sich wohl um eine Lungenentzündung handelte. Eine solche Erkrankung ließe sich normalerweise mit einigen Tagen Bettruhe und einem ordentlichen Kräuter-Honig Sud heilen, doch angesichts der Eiligkeit unserer Mission bat ich die Göttin um Hilfe und schon am selben Abend konnte mein Gefährte wieder schmerzfrei handeln.

Kurz bevor die Sonne am nächsten Tag den Zenit überschritt erreichten wir Temphis. Wir gaben das Päckchen bei genannter Adresse ab wurden aber nicht hineingelassen. Beim einem sehr nötigen Besuch in einem der berühmten tulamydischen Badehäuser erfuhren wir mehr über diesen ``Alfessir'' bei dem wir unsere Ware abgeliefert hatten. Er war ein regional sehr bekannter Edelsteinhändler, was uns angesichts unserer Versuche Mondsteine ausfindig zu machen gerade zu Pass kam. Frisch gewaschen und in mein offizielles Ornat gekleidet gewährte man mir eine Audienz bei ihm. Leider besaß er selbst nur kleinere Mondstein Splitter, allerdings konnte er uns nach Samara verweisen, wo es angeblich einen sehr großen Stein im Besitz des Efferdtempels gab.

Nach einer ausnahmsweise sehr erholsamen Nacht in Temphis brach die Gruppe Richtung Ehristar auf. Der sonst so schweigsame Firnin ließ uns jedoch schon nach einigen Stunden an einem alten tulamydischen Hügelgrab halten, durch das seiner Aussage nach ein ``Kraftlinie'' floss. Das aus Entfernung eher unscheinbare Konstrukt stellte sich als 4-stufige Pyramide heraus, die von Tierstatuen umgeben und mit, nach Temyrs Aussage, urtulamydischen Runen besetzt war. Er schätzte das Alter auf etwa 1500 Jahre. Auf der zweiten Stufe der Pyramide fanden wir noch eine Inschrift, die vor großer Gefahr beim Betreten warnte, doch meine übereifrigen Gefährten Rezzanjin und Ragnos hatten bereits einen versteckten Mechanismus entdeckt und durch eindrücken der zweiten Ebene tat sich eine Öffnung in die Tiefen des Grabes auf. Es stellte sich heraus, dass man die Warnungsinschriften wohl doch hätte ernst nehmen sollen, denn schon nach ein paar Schritt in den unterirdischen Gang blieb mir die Luft weg und nur geradeso schfften wir es alle wieder aus der Pyramide heraus\dots

Es stellte sich heraus, dass man die Warnungsinschriften wohl doch hätte ernst nehmen sollen, denn schon nach ein paar Schritt in den unterirdischen Gang blieb mir die Luft weg und nur geradeso schafften wir es alle wieder aus der Pyramide heraus\dots

Ein anderer Eingang der Pyramide stellte sich als vielversprechender heraus. Ohne weitere Probleme drangen wir tief in die Eingeweide der Pyramide vor und entdeckten die längst vergessene Grabkammer eines lange verstorbenen Fürsten Goriens. Zuerst dachten wir, das Grab sei längst geplündert worden, denn nur noch einige zerschlagene Steinkrüge deuteten auf die einst prächtigen Grabeinlagen hin. Doch schnell merkten wir, dass dies nur ein Trick war, um Eindringlinge vom Auffinden der richtigen Grabkammer abzuhalten, denn in den Sarkopharg eingebettet war ein Mechanismus, um eine weitere Treppe in die Tiefe zu öffnen.

Und tatsächlich hatten wir Glück! Die List des Sultans hattte tatsächlich die Plünderer Goriens davon abgehalten, seine Grabkammer zu entweihen und wir stießen auf ein perfekt präserviertes Grabmahl. Wir konnten auch einige Komponenten des Bannes dort finden, denn wie wir gehofft hatten, handelte es sich bei diesem Fürsten um den Sultan Shamscherib, der lange Zeit mit der Instandhaltung des Bannes beauftragt war.

Doch Rezzanjins Gier trieb uns fast in den Ruin. Er konnte seine Finger nicht von den prächtigen güldenen Grabeinlagen lassen und als er sie berührte, rumpelte es tief in der Kammer. Tiefe Risse brachen in der Decke auf und die Grabkammer begann einzustürzen. Während wir alle Hals über Kopf aus der Pyramide flohen, kam Firnen die überaus intelligente Idee, sich versteinert vor den herunterstürzenden Trümmern zu retten. Doch dies führte nur dazu, dass die Trümmer seinen starren Leib einquetschten.

Draußen hielten wir Rat und Temyr blieb nichts andered übrig, als das mächtige Elementarwesen zu rufen, dessen wahrer Name ihm aus Maraskan noch bekannt war. Das Wesen lies sich überzeugen, Firnen aus den Trümmern zu bergen, doch dieser war von den Steinen schwer verletzt und kaum noch am Leben.

Verzweifelt, denn auch meine karmalen Fähigkeiten waren langsam erschöpft, suchten wir nach einer Stadt oder wenigstens einer Karavanserei, in der wir uns erholen könnten. Und, wie es in tulamidischen Märchen manchmal erzählt wird, fanden wir auf einmal in der Mitte der kargen Steppe Goriens die sagenumwobene Stadt Al'Ahabad. Wie ein Märchen erschienen uns die alabasternen Arkaden und grünen Gärten, die aus dem rostroten Sand in die Höhe ragten. Dort fanden wir eine Unterkunft und Firnen wurde von den Heilern des Sultans Hasrabal gesund gepflegt.

Wir wurden zu einer Audienz von eben jenem Herrscher geladen und ich muss sagen, der Herrscher Goriens verdient die Bewunderung, die man ihm zukommen lässt, auch wenn er ein Ungläubiger ist. Von stattlicher Statur ist er, aber seine wahre Macht liegt in seiner diplomatischen Art und in seiner Zaubermacht begründet. Er bot uns seine Hilfe an, aber im gegenzug sollten wir ein Buch für ihn holen. Das einzige Problem dabei war, dass dies in der Akademie von Rashdul zu finden sei, die seit der Übernahe durch Belizeth Dschelefsunni verschlossen war\dots

\subsection{Rashdul nach Temyr ibn Sahid}

Nachdem wir uns also solcherart der Hilfe des Sultans versichert hatten, verbrachten wir noch eine kurze Zeit im Schoße der fürstlichen Hofanlagen, namentlich, um die Gastfreundschaft und Wunder dieses Ortes aufzunehmen, eigentlich, um Firnens Wunden auszukurieren\dots

So brachen wir denn am nächsten Tag gegen Rashdul auf, um unseren Teil des Geschäfts ein- und ``Das Große Elementarium'' in unseren Besitz zu bringen. Stippen indes, der als Sprössling des Khunchomer Herrschaftszweiges Gefallen an Hasrabals Hofe zu finden hatte, blieb in sicherer Verwahrung des Magierpotentaten. Später sollten wir erfahren, dass die Annehmlichkeiten der hasrabalschen Fürsorge sich durchaus auch auf dessen Tochter erstreckten, und alsbald wurde die Vermählung zweier Dynastien aus Khunchom und Gorien bekannt gegeben.

Wie dem auch sei, führte unser Schicksal zunächst weiter den verschlungenen Mhanadi herauf, von Yostra und Palatechk über Kansul und Sertar und weitere Hundert-Seelen-Nester, deren Name auf keiner Karte verzeichnet ist; am Ende des staubgedeckten Weges jedoch lag Rashdul, ``Deren Alter nicht zu ermessen ist''. Vor dem Gang zur Akademie stand jedoch die Suche nach dem nächsten Mondstein an, der ja für die Forschungen der Khunchomer von so großer Bedeutung war. Es zeigte sich jedoch schnell, dass uns diesmal jemand bei unserer Suche zuvorgekommen war: Ein liebfelder Adliger namens Bravaldi hatte offensichtlich schon vor Wochen Erkundigungen über Ruinen außerhalb der Stadt eingezogen und ein Expeditionskorps für Ausgrabungen zusammengestellt. Und tatsächlich befand sich einige Meilen vor den Toren Rashduls die schwer gesicherte Grabungsstätte, in welcher der Horasier Stellung bezogen hatte. Bravaldi zeigte sich nicht gerade begeistert von unserem Begehren, verweigerte uns die Herausgabe des Mondsteins und entließ uns mit der Bemerkung, bei der nächsten Störung von seinen Söldlingen Gebrauch zu machen. Während dieser Unterredung wich ihm eine junge Frau keinen Augenblick von der Seite, die eindeutig nicht aus dem Horasreiche zu stammen schien. -- ich erwähne sie nicht nur deshalb, weil sie so außerordentlich fehl neben dem Liebfelder wirkte, sondern auch, weil wir es noch öfters mit ihr zu tun haben werden. So verlegten wir unsere nächsten Schritte denn auf den Schutz der Nacht, um den Mondstein auf Diebesfüßen seinem Besitzer zu entwenden. Die Sache ging so lange gut, bis wir unvorsichtigerweise beim Öffnen einer Truhe eine magische Falle auslösten, wodurch das Lager in helle Aufruhr geriet. Die Falle stammte von der bereits erwähnten jungen Frau, die ganz offensichtlich der magischen Zunft angehörte und eine Tochter Madas war. Hätte Ragnos nicht mit Pfeil und Bogen ausgeholfen, wir wären aus der Sache nicht eben lebend herausgekommen. Mit einem weiteren Mondstein im Gepäck begaben wir uns nach Rashdul zurück. Es fiel uns nicht gerade schwer, die Hallen der Akademie im Gewirr der Straßen und Gassen auszufinden - denn traditionell stellen die Schulen der magischen Zunft im Süden die größten und prachtvollsten Gebäude einer Stadt - , wohl aber, hinein zu kommen. Das kommt so: Seit der Zeit der Magierkriege sind die Pforten der Akademie von Dschinnen verschlossen und öffnen sich schlechterdings vor keiner Macht der Welt, am wenigsten aber für unerwünschte Besucher. Glücklicherweise aber hieß man uns als Kollegen und forschungswillige Besucher, welche die Wunder der Rashduler Bibliothek mit eigenen Augen zu erblicken wünschten, willkommen, und verbrachte uns mit Lastkörben ins Innere der Akademie. Dort trafen wir denn auf die Spektabilität und Tochter unseres guten Freundes Dscheleff, Beliseth, deren magische Fähigkeiten, insonderheit der Dämonenbeschwörung, schon reichlich Kunde im ganzen Land gemacht hatten.

Ich untertreibe nicht, wenn ich den oft bemühten Schauer zitiere, der mir während der nachfolgenden Unterhaltung mit der Spektabilität den Rücken hinab lief: Von betörender Schönheit, aber unnachgiebiger Härte und Kälte des Gemüts, erwies sich Beliseth als die Frau, unter deren Augen man am allerwenigsten einen Raubzug, dazu noch im Auftrage eines befeindeten Magierfürsten, durchführen wollte. Nachdem sie die Führung der Akademie aus den Händen ihres Vaters gerissen hatte, war das traditionelle Gleichgewicht der elementaren und beschwörenden Zweige in Rashdul arg aus den Fugen geraten, und die verbliebenen Magister der Elementarschule hatten sich unter Beliseths Knute fügen müssen. Solcher Geschichten eingedenk bezogen wir also Quartier in den weitläufigen und angenehm gelegenen Gästezimmern, bevor wir uns in das Gewirr der Regale und Lesealkoven der Bibliothek stürzten. Doch selbst mit der tatkräftigen Hilfe der vertrauten Archivare war das gesuchte Exemplar nicht nur nicht aufzufinden, es schien überhaupt niemals im Bestand gewesen zu sein. Weitere Nachforschungen brachten uns dann jedoch auf den Gedanken, dass Beliseth möglicherweise das Buch konfisziert und in ihren Privatgemächern unter Verschluss gehalten hatte. So fassten wir den Plan, unserem Raubzug auch einen echten Einbruch hinzuzufügen\dots

Dazu bedurften wir der Hilfe eines Bekannten von Dscheleff, welchem wir uns zu erkennen gaben: Im Schutze eines Wettstreits, welchen die Elemenmtarmagier vom Zaun zu brechen gedachten, wollten wir den Streich wagen. Und so geaschah es, dass wir am nächsten Tag im Schutze eines gewaltigen Spektakels erwartungsvoll durch die prachtvollen Gänge des Hauptgebäudes schlichen, bis wir schließlich die Gemächer Beliseths erreicht hatten. In Windeseile drangen wir in den nun leeren Raum ein, und Firnen entdeckte vermittels eines Odem ein magisch verschlossenes Geheimfach. Mit einem wohlgewirkten Foramen wurden wir des Inhaltes gewahr - des gesuchten Buches - , verschlossen den Mechanismus eilig wieder und flohen das Gemach noch bevor Beliseth von ihrer Strafaktion zurückgekehrt war. Da wir den Folio schnell aus den Mauern der Akademie bringen mussten, überwanden wir die Mauern mit Magie und flohen der Aufregung und der Rache Beliseths, ohne zurück zu blicken.
Hasrabal zeigte sich sehr erfreut über unsere Anstrengungen und schickte sich an, seinen Teil der Abmachung zu erfüllen.

\subsection{In Merwhed nach Boronos te Partholon}

\paragraph{28. Efferd 1019 BF}
Am Morgen nach dieser götterverlassenen Nacht machten wir uns auf den Weg in das ungläubige Mherwed, was ich ja nicht mehr betreten durfte.
In der Sonne des scheidenden Tages erhob sich vor uns endlich die gewaltige Brücke Bastrabuns. Ich verabschiedete mich von meinen Weggefährten und schlug mein Lager in Sichtweite der Brücke auf. Währenddessen suchten die Anderen die Mherweder Zauberschule in der alten Sommerresidenz des Kalifen auf. Sie wurden beim Akademieleiter vorstellig und rangen dem Rastullahanbeter eine Erlaubnis zum Besuch der hiesigen Bibliothek ab. Sie quartierten sich in einen der Gasthäuser in der Mitte von Mherwed ein.

\paragraph{29. Efferd 1019 BF}
Temyr und Firnen versuchten einen Dispens zu erwirken, der ihnen erlaubte in der Stadt zu zaubern. Der Akademieleiter erlangte dafür aber die Konvertierung zum Rastullahglaube, was beide -- den Götter sei Dank -- ablehnten. In der Bibliothek fand man heraus, dass eine geheime Sekte in Anchopal, Die Hüter des Netzes des Bastrabun, gibt. (Vielleicht finden wir mehr Informationen in Keshal Rohal.) Außerdem gibt es anscheinend einen Statuette aus Mondstein, die der Schlüssel zum Bannnetz sei und damit eine zentrale Komponente des Banns. Deren Aufenhaltort könnte Yash-Hualay, Elem, Nebachot, Yol-Fasar, Zamorrah oder Yasra sein.

Zum Mittagessen speisten sie mit den Magistern und begangen ein paar Unhöflichkeiten, für die sich sich später entschuldigten.
Magister Omaran sprach sie nach dem Essen an, denn er habe Informationen, dass auf dem alten Boronsanger vor der Stadt entweder Bastrabun oder einer seiner engsten Vertrauten begraben sei.

Am Abend holten mich die Anderen ab und wir begaben uns zum Treffpunkt auf dem Anger. Kurz nach dem Eintreffen Omarans tauchten Ghule auf und im entstehenden Chaos flüchtete Omaran. Nachdem Kampf, bei dem Rezzanjin, Temyr und Firnen veletzt wurden, segnte ich den Friedhof neu ein.

\paragraph{30. Efferd 1019 BF}
Firnen und Temyr waren von einer seltsamen Krankheit ergriffen, die sich als Schlafkrankheit, welche Ghule durch ihr Gift übertragen, herausstellte.

\paragraph{31. Efferd 1019 BF}
Nachdem Firnen und Temyr wieder auf dem Beinen waren, plante man, wie man Omaran einen Besuch abstattet und den Mondstein von der Brücke raubt. Unter Einwirkung eines Bannbaladin erzählte Omaran den Anderen, dass er mit der Asfaloth-Domäne gearbeitet hatte. Eine Person, die sich selbst Meister nannte(sehr wahrscheinlcih Abu Terfas von Al'Churam, dieser soll einen verwunschenen Palast im Khoramgebirge haben und ein Asfaloth-Paktierer sein), hatte ihn kontaktiert und Informationen zu Asfaloth gegeben, als Gegenleistung sollte er Bücher aus der Bibliothek entwenden und der Hexe Acharia Zugang zur Bibliothek verschaffen.

Den Stein entwendeten Temyr und Firnen mit Hilfe der Zauber Ignorantia Ungesehn und Motoricus Geisteshand. Danach brachen wir nach Al'Ahabad auf, um Sultan Hasrabal von Gorien das Gantorana zu bringen.

\paragraph{3. Travia 1019 BF}
Nach zwei beschwerlichen Tagen durch Gorien erreichten wir die Oase um Al'Ahabad. In der Stadt gaben wir das Gantorana dem Sultan. Er sagte, dass jeder an den Dschinn nur eine, eine einzige Farge stellen kann und dass der Dschinn alles Gesprochene als Frage auffasst. Wir sollten nach einer Stunde wiederkommen.

Wir wurden in Hasrabals Beschwörungszimmer geführt, auf dem Boden war ein gewaltiges Hexagramm gezeichnet. Um das Hexagramm waren sieben Pentagramme gezeichnet und Hasrabal wies uns an, in jeweils ein Pentagramm zu gehen und es nicht zu verlassen. Hasrabal fing an uralte Zauberformel zu intonieren und langsam entzündeten sich der Holzkreis im Hexagramm. Es manifestierte sich eine zwei Schritt hohe Flammensäule, aus der zwei kräftige Arme und ein Kopf entstanden.

Wir stellen unsere Fragen an den Dschinn:
\begin{enumerate}
    \item Wo befindet sich das magische Artefakt in Form eines silbernen Handschuhes, das Bastrabun zu seiner Lebenszeit erstellte und bei sich trug, heute?
    \item Wo befinde sich die Foki, die Bastrabun benutzte um seine magische Bannmauer zu erstellen, heute?
    \item Wo befindet sich die Statuette, die der Schlüssel zur Veränderung von Bastrabuns Bann ist, heute?
    \item Wo genau sind die sterblichen Überreste des Bastrabun begraben worden?
    \item Welche Bewegungen, Sprüche und Zeichnungen hat Bastrabun genutzt um das Ritual zur beschwörung seiner Bannmauer durchzuführen?
    \item Was ist die Verbindung zwischen Assarabad und Borbarad?
\end{enumerate}

Dies waren seine Antworten:
\begin{enumerate}
    \item Er befindet sich am Arm des einhändigen Frevlers.
    \item Die Foki werdet ihr in Al'Churam im Khoramgebirge, in Samra, Khunchom, Anchopal und einen Zeltlager vor Mherwed finden.
    \item Ich brachte sie einst einem in den Palast der drei Herrscher, sie liegt am Eingang zu einer Welt, die nicht sein kann , die nicht sein darf.
    \item Sein Körper liegt in der unheiligsten Stadt der Echsen, dessen Geheimnisse er erforscht, ewig vor Satinavs Griff geschützt.
    \item Viele Hundert, die simultan an verschieden Ort in den Tulamidenlanden ausgeführt werden müssen. Die Anweisungen habt ihr gefunden und wieder verloren, ihr werdet aber zwei treffen, die die Anweisungen gefunden haben.
    \item Die Stadt der drei Herrscher gehörte einem, gehörte schon immer einem, und der ist ER allein.
\end{enumerate}


Am folgenden Tag brachen wir Richtung Zamorrah auf.

\subsection{Zamorrah nach Rezzanjin al'Ahjan}


\paragraph{Tagebucheintrag Ende Effert/Anfang Travia}
Nachdem wir die letzten 4 Tage durch die staubige Landschaft Goriens gereist sind, es geschafft hatten unbemerkt an Merwhed vorbei zu kommen, wo man unseren Diebstahl zweifelsfrei erntdeckt hatte und nach uns suchte, herausgefunden hatten, dass uns Bravaldi um mindestens einen Tag voraus war, uns bei einem Haiamundi in Manessipur eine von den normalen Geschichten über den Untergang Zamoras abweichende Variation angehört hatten, kamen wir heute, nach einem weiteren Tag anstrengender Reise, in dem Ort Borbra an, welches interessanterweise sogar einen Tempel von Schwester Tsa beherbergte. Gerade als wir in Borbra am Hauptplatz ankamen und voller Verwunderung die auf dem Hauptplatz stehende Eiche, die so gar nicht in das sonst recht tulamidische Bild des Dorfes hineinpasste, sahen wir den Geweihten von Schwester Tsa, einen Mann mit roten Haaren vor dem Tempel kleinen Kindern Geschichten erzählen. Wir suchten uns eine Herberge und gingen zu den örtlichen Händlern um unsere Vorräte aufzustocken. Bei diesem kleinen Stadtrundgang entdeckten wir zwei weitere Schreine der zwölf Geschwister. Einen von Bruder Praios und einen von Schwester Peraine, den Toran am Abend auch gleich zu einem Gebet ansteuerte. Firnen dagegen übernahm die Aufgabe herauszufinden, was die örtlichen Geschichtenerzähler über die Historie des kleinen Örtchens selbst zu berichten wussten. Wie er uns eben erzählte wurde er an den Geweihten von Schwester Tsa weiter gewiesen. Dieser erzählte ihm wohl unter anderem die Geschichte über die Entstehung der Eiche, welche stark mit der Lokalbekanntheit Tarlisin von Borbra zusammenhing. Aus dem zerbrochenen Magierstab von diesem sei der Eichenbaum entstanden, welcher heute als Heiligtum von Schwester Tsa gesehen wird. Was er erzählte war jedoch alles in allem recht uninteressant, bis auf die Tatsache, dass Bravaldi wohl vor einigen Tagen hier vorbeigekommen war.

\paragraph{Tagebucheintrag einen Tag später}
Nach einem halben Tag später kamen wir im Dorf Samra an, welches von Zamora aus auf der gegenüberliegenden Seite des Mahandi liegt. Wir mussten feststellen, dass Bravaldi sein Lager schon auf der Seite Zamoras aufgeschlagen hatte. Doch wir wussten, dass sich eine weiterer Mondstein wohl in Samra befinden würde. Doch zuerst fragten wir bei einem Wirt nach einer Möglichkeit zum Übersetzten nach Zamora. Dieser warnte uns zugleich diese verfluchte Stadt zu betreten. Eine Warnung, die wir weder zum ersten noch zum letzten mal gehört hatten. Danach wollten wir den Tempel von Bruder Effert aufsuchen, zu dem man uns gewiesen hatte, weil dort ein großer Mondstein sein sollte. Wir fanden nur einen Tempel von Schwester Peraine, den wir zugleich betraten. Dort entdeckten wir schnell einen Mondstein, der genau die richtige Größe hatte und in den Händen einer Statue lag. Firnen bestätigte, dass es sich um einen magischen Mondstein handele. Nach etwas längeren und zähen Verhandlungen mit den herbeigeeilten Geweihten der Peraine durften wir den Mondstein im Austausch gegen 25 Maravedi unser Eigen nennen. Am Abend besuchten wir noch einen einen blinden Haiamund, der zwar unsere Frage nach der Stadt der drei Herrscher nicht beantworten konnte, uns jedoch viel über den angeblich unter Zamora hausenden Golem und über die Stadt selbst mitteilen konnte. Auch er sprach eine Warnung aus Zamora nur bei Tag zu betreten, wenn wir es schon betreten wollten. So konnten wir uns beruhigt ins Bett legen mit dem Bewusstsein einen weiteren Mondstein im Gepäck zu haben.

\paragraph{Tagebucheintrag einem weiteren Tag später}
Am Morgen des nächsten Tagesbrachen wir dann auf Richtung Flusshafen und fanden bald einen Fischer, der bereit war uns für relativ viel Geld überzusetzen. Wir ließen uns weiter flussabwärts bringen, wo wir gleich hinter einem Hügel, versteckt vor den Blicken Bravaldis und seiner Söldner, ein kleines Lager aufbauten. Dann wagten wir uns in die Stadt.

Zu Beginn suchten wir relativ sinnlos in der Gegend herum und konnten uns durch intelligentes Verstecken und ein paar Zauber auch der Begegnung mit Bravaldi persönlich durch die Gefangennahme durch einige seiner Söldner entziehen. Dann beschlossen wir zu schauen, was Bravaldi vorhat und schlichen uns zu seinem Lager. Der Versuch daran vorbei zu schleichen scheiterte allerdings, da Temyr und Toran entdeckt wurden. Der Rest ging ziemlich schnell, Temyr zauberte kurz etwas, verschwand dann auf einmal und wir fanden uns auf der Flucht vor den heranstürmenden Söldnern, die wohl leicht wütend waren, da wir anscheinend ein paar ihrer Kumpane getötet hatten. Nun hieß es um unser Leben zu laufen und das taten wir. Toran bog als Erster vor den schnell aufholenden Söldnern in eine Seitengasse ab. Zwei der sechs Söldner, die uns verfolgten hechteten hinter ihm her. Zwei weitere taten dasselbe mit Boronos, als dieser die größere Hauptstraße verließ. Ragnos und ich, die wir noch am schnellsten rannten, legten den Söldnern einen Hinterhalt. Als sie nichtsahnend nach uns suchten hatte nur der, den Ragnos nicht getroffen hatte eine Chance zu kämpfen, doch er war zu langsam um meiner Klinge zu entkommen. So machten wir uns auf die Suche nach Boronos, den wir im Kampf gegen einen der Söldner in großer Bedrängnis vorfanden. Auch diese Söldners entledigten wir uns mit einem Pfeil recht schnell. Ich konnte das komische Gefühl, dem Söldner jetzt sein Herz herauszureißen zu wollen schnell abschütteln, auch wenn es mich ein wenig überraschte so zudenken. Nachdem Boronos seine Wunden geheilt hatte, wollten wir weiter nach Toran suchen, als eine kleine Katze plötzlich vor uns entlang spazierte. Als Boronos diese, er hielt sie für einen ruhelosen Geist, schon entschwören wollte, erklang plötzlich Torans Stimme in meinem Kopf und ich erinnerte mich daran, das diese Katze sein Zeichen war. Er klärte uns über seinen Aufenthaltsort auf und wir fanden unter Führung der Katze schnell zu dem bewusstlosen Toran, der den Wächter, der etwas entfernt von dem Gebäude stand, in seiner Katzenform schon verjagt hatte. Ragnos bezog einen guten Aussichtspunkt und legte sich auf die Lauer. Wir kamen dann zu Toran und die Katze verschwand und Toran erwachte. Wir tüftelten einen Plan aus, wie die Wächter zu bezwingen seinen und baten Toran in Katzenform nach der Hexe zu suchen. Er fand sie nicht, konnte aber die Wächter verjagen. Ich schlich zu Ragnos um die Lage zu überblicken als plötzlich die Hexe inmitten des Loches auftauchte, das die Arbeiter gebuddelt hatten und das große Tor öffnete. Die Arbeiter und Söldner rannte bei diesem Anblick zurück zu ihrem Lager, wir jedoch folgten der Hexe, wissend, das sie zu Firnen oder Temyr aufschließen würde, die Toran mit einer seltsamen neuen Gabe seines Zeichens angeblich dort unten aufgespürt hatte. Wir versuchten der Hexe zu folgen, doch wir verloren sie in den verzweigten Gängen schnell. An einer zweiten Tür, die offensichtlich vor nicht allzu langer Zeit geöffnet wurde hielten wir an, da wir erkannten, dass auch wir verfolgt wurden. Als plötzlich Bravaldi auftauchte. Er erklärte, dass die Hexe ihn betrogen hatte und wir verfolgten nach geschlossenen Frieden der Hexe weiter. Dann kamen wir endlich in einem Raum mit einer komischen riesigen Statue, die am Baden lag. Wir entdeckten sofort Temyr und Firnen, die uns nach einer kurzen Begrüßung erzählten, dass die Hexe in einen Gang geflüchtet war, der von einem Dämon bewacht wurde und wir doch in die Schutzkuppel um Firnen kommen sollten. Nach einen Zauber war der Dämon verschwunden und auch Bravaldi, der sich erst vor dem Dämon gefürchtet hatte, folgte uns mit seinen Schergen. Doch eine weiteres Hindernis versperrte uns den Weg in den Raum, in dem die Hexe sich befand. Eine unsichtbare Schutzkuppel war aufgespannt worden, die wir mit unseren Waffen nicht zu durchdringen vermochten. In dem Raum selber bot sich uns ein schrecklicher Anblick. Die Hexe hatte eine kleine Statue in der Hand die sich langsam verformte. Plötzlich stand Firnen mit Temyr in der Mitte, des Raumes, der Wohl die Grabkammer Bastrabuns darstellte. Doch Firnen schien nicht er selbst zu sein. Als die Hexe auf ihn einredete, schien er das Angebot von Macht, das sie ihm machte annehmen zu wollen. Auch sein Akzent hatte sich verändert, er benahm sich auf ganz anders und wollte wohl zur Hexe um sich mit ihr weg zuteleportieren. Doch Toran, vorher leise im Gebet versunken stürmte durch die Schutzkuppel, zerstörte sie und warf sich auf Firnen, der dem Gewicht des Geweihten nicht widerstehen konnte und umfiel. Die Hexe war auf einmal weg und Toran, dem Temyr erzählt hatte, dass Sulhamid, der Geist, der in seinem Rubin-Auge saß, von ihm Besitz ergriffen hatte, vertrieb diesen aus Firnens Geist. Bravaldi war sehr erbost über das Verschwinden der Hexe, hatte sie ihn doch betrogen, als sie ohne ihn in die Katakomben von Zamora ging. Immer noch ein wenig erstaunt von der schnellen Wendung der Ereignisse erzählten wir Bravaldi von unserem Auftrag\dots

% \subsection{Der Angriff auf Borbra nach Rezzanjn}
% 
% \dots Wir gingen erstmal wieder an die Erdoberfläche und gesellten uns in sein Lager. Dort besprachen wir die Ereignisse und klärten Bravaldi über die Umstände unseres Handelns auf und fanden geraus, dass er erstaunlich viel wusste. So konnte er uns mitteilen, dass die Hexe einen Begleiter hat, einen Krieger in schwarzer Rüstung. Es war gut zu wissen, auf was wir uns bei unserem nächsten Treffen mit ihr vorzubereiten hatten. Letztendlich beschlossen wir, dass Bravaldi seine gesammelten Informationen an die Kunchomer Magierakademie weitergeben sollte und wir erstmal nach Anchopal weiterreisen würden. Wir gingen in Richtung Süden, den Fluss entlang, um zu unserem Lager zu gelangen, doch wurden wir noch vom Söldnerführer aufgehalten, der Blutgeld für seine dank uns gefallenen Kameraden forderte. Wir zahlten ihm was er wollte und gingen dann weiter in Richtung unseres wohl unnötig aufgebauten Lagers und packten die Sachen zusammen und mussten nicht lange auf den Fischer, der uns hingebracht hatte warten. Er brachte uns auch wieder zurück. Er fragte, was wir gemacht hätten, was mich sehr verwunderte, da er eigentlich nichts besonderes von uns mitbekommen haben sollte, immerhin liegt Samra ja auf der anderen Flussseite. Doch als wir auf der Flussmitte waren, dämmerte es mir. Zehn Reiter der örtlichen Reiterei warteten auf uns, um uns in Gewahrsam zu nehmen, da wir, bzw. die Hexe mit dem Öffnen der Tore wohl einige Geister aufgeschreckt hatten, die wiederum die Dorfbewohner aufgeschreckt hatten. Wir konnten letztendlich noch mit dem Hauptmann darüber verhandeln, dass er uns freiließ und so zogen wir Richtung Anchopal. Doch freilich kamen wir nicht innerhalb des Tages dort an und so machten wir uns daren eine Lagerstelle zu finden. Diese war schnell entdeckt und dazu noch zwei weitere Gestalten, Steinmetzgesellen, mit denen wir zu Abend speißten. Wie Temyr feststellte unterhielten sie sich über einen Fund, sogenannte Käfereier, den sie erst kürzlich gemacht hatten. Es waren Relikte vom großen Schwarm, in Sandstein eingeschlossen. Gerade als einer der beiden Gesellen in das Zelt ging, um etwas zu holen, schien das Käferei in der Hand des anderen zu bröseln. Plötzlich brach ein Käfer aus dem Ei heraus und griff den Gesellen an. In kürzester Zeit attakierte der Käfer dessen Gesicht und riss ihm die Haut von diesem. Firnen reagierte schnell und tötete den Käfer auf Magische Weise, bevor er den Gesellen zerfleischt hatte. Doch bevor wir uns über diese wahrhaftig grausame wie merkwürdige Schauspiel wundern konnten, stürtzte eine zweiter Geselle, über den gesamten Leib mit Käfern gespickt aus dem Zelt. Er starb innerhalb kurzer Zeit und hätten wir Firnin nicht dabeigehabt, so wäre es auch unser Ende gewesen, da man solche Massen an Käfern unmöglich mit dem Schwert bekämpfen konnte. Er tötete auch diese Käfer mit einem Zauber, woraufhin sie zu Staub zerfielen. Der zweite Steinmetz rannte weg, doch wir hatten ihn wieder eingesammelt und werden ihn nach Anchopal begleiten. Im Moment grübeln die Magier noch, woran es gelegen sein könnte, dass die Käfer wieder lebendig wurden, doch eine Vermutung hatten sie: Es konnte nur Abu Terfas dahinter stecken. Jetzt wo er den Schlüssel zum Schwarm hatte, war er der einzige, der so etwas heraufbeschwören könnte.
% 
% \paragraph{Tagebucheintrag Anfang Travia in Borbra}
% Kurz ist die Zeit um die zurückliegenden Ereignissen niederzuschreiben. Nachdem wir in Anchopal angekommen waren teilten wir uns auf: Die Magier gingen zur Ordensburg der grauen Stäbe wir anderen zum Perainetempel. Dort erfuhren wir, dass wir nicht die einzigen waren, die von Käfern angegriffen worden waren. Wohl eine flächendeckende Auswirkung des Rituals. Jedoch hatten wir mit der Suche nach Mondsteinen und dem Aufenthaltsort von Abu Terfas, den wir wohl finden müssen, um das Grauen zu beenden, keinen Erfolg. Die Magier jedoch schienen zumindest einen Teilerfolg zu haben. Der Leiter der grauen Stäbe, Tarlisin von Borbra, würde sich um seine Verbindungen bemühen, um den Aufenthaltsort von Abu Terfas herauszufinden. Auch würde er die Giftkammern der Ordensburg nach dem Mondstein durchsuchen lassen. Am nächsten Morgen sollten wir uns mit ihm im örtlichen Badehaus treffen. Dies taten wir auch. Im Dampfbad erwartete er uns. Gerade als wir die Angelegenheiten besprechen wollten, schien er plötzlich einen Hustenanfall zu haben. Und auch ich bemerkte, dass es schwerer wurde zu antworten. Doch bevor irgendwer etwas tun konnte stürmten zwei in schwarz gekleidete Gestalten den Raum. Ich vollkommen nackt, schaffte es gerade so den einen in Schach zu halten, nachdem der Dolch, den er in der Hand gehalten hatte plötzlich, wie von Geisterhand bewegt, wegflog. Doch nachdem ich ihn mit den effektiven Hruruzattritten, die mir mein Meister einst gezeit hatte beseitigt hatte, erkannte ich das wahre Ausmaß des Anschlags. Boronos hatte eine böse Wunde am Hals davongetragen und lag keuchend am Boden. Auch der Leiter der Grauen Stäbe hatte eine schlimme Stichwunde davongetragen. Doch Toran und die Magier schienen die Lage unter Kontrolle zu haben. Während sie die Sterbenden versteinerten, um die Wirkung des Giftes aufzuhalten, hörte ich in dem eigentlich abgeschossenen Raum, in dem unsere Sachen lagerten, Geräusche. Ich lauschte erst kurz an der Tür und öffnete diese dann schwungvoll, nur um einen verdutzten Maraskaner in schwarzer Rüstung zu sehen, der bei meinen Anblick so erschreckte, dass er schnell das weite suchte. Als die Verletzten wieder aufwachten, waren sie anscheinend gesund und Tarlisin erklärte uns, dass ein Stamm von Ferkinas im Khoramgebirge wüsste, wo sich Abu Terfas Palast befinden würde. Auch der Mondstein sei gefunden worden und werde auf Wunsch nach Kunchom geschickt. Der Attentäter, den ich unschädlich gemacht hatte erzählte uns, dass der Maraskaner sie angeheuert hatte und eine stattliche Menge an Geld geboten hette für unseren Tod. Auch mit dem Gift hätte er sie ausgestattet. Bestürtzt von den Ereignissen, auf uns war ein Giftanschlag verübt worden, vermutlich von dem Begleiter der Hexe, zogen wir uns an und wollten zurück zur Burg der grauen Stäbe, da liefen wir direkt in ein Menschenmenge hinein. Der dem Sie lauschten erzählte von Chimären, die sich in Massen um das Koramgebirge befinden sollen. Wir kämpften uns durch die Menschenmenge und sprachen mit ihm persönlich. Er erzählte uns, dass er vor zwei Tagen aus dem Tal in dem die Ferkina wohnen, die wir besuchen wollen, die Chimären kamen. Er habe gerade noch fliehen können. Wir gingen weiter zur Ordensburg und beschlossen unsere gesamten Mondsteine von dort aus nach Kunchom zu schicken. Dann begann am Mittag der Eilritt Richtung Samra, und dann, nachdem wir erfahren hatten, dass die Chimären Borbra angegriffen hatten, es jedoch noch nicht gefallen war, ritten wir weiter nach Borbra. Dort fanden wir das Dorf von Verteidigungsringen aus Holzgerümpel umrandet. Wir wurden reingelassen und ließen unsere vom Eilritt erschöpften Pferde außerhalb der Palisaden. Der Dorfrat tagte unter der Eiche in der Dorfmitte und wir setzten uns dazu und berieten sie bei den Verteidigungsmaßnahmen. Zusätzlich boten wir an ihnen zu helfen. Die Chimären würden wohl am Abend angreifen und das Dorf hatte wenige Kämpfer, die es verteidigen können. Wir halfen nach der Besprechung dabei die Frauen und Kinder sicher in den halbwegs pasabel gesicherten Häusern unterzubringen. Wir beschlossen besonders den Dorfplatz zu schützen. Kaum hatte die Dämmerung begonnen, kamen die Chimären, es waren zum Teil Wesen, den Khoramsbestien ähnlich, zum Teil auch Bären mit Stierköpfen. Der Anführer war ein Mantikor von großen Ausmaßen, den wir, so hatten wir es beschlossen, zu vernichten versuchen würden. Boronos weihte den Boden um die Eiche. Der Angriff kam schnell und von allem Seiten, war aber auch wieder schnell verebt, nachdem sie bemerkt hatten, dass die Eiche durch den Schutzkreis geschützt wurde. Bronos meinte der Schutzkreis würde noch etwa anderthalb Stunden halten. Diese Zeit nutzten wir. Die Chimären, die noch in der Stadt waren un den Rückzug noch nicht mitbekommen haten, töteten wir, die Dorfbewohner wiesen wir an, an verschiedenen Stellen Hinterhalte zu Legen. Die Magier nutzten die Zeit um drei mächtige Dschinne zu beschwören und wir bereiteten uns vor die jetzt hauptsächlich von Süden kommende Chirmärenarmee aufzuhalten. Die Dschine sollten sich, so die Anweisung, auf den Mantikor konzentrieren. Ich postierte Ragnos auf den Dach und mich daneben, um ihn zu schützten. Kaum waren wir mit unseren Vorbereitungen fertig, da stürmten die Chimären schon auf den Platz. Die, die sich unten auf dem Platz befanden, bekämpften sie, doch bald wurden es zu viele und ich sprang vom Dach, um ihnen zu helfen. Anscheinend hielten die Dschinne viele Chimären in Kämpfen rund um das Dorf in ausreichender Entfernung zum Dorfplatz, sodass der Ansturm leichter als erwartet ausfiel. Dennoch waren wir mit den vorhandenen Chimären beschäftigt genug beschäftigt, als plötzlich ein Knäuel aus Dschinn und Mantikor auf dem Dorfplatz landete. Der Mantikor zerschmetterte den Dschinn und griff den Tsageweihten an. Sofort wussten alle, dass wir den Mantikor jetzt angreifen mussten. Ich versetzte mich ins N`churr und machte mit. Schnell versetzten wir dem Mantikor teils schwere Wunden, doch er schien sich fast mühelos im Kampf gegen mehrere Leute unsererseits verteidigen zu können. Mitlerweile gingen die anderen Chimären dazu über uns anzugreifen, und damit ihren General oder Anführer zu verteidigen. Doch der Mantikor bekam mitlerweile Probleme und taumelte schwer von Boronos getroffen Richtung Eiche. In einem letzten verzweifelten Aufbäumen rammte er seinen Stachel in die Eiche\dots
% 

\paragraph{Teil 7 nach Toran Ostik}

\dots Wieder zurück an der frischen Luft konnten wir uns näher mit dem Horasier, der sich als ein gewisser Antonio Bravaldi herausstellte, befassen. Länger als erwartet tauschten wir Informationen aus und konnten uns letztendlich darauf einigen, Bravaldi alleine nach Kunchom zu senden, damit er die Akademie informierte, während wir uns der Hexe annehmen würden. Dieser recht frische Plan stockte schon in seinen Kinderschuhen, als wir beim verlassen des Lagers einen uns nur zu sehr bekannten Söldnerhauptmann trafen. Unwillig uns näher mit einem solch abscheulichen Individuum zu befassen zahlten wir ihm sein Blutgeld und setzten unsere Reise fort. Dies sollte jedoch nicht das letzte Problem mit den militärischen Einheiten der Gegend gewesen sein. Schon an der Brücke trafen wir auf die Soldaten der Stadt, die es sich nicht nehmen lassen wollten uns festzusetzen. Nach einigen erhitzen Diskussionen zwischen Temyr und dem Kommandanten der Reiterei in für mich viel zu schnellem Tulamidya zogen sich die Truppen zurück und man nahm uns das Versprechen ab sofort die Stadt zu verlassen. Endlich konnten wir auch das Lager der Stadttruppen hinter uns lassen, nicht jedoch ohne das Ragnos ein merkwürdiger silberner Handschuh im Zelt des Kommandanten auffiel. Unfähig mehr zu diesem Thema zu tun lenkten wir unsere Schritte gen Anchopal. Nach einem ereignisarmen Nachmittag trafen wir am Abend zwei Steinmetzgesellen, die Säcke voll Sandstein bei sich trugen, in den Insekten eingeschlossen waren. Sie schienen diese verkaufen zu wollen, doch daraus wurde nichts, denn als wir uns mit ihnen zum Abendmahl niederließen brachen die totgeglaubten Fossilien krachend auseinander und wir mussten uns einem Schwarm wütender Insekten stellen. Einer der armen Gesellen wurde bis auf die Knochen von ihnen zersetzt bevor Firnin die meisten mit einer Welle des Schmerzes töten konnte.

Nach nur minimalen Ehrerbietungen für den toten Gesellen mussten wir am nächsten Morgen weiterziehen. Der zweite der beiden hatte nur minimale Verletzungen davon getragen und wir ließen ihn mit ein wenig Proviant alleine zurück.

Von hier an reisten wir einige Tage ereignislos nach Anchopal.

Der Anblick des grün/weiß schimmernden Hains der Mutter Perain ließ die Anstrengungen der Reise schnell von uns fallen, als wir Anchopal erreichten. Das Bollwerk des göttlichen Gartens gegen das Voranschreiten der Wüste, war schöner als ich es mir je vorgestellt hatte. Zudem schien wir gerade zu einem wichtigen Zeitpunkt eingetroffen zu sein, denn die Straßen waren gefüllter den je mit Pilgern und Ladenpredigern, die ihre Gebete von zweifelhafter Herkunft mit vorbildlichem Eifer vortrugen. Nachdem sich die Gruppe einquartiert hatte teilten wir uns auf. Die Magier gingen zur Burg des Ordo Defensores Lecturia und der Rest besucht den Peraintempel. Nachdem man die Magier an den Türen der Burg zum Archiv verwies machten wir uns auf den Weg zum Rand der Stadt, wo uns die Geweihten der Perain freundlichst empfingen. Beunruhigender Weise konnte uns einer der Brüder von mehr Vorfällen wie dem auf unserer Reise nach Anchopal berichten. Unterdessen hatten Temyr und Irian die Chance sich mit der Archivarin des Ordo zu unterhalten. Betreffend die ``Stadt der drei Herrscher'' verwies diese auf Borbra oder Elem. Zu Al'Churam konnte sie uns nichts sagen doch sie riet uns Abu Terafs über seine ungewöhnlichen alchemistischen Zutaten ausfindig zu machen. Über die Mondsteine wusste sie ebenfalls nichts.

Am nächsten Tag trafen wir uns mit Tarlesin von Borbra in einem tulamydischen Dampfbad. Offensichtlich wollte jemand nicht, das wir uns mit ihm unterhielten, denn schon im eigentlichen Bad wurden wir von Meuchelmördern überfallen die mit schwer vergifteten Dolchen auf uns losgingen. Zwar war Rezzanjin ihnen haushoch überlegen, doch das Gift auf den Dolchen stellte sich als so starke alchimistische Tinktur heraus, dass nur mit viel Hilfe der Göttin der Tod von Tarlesin abgewendet werden konnte. Als wäre das nicht genug gewesen traf Rezzanjin noch eine schwarz vermummte Gestalt in den Umkleideräumen, die jedoch flüchtete bevor sie irgendetwas stehlen konnte. Nach dieser Aufregung verließen wir das Bad nur um draußen einen verzweifelten Flüchtling anzutreffen, der uns von schrecklichen Chimären berichtete, die Aborea und andere Städte in der Umgebung angegriffen hatten. Sofort brachen wir gen Samara auf.

In Samara verwies der Kommandant der Wache uns nach Borbra.

Als wir die Stadt erreichten finden wir sie verbarrikadiert und nur nach einigem hin und her ließ man uns ein. Die Bedrohung durch die Chimären schien hier haut nah zu sein und wir trafen uns mit den Dorfältesten und dem Tsa-Priester um die bevorstehende Verteidigung zu planen. Bei Sonnenuntergang folgte der Angriff. Boronos, der Dorfpriester Tsadika und ich zogen Schutzkreise um die Eiche in der Mitte des Ortes was die Chimären zum Rückzug zwang. Temyr nutzte die Zeit um 3 Dschinne zu beschwören, die sich bei nächsten Angriff voller Wut auf die Dämonen stürzten\dots

\emph{Ab hier sind die Aufzeichnungen beschädigt. Aus Randnotizen und Erzählungen vermochte ich zu rekonstruieren dass die Gezeichneten auf dem Weg zum Palast der Abu Ter'Fas einem Stamm Ferkinas begegnet waren. Allerdings ist nicht mehr zu rekonstruieren was bei diesem Treffen geschah.}

\subsection{Der Palast von Abu Ter'Fas nach Firnen Wulfgrimm}

Die Chimärenhorden zogen sich nach dem Tod ihres Anführers zurück, doch mit der Zerstörung der Tsaeiche, scheint mir der angriff allgemein von Erfolg gekrönt gewesen zu sein. Noch in den frühen Morgenstunden brachen wir auf, die Wurzel des Bösen, Abu Terfas, zu finden, und aus zu reißen. Den Weg zu finden sollte nicht sehr schwer werden, immerhin hinterlässt ein ganzes Heer tonnenschwerer Chimären, gewisse Spuren in der Landschaft.
Wie sich herausstellte schienen die Chimären jedoch einen riesigen Umweg gelaufen zu sein, und wir beschlossen einen kürzen Weg quer über die engen Bergpfade zu finden.
Eine Horde Ferkinas (die wilden Bergstämme des Südens) lauerte uns auf, und schien von Rezzanjins pathetischer Rede, von wegen Befreiung vom bösen Chimärenmeister, nicht begeistert zu sein. mit göttlichem Wirken unterband Toran einen Kampf, allerdings wurden wir nun als Geister zu ihrem Schamanen geleitet, dass er uns entschwöre.
Wir erläuterten dem Schamane, mit Abu Terfas Handel treiben zu wollen, und gaben 2 der Ferkinas in Folge dessen. Tabak und maraskanische Trockenfrüchte mit. Ich teleportierte mich aus dem Lager und folgte ihnen unauffällig, bis ich auf dem glatten Stein am Rande einer Schlucht ausrutschte und fiehl. Phex war mein Hirte und Peraine meine Rettung. Ein Fellsforschprung 10 Meter tiefer stoppte meinen Sturz und Toran heilte mein Bein gebrochenens Bein. Sie hatten sich im Schutz der Nacht auss dem LAger der Ferkinas fortgeschichen, um mir zu folgen, und fanden mich schließlich. Wir gingen den ``Pfad'' den auch die beiden Ferkinas genutzt hatten weiter entlang, und gelangten schlussendlich in ein Tal mit einer alten tulamidischen Festung. Nach dem wir einer wilden Stierhorde die dort zu grasen gedachte entgangen waren schwebten wir mit Hilfe des Nihilogravo über die Mauer. Ein sonderbarer Gärtner begegnete uns im langem ``Garten'', der den Platz zwischen Mauer und Hauptgebäudekomplex füllte. Wir erzählten uns im Gebirge verirrt zu haben, und vor der Stierhorde über die Mauer geflüchtet zu sein. Ich denke nicht, dass er uns wirklich glauben schenkte, doch er brachte uns in ein Zimmer im Gästehaus. Die Geweihten bat er jedoch vor dem Haus zu warten. Boronos und Toran folgten uns nach, nachdem sie im Garten auf die hexische Gehilfin von Abu Terfas gestoßen und nur knapp ihrem Anfriff entgangen waren. Weiter hinten im Gästehaus führte ein Gang in ein tieferes Kellergewölbe, und von da aus in ein unterirdisches Kammernsystem, das als Arbeitsstätte zu dienen scheint. Beunruhigend wirkten vor allem die übernatürlich ghohen und breiten Durchgäng zwischen den einzelnen Raumabschnitten. Hinter einer großen Ritualkammer, fanden wir ein Verlies mit den sonderbarsten Versuchsergebnissen. Besonderes Mitleid erregte ein kleiner verwirrter Junge, der an Stelle von Armen nun große Fledermausflügel besaß. Einziges noch zurechnungsfähiges Individuum dieses Gefängnisses stellte ein junger, wie sich herausstellt Phexensjünger dar, den Temir sogleich als den jungen Da Merinaldie wiedererkannte, der vor einigen Jahren den Überfall am Rande der Gor überlebte. (sieh Staub und Sterne) Er hatte das große Chimärenbuch aus der Akademie zu Rashdul entwendet und an Abu Terfas für 666 Dukaten verkauft. Als die Akademie eine Blohnung von 777 Dukaten auf die Wiederbeschaffung aussetzte, versuchte er das Buch zurück zu stehlen. Abu Terfas fand ihn, verkrüppelte ih nachhaltig die Hände, und lies ihn in dieser Zelle versauern. Wir gaben ihm zu Essen und zu Trinken, und durhsuchten weiter die Kaverne. Abu Terfas fanden wir zusammen mit der Hexe und einem Haufen Gepäck in einem Beschwörungsraum. Abu Terfas lachte uns aus, befahl einer riesigen Drachen-Troll-Chimäre in Borbaradianergewandung uns zu töten und verschwand mit der Hexe und dem Gepäck durch den Limbus. Wir erwerten uns nach bestem Können dem Chimärenmonstrum und gingen am Ende auch tatsächlich siegreich aus der Auseinandersetzung hervor. In den persönlichen Gegenständen des monsters, das ganz offensichtlich als Abu Terfas persönlicher Diener gearbeitet hatte, fanden wir eine Packliste, auf die kurioser Weise auch ein Tsageweihter gesetzt worden war. Angesichts dieses Gepäcks und dem Angriff auf Borbra und die Tsaeiche scheint sein Reiseziel eindeutig. Doch ist es nun schon zu spät um noh durch die Bergschluchten zu wandeln. Wir werden die Nacht über noch im Palast des Chimärenmeisters verbringen, und morgen nach Borbra zurückkehren. Hoffentlich kommen wir noch rechtzeitig um das Schlimmste zu verhindern.

\subsection{Unter Borbra nach Boronos te Partholon}

\paragraph{Ende Travia 1019 BF Al'Churam, Khoramgebirge}
Nachdem wir diese Monstrosität aus Drache und Troll besiegt haben, untersuchen wir Abu Terfas Anwesen. Hinter dem Bild eines Mannes, der anscheinend Abu Terfas darstellen soll (deutlich kleiner und älter, aber mit Bastrabuns Hand), was uns schon vorher aufgefallen ist, finden wir einen Tresor mit den Büchern: borbaradianisches Geflüster (ungekürzt) und das Hybridiarum, welches den Chimärenbau auf höheren Niveau behandelt.
Danach verlassen wir Al'Churam.

Sechs Tage später kommen wir endlich in Borbra an und verängstigte Bauern erzählen uns, dass vor sechs tagen ein Mann, Abu Terfas, und eine Frau; Acharia, erschienen sind und in den Tsa-tempel gegangen sind. Einen Tag später folgte ihnen Tarlisin von Borbra und eine Söldnerin. Tarlisin befahl den Tsa-Tempel zu verbarrikadieren. Wir dringen in den Tempel ein finden aber nur ein heilloses Chaos vor und ein Loch im Boden vor dem Altar, auf dem ein geopfertes Kind liegt. Wir steigen in das Loch hinab.
Nach einer kleinen Ewigkeit erreichen wir einen kleinen Saal, die Einrichtung ist alt modisch. Wir untersuchen die Gewölbe und uns erscheinen Eiblicke in das Leben dieser Gewölbe vor 500 Jahren. Am Ende der Gewölbe finden wir eine Treppe, die weiter in die Tiefe führt. Dort treffen wir wieder auf Gewölbe, die selben nur diesemal zur Zeit von Assarabad, und es geht weiter hinab. Die letzten Gewölbe sind aus der Zeit der Echsen. Dort treffen wir in einem großen Saal auf Abu Terfas und seine verrückte Hexe, welche eine Pforte des grauens in die Domäne der Asfaloth geöffnet haben.
Nach einem harten Kampf, indem Ragnos seinen Unterarm verliert und Echsen-Chimären uns stark mitnehmen, stößt Ragnos Abu Terfas in die Pforte des Grauens, nachdem er ihm den Handschuhe Calamans abgenommen hat, und kann diese Schließen.
Tarlsin von Borbra eilt uns zur Hilfe, doch entpuppt sich, dass Borbarad von ihm Besitz ergriffen hat, da Tarlsin einen Zauber aus dem bobaradianischen Cantus benutzte. Borbarad verhöhnt uns, bevor er uns zum Sterben in der einstürzenden Globule zurücklässt. Mit letzter Kraft schaffen wir es zu flüchten. Im Dorf kann uns dann der Tsa-Priester heilen.

\chapter{Schatten im Zwielicht}

\section{Geleitwort}

\section{Die Tagebücher}

\subsection{Der Aufbruch nach Rezzanjin al'Ahjan}

\paragraph{Beilunk (Unbeknnates Datum)}
Auf dem Weg nach Klammsbrück kam ich heute mit Firnen zusammen in Beilunk an. Das Wetter hatte seine eigene Schönheit, es war nebelig und kalt. Vermutlich auch deshalb feierte Beilunk rein zufällig an diesem Tag einen Feiertag für einen Praiosheiligen. Mit Sonne sähe diese Stadt bestimmt viel schöner aus. Wir wollten für die Nacht in einer Gaststätte unterkommen. Wie sich herausstellte hatten wir Glück, denn wir passten den Wirt noch ab, bevor er zum Gottesdienst ging. Doch konnten wir nicht in der Herberge bleiben, weshalb wir beschlossen auch zum Gottesdienst zu gehen. Firnen hatte unser maraskanisches Prinzip mit der Zahl 2 wohl etwas falsch erstanden. Er kaufte für den Gottesdienst 48 Kerzen, für jeden Gott vier, die wir dann unter uns aufteilten. In der Kirche wurden wir dann etwas schief angeguckt, stellten unsere Kerzen dann in eine Seitenkapelle, zündeten sie an und setzen uns in die Reihen zu den einfachen Leuten. Doch unglücklicherweise schienen sich die Priester thematisch vor allem auf die böse Magie zu versteifen. Irgendwann eskalierte die Situation und Firnen wurde von einigen Bauern tätlich angegriffen. Durch die rettenden Worte von Amando la Conda da Vanya, der die Bauern zurückscheuchte und uns nach draußen begleitete, wurden wir vor ein paar Problemen bewahrt. Er erzählte uns, dass er demnächst wohl nach Maraskan gehen würde, um dort für Ordnung zu sorgen. Dann eilte er auch schon davon. Wir stellten fest, dass wir es etwa zweieinhalb Stunden bei mehreren langweiligen Predigten in dieser Kirche ausgehalten hatten. Wir gingen zurück und warteten auf den Wirt.

\paragraph{Mirham}
Am nächsten Tag, machten wir uns auf, um das Kontor von Quendan Gorbas am Hafen zu besuchen um uns Pferde zu kaufen oder zu leihen. Der Schreiberling wollte uns zunächst als Boten einsetzten, doch als wir unsere Namen nannten, winkte er uns zu seinem Herrn Quendan Gorbas durch. Dieser kannte offensichtlich unsere Namen und wollte von uns mehr über die Gefählichkeit der Region rund um die Tulamidenlande erfahren. Wir erzählten ihm alles was wir wussten, was wahrhaftig nicht viel war. Er bot uns an am Abend zu seinem Haus am großen Marktplatz zu kommen, in welches er uns zum Abendessen einlud. Den Rest des Tages verbrachten wir in Beilunk, wir fanden sogar eine Taverne, in der Lieder über unsere Heldentaten gespielt wurden, wenn sie auch ein wenig falsch wiedergegeben wurden. Als wir am Abend am Haus ankamen, wurden wir gebührend empfangen, doch bevor wir richtig mit dem Essen anfangen hatten, frischte draußen der Wind auf, ein nackter Mann stand plötzlich hinter unserem Gastgeber und rammte ihm einen Dolch in den Hals. Bevor ich reagieren konnten, stand ich wohl unter einem Zauber und war total desorientiert. Ich fühlte mich extrem hilflos und eigentlich durfte mir so etwas nicht passieren, da es genauso gut Borbaradianer sein konnten, die uns auf diese Weise bedrohen konnten. Als ich wieder wusste wo ich war, stand ich vor einer zierlichen Frau, die mich einem Stab in der Hand haltend fragte, ob ich mich fesseln lassen würde. Meine anfängliche Skepsis war verflogen als sie Firnen, der Wiederstand leistete, in Stein verwandelte und ich ließ mich fesseln. Mir wurden die Augen verbunden und ich wurde aus dem Haus geführt. Ich fühlte plötzlich, dass ich fiel. Danach wachte ich an einem Stuhl gefesselt in einem Raum auf, in dem mir ein langer Mann mit bunten Haaren gegenüber saß. Er stellte sich als Salpikon Savertin, Leiter der schwarzen Gilde, vor. Als wir unsere Namen nannten, erklärte er, dass wir bei dem Abendessen mit dem Händler vergiftet werden hätten sollen. Zwei Agenten der Spezialeinheit der Schatten, die sich die Bekämpfung von Borbarad und Borbaradianern auf die Fahne geschrieben hatten, hatten uns bei dem Auftrag diesen zu töten wohl mitgenommen. Er stellte uns die beiden auch vor, Adaque war die kleine Frau, Pedresco nannte sich der Mann, der nicht aussah, wie ein Magier und wohl dennoch magisch begabt war. Anschließend eröffnete man uns, dass wir uns in den Gewölben der Schatten in Mirham, einer Stadt von der ich nur wusste, dass sie irgendwo im Süden lag, befanden und uns entweder den Schatten anschleißen konnten, wenigstens für eine kleine Zeit, oder ohne Gedächnes wieder zurück nach Beilunk geschickt werden würden. Wir entschieden uns für ersteres und bekamen gleich unseren ersten Auftrag. Ein Kämmerer war von der Mission noch mitgebracht worden. Wir sollten herausfinden wohin genau sein Herr die großen Mengen an Geld brachte, die ins Horasreich transferiert wurden. Es war recht schnell geklärt, dass das Geld zu einem Waffenhänder in Drol ging. Als nichts weiter aus dem Kämmerer herauszuholen war, kam Adaque herein und erklärte, als wäre es das selbstverstandlichste der Welt, dass er jetzt getötet werden würde und seine Organe verkauft werden würden. Ich war fassungslos, aber mir wurde klar, dass hier in einer schwarzmagischen Akademie solche Grausam- und Ehrlosigkeiten an der Tagesordnung waren. Beim Abendessen wurden wir den anderen Schatten vorgestellt. Insgesamt sind sie ein bunter Haufen, hauptsächlich aus Magiern der Mirhamer Akademie bestehend. Doch auch ein Meuchelmörder aus Al´Anfa und ein jetzt gesuchter ehemalig weißmagischer mittelreichischer Adeliger waren dabei. Es wurde ein netter Abend mit vielen lustigen Geschichten, an dem uns viele der Schatten sympatisch wurden. Am nächsten Tag wurden einige der Schatten auf eine Mission zum Waffenhändel geschickt. Wir wurden während der vier Tage, die die Mission dauerte durch die Räume der Schatten geführt und bekamen die Akademie und die Stadt gezeigt. Ich unterhielt mich auch mit dem Schatten Sherianus von Darboniam über die Informationsquellen der Schatten, da sie erstaunlich viel über unsere Gruppe wussten. So meinte Sherianus, dass die Rashduler Fürstin uns fast an Borbaradianer verkauft hätte, wenn jemand von diesen genug Geld geboten hätte. Ich spielte mit ihm noch ein paar Runden ein Würfelspiel, da er doch sehr über seine Langeweile jammerte. Ich fand auch heraus, dass Adaque so wie ich Ruruzat konnte und übte ein wenig mit ihr, sie war erstaunlich gut. Am vierten Tag rief uns Savertin in den Besprechungsraum und teilte uns mit, dass die Mission einigermaßen erfolgreich war. Der Waffenhändler selbst war nicht anwesend, jedoch fand man einen Brief, der einen gewissen Herrn von Dorkstein, einen Esquirio, während einer Jagd einen borbaradianischen Kontakt versprach. Salpikon wollte uns zusammen mit Adaque und Pedresco zu dieser Jagd schicken um herauszufinden, wer den Esquirio kontaktieren wollte und um den Kontakt zu behindern oder den Sinn herauszufinden. Dafür konnten wir nicht als einfache Bürger hingehen, sondern mussten verkeidet erscheinen, ich als tulamidischer Sultan oder Emir, Firnen als mein Leibmagier, Pedresco ironischerweise als mein Leibwächter, obwohl er gar nicht kämpfen kann und Adaque als meine Konkurbine. Morgen geht’s los. Ich bin mir fast schon sicher, dass ein paar der Schatten schon Wetten abgeschlossen hatten, wie lange ich die Rolle waren kann, ohne von den horasischen Adeligen entlarft zu werden.

\subsection{Die Ereignisse in Kuslik nach Firnen Wulfgrimm}
Da waren wir nun also in den Schatten. Dem seltsamstem und unvertrauenswürdigstem Zirkel, mit löblichen Zielen, den ich je gesehen habe.

Und wir, als temporäre Mitarbeiter, würden helfen, eine Verschwörung der Borbaradianer auf zu decken. Rezzanjin und ich werden morgenfrüh, mit der hüpschen Adaque und dem Druiden Pedresco nach Kuslik aufbrechen (bzw. dort erscheinen) um wärend der Jagd, den Esquirio im Blick zu behalten, und seine borbaradianische Kontaktperson kennen zu lernen. Wir werden als zusammen gehörende Gruppe, an der Jagd teilnehmen. Rezzanjin miemt einen tulamidischen Adligen: Abdul iben Achmet al Schamier . Ich stelle seinen Leibmagier, Omar, dar (Absolvent der Dracheneiakademi). Adaque geht als Raschar, die persönliche Kurtisane. Und Pedresco ist Pedro, Leibwächter und Diener. Er trägt Rezzanjins Schwert und Rüstung.


Was ein ereignisreicher Tag. Heute Morgen sind wir mit Adaque aufgebrochen. Mit der Hilfe so genannter Durthanischen Sphären (große Kristallkugeln,mit Eisenverstrebungen) sind wir durch den Limbus nach Kuslik gereist und haben uns unter die Jagdteilnehmer gemischt. Der mirhasche Gesante hat uns dazu eingeladen. Wir als Diener, aber auch Rezzanjin wurden hart bedrängt. Man löcherte uns mit Fragen noch und nöcher, um unseren politischen Standpunkt zu erkennen (bezüglich der horasischen, politischen Situation). Rezzanjin beharrte jedoch auf Habdelsbeziehungen, und wir als seine Vertrauten schlugen in die selbe Kerbe. Bei der Mittagsrast auf einem der freien Bauernhöfe, verdrückte sich der Esquirio. Ich folgte ihm unauffällig, und auch Pedresco berichtete, dass er ihn magisch überwacht habe. Es stellte sich heraus, dass er sich lediglich mit der Bauerstochter in der Scheune verlustierte. Rezzanjin, der die ständigen Beleidigungen des Kronprinzen nicht mehr ertragen konnte, beziehungsweise annahm, dass auf die reine Quantität der provozierenden Bemerkungen eine Duellforderung von ihm erwartet würde, hatte unterdess als logischen Schluss seiner Überlegungen, den Kronprinzen zum Duell gefordert und bestimmte mich just, als ich mich wieder unter die Gesellschaft mischen wollte, zu seinem Sekundanten. Obwohl das Rapierfechten nicht in Rezzanjins herkömmliche Kampftechnik fällt, war er mit seinen Kentnissen dem Kronprinz noch immer weit überlegen, was ihr Verhältniss auch im Anschluss nicht gerade nachhaltig entspannte. Am Abend lies die gesammte Jagdgesellschaft ihre Zellte auf einem der Hügel im Effert vor Kuslik aufschlagen. Und siehe da, das Vögelchen flog aus.

Als der Esquirio sich anschickte nach Kuslik aufzubrechen, meldeten wir uns beim mirhamer Gesanten für das Abendmahl ab, und folgten dem Esquirio nach Kuslik. Dieser begab sich geradewegs zum Palazzo der Fürstin. Wir hatten einige Probleme das Hofpersonal los zu werden, aber schlieslich gelang es uns das Gespräch des Esquirios mit einer jungen Frau zu belauschen. Irgendwoher kannte ich diese Stimme. Pedresco begab sich als Page illusioniert in den Raum und erhaschte einen Blick auf eine junge, hellhäutige, elfische Frau, bevor diese ihn unter einen elfisch gewirkten Banballadin stellte und aus dem Raum schickte. Allerdings ist sie trotdem mistrauisch genug gewesen eine algemeine Überprüfung der magischen Strukturen ihres Umfeldes vorzunehmen. Entweder sie erkannte die Ilusion, oder sie bemerkte die allgemeine Anwesenheit von Magieren, oder Beides. Jedenfalls fluchte sie Laut und floh.

Ich hatte mich unterdess mit Rezzanjin in den Emfangssaal teleportiert, wo uns die Elfe wieder begegnete. Mit magischer Geschwindichkeit rannte sie an uns Vorbei und nahm sich lediglich die Zeit die Palastwachen zu verzaubern. Zwei arme Individuen, mit schäumenden Mündern und irrem Blick stürtzten sich uns entgegen und deckten damit den Rückzug der Elfe. Ich hatte lediglich die Möglichkeit ihr einen Fulminiktus hinterher zu schläudern, bevor ich bald von den Wachen zerhackt worden wäre, bis Rezzanjin endich in den Bewitz seiner Waffe kam. Getroffen habe ich die Elfe trotzdem wohl recht gut, zumindest war Pedresco sehr damit zufrieden, als er wohl der Elfe gehörende Bluttropfen von der Treppe sammelte.
Adaque hatte sich unterdess darum gekümmert jegliche Dokumente aus dem Zimmer des Esquirios zu entwenden, und wir begaben uns nun allesamt zurück zu unserem magischen Reisegefährt.

In dem Gespräch zwischen dem Esquirio und der Elfe, bot diese ihm an, Borbarads Herrführer zu werden (Sie sprach von einem Sturm. Außerdem gab sie ihm eine Liste mit gewissen Dingen, die er besorgen sollte. Die Liste hatte Adaque mitgebracht, aber der Inhalt schien verschlüsselt und ohne jedes System zu sein. Wie Pedresco herausfand, handelte es sich um eine magische Verschlüsselung, mit einem uns unbekannten Wortschlüssel. Pedresco besitzt zwar nach eigenen Aussagen einen Zauber, der es ihm ermöglichen sollte, den Wortschlüssel heraus zu finden, doch sah er sich nicht mehr in der Lage genug astrale Energie auf zu bringen, so dass ich vor erst mit einem Xenographus aushalf und den Inhalt der Liste vorstellte. Es handelte sich um Angaben zu Edelsteinen (verschiedene Arten, Anzahl der benötigten Exemplare, jeweiliges Gewicht, jeweiliger Schliff).

Die Rückreise durch den Limbus verlief weniger glimpflich als der Hinweg. Als wir den vollkommen geräuschlosen Limbus hinter uns liesen, musste ich feststellen, das mir mein Hörvermögen abhanden gekommen war. Nach der Behandlung durche einen Alchimisten der Akademie, kann ich war schon wieder hören, dafür plagen mich nun starke Ohrenschmerzen.

Rezzanjin ist vor etwa einer Stunde vorbeigekommen, um mich auf den aktuellen Stand der Dinge zu setzen:
Eine Kristallomantische Nutzung der Edelsteine ist unwwahrscheinlich. Die Experten in diesen Fachgebieten gaben zu bedenken, das die Steine zu groß seien, und neben der Form auch die reine Masse unsinnig. Savertiens Echse füge noch an, das man für die Kristallomanie eine wesentlich höhere Schliffdiversität benöigen würde.
Am wahrscheinlichsten schien uns doch eher die Bezugnahme einer Dämonologin, die die Edelsteine als Donarie für Beschwörungen deklarierte.
Demnach vereinfachen die Edelsteine die Beschwörung. Einige der Edelsteine rechnete sie der allgemeinen Größe und Güte des Rituals zuträglich an. Andere seien ihres Wissens nach speziel auf Charyptoroth (Herrin des Wassers), Agrimoth (Schänder der vier Elemente: Luft,Feuer,Humus,Erz), und Thargunitoth (Herrin des Lebenden Todes) ausgelegt. -> Ein gigantisches Ritual, zur beschwörung hoher Dienern dieser drei Domänen und die gleichzeitige Anwerbung eines Heerführers: Borbarad plant einen Krieg, mit Legionen von Untoten nebst seinen weltlichen Anhängern, dämonisch pervertierter Ausrüstung (Agrimoth) und Dämonenbarken und Seeungeheuern, um von Maraskan über zu setzen. Fragt sich nur wo er zu Landen gedenkt? (Die asphalotische Präsenz im Tulamidenlande hat ihn gestört und er hat etwas dagegen unternommen. Möglicherweise plant er dort hin zu gelangen.)

Unser nächster Anhaltspunkt sind die Minen von Fasar, denn egal wer jetzt die Edelsteine besorgt, die Fasarer Minen sind seine wahrscheinlichste Anlaufstelle, sind sie doch recht ertragreich, und für jeden mit genug Dukaten in der Taschen auch zugänglich.

\emph{Der Bericht über die Ereignisse in Fasar ist leider verschollen.}

\subsection{Die Ereignisse in Warunk nach Rezzanjin al'Ahjan}

\emph{Daten unklarer Präzision}

\paragraph{19. Boron}
Nach dem Kampf war Firnen verletzt und der Paktierer war mit seinem Dämonen weggeflogen. Ich eilte zu Firnen hin - sein Arm war schlimm verwundet. Ich wollte mich schon ans Verbinden machen, doch Adaque hatte für Firnen glücklicherweise noch einen Balsam übrig, sodass Firnen nicht unter meinen bescheidenen Heilfähigkeiten leiden musste. Dennoch war Firnen noch nicht komplett geheilt und wir mussten irgendwo unterkommen. Die Akademie stand nicht zur Auswahl, da wir dort nicht hinein kommen würden. Ich erinnerte mich, dass die beiden Tempel von Bruder Boron sich bei unserem letzten Besuch ziemlich um uns gestritten hatten und so schlug ich vor, dass wir uns im Al-Anfanischen Tempel einquartieren. Die Boronis hatten nichts dagegen und so versorgten sie unsere Wunden und stellten uns Betten bereit. Wie sich herausstellte war auch Pedresco verletzt uns so übernachten wir im Tempel von Bruder Boron.

\paragraph{30. Boron}
Heute früh berieten wir uns über unsere Möglichkeiten und kamen zum Schluss, dass wir herausfinden mussten wo der Mann hingeflohen war. Wir beschlossen, dass wir den Paktierer durch ein Ritual, welches Perdresco beherrscht, kontrollieren würden, um ihn so daran zu hindern die Edelsteine zu überbringen. Wir konnten auch nicht im Borontempel bleiben, weshalb ich vorschlug, dass wir uns in einer etwas edleren Herberge, die im mittelreichischem Stil gehalten ist, einquartieren. Dort heilte Firnen Pedresco. Da wir planten durch die Sphären wieder zurückzureisen, mussten wir es schaffen in die Akademie von Rashdul zu kommen. Firnen teleportierte sich in die Akademie und kam kurz darauf wieder und berichtete, dass er eine Nachrcht an Mirhiban geschickt hatte, die noch immer in Fasar verweilte, dass sie sobald möglich nach Rashdul kommen sollte. Das ist also geklärt. Am Abend merkte Pedresco, dass das Blut des Paktierers, welches er für das Ritual benötigte, verschwunden war. Da wir ohne weiteres auch kein neues auftreiben konnten, mussten wir uns damit begnügen, dass Pedresco durch die Augen des Paktierers schaute, um anhand der Landschaft zu erkennen, wo er sich befand. Anhand der Beobachtungen konnten wir zumindest die Richtung bestätigen, die er mit seinem Flugdämon nahm. Er schien Richtung Süden zu entfliehen, in einer Geschwindgkeit, die jeder Menschlichkeit zu entbehren schien. Doch wir konnten nicht auf sein Ziel schließen, weshalb wir wohl keinen Schritt weiter gekommen waren. Wir beschlossen darauf zu hoffen, dass Salpikon Savertin weitere Informationen hat.

\paragraph{7. Hesinde}
Heute sind die Spektabilitäten der anderen schwarzmagischen Akademien in Mirham angekommen. Laut Savertin ist das Konvent der schwarzen Gilde die letzte Hoffnung weitere Informationen über das Ritual der Borbaradianer zu erlangen. Savertin hatte uns nach seiner Rückkehr eröffnet, dass er glaubte das Ritual diene zum Erbauen einer neuen schwarzen Feste. Des Weiteren bat er Firnen und mich um Unterstützung bei dem Überreden der anderen Spektabilitäten, da Savertin bei diesem Konvent die Meinung der schwarzen Gilde im Bezug auf die borbaradianische Bedrohung eindeutig klären wollte. Leider hatten die Schwarzmagier nicht eine eindeutige Meinung, weshalb Savertin ihnen die eindeutige Bedrohung durch Zeugen - in Form von uns - klar machen wollte. Die ganzen Festaktivitäten, die das Jonvent begleiteten, hörte man teilweise bis in den Keller der Schatten hinein. Wie mir einer der Schatten erklärte würden zu Beginn des Konvents eine ganze Menge alter Rituale und dergleichen erfolgen, irgendwannn würde es dann zu den Diskussionen kommen, aufgrund derer dieser Konvent einberufen wurde.

\paragraph{11. Hesinde}
Gestern wurde Firnen zur Befragung durch die versammelten Spektabilitäten berufen. Anscheinend lief es nicht so gut, denn kurz danach gab es ziemlichen Radau und Savertin scheuchte die Spektabilitäten, die wie es hieß, dem Vorschlag zugestimmt hatten, Borbarad zu kontaktieren, aus der Akademie. In den Gewölben der Schatten wurde die Stimmung erst wieder besser als Sherianus, mit der Absicht sich zu betrinken, mit mehreren Flaschen Wein in den Speisesaal kam - die anderen machten natürlich mit und so schloss ich mich an.

Heute morgen überraschte uns Savertin mit einer neuen Mission. Er hatte vor Pedresco, Firnen und mich nach Tobrien zu schicken, um einen Druiden zu finden, der ein überläufiger Borbaradianer war und sich angeblich irgendwo in der Nähe von Warunk befinden sollte. Sein Name war Sadragon und unser einziger Anhaltspunkt ihn zu finden war die Philosophenschule in Warunk. Es traf sich ganz gut, dass die Reise mit der Sphäre sehr lange dauerte, denn wir waren alle noch verkatert. Als die Sphäre aus dem Limbus wieder in unsere Welt trat befanden wir uns in einem Kellergewölbe. Als Firnen seine Fackel anmachte, befand sich um uns herum ein altes in Spinnenweben eingehülltes Gewölbe. Bevor wir uns fragen konnten wie viele Spinnen es für so ein gigantisches Spinnennetz braucht, hörten wir auch schon Chitinpanzer knacken und die eine Spinne, die für das verantwortlich war kam herangerauscht. Wir schafften es gerade so hinaus, jedoch teleportierte sich Mirhiban, als die Spinne sie verletzt hatte an einen uns unbekannten Ort. Als wir aus dem Gebaüde hinausfanden befanden wir vor den Stadmauern Warunks. Wie wir später noch herausfinden sollten waren wir in dem seltsamen Berg einiger ehemaliger Magier gelandet. Zurückzufinden wird wohl ein Problem werden. Wir suchten nach einer Herberge und fanden diese recht schnell und machten uns auf zur Philosophenschule, die geschlossen hatte. Das einzige Problem das wir jetzt noch haben ist Mirhiban zu finden.

\paragraph{12. Hesinde}
Die Reise durch den Schnee ist gewöhnungsbedürftig, jedoch seit dem harten Monat Reise durch Weiden nicht weiter problematisch. Der komische Knilch, der die Tür geöffnet hatte, als wir an die Philosophenschule geklopft hatten, hatte uns zu dem Dörfchen Viereichen, nahe des Yslisee gewiesen. Noch ist es nicht in Sicht und da wir uns dafür entschieden hatten ohne Pferde zu reisen wird es wohl noch ein wenig dauern, bis wir es erreichen.

\paragraph{18. Hesinde}
Heute sind wir in Viereichen angekommen. Meine Hoffnungen auf eine baldige Rückkehr nach Warunk haben sich nicht bestätigt. Der Druide hat es sich anscheinend in einem Wald gemütlich gemacht. Leider eine Tagesreise vom Dorf entfernt. Doch auch der Wald verhieß nichts gutes. Sollte er wirklich so nebelig sein, wie sein Name behauptet, dann müssen wir morgen aufpassen.

\paragraph{19. Hesinde}
Wahrhaftig ein Tag zum Vergessen. Nicht nur mussten wir einen ganzen Tag durch einen nebeligen verschneiten Wald stapfen, nein, eine Begegnung mit Borberadianern war natürlich nicht zu vermeiden. Zwar wusste der Druide nichts über das Ritual, jedoch vermeldete er, dass Borbarad angeblich auf Maraskan verweilte. Auch wollte er wissen, dass ein Angriff eines borbaradianischen Invasionstrupps auf Tobrien am 11. Ingerimm stattfinden würde. Insgesamt war er ziemlich paranoid und schien uns für Borberadianer zu halten. Von einem buckeligen Exemplar dieser wurde er angeblich angeworben, um als Herrscher der Druiden aventurienweit aufzusteigen. Jedoch fühlte er sich betrogen und falschen Versprechen auf dem Leim gegangen zu sein, weshalb er wohl floh. Just im Moment, als es schien, dass er uns alles veraten hatte, klopfte es an der Tür und jemand verlangte Einlass. Der Druide erreichte den Höhepunkt seiner Paranoia, was dazu führte, dass er uns beschuldigte Borbaradianer zu seinem Haus geführt zu haben. Ich öffnete die Tür und sah einen dieser. Nach einer kleinen Konversation beschloss er sein Schwert in die Tür zu rammen. Doch auch als er es herausgezogen hatte lief es nicht besser für ihn. Er konnte meinen Schlag nicht parieren und lag schwer verwundet am Boden. Doch als eine Stimme im Nebel schrie: `` Los meine kleinen Legionäre'', war ich auf Männer gefasst und nicht auf zwei kleine Kinder. Doch wie sich herausstellte waren es nur Kinderkörper, geführt von Dämonen, die während des Kampfes langsam aus ihren kleinen Gefängnissen herausbrachen. Ich musste meine gesamte Kraft aufbieten, eine Ladung des Axxeleratus Ringes raushauen und mich in eine Echse verwandeln, um die um Gnade flehenden Kinder von ihrem Leid zu erlösen. Der Druide verstarb dabei und der Magier, es war vermutlich der buckelige mit Namen Xeraan, konnte knapp entfliehen. Der Versuch den schwer verwundeten Schergen wiederzubeleben, um ihn zu befragen misslang. Wir müssen wohl mit der Hütte, die vom Kampf in Mitleidenschaft gezogen wurde Vorlieb nehmen, da wir innerhalb der Nacht wohl nicht nach Viereichen zurückkehren können.

\subsection{Die Ereignisse auf Altoum nach Rezzanjin al'Ahjan}

Nach diesen schrecklichen Ereignissen, waren wir auf Grund der Kälte, und der späten Tageszeit gezwungen, die Nacht noch im Haus des Druiden zu verbringen. Selbigen hat Pedresco in der Zwischenzeit am Steinkreis begraben.

\paragraph{27. Hesinde}
Nach etwa drei wöchiger Abwesenheit sind wir heute Abend in die Gewölbe der Schatten zurück gekehrt. Savertin bot uns an, uns einer Expedition unter der Leitung von Kalmann anzuschließen, die morgen gen Altoum auf zu brechen gedenke, um noch einmal Nachforschungen in Altair, dem vor einiger Zeit unter mysteriösen Umständen vernichtete Götterorakel (Hesinde, Effert, Phex), anzustellen. Wir nahmen gerne an.

Nach dem Abendessen luden uns Adaque und \dots zu einer gemütlichen Runde in Adaques Gemächern ein. Wir saßen noch fröhlich schwatzend in gemütlicher Runde, als ein Schatten aus der Wand brach. Es handelte sich um Magister Varagaun, der vor einiger Zeit bei einer Mission in Selem, von Borbaradianern gefasst worden war, und nun auf diese Weise ihnen entkommen war. Er warnte uns vor einem Verräter innerhalb der Schatten, doch noch bevor er näheres erläutern konnte brachen dämonische Höllenhunde aus ihm hervor (Hetzer der Nagrach und Lolgramoth Domäne). Wir erwehrten uns ihrer mit Mühe und Not, und weihten im Anschluss Savertin ``unter vier Augen'' in Varianz Anliegen ein. Eine offizielle Durchsuchung der Schatten wäre zu gefährlich. Der Verräter könnte versuchen, die versammelte Mannschaft größtmöglich zu vernichten Savertin bat uns daher wachsam und vorsichtig zu sein und niemandem auch nur ein Wort davon zu erzählen.

\paragraph{28.Hesinde}
Heute Morgen sind wir mit der Expedition aufgebrochen.
Die Gruppe besteht aus:
Kalman, als Leiter. Adaque, Magister Dagenfeld und mir als Analysemagier. Curilian, dem Alchimisten, Pedresco, Rezzanjin, und dem Alanfaner, Tiratio. Nemris und Virondil, zwei Adepten, die bereits in örtliche Aktionen der Schatten involviert waren. Einem Golem aus Ton. Und zwei Sklaven.

\paragraph{30. Hesinde}
Im mirhamer Hochseehaven (etwa zwei Tagesreisen von Mirham entfernt) haben wir die ``Opalglanz'', ein prunkvolle Zedrake, bestiegen und segeln nun hinaus in Charyptik Hoheitsgebiet der Piraten und Bukanier. Es ist schön endlich wieder Seewind um die Nase zu haben.

\paragraph{02. Firun}
Der Kapitän glaubt an Geschwindigkeit zu verlieren, und auch ich habe das Gefühl wir lägen tiefer im Wasser, als zu Beginn unserer Fahrt.
Tatsächlich stellte sich heraus, dass jemand Vitriol aus der Kiste entwendet haben muss, um damit ein Loch in den Schiffsrumpf zu ätzen. Ein Verräter weilt unter uns. Ist es vielleicht sogar Varagauns Verräter?

\paragraph{03. Firun}
Der Verräter hat wieder zugeschlagen: Mit rohem Fleisch, gewürzt mit Ingrimm hat jemand einen \dots Wurm, von einer offenen Luke bis in die Kabine der Magister gelockt. Es konnte jedoch keinen größeren Schaden anrichten.

\paragraph{04. Firun}
Wohlbehalten haben wir Charypso auf Altoum erreicht, von dort aus gedenken wir per Flussschiff bis Altair zu gelangen.

\paragraph{05. Firun}
Letzte Nacht hatte ich einen sonderbaren Traum:
``Eine Halle mit Säulen, die Schlangenleibern glichen. Ein Relief: drei Gestallten in schwarz. Grünes Licht erstrahlt und drei Personen, komplett aus Jade, treten daraus hervor. Ein gigantischer Feuerball explodiert und die Jadefiguren zerspringen in abertausende von Splittern.''

Mit unlauteren Methoden überzeugten wir einen Flussschiffer uns doch in die ``verfluchte'' Gegend von Altair zu bringen. Seine Ruderer flohen, als sie das Ziel erfuhren, und als der Bannbaladin den Schiffer nicht mehr für uns einnahm, sah Kalman sich gezwungen den armen Mann ertränken zu lassen.

Nun halte ich das Steuer und die beiden Sklaven und Tiratio und Rezzanjin staksen.

\paragraph{08.Firun}
Wir haben gegen Abend eine Stelle mit vielen Stromschnellen, und Wasserfällen erreicht. Wir befinden uns noch ca. eine Tagesreise von Altair entfernt, aber mit dem Schiff ist hier kein Vorankommen mehr.

Nach dem Entladen der Expeditionsgüter, geriet ich mit Kalmann aneinander. Ich schlug den dringenden Wunsch den riesigen Badezuber, trotz zu knapper Packkapazität, mitzunehmen aus, woraufhin sich Kalmann in seiner Position, als Expeditionsleiter, angegriffen fühlte. Um ich wieder ein zu Norden, wollte er mich ,Kraft seines Amtes, den Zuber tragen lassen. Ich weigerte mich und er rächte sich mit einem Zauber des Cantus: Corpofesso-Gliederschmerz. Ich spiegelte seinen eigenen Zauber auf ihn zurück, und erbost ließ er mich verhaften, nach dem ich für ein vorzeitiges Ende seines Zaubers gewirkt hatte. Adaque belegte mich mit einem ``Granit und Marmor''. Als sie jedoch beschlossen, Sklaven in Carypto zu erhalten, war es mir eine Freude, als Kalmann selbst, meine Verwandlung unterbrechen musste. Während der Rest schon einmal versucht einiges des Gepäcks zu den Ruinen Altairs zu bringen, werde ich mich mit den beiden Sklaven und dem Schiff zurück begeben und weitere Arbeiter anheuern.

\paragraph{13. Firun}
Wir sind zu den Stromschnellen zurück gekehrt, und werden morgen das letzte Gepäck nach Altair bringen. Während meiner Abwesenheit scheint sich einiges im Lager ereignet zu haben.

Rezzanjin berichtete, das in der Nacht auf den 11. Firun die Truhe unter Kalmanns Bett, mittels magischer Destruktion, aufgebrochen worden sei, und der Expeditionsbericht der Hesindekirche, der als Grundlage für unsere weiteren Erkenntnisse dienen sollte, so wie Kalmanns Nordnadel (im Süden Kompass geheißen), sollen entwendet worden sein. Zum öffnen der Truhe wurde, allem Anschein nach, ein kleiner Artefakthammer verwendet, der aus dem Materialzelt entwendet wurde. Nur Expeditionsteilnehmer konnten dieses Zelt unbemerkt betreten. Der Verräter weilt noch immer unter uns.

Als würde diese Bedrohung noch nicht reichen, ist Rezzanjin in der Nacht auf den 12. Firun, bei seinem Wachrundgang, auf einen Eingeborenen gestoßen, der versuchte sich Eintritt in das Zelt der Magister zu verschaffen. Der Einbruch wurde verhindert, doch der Eingeborene konnte entkommen. Hoffen wir das morgen nicht der ganze Stamm bei uns auf den Planken steht.

\paragraph{14. Firun}
Wir sind in Altair angekommen, und haben das Lager fertig errichtet. Ab morgen beginnt die Forschung. Als Reaktion auf die jüngsten Ereignisse, beschlossen Rezzanjin, Pedresco und ich, dass es sinnvoll wäre, in der Nacht zu wachen. Tiratio, den wir fragten, erklärte sich bereit für die erste Wacht, Rezzanjin, sagte die weite zu, und ich übernahm die dritte Schicht.

\paragraph{15. Firun}
Ich habe mich während meiner Morgenschicht ein wenig mit Zulhamid unterhalten, und dieser zeigte mir mittels des Auges ein chaotisches und vollkommen zerstörtes Netz von Kraftlinien, die sich durch Altair ziehen.
Interessant ist der Hesindetempel. Noch recht gut erhalten ist er umzingelt von allen möglichen Arten echsischen Getiers. Rezzanjin versuchte sich ihm zu nähern, aber die Tiere scheinen ihn zu bewachen.
Am Nachmittag bemerkte Rezzanjin eine Gruppe Piraten, die versuchten den Damm zu brechen. Wir konnten sie rechtzeitig aufhalten, und erlitten nur eine kleine Überschwemmung des Lagers. Bevor er starb konnte der Anführer der Bande noch berichten, von einer Frau in Charypso beauftragt worden zu sein. Die Verräterin ist also nicht nur unter uns, sondern außerdem noch weiblich. (Damit bleiben nur noch Adaque (sehr unwahrscheinlich), und Nemris (möglich))

\paragraph{16. Firun}
Heute Morgen fanden wir Magister Dagenfeld tot in seinem Bett, keinerlei Spuren, seine magische Aura hatte ihn bereits verlassen. Möglicherweise ist er tatsächlich seinem hohen Alter erlegen.

Während weiterer Studien an Fundstücken, gelang es mir ein detaillierteren Blick in das magische Gefüge zu werfen, der es mir ermöglichte die chaotische, magische Überstrahlung aus zu blenden: Auf der Orakelinsel hat ein Kampf zwischen ordnenden und chaotischen Kräften stattgefunden. Das chaotische und zerstörerische überwog. Als Vermischung asphalotisch und arimothischer Macht, ein chimärisches Wesen dämonischen Feuers, waltete es mit gewaltiger Macht, aber dazu verdammt, aus sich selbst heraus zerrissen zu werden. Es zerstörte die Stadt und entschwand gen Osten. Möglicher Weise dort, wo Rezzanjin eine gigantische Schneise im Wald ausmachen konnte. Eine der Ursprünglichen Kraftlinien, die sich in Altair kreuzen, ähnelt in verdächtiger Weise, jener die den Nachtschattenturm traf.

Pedresco flog als Nebelfetzen in den Hesindetempel. Die Echsen (von der Eidechse bis zum Alligator) waren inzwischen damit beschäftigt sich selbst zu bekriegen. Nach seiner Rückkehr war es ihm möglich eine Radierung des Reliefs anzufertigen. Es entsprach jenem Exemplar aus meinem Traum in Charypso, und auch die Beschreibung des Hesindetempels mit den schlangenleibigen Säulen. Es stellte sich heraus, das sowohl Pedresco, als auch Rezzanjin, jenen prophetischen Traum ebenfalls gehabt hatten. Nun besaßen wir auch eine genaue Darstellung des Reliefs. Es zeigte drei Gestalten, die je einen Gegenstand in den Händen hielten. die rechte Figur einen Ring, den wir als den heute im Phextempel gefundenen Jadering identifizierten, die mittlere Figur einen Tropfen (möglicherweise Effert), und die linke Figur eine Kugel (Es gibt Berichte um eine Kugel der Hesinde, die erwiesene Wunder gewirkt habe). Vielleicht kommen wir weiter wenn wir alle drei Artefakte finden. (Die Artefakte wurden ursprünglich wohl von den drei Jadestatuen des Orakels getragen (jedem Gott eine Statue.) Und sie beantworteten einen drei Fragen. Phex log, Effert legte frei aus, und Hesinde sagte stets die Wahrheit. Welche Aussage welchem Gott angehörte blieb dem Fragenden dabei stets verborgen.)

\paragraph{17. Firun}
Schon wieder ein Traum den wir drei uns teilen: ``Eine Borongeweihte verbrennt im Feuer eines Drachen, der einen Samen verschlingt, ein wichtiges Puzzelteil, das wir verzweifelt zu finden gedenken. Eine Frau geht um und meuchelt.''

Resümee des Tages: Sieg auf \dots halber Linie. Doch alles der Reihe nach:
Heute Morgen fanden wir vier unserer sechs Arbeiter mit Blasrohrgiftpfeilen getötet in ihrem Zelt. Glücklicherweise fanden sich die Eingeborenen trotzdem bisher nicht ein.
Pedresco, der über Nacht nicht zu Rande gekommen war, den Geist von Nemris zu prüfen, wurde etwa zu Mittag, plötzlich von dem Golem angegriffen. Kalmann war nicht in der Lage ihn zu kontrollieren und der Golem begann das gesamte Lager zu verwüsten. Mit purer Gewalt stoppten ein Großteil der Magier und Rezzanjin den Golem. Pedresco hingegen schnappte sich Nemris und drang in ihren Geist ein. Siehe da, sie war die Verräterin, die unsere Expedition zu verzögern und zu erschweren suchte. Im Zelt der Adeptin fanden wir auch den Effertstropfen.

Noch gewahr des letzten Traumes, machten wir uns auf zur Waldschneise im Osten. Welches Drachenwesen, soll im Traum gehandelt haben, wenn nicht jenes dämonische Feuerwesen, das Altair verwüstete und gen Osten ging. Wir kamen an die Schneise, und umgeben von dämonischer Hitze fanden wir eine Chimäre, halb Spinne und halb Drache. Von ersterem den Leib und von letzterem Kopf und Flügel. Ein magische Analyse ergab einen mächtigen Dämon, der im inneren der Chimäre gebunden blieb, und eine Ordnende Kraft, die sich ebenfalls in der Chimäre zu befinden schien. ``Der Drache, der Verschlang den Samen, ein Puzzelteil, welches wir verzweifelt zu finden suchen.'' Wir schafften es auch mit vereinten Kräften nicht den Dämon zu entschwören, doch auch so gelange es mir die heilige Kugel der Hesinde aus der Chimäre zu holen.

All unsere Versuche irgendetwas mit den drei göttlichen Artefakten zu erreichen schlugen fehl.

\paragraph{18. Firun}
Unsere Mission ist beendet und erfolgreich verlaufen. In den frühen Morgenstunden ereignete sich uns gar Sonderbares: Pedresco, Rezzanjin und ich erwachten, und es zog uns zur Orakelinsel, die von einem blauen Glanz umgeben schien. Während ich ins Gebet an die Zwölfe versank, errichteten Rezzanjin und Pedresco die Jadestatuen, aus den einzelnen Splittern neu. Und das Orakel wurde uns offenbar. Drei Fragen gewährten sie jedem von uns, und beantworteten sie mit den ihnen eigenen Charakterzügen. Phex mit Lüge, Hesinde mit Wahrheit und Effert mit Auslegung.

\begin{enumerate}
\item Wo wird das Ritual stattfinden?
\begin{itemize}
\item Überall, um uns herum.
\item Dort, wo das verfluchte Wasser auf schroffe Klippen stößt.
\item Auf der Spitze eines Berges, in einer Welt aus Sturm und Feuer.
\end{itemize}
\item Wann wird das Ritual stattfinden?
\begin{itemize}
\item Es ist nicht mehr aufzuhalten.
\item Wir werden kommen zur rechten Zeit.
\item Wenn Sonne und Mond sich vereinigen.
\end{itemize}
\item Nächstes Eroberungsziel des Dämonenmeisters ?
\begin{itemize}
\item Ein altes und neues Land, von dem aus er Herrschaft über den gesamten Sphärenkreis erlangt.
\item Erst das Meer, dann das Land. Die gesamte Heide und jede Rose.
\item Er wird stürmen, über alles, was der Mensch erbaut
\end{itemize}
\item Warum will der Dämonenmeister die Herrschaft über Aventurien?
\begin{itemize}
\item Er will eine neue Herrschaft und neue Throne.
\item Seines und des Anderen Ziel sind sich sehr ähnlich, und doch verschieden. Sein Ziel ist nicht so kleiner als ihr denkt, aber größer als ihr es euch je vorzustellen vermögt. Geblendet und verraten, begeht er den schlimmsten Verrat.
\item Er will ein altes Netz nutzen, um ein neueres zu zerstören, und etwas neues zu schaffen, das ewig währt.
\end{itemize}
\item Stimmt der 20. Boron, als der Zeitpunkt an dem Borbarad den Sturm auf Aventurien eröffnet?
\begin{itemize}
\item Nein (Phex)
\item JA (Hesinde oder Effert)
\item JA (Hesinde oder Effert)
\end{itemize}
\item Was ist das Ziel des Rituals?
\begin{itemize}
\item Vernichtung.
\item Ewige Herrschaft.
\item Die größte Festung aller Zeiten.
\end{itemize}
\item Ist ein roter Apfel rot?
\begin{itemize}
\item Nein. (Phex)
\item Rot und nicht rot. (Hesinde?)
\item Egal, wenn man ihn gegessen hat. (Effert?)
\end{itemize}
\item Wer ist Effert?
\begin{itemize}
\item Nicht der dritte von uns. (Phex!)
\item Einer meiner beiden Brüder. (Hesinde)
\item Ich. (Effert)
\end{itemize}
\end{enumerate}
Bevor die Statuen endgültig zerfielen sprachen sie noch je eine Letzte Prophezeihung.

\subsection{Die Ereignisse auf Andalkan nach Firnen Wulfgrimm}

\paragraph{18. Firun}
Am späten Morgen haben wir das Lager abgebrochen, und uns zurück zu unserem Boot begeben. Die drei göttlichen Reliquien blieben verschwunden, was Kalmann nicht gerade in beste Laune versetzt hat. Wir mussten 2 mal Gehen um allen Kram wieder zu unserm kleinen Flussschiff zu transportieren, und dabei muss man uns beobachtet haben. Wir waren gerade fertig mit dem Verladen und Verstauen, als Rezzanjin eine Bemerkung am Rande des dichteren Urwalds auffiel. Seine Warnung kam gerade rechtzeitig, damit die meisten sich vor der Salve Blasrohrpfeilen in Sicherheit bringen konnten. Virondil schaffte es nicht und litt nach anfänglichen Lähmungserscheinungen, noch zwei Tage an spontanen Muskelkrämpfen. Doch was ist schon eine Horde Waldmenschen, gegen einen absonderlichen Haufen Schwarzmagier. Die ersten kleineren Zaubereien schon, schlugen das abergläubische Pack in die Flucht, und wir konnten die Mühselige Rückreise noch vor dem späten Mittagsmahl beginnen.

\paragraph{20. Firun}
Wohlbehalten haben wir Charypso erreicht. Die Opalglanz lag noch vor Anker und konnte noch heute Beladen werden. Morgenfrüh mit der ersten Flut des Tages laufen wir aus. Zu klären bleibt noch wohin.

\paragraph{21. Firun}
Nach reiflicher Überlegung sind wir gemeinsam zu dem Schluss gekommen, besser nicht zu überstürzen zu handeln. Also nicht nach Andalkan sondern nach Mirham zurück zu kehren.

\paragraph{26. Firun}
Erneut stehen wir vor den Toren Mirhams, doch die Akademie scheint während unserer Abwesenheit gewachsen zu sein. Es sind einiges mehr Leute in der Akademie unterwegs, und ein regelrechtes Zeltlager hat sich gebildet.
Nach unserer Berichterstattung hat Savertin uns erläutert, was es mit dem Massenauflauf auf sich hat. Kalmann hatte ihn bereits per dämonischem Boten von unserem Ergebnis berichtet, und bis zu unserer Rückkehr hat er alle Mitglieder der Schatten (inklusive derer, die normalerweise nur Auswertig agieren) um mit ihnen gemeinsam bei unserem weiteren Vorgehen zu handeln.

Am Abend versammelte Savertin die gesamten Schatten in den Gewölben, und holte eines der Schwarzen Auge, das sich in Besitz der mirhamer Akademie befindet, hervor. Die Spannung und Faszination unter den Magiern war fast greifbar. Mit Hilfe des mächtigen Artefaktes lenkte Savertin unseren Blick auf Andalkan. Eine kleine Insel hinter Maraskan. Grundsteine einer gigantischen Anlage. Sklaven und Dämonen, die zu hunderten arbeiten müssen. Die Informationen waren ausreichend. Unser Plan war simpel, aber brauchbar. Ein Stoßtrupp würde mit den Sphären nach Talusa reisen, und die dort vor Anker liegenden drei Schiffe der Perlenmeerflotte übernehmen. Die restliche Mannschaft würde mit Opalglanz, die man dämonisch beschleunigen würde nachkommen. Treffen würde man sich in drei Tagen vor der Insel \dots . Von dort aus würde die kleine Armada gemeinsam bis nach Andalkan segeln. Nach der Landung sah der Plan vor so viel wie möglich zu zerstören und wenn möglich eine Art Zentrum auszumachen und ebenfalls zu vernichten. Wir schlossen uns dem Stoßtrupp an, der in zwei Tagen nach Talusa reisen würde.

Endlich wieder auf See. In Talusa fanden wir alles vor wie beschrieben. Mit Hilfe einer überzeugenden Halluzination wiesen wir uns vor der Conta Admiralin als KGIA-Agenten aus, und überzeugten sie mit uns zu segeln. War leichter gewesen, als gedacht.

Wie verabredet haben wir den andern Haufen mit der Opalglanz vor \dots getroffen. Andere Schiffe der Seeblockade machen uns keine Schwierigkeiten, schließlich gehören wir zu ihnen - denken sie, reicht uns.
Savertin hat das Hauptschiff übernommen. Rezzanjin, Pedresco und ich wurden auf die Opalglanz umgesiedelt. Savertin ließ durchblicken, dass der Verräter wieder zugeschlagen habe.

Tatsächlich der Verräter und wir hätten ihn nie verdächtigt. gegen Mittag des heutigen Tages erschienen zwei Skelette aus dem Laderaum kommend. Geballte schwarzmagische Macht verbrannte sie zu Staub. Wie sich in der folgenden Analyse der verkohlten Leichen zweier Novizen ergab, hatte Zurlan sie mit einer Illusion belegt. Zurlan erstach Kalmann noch bevor dieser ihn entlarven konnte, verriet sich damit aber selbst. Leider brachten wir Zurlan um bevor wir ihn befragen konnten.

Meine Zeitrechnung ist mir wieder aufs sträflichste Abhanden gekommen, aber was soll's. Anders als erwartet schien Andalkan nicht großartig in Bebauung zu sein, die zwei Piratenschiffe, die wir im schwarzen Auge gesehen hatten lagen allerdings tatsächlich in der Bucht. Sie schienen uns bemerkt zu haben und zogen sich zurück hinter eine kleine Landzunge und dann geschah es. Die zwei kleinen Kriegszedraken, sollten sich eigentlich absetzen und wo anders an Land gehen, doch wie sie sich aus unserem Konvoi lösen wollten, tauchten in unserem Rücken eine ganze Schar kleinerer Piratenschiffe auf und eröffneten das Feuer. Brandtöpfe, Steine, Speere, und Spliterkugeln. schlugen in unseren Schiffsverband und drohten ihn binnen kürzester Zeit zu versenken. Wir erreichten das Dämonenschiff mit den schwarzen Segeln, und sie begannen uns zu Entern. Die folgende Schlacht war lang und blutig, wer lebendig über Bord ging, den verschlang das dämonische Wasser. Rezzanjin, schlug sich wacker gegen eines von Xeraans Dämonenkindern, ausgerüstet mit dem Charyptorothschwert. Besiegen konnte er ihn erst nach dem er sich in eine fast drei Schritt große Echse verwandelt hat. (Seit der Zeichnung auf Maraskan ist er deutlich seltsamer geworden. Ich habe keine Ahnung wie er das macht.) Zusammen mit ein paar alanfaner Magiern erschuf ich einen Schild um Savertin zu decken der sich mittels eines Artefaktes daran machte, die riesige Dämonenspinne auf dem Piratenschiff zu vernichten. Nach dem dies nicht vollständig gelang, erledigte rohe Gewalt das Monstrum, doch der Kampf war nicht zu gewinnen. Als zu allem Überfluss noch drei Dämonenbarken auftauchten, setzten wir die Opalglanz endgültig in Brand, kappten die Taue und agierten von dem gekaperten Piratenschiff aus. Doch eine Flucht war unmöglich. Arg beschossen, und von allen Seiten bedrängt, (selbst vom Himmel her griffen uns Irrhalken mit eventuellen Reitern an,) setzten wir eines der Beiboote ab und entkamen mit letzter Kraft an den Strand und in den maraskanischen Inselwald.

\subsection{Das Ende nach Firnen Wulfgrimm}

Welch ein Fiasko! Die Schatten wurden komplett aufgerieben. Nur Salpico Savertin konnte auf seiner Echse entfliegen. Wir, Rezzanjin, Pedresco, ein alanfaner Magus und ich, haben uns von dem dämonischen Piratenschiff aus, mit einem Beibot auf das Eilande Andalkan absetzen können.

Bei unserer Flucht durch den Dschungel Andalkans fanden wir echsische Tempel, drei an der Zahl. Der erste scheint von echsischen Priestern der Charyptoroth frequentiert zu sein, die überdies in regem Austausch mit den, die Strände bewohnenden Piraten und Borbaradianern, zu stehen scheinen. Der zweite Tempel scheint der Hesinde, beziehungsweise dem echsischen Äquivalent zugehörig (Schlangen in Statuen und lebensecht überall). Rezzanjin entfloh diesem Tempel auf Grund, so sagte er, der Echse in ihm, die Unbehagen verspührte, angesichts der echsischen Priester dieses Tempels.

Am dritten Tempel verweilen wir nun, nach der gestrigen entbehrungsreichen und knappen Flucht , auf der wir mehrfach getrennt wurden,(Verstecken auf Bäumen, Kampf mit den Borbaradianerpiraten und ihren Bluthunden), haben wir uns hier eingefunden. Der Tempel besteht eigendlich nur aus, auf einer Lichtung aufgestellten, Kristallen und die beiden Echsenpriester beschäftigen sich unentwegt mit der Begutachtung verschiedenster Edelsteine. Trotzdem scheint auch dieser Tempel, von den Piraten gemieden zu werden, die stets jeweils einen großen Bogen um die drei Tempel machen. Die beiden Echsenpriester haben versucht mit uns Kontakt auf zu nehmen. Ich vermittelte ihnen, dass wir in Frieden hier ausruhen wollten, sie quietierten dies mit der Anweisung, sie nicht zu stören. Nach einer Weile kam der kleinere der beiden Echsen erneut auf uns zu. Er zeigte auf das Schwert der Charyptoroth, das wir erbeutet hatten, und sagte etwas, mir größtenteils unverständliches, das irgendetwas mit opfern zu tun hatte.

Um sie nicht zu verärgern, überliesen wir ihm das Schwert, mit dem er von dannen zog. Hoffentlich um es zu opfern oder zu zerstören. Morgen müssten wir eigendlich wieder ausreichend zu Kräften gekommen sein, obwohl ein weiterer Tag Pause auch nicht schaden würde.

Der zweite Tag Pause blieb Wunschtraum. Wir waren gerade dabei die Reste des erlegten Bluthundes zum Frühstück zu verzehren, als die Charyptorothechsen auftauchten und mit den beiden Echsenpriestern zu reden begannen. Worübr sie verhandelten konnte ich nicht verstehen, doch wir beschlossen, dass eine schnelle Flucht angebracht sei. So wie wir den Tempelbereich verließen, schossen rings um uns herum Wassertentakelarme aus dem Schlamm und versuchten uns zu greifen. Rezzanjin wurde zwar erwischt, konnte sich aber dem Griff entwinden.

Um die Verfolger, die zu hunderten unterwegs zu seien schienen, zu verwirren zauberte der Magus einen großen Raum magischer Dunkelheit. Doch von oben herab tönte eine Stimme, wir seien umzingelt und sollten uns ergeben. Pedresco verwandelte sich, wie ich später erfuhr in seine Nebelgestallt und entkam. Der al'anfanische Magier ließ sich gefangen nehmen und vor die Anführer bringen. Dort sprengte er sich und die Umstehenden, in einem letzten aufopferungsvollen Akt, mittels seines magischen Könnens, in die Luft. Ich schnappte mir Rezzanjin und teleportierte mich mit ihm zum Strand, und anschließend auf das entfernteste Schiff, dass ich ich ausmachen konnte. Wir materialisierten uns direkt vor dem Schiffsmagier, den ich sofort unter einen Bannbaladin stellte. Er fragte, ob Rotkappe uns schicke, wir bestätigten dies und nannten den Auftrag, nach Talusa gebracht zu werden. Die blutige Hure, wie das Schiff geheißen wurde wendete und schiickte sich an mit uns an Bord die Bucht entgültig zu verlassen. Die anderen Schiffe eröffneten das Feuer auf uns, konnten allerdings keine schwereren Schäden verursachen. Den Schiffsmagier baten wir allsbald in seine Kajüte und schlitzten ihm die Kehle auf. Wir warfen ihn über Bord mit der Begründung, Rechnung mit Rotkappe, ein Bannbaladin auf den trunkenen Kapitän regelte den Rest. Das einzige was mir noch Sorge bereitet ist der sonderbare Husten, den sich Rezzanjin seit gestern zugezogen zu haben scheint. Hoffentlich bereitet der uns nicht noch schwerwiegende Probleme.

Heute morgen ist die Stahlfeder hinter uns am Horizont aufgetaucht. Scheinbar verfolgen sie uns doch noch (Auch die Stahlfeder gehörte zu den Borbaradianer-Piratenschiffen)

\paragraph{13. Tsa}
Wir haben Talusa erreicht. Etwa zehn Meilen vor der Küste hohlte uns die Stahlfeder ein, und eröffnete das Feuer auf die blutige Hure. Ich teleportierte mich mich mit Rezzanjin an den talusischen Strand.

Eine Stunde später, wir irrten nich durch die taluser Straßen, um am Hafen ein schiff nach Kunchom zu finden, trafen wir unvermittet auf Pedresco. Er war in Nebelgestallt bis auf die Stahlfeder geflogen. Sah die blutige Hure aus der Bucht entfliehen, und ordnete an, die Verräter zu verfolgen. Anschließend beherrschte er den Kapitän, versengte die blutige Hure vor Talusa, verwickelte die Stahlfeder in ein aussichtsloses Gefecht mit der talusischen Küstenwache, und flog als Nebelleib nach Talusa rein, wo er uns nun zufällig wiedergetroffen hat.

Pedresco entschied zurück nach Mirham zu gehen, Rezzanjin, der noch arg mit seinem Husten zu kämpfen hat, will nach Kunchom, um bei einem Fähigen Meister seine Schwertkunst zu vollenden.
Ich werde mit nach Kunchom kommen und vvon dort aus weiter nach Klammsbrück reisen. Wenn alles glatt läuft, sollte ich etwa am 5. Phex, nach der Schneeschmelze dort eintreffen.

Die Festung war nur eine Illusion. Den Grundstein seiner Herrschaft, hat Borbarad wohl dieses Mal in Form einer Flotte gelegt. Wer weiß wie viele dieser Dämonenarchen er noch hat, beziehungsweise bis zum 21.Ingrimm haben wird.

\chapter{Schlusswort}