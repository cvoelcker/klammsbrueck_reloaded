\chapter{Euch zum Geleit}
So viele Schlachten\dots Ich glaube, wenn ich an die G7 denke, dann hat dieser 

\chapter{Erinnerung an bessere Zeiten}
Lieber Waldemar,

ach könnte doch jeder Tag so schön sein wie unser letzter in Punin!
Ich möchte mich noch einmal sehr für deine Gastfreundschaft bedanken, und natürlich auch für die wundervollen Flaschen edlen Almadischen Weins!
Leider glaube ich das Tarlisin meinen kleinen geheimen Weinkeller im Hauptquartier der Grauen Stäbe gefunden hat, ein, zwei Flaschen sind sonderbarerer Weise verschollen!?

Es hat mir sehr große Freude bereitet von den neuen Eleven zu hören. Zwei Gildensiegel tragen zu dürfen, und das von so renomierten Akademien wie Punin und Klammsbrück ist ein großartiges Privileg. Außerdem, und das darf ich natürlich nicht verraten, habe ich letztens die große Freude gehabt mit Walpurga von Weiden zu speisen. Die Herzogin wünscht sich, dass die Akademie bald wieder eine eigenständige Präsenz in den wieder eroberten Gebieten erhalten soll!
Natürlich ist dies noch wilde Spekulation; Eleven in aktives Kriegsgebiet zu bringen scheint mir eine fürchterlich schlechte Idee.
Deine alten Gefährten machen sich einen großen Namen im Kampf gegen die Erben des Dämonenmeisters! Doch ich muss zugeben dass deine Entscheidung der akademischen Laufbahn treu zu bleiben mich immer mit besonderem Stolz erfüllt hat. Aber verrate das bitte nicht den anderen. Es ist die leidliche Art alter Lehrer nur jene zu lieben, die selbst ein Lehramt übernehmen.

Ich wollte diesen Teil des Berichtes dennoch mit freudigen Nachrichten beginnen, denn ich denke dieser wird am schwersten zu lesen werden. Denn er erhält vor allem Berichte vom Krieg aus Tobrien, beginnend mit der Schlacht von Kurkum, bis zum bitteren Ende als Grunewaldt im Drachenfeuer verbrannt.
Leider werden die Berichte aus dieser Zeit Bruchstückhafter. Man kann den Gezeichneten wohl nicht verdenken, dass sie in diesen Tagen wichtigeres zu tun hatten, als lange Reiseberichte zu verfassen!
Ich habe an einigen Stellen Details aus meinen eigenen Erinnerungen ergänzt, aber die vielen Jahre sind auch an meinem Erinnerungsvermögen nicht spurlos vorbei gezogen.
Bitte vergieb mir also, wenn Teile der Erzählung nicht ganz der Wahrheit entsprechen sollten.

Dein alter Lehrmeister,

Spektabilitus Emeritus Iliricon Tannhaus, Ordo Defensores Lecturia
\chapter{Goldene Blüten auf Blauem Grund}

\section{Geleitwort}


\begin{flushright}
Claas Völcker, Toronto, den 28.05.2025
\end{flushright}

\section{Die Ballade von Kurkum}

Dunkle Wolken über den Gipfeln\\
im Sommer ein frostiger Wind zieht herauf\\
Die Horden des Dunklen sind eingefallen\\
das Land liegt still im Blutesrausch.

Der Feind steht stark, stützt seine Mach\\
auf Stahl und manch Gezücht der Nacht\\
Der Niederhöllen plagend Qual\\
und dreifach in der Überzahl.

Das Tal liegt rauchbedeckt und leer\\
Die Siedlungen brennen rings umher\\
Und letzte Schaaren sieht man zieh' n\\
zur Burg im See hernieder flieh' n.

Und einsam erhebt sich gegen das Meer\\
aus Feinden, der Göttin Streiter Wehr\\
Die Mauern steh' n stark, das Tore hält\\
bis dass der letzte Streiter fällt.

Bewaffnet alle Dörfler und Bauern\\
Beistand den Kriegerinnen auf den Mauern\\
Auch Fremdländische Helden finden sich ein\\
in der Verteidiger dünnen Reih' n.

Ein Priester, der Göttin der Ernte zu Diensten\\
Vom Reiche ein adliger Rittersmann\\
Ein Krieger von Maraskans fernen Inseln\\
Ein tobrischer Junker und Edelmann.

Ein Dschinnrufer aus dem warmen Süden\\
der Gegenstände verzaubern kann\\
Ein Magus, halb elfisch, vom hohen Norde\\
bewandert in der Magie, Hellsicht und Bann.

Und gemeinsam erhebt sich gegen das Heer\\
aus Feinden die bunt gemischte Wehr\\
Die Verteidiger wachen und steh' n bereit\\
zu halten die Burg zu zerschmettern den Feind.

In der Burg wird' s enger und doch liegt sie still\\
Drei Tage zu Essen, gut und gern\\
Gleichwohl der ein Schmied vom Volk der Kurzen\\
gen Heimat zog, bleibt Hilfe fern.

Der Feind stellt Lager rings umher\\
den Kreis geschlossen mit Bedacht\\
Nur die Fremden, frohen Mutes\\
fallen aus in dunkler Nacht.

Die Zelte steh' n entflammt Reihe\\
bald brennt das Katapult im Nu\\
Der Feind verhöhnt vor' m ersten Schlage\\
Man gönnt ihm Nächtens keine Ruh.

Und gemeinsam erhebt sich gegen das Meer\\
aus Feinden die bunt gemischte Wehr\\
Die Verteidiger wachen und steh' n bereit\\
zu halten die Burg, zu zerschmettern den Feind.

Untätig bleib auch das Böse nicht\\
sät Angst und Schrecken zwischen den Mauern\\
Mit Dolch und Gift zum Brunnen schlich\\
Der Kampf, der soll nicht lange dauern.

Die Fremden aber steh' n bereit\\
vereiteln jeden Schlag gescheit\\
Mit Hilfe mancher Wesenheit\\
und auch den Göttern an ihrer Seit.

Doch die Häscher des Bösen sind ohne Zahl\\
Die Verräter unerkannt allzumal\\
In den Gewölben des Bergfrieds, versteckt und gebunden\\
zwei dämonische Schwerter sind verschwunden.

Und einsam erhebt sich gegen das Meer\\
aus Feinden, der Göttin Streiter Wehr\\
Die Mauern sind stark, das Tore hält\\
bis das der letzte Streiter fällt.

Mit beginnendem Tag, der Feind stellt sich auf\\
mit bloßem Auge nicht zu zähl' n\\
Getöse und Rauch erhebt sich vom Lager\\
ihre böse Absicht zu verhehlen.

Aus Kraft und Energie geboren\\
aus Schatten ein Dämon kriecht hervor\\
ist langsam, doch stetig und zielgerichtet\\
Gerufen zu zerstören das Tor.

Welch höllisch Kraft die Balken zersplittert\\
die Angeln schonquetschend im Mauerwerk dreh' n\\
Doch der nordische Magus steht auf dem Torhaus\\
der Dämon kann ihm nicht widersteh' n.

Und gemeinsam erhebt sich gegen das Heer\\
der Feinde, die Streiter auf der Wehr\\
Mögen kommen ins Tal, alle Niederhöll' n\\
am Fuß dieser Mauer, soll' n sie zerschell' n.

Auch wer nicht kämpft, der ist betroffen\\
Mord und Grauen Hand in Hand\\
Dunkle Zeichen, ein Greis den Tod\\
durch einer Gottheit Statue fand.

Verräter gehen weiter um\\
vergiften, morden, sähen Zwietracht\\
Perain alleine trotzt der Qual\\
lindert Pein, hält die Moral.

Die Verteidiger stehen ratlos, machtlos\\
wo man kein Feind zu sehen vermag\\
Und auch im Kampfe auf den Mauern\\
sterben viele Tag für Tag.

Und allein steh' n die Streiter auf der Wehr\\
umschlossen in Mitten des Feindes Heer\\
Wer weiß, wie lang die Mauer hält\\
Doch die Stunde rückt nah, da der Letzte fällt.

Keine Belagerungswaffen, die Dämonen entschwor' n\\
bleibt dem Feind nur der Kampf, Mann gegen Mann\\
Jeder Zoll auf den Mauern wird teuer erkauft\\
doch ein Preis, den der Feind sich leisten kann.

Die Verteidiger wanken, der Feind rückt nach\\
Eine Mauer besetzt, das Ende hat schon begonnen\\
Ein Hornstoß vom Berg, in der Not höchster Stunde\\
die Zwerge zur Hilfe sind gekommen.

Die Horden zieh' n von der Burg abgelenkt\\
bei den Lagern, die Zwerge nun arg bedrängt\\
Doch den Freund nie verraten, ist der Rondra Brauch\\
Die Kriegerinnen reiten aus.

Und gemeinsam setzen sich gegen das Heer\\
der Feinde, die tapferen Streiter zur Wehr\\
Die Stunde mag kommen, da der der letzte fällt\\
doch noch ist sie fern, die Mauer hält.

Ein glorreicher Ritt, viel Ruhm und Ehre\\
doch wenige kehren zur Burg zurück\\
Einer, nach dem Ander' n, vernichtet, erschlagen\\
Die Verteidiger fallen Stück für Stück.

Die Haut mit eisigem Frost überspannt\\
gefrierend, lähmt seiner Gegner Hand\\
Ein gewaltiger Hüne auf den Mauern erscheint\\
Zerschmettert Stein, Stahl und Gebein.

Sein Schwerte blut' ge Kreise zieht\\
der Maraskaner den Krieger sieht\\
Mit wildem Geschrei stürmt er los\\
verwandelt zur Echse, drei Schritt groß.

Und unter Aufgebot ihrer gesamten Kraft\\
wird der Feind geworfen zurück in die Nacht\\
Und wenn auch die Mauern in Flammen vergeh' n\\
die tapferen Streiter bleiben weiter steh' n.

Das letzte Brot wurde angebrochen\\
der letzte Balken am Tore angebracht\\
Und in Stille beginnt, vor dem Tag des Gott\\
ohne Namen, die allerletzte Nacht.

Mitternacht zieht herauf, die Burg liegt verschanzt\\
Von den Göttern geschieden, den Mut ungebrochen\\
Ein Schaben und Kratzen, im Tal aus den Gräbern\\
lebendig die Toten kommen gekrochen.

Und der Feind zieht aus, mit allen Mann\\
Die Lager bleiben des Schutzes bloß\\
Was soll es auch schaden?, Wer hält ihnen stand\\
Der allerletzte Sturm bricht los.

Und mit Gewalt brandet gegen die Wehr\\
das siegreiche, feindlich, gottlose Heer\\
Die Verteidiger steh' n, doch das Herze bang\\
Diesen Kampf überlebt keiner, noch sonderlich lang.

Die Toten am Tor, nur das flammende Schwert\\
des Magus sie von den Durchbruch abhält\\
Doch der, auf den Mauern, der Deckung entbehrt\\
durchbohrt im Hagel aus Pfeilen fällt.

Die Mauer zerspringt und im fliegend Gestein\\
ein riesiger, nachtschwarzer Dämon tritt ein\\
Gegen dies Wesen der Höllen, hilft Wehr nicht, noch Trutz\\
Nur der Tempel der Göttin bietet noch Schutz.

Da tritt Yppolita ihr Schwert in der Hand\\
hervor, alles starr, die Zeit wie gebannt\\
Die Amazonenkönigin stimmt an\\
den Gesang, beginnt ihren letzten Kampf.

Ihr Ross unterdessen trabt zum Geweihten\\
Peraines, von den Göttern gelenkt\\
Nimmt ihn auf, trat los, und entschwindet\\
gen Himmel, mit Flügeln ward' s beschenkt.

Auch die Königin muss dereinst nun fallen\\
unergründlich die Launen des Schicksaals walten\\
Doch vermochte sie, ein paar Minuten\\
das Verderben noch aufzuhalten.

Ihr Opfer indessen nicht ohne Sinn\\
Ein rauschen am Himmel und feuriger Odem\\
Der Geweihte kehrt zurück\\
auf einem Drachen kommt geflogen.

Steine glühen, die Festung, sie schmilzt\\
Dämon und Heer in Flammen vergeh' n\\
Nur den Tempel lässt das Geschöpf\\
der Göttin, heil und unversehrt steh' n.

Die Schlacht gewonnen, kein glorreicher Sieg\\
und doch werden Barden es einst besingen\\
Auf Kurkums Feldern indessen klagend\\
Totenlieder leise verklingen.

\section{Die Tagebücher}

\subsection{Von Klammsbrück nach Beilunk nach Firnen Wulfgrimm}

\paragraph{2.Ingrimm}
Der Tag, da der Sturm über Tobrien und unsere kleine Akademie herein bricht, mag noch fern scheinen, doch die Düsternis greift bereits mit ihren langen Fingern nach uns, um uns bereits in einen Vorgeschmack der kommenden Finsternis zu hüllen. Erwen, der unverbesserliche Schlaumeier und Missionar in bezüglich aller Themen, von denen man noch nie etwas hören wollte, erschien nicht zur morgendlichen Mahlzeit. Zuerst witzelten wir noch über seinen Verbleib und Lobten die uns nun gegönnte morgendliche Unbelehrtheit, doch nach einer Weile begannen wir uns ernsthaft zu sorgen. Die junge Elevin, die wir schickten, um ihn zu holen, traf ihn nicht an. In tiefster Sorge machten wir uns auf die Suche, erst innerhalb der Burg, dann auch außerhalb. Die beiden Bauernfamilien konnten uns nicht weiterhelfen, doch einer der jungen Jagdgesellen, die Ragnos unter seine Fittiche genommen hatte, brachte uns schließlich einen blutigen Stofffetzen, den wir Erwens Gewand zuordnen konnten. 

Bestürzt folgten wir ihm zum Fundort, im Wald, in der Nähe des Tores. Ragnos folgerte aus den Spuren, die er zu finden vermochte, dass jemand, mit kleinen Füßen, von der Mauer her kam, und weiter Richtung Felsen ging. Eine magische Analyse ergab, dass es ein Magus gewesen sein muss. den am Ort des Zusammentreffens mit dem Hesindegeweihten, waren die Reste eines Silentiumzaubers zu erkennen. Im folgenden Verlauf der Spuren, mischte sich eine dämonische Komponente hinzu. Wir folgten der Spur bis zu einer kleinen Höhle. Neben Erwens zerbrochenem Amulett fanden wir dort Reste eines Heptagramms, und die Analyse inklusive weiter Vermutung, legt nahe, dass dort ein mächtiger Magier (er benutzte keinerlei Paraphernalia), einen höheren Diener des Armatheroth beschwor. Was im Anschluss mit Erwen und seinem Entführer geschah, bleibt uns rätselhaft. Iliricon war außer sich vor Trauer und Wut, doch ich denke er wird sich wieder fassen.

Die Ereignisse des Tages hatten sich damit jedoch bei weitem nicht erledigt. Kurz vor dem Mittagsmahl, traf Flammenzunge, nun im Ordinat eines Rondranovizen, ein, uns eine Botschaft Ayla vom Schattengrunds zu überbringen, die uns einlud schnellst möglichst nach Perricum zu kommen.

Iliricon war nur mäßig begeistert schon wieder allein die Führung zu tragen, und Temyr war schon kurz davor, ob des erneuten schweren Verlustes, das Vertrauen in Iliricon zu verlieren, doch es bleibt alles wie bisher. Iliricon übernimmt die Leitung und wir reisen übermorgen in der Früh ab. Wir werden mit Borgil und der Kutsche, bis nach Beilunk reisen und dann mit einem Schiff nach Perricum übersetzen.

\paragraph{14. Ingrimm}
Wir haben Beilunk erreicht, ab morgen bin ich endlich mal wieder auf See.

\paragraph{18. Ingrimm}
Heute Morgen sind wir im Haven zu Perricum eingelaufen. Eine beeindruckende Stadt. Mit einer deutlich bemerkbaren Truppenpräsents.
Flammenzunge hat uns sofort zur Löwenburg gebracht, doch dort mussten wir feststellen, dass Ayla noch nicht aus Gareth zurück gekehrt sei. -- und nur noch zwei Tage bis zum Sturm -- Beim Mittagsessen haben wir Oleg auf unseren neusten Stand gesetzt (und die anderen Geweihte die unseren Ausführungen gespannt folgten). Den restlichen Nachmittag, haben wir uns mit hohen Militärs und Reichsdienststellen (KGIA und Flottenadmiral) herumgeschlagen, die weder bereit waren uns zu glauben, noch uns auch nur einmal richtig zu zuhören. Alles in allem, nicht gerade erfreulich. Auf dem Rückweg holte uns Phexens Rache, für unser Glück auf Andalkan, wieder ein. Ein Kutsche, ganz in Schwarz, fuhr uns fasst über den Haufen, Temyr brach sich eine Rippe, und in einer nahen Nebengasse fanden wir eine verwüstete Apotheke. Der Apothekarius war mittels eines Fulminiktus zu Strecke gebracht worden. Und wie uns seine Gehilfin mitteilen konnte, hatte er zu Letzt eine schwarzhaarige Elfin bediente, die frevlerische Güter verlangte, und wahrscheinlich Meteoreisen, Mondstein, Onyxe und Alraune entwendete, nach dem sie den Apothekarius getötet hatte. Vielleicht ist sie die Insassin dieser schwarzen Kutsche? In zwischen sitzen wir wieder wohl behalten auf der Löwenburg. Rezzanjin hat noch eine Unterredung mit einem Recken von Rodenstein, der unsere Taten aufschreiben möchte, und nun unseren Reisebericht noch einmal wollte.


\paragraph{19.Ingrimm und noch einen Tag bis zum großen Sturm}
Ayla wird morgen zurück erwartet, solang bleibt uns nur Untätigkeit. Ragnos hat sich schon früh aufgemacht, die Straßen der Stadt zu erkunden und auch Temyr und ich wollten uns noch einmal in die Stadt begeben, um zu fragen, ob die Stadtwache, weiteres Entwendetes feststellen konnte. Am Tatort fanden wir eine Wache, die uns jedoch nichts Neues berichten konnte. Temyr philosophierte noch über die Verwendungsmöglichkeiten, der geraubten Güter, als zwei Adepten der hiesigen Magierakademie die Straße herab kamen, uns bemerkten und mit uns ins Gespräch kamen. Wir erzählten von unseren Erkenntnissen bezüglich des Überfalls, und Temyrs Vermutung der Verwendung der entwendeten Güter für Zwecke der Beherrschung. Gewahr dieses Details, baten uns die Beiden, sie zu ihrer Akademie zu begleiten.

(Gestern die Soldaten, heute die Rohalswächter -- wenn man unter Strafe gesucht wird, weiß jeder davon, wird man aber wieder begnadigt, kriegt das niemand mit.)

Nach dem das Missverständnis, mit dem Rohalswächter, an den Akademiepforten geklärt war, berichteten wir der Spektabilität zu Perricum von unseren Analysen und Schlussfolgerungen, bezüglich des Einbruchs, und seine Spektabilität, klärte uns auf, warum uns die beiden Adepten anschließend hier her geschickt hatten: Zusammen mit einer jungen, begabten Magistra, waren der Akademie vor wenigen Tagen zwei sehr teure Bücher, eine Originalausgabe über Beherrschungszauberei, von Galotta persönlich, und die Laboratorien Galottas, abhanden gekommen. Die Ingredienzien, die aus der Apotheke gestohlen wurden, hängen vermutlich mit den Diebstählen aus der Akademie zusammen.

Die junge Magistra hat wohl einen Bruder, der eine Reederei betreibet. Wir werden ihm nach einem Mittagsimbiss einen Besuch abstatten, vielleicht kann er helfen, auf zu klären, was mit seiner Schwester geschehen ist.

Der ebenfalls noch recht junge Leomar von Schafspitz, den sein Reichtum sichtlich in die Breite getrieben hatte, empfing uns freundlich, wurde aber unruhig, als wir auf seine Schwester zu sprechen kamen. Bannbaladin sei Dank, erzählte er uns, das seine Schwester sehr merkwürdig geworden sei, seit sie vor zwei Tagen zu ihm gekommen sei, um ihn um Hilfe zu bitten. Seit dem versteckt er sie in seinem Keller. Wir baten, zu ihr gelassen zu werden und er gewährte uns den Wunsch. Schon durch die massive Holztür hindurch konnte ich die Lage der Situation deutlich erkennen. Ein niederer Besessenheitsdämon, versuchte noch immer endgültige Herrschaft über den Geist der Magistra zu erlangen. Sofort schickte ich den Reeder los, Hilfe aus der Akademie zu holen. Mittels Gardiani geschützt, betraten Temyr und ich den Raum, um zu helfen, oder Notfalls den Dämon lang genug auf zu halten. Der Kantus Pentagramma konnte mir hier nicht mehr helfen, wollte ich verhindern, dass die Magistra mit in den Niederhöllen verschwand. Der Dämon hielt inne, als wir den Raum betraten. Dann wand er sich mir zu, beziehungsweise Zulhamid, den er stets mit Groß Kophta ansprach, ich antwortete ihm. Hinterher erfuhr ich in der Akademie, dass es mir nie hätte möglich sein sollen, den Dämon zu verstehen. Wahrscheinlich liegt es an Zulhamid. Bis die Hilfe aus der Akademie eintraf, befragte ich den Dämon, und erfuhr, dass sich seine Beschwörerin zu Letzt in einem fürstlichen Gasthaus, am Marktplatz aufgehalten habe, und dass sie eine südländische mächtige Magierin gewesen sei.

Mit Hilfe zweier Adepten, brachten wir den Dämon und damit auch die Magistra, durch den Limbus in ein Entschwörungszimmer in der Akademie. Temyr und ich brachen sofort auf, unsere Gefährten zu suchen und dem Gasthaus einen Besuch abzustatten.
Im Gasthaus fanden wir zwar nicht mehr die Beschwörerin, denn sie war bereits abgereist. Doch wohl fanden wir vergessenen Hinterlassenschaften, wie etwa einen Brief Galottas, an Azariel Scharlachkraut. Auch konnten Sie und ihre halbelfische Schülerin eindeutig der schwarzen Kutsche zu geordnet werden. Dem Brief zufolge würde sich Azariel und ihre Begleiterin nach Beilunk begeben, am Arvepass mit Dienern Galottas zusammen treffen, um ein mächtiges Wesen zu besiegen. Anschließend würden sie dort ein Artefakt finden, und einen neuen Ogersturm herauf beschwören.

\paragraph{19. Ingrimm (späterer Eintrag)}
Im Laufe des Abends, machten wir noch eine sehr merkwürdige Begegnung mit einem mittelreichischem Söldling. Wir hatten gerade Ragnos, auf dem Marktplatz gefunden und wollten uns zurück zur Löwenburg begeben, als ein gut, aber einfach gekleideter Mann, wohl im goldenen Höhepunkt seiner Jahre, in Temyr stolperte. Er fing sofort an, sich wortreich und mit wohlgewählter Sprache zu entschuldigen und schaffte es tatsächlich, Temyrs tulamidisches Gemüt ausreichend zu besänftigen. An seiner Seite fiel mein Blick, auf ein wohlgearbeitetes Langschwert, was ihn in dieser Stadt, eindeutig als Krieger, oder Adeligen auswies, oder als Betrüger, was mir jedoch unwahrscheinlich erscheint. Eine sofortige Analyse, seiner arkanen Struktur, zerstreuten meine Bedenken, bezüglich eines möglichen, borbaradianischen Spions, restlos, und offenbarten deutlich, den häufigen Gebrauch von Metall. Wohl ein ausgebildeter Kämpfer, der das Tragen von Schwert und Eisengewand gewohnt ist. Der wahre Entschuldigungsschwall gipfelte schließlich in einem Getränkehandel, der sich, Dank Temyrs wahrhaft tulamidischem Geschick, durch eine freien Runde in der kleinen Schenke, zum Eberkopf, auszeichnete. Das Geplauder ergab, dass er fahrender Krieger sei, der sich als Geleischutz für Kaufläuter engagiere. Die Rondrakirche habe sein Schwert, ohne Beitritt, abgelehnt, und jetzt warte er auf einen wohlhabenden Mann, der Beistand wohl gebrauchen könne. Ausgebildet wurde er in Rommylis. Trotz allgemein gemütlicher Erzählrunde, kam Rezzanjin nicht umhin, sich mit ihm, er nannte sich Irian Falkberg, zu einem Duell morgen Vormittag zu verabreden. Zu späterer Abendstunde trennten sich unsere Wege wieder, und wir kehrten ohne weitere Zwischenfälle zurück zur Rondrafeste und damit zu unseren Schlafstellen.
Wir hatten uns gerade erst zur Ruhe gelegt, als ein Tumult auf dem Hof uns auch schon wieder aus den Betten trieb. Ihre Erhabenheit Ayla vom Schattengrund ist zurückgekehrt. Wir versuchten noch zur gleichen Stunde unsere Audienz zu erhalten, doch man verwies uns auf Morgen nach der ersten Messe.

\paragraph{20.Ingrimm}
Bis gestern ward der Sturm heraufgezogen, ab heute sollte er mit aller Macht über uns hereinbrechen.
Nach der Messe kamen wir, wie vereinbart, in die Sakristei (, eher eine Waffenkammer, als eine klerikaler Gewandunsstube.) Wir legten in groben Zügen unsere Erkenntnisse, über den beginnenden Krieg und die jüngsten Ereignisse dar. Trotz deutlichem Missmut in einigen führenden Positionen, wurden wir von ihrer Erhabenheit zur obersten Krisensitzung der Rondrakirche herbeigezogen, um von unseren Erfahrungen zu berichten.
Jäh unterbrochen wurde die Runde, als Ayla und Toran göttliche Botschaft aus Mendena erhielten. Binnen weniger Stunden war die Hauptstadt Tobriens überrannt, jede Gegenwehr vernichtet worden.
Der Rat hat getagt, die Erhabene hat entschieden. Vor ca. 4 Jahren hatte die Rondrakirche ihre Hilfe im Kampf gegen Borbarad versprochen, nur waren sie ohne einen konkreten Feind nicht handlungsfähig gewesen. Jetzt aber wo sich der Schatten erhebt und Gestalt annimmt, wird sich auch die Kirche der Leuin erheben. -> Die Rondrakirche ruft den Kriegszustand aus.
Ayla hat sich zum Gebet zurückgezogen. ihre Leibgarde hält Stundenwache, in ca. sechs Stunden wird der Krieg mit einem großen Götterdienst zu Ehren der Leuin begonnen. In frühestens einem halben Jahr wird die Streitmacht in voller Stärke ausgehoben und bereit sein.
Uns übergab Ayla indes ihr Siegel, um die Vertreter des Militärs und der KGIA zu überzeugen, und in Kenntnis zu setzen. Anschließend wird sich Toran auf seine Rede vorbereiten, die er in der großen Messfeier halten soll.
Während die Rondrakirche ihre Armeen aus hebt, werden wir Perricum wieder verlassen. Morgen in der Frühe werden wir aufbrechen, in Richtung Arvepass um zu versuchen Azariel aufzuhalten. zu Pferde sollte es uns eigentlich möglich sein, die Kutsche einzuholen.
Den Söldling Irian, der eine nicht schlechte Figur beim Duell gegen Rezzanjin machte, konnte ich überzeugen uns sein Schwert zur Verfügung zu stellen. Nach Arngrimms Tod können wir ein gutes Schwert in unseren Reihen wohl gebrauchen. Und das uns so ein Glücksritter folgt und seinen Anteil kassiert, scheint mir wahrscheinlicher, als dass die Boronkirche uns stetig etwaige Begleiter gewährt.

\paragraph{21. Ingrimm}
Wir sind den ganzen Tag geritten. Die Pferde sind erschöpft das Dorf in dem wir rasten nennt sich Dergelsmund. Der Kutsche sind wir noch auf den Fersen, die meisten, die wir fragten hatten die merkwürdigen Reisenden bemerkt.
Morgen werden wir uns der Küstenlinie folgend in die Trollzacken begeben, und müssten gegen Abend den Arvepass erreichen.

\paragraph{22. Ingrimm}
Begegnungen sonderbarer Art scheinen nicht selten zu sein dieser Tage. Im Gegensatz zu Irian, war die betreffende Person dieses Mal jedoch nicht gewillt uns zu begleiten.
Noch recht zu Anfang des, in langsamem Tempo sogar bereitbaren Pfades ins Gebirge, hörten wir Rufe und konnten einen kleinen Steilhang hinab, eine zerschellte Kutsche und eine Person davor ausmachen. Wie sich herausstellte war die Kutsche tatsächlich jenes schwarze Exemplar, dem wir bisher gefolgt waren, die Insassen allerdings, waren auf den Zugpferden entschwunden. Die Person, die verzweifelt um die Kutsche herum lief, war Arves vom Arvepass. In Halsbrecherischem Tempo war die Kutsche oben auf dem Pfad heran gesaust, und vom Weg abgekommen. beim Sturz in die Tiefe, hatte sie seinen Pagen erwischt. Während Toran und Peraine sich um den Verwundeten Pagen kümmerten, erzählte Arves von seinem früheren Leben. Wie sich herausstellte kannte er Xeraan, dem er einst bei einem Zwischenfall in Kurkum, einer Amazonenfeste in den beilunker Bergen, begegnet war.

Nun nächtigen wir im Freien, irgendwo in den Trollzacken, nicht unweit vom Arvepass. Ich werde die dritte Wache halten, Irian übernimmt die erste, Rezzanjin hat die zweite. Wir werden sehen, was der Morgen bringt.

\paragraph{23. Ingrimm}
Der Morgen brachte weiter kaum spektakuläreres, als die Natur an sich eh schon darstellt, doch Boron gönnte uns nicht die Ruhe der Nacht, statt dessen hielten die Götter sowie die Erzdämonen, einiges an Überraschung bereit für die Nacht.

Es hatte gerade die zweite Wache begonnen, da riss uns ein lauter Schrei aus dem Schlaf. Hoch in den Wolken, über dem Nachbartal, erhob sich in güldenem Glanz ein Greif, verwickelt in einen Kampf mit Irrhalken, vier an der Zahl. Der Prajosdiener, der den Pass bewacht, den Legenden nach Herophan genannt, streitend gegen die dämonischen Diener Galottas, die wie er schrieb, Azariel und ihren Begleitern helfen sollten. Wir griffen zu all jenem, was wir zu brauchen glaubten, und eilten los, dem majestätischen Geschöpf im Kampf gegen die Finsternis bei zu stehen. Irian, der im Kampf mit dem Schilde bewandert zu seien scheint, habe ich in aller Kürze den Artefaktschild überantwortet, und die Verwendung erläutert. Wenn wir auf Zauberer treffen, kann es nicht schaden, wenn es jemanden in unseren Reihen gibt, der diesen Schild zu führen weiß. Auf halber Strecke schnitten uns zwei der Irrhalken den Weg ab und kesselten uns ein. Nur unter gemeinsamem Aufbieten all unserer Kräfte gelang es uns schließlich sie zu vernichten. Unterdes hatte sich auch der Greif seiner Angreifer entledigt, und kam zu uns hernieder. Die Berge schienen zu vibrieren, und der Fels schien zu leuchten, als er sprach:

Keiner von euch ist frei von Schuld! Keiner ist frei von Schuld, außer Prajos. Ein Schatten erhebt sich im Osten und seine Diener sind siegessicher, doch ihr steht gegen die Finsternis. Ihr habt euch wacker geschlagen Sterbliche."

\subsection{Irian von Rabemund nach Firnen Wulfgrimms}
\paragraph{3. Rahja}
Gegen Mittag haben wir Beilunk erreicht. Unser Gesuch dort bei Hofe vorsprechen zu können verlief weniger informatif und glimpflich, als gedacht. Zwar hörte man uns an, doch Irian machte sich des Vergehens, der Beleidigung hochrangiger Personen schuldig und wurde verhaftet, und ich bin im Audienzsaal direkt Saldor Foslarin in die Arme gelaufen und befinde mich nun in einer Zelle im Hochsicherheitstrakt der hiesigen Magierakademie. Immerhin steht mir ein weiterer Prozess bevor.

\paragraph{4. Rahja}
Heute Vormittag ist der Prozess gelaufen. Geglaubt haben sie mir nicht wirklich, aber sie haben mich vorerst frei gestellt. Bis zum allaventurischen Konvent am 15. Ingrimm 1020 BF habe ich Beweise zu liefern, die meine Unschuld bezeugen können.

Gen Mittag wollten wir unseren neuen Gefährten Irian Falkberg aus dem Schuldturm holen, und bekamen statt dessen einen KGIA-Agenten Irian von Rabemund, der auf uns angesetzt worden war um ein Auge des Reiches auf uns zu haben, besser gesagt, auf den Maraskaner und den Expaktierer. Eine Befragung einer der Begleiterinnen Azariels, die der Magierakademie in die Hände gefallen war, brachte keine neue Erkenntniss. Morgen werden wir uns gen Ogerforst aufmachen.

\subsection{Das Ogerkreuz nach Firnen Wulfgrimm}
\paragraph{5. Rahja}
Unser Zwischenstopp lässt uns die Nacht in Grünau verbringen, ein Dorf, wo sich Fuchs und Hase "Gute Nacht" sagen. Alle außer den letzten vier Uralten hatten das Dorf längst verlassen, um gen Westen zu flüchten. Uns als Fremde beäugte man kritische, doch man gewährte uns Obdach für die Nacht. In Samahan so erzählte man uns wurde vor etlichen Jahren ein Rondratempel nieder gebrannt, von setsammen Kultisten, die das Dorf personell durchsetzt hatten. Inzwischen sollen dort aber wieder normale Leute leben. Verspricht ein interessanter Ort zu werden. Mitten im Ogerforst gelegen wird sich dort in der Nähe zuindest, auch Azariel und ihre Begleiter finden lassen.

\paragraph{6. Rahja}
Es war tatsächlich noch ein vollständiger Tagesritt bis Samaham. Wir wollten uns schon nach einem geeigneten Ort für ein Nachtlager umsehen, als uns ein paar Holzfäller mitteilten, das wir bereits eine halbe Stunde vor Samaham wären. Jetzt sitzen wir im hiesigen Gasthaus am Marktplatz, "Zum Schwert und Schild" geheißen, und konsumieren abartig teures, und ebenso abartiges Safrangebräu, eine Art Gewürzwein. Die Besitzeren der Taverne konnte uns neben einem Zimmer für die Nacht, auch bezüglich dr Gesuchten weiterhelfen. Sie hatten ebenfalls hier übernachtet, und sind vor etwa zwei Tagen in den Ogerforst aufgebrochen. Um genaueres über die dierekte Umgebung und den Ogerforst zu erfahren, verwies sie uns an Karen, ein "Waldläuferin", die zur Zeit im Dienste der Safranhändlerin die Karren sicher durch den Forst geleitet. Wir werden sie morgen einmal aufsuchen.

\paragraph{7. Rahja}
Wir haben Karen schließlich bei der Safranhändlerin angetroffen, aber sie konnte uns auch nicht weiterhelfen. Unser Problem hat sich bisher zwar immer noch nicht gelöst, wurde aber von einem größeren überlagert. Noch in den Morgenstunden kam ein Amazone schwer verwundet angeritten, und berichtete vom Fall ihrer Heimatburg "Löwenstein", und dass die auf dem Weg nach Kurkum sie, ihre Königin zu warnen. Ihr folge ein Banner Söldlinge auf den Fuß, und der Heerwurm unter Führung eines barbarisch gekleideten Hühnen sei sicher auch bad hier. Nach dem Toran ihre Wunden geheilt hatte, versprach sie, uns zu helfen die Dörfler zu beschützen, auf dem Weg durch den Ogerforst. Diese wollten versuchen sich in Höhlen dort zu verstecken. Bis alle bereit waren, brach schon der Abend an, aber wir zogen trotzdem schon etwa drei Stunden in der Forst hinein. Eine halbe Stunde vor der, für das Nachtlager angepeilten, Lichtung schälten sich etwa 20 Oger aus dem Forst. Sie verlangten die Keule des Azag zurück, oder alle fressen zu dürfen. Eine inspektion des Dorfes, durch Temyr und mich ergab folgende Lage: Das Banner, teils beritten, eingetroffen. der Hühne unter ihnen. Azaril, inklusive Galotta höchst selbst, vor Ort. Die Ogerkeule in ihrem Besitz, als Bestandteil des Rituals. Die zurück geblieben Dörfler, alle in Gefangenschaft. Wir werden uns nun zusammen mit dre Amazone, sie nennt sich Roana, und einigen der Oger zum Dorf begeben, um zu sehen, was sich machen lässt.

\paragraph{8. Rahja}
Es lief alles mehrfach nicht so wie geplant, aber wir verzeichnen einen Sieg: Die Hälfte des Banners aufgerieben, Azariel und Galotta geflohen, der Hühne tot, fünf der Oger gestorben, wir und die Amazone am Leben. Die Ogerkeule nun in unserem Besitz.

\subsection{Die Ankunft in Kurkum rekonstruiert von Iliricon Tannhaus}

Stichpunktartige Inhalts Summierung meinerseits. Wer auch immer das zu schreiben hat.

- Ritt zu den Amazonen.
- Evakuierung der Dörfer
-> inkl. Aussendung des Boten zu den Zwergen der umliegenden Berge.
- Ausritt zur Erkundungsmission
- gescheiterter Erkundungsversuch

Schon die Tatsache, dass wir es zurück bis nach Kurkum geschafft haben, scheint mir ein Wunder.
Dämonische Hunde hetzten uns drei Tage lang, Tag und Nacht. Vor Gegenwehr sofort fliehend, zermürbten sie uns mit unzähligen kleinen und auch größeren Wunden, Schlaflosigkeit und eine anhaltende Angst vor ihrer Wiederkehr. Im Tal vor dem Pass gen Kurkum trafen wir auf einen Feind, der bald besser, als Ragnos, es verstand mit dem Bogen umzugehen, und hätte ich nicht solch treue Freunde, die Stunden nach mir gesucht haben, läge ich wohl immer noch ohne Bewusstsein im Wald, oder inzwischen gar im Lager der Feinde.

Doch Dere hat uns wieder. Die eiskalte Meute, des Herren des ewigen Eises, konnten wir, wenn auch knapp, noch besiegen, und ein wahres Wunder der Peraine ließ uns, in Torans heilenden Armen, schnell genesen.

\subsection{Vor der Belagerung nach Firnen Wulfgrimm}

\paragraph{20. Rahja}
Ich habe mich mit entkräftenden Stunden in astraler Meditation, soweit an den Rande der Erschöpfung gezaubert, dass ich mich zur Mittagsstunde bereits zur Ruhe legen konnte. Frisch erwacht, war ich aber noch lange nicht fertig. Bis mein Stab wieder vollständig präpariert, und ich ausreichend astrale Kraft aufgenommen hatte, war ich der Ohnmacht durch gänzliche Ermüdung, schon so nahe, wie Langschiff dem zerschellen an den Klippen bei Pieroth , während des Meerjungferngesangs. Das reichhaltigere Mahl der Amazonen, welches die Flüchtlingsrationen deutlich zu übersteigen scheint, war gerade recht, um mich ausreichen zu stärken.

Meine Gefährten, die ich, bis auf Irian, beim Mahle vermissen musste, waren über die Feuerstellen der Flüchtlinge verteilt. Toran, der seit des Perainwunders, und seiner allgemeinen Hilfsbereitschaft, großen Respekt der Flüchtlinge, besaß hatte sich der Einladungen wohl kaum Erwehren können. Zudem gibt sein unerschütterliches Vertrauen in die Götter, den normalen Menschen, die nicht für den Kampfe leben, den Mut, der ihnen längst verlorenen gegangen wäre, in dieser hoffnungslosen Lage. Rezzanjin speiste bei seiner Einheit, ein paar junge Männer, der umliegenden Dörfer, die bei der Verteidigung helfen wollten, und nun bei Rezzanjin in die Lehre gingen. Ragnos muss ich gestehen, konnte ich bisher nicht finden, aber er wird wohl auf der Burg weilen. Am Nachmittag trainierte er noch einige junge Burschen im Bogenschießen. Temyr fand ich etwas abseits an einem Feuer, allein mit einer jungen Frau, die noch nicht mal dreißig Lenze zählen kann (dieser alte Schamör, aber Hesinde allein weiß, vielleicht verschenkt er ja doch tatsächlich noch einmal sein Herz). 

Etwas abseits von Temyr und Abendbegleitung, bemerkte ich im Schatten eine junge Frau, vielleicht vier oder fünf Lenze jünger als ich, die die Szenerie am Feuer zu beobachten schien. Ich muss gestehen, dass mich dieser Umstand misstrauisch stimmte. Eine kurze Analyse offenbarte eine eindeutige, geschulte, wenn auch schwache, astrale Begabung. Wie ich im Gespräch herausfand, ist sie magisch begabte Alchimistin vom Bund des roten Salamanders zu Festum. Ihrer Mithilfe verdanke ich die Heilung gestern Abend. Zutiefst in Scham wegen meines anfänglichen Misstrauens, verabschiedete ich mich nach kurzer Zeit. Meine ermatteten Glieder lechzen geradezu nach erholsamem Schlafe.

\paragraph{21. Rahja}
Als ich gegen Morgen unser Quartier verließ, war ich nur halb so ausgeruht, wie gestern Abend noch erhofft. Schon längst nicht mehr auf Dere weilend, hetzte mich die eisige Meute dennoch durch die endlosen Weiten jener Wüste, in der mir der Herr der ewigen Rache gegenüber trat.

Zu allem Überfluss fiel das erste Grußwort diesen morgen, durch den Waidmann, der sich nun schlafen legte, in einen Satz mit: "Sie sind eingetroffen. Das Heer hat das Tal erreicht.", wirklich ermunternd. Auf dem Weg zum Frühstück begegneten mir Toran und Daena, die Alchimistin von gestern Abend. Sie waren unterwegs, den Nuttierstall auszumisten, um das Lazarett dort hin zu legen. Toran bot mir an ihnen zu helfen, doch ich lehnte ab, und begab mich erst mal auf die Suche nach einem Morgenmahl. Im Anschluss wechselte ich mein schlichtes Reisegewand gegen tauglichere Lederkleidung, und begab mich zu den Ställen um tatsächlich ausmisten zu helfen. Die Anwesenheit der jungen Dame, spornten mich an meine schlechte Laune herunter zu schlucken, und es gelang mir erstaunlicherweise bei der Arbeit, und der netten Unterhaltung meinen Traum und allgemeine Lage zeitweilig zu vergessen. Gegen Nachmittag stieß Temyr zu unserer Arbeitsgruppe hinzu, er war des Apfelzählens überdrüssig geworden.

Als ich Daena am Abend nirgends finden konnte beschloss ich mir eine stille Ecke zu suchen, um ein wenig alleine zu grübeln.

Wer sucht der findet nicht, wer die Suche aber aufgibt wird gefunden werden. Im Laufe des Abends kam Daena auf mich zu. Sie bat mich etwas fragen zu dürfen. Ich freute mich über ihre Anwesenheit, und gewährte ihr ihre Bitte nur zu gern. Oh ihr Götter, was treibt ihr doch für Schabernack mit den Sterblichen. Sie wollt mehr über Temyr erfahren. Im folgenden blieb ich wohl nicht ganz redlich, obgleich ich nicht log, doch ist sie mir zu sympathisch, um sie der Ungewissheit um Temyrs Liebschafften auszusetzen, und so ging ich in meiner kurzen Umschreibung vielleicht etwas zu hart mit ihm ins Gericht, was Dana, zur Vergrößerung meiner Schuldigkeit, missmutig zu stimmen schien. (Möge Rahja geben, dass da noch nicht alles verloren ist.) Ohne jegliches Gefühl für die Situation, wie mir nachträglich scheint, und getrieben von reiner Neugier, bat ich sie im Gegenzug, mir von ihrem Leben zu erzählen, sie tat dies bereitwillig, und es gelang mir überdies sie für kurze Zeit von ihrem aufkeimenden Kummer abzulenken:

Als Tochter einer Magae und eines Magus zu Festum, war sie zu großen Enttäuschung ihrer Eltern kaum magisch begabt. Ihre geringe Begabung jedoch schaffte sie sich zu bewahren in der Ausbildung zur Alchimistin. Nach dem Tod ihrer Mutter an Pocken, zog sie aus sich der Heilkunst zu widmen. In diese Tal verschlug es sie auf der Suche nach exotischen Kräutern. Seit zwei Wochen ist sie nun hier und kann, Dank des feindlichen Heeres, nicht mehr weg.

Ich versuchte ihr etwas aufmunternd zuzusprechen ob ihres Schicksaals und des unergründlichen Schiedsspruches der Götter. Ich denke es gelang zumindest ansatzweise. Sie verabschiedete sich recht bald darauf und auch ich werde mich nun zur Ruhe legen.

\subsection{Die Belagerung Kurkums nach Firnen Wulfgrimm}

\paragraph{22 Rahja}
Das Heer hat sich geteilt, und die Burg mit drei Lagern umstellt. Irian und Toran vermuten, dass die nächtlichen Aktivitäten im Lager auf Untote zurückzuführen sei, die das Tageslicht scheuen. Denn die Borongeweihte von Waldheim, sprach von einer Vision dies bezüglich. Doch gestärkt im Vertrauen auf die Macht der Götter, werden wir uns jedem Problem stellen, das die Niederhöllen uns entgegen schleudern. Denn ein jeder von uns ist ein kleiner Stein, der Macht des Bösen in den Weg gelegt.

Früh bliesen die Hörner heute Morgen zum Göttinnendienst im Rondratempel, und die Göttin selber offenbarte sich uns und stärkte unseren Mut.

Toran beklagte sich über mangelnde Kräuterversorgung, und ich beschloss Temyr aufzusuchen, um unseren Beitrag am Überleben der nächsten Tage, voran zu treiben. Tatsächlich störte ich Temyr schon zu dieser noch recht frühen Stunde bei einem Tête a Tête mit seiner jungen Abendbekanntschaft, Larona wie er sie mir vorstellte. Ich konnte ihn allerdings loseisen, und gemeinsam begaben wir uns in unseren Turm, einen Dschinn des Humus zu bitten uns einige Kräuter zu gewähren. Der Dschinn wirkte fast schon beleidigt ob dieses nichtigen Auftrags, doch angesichts der schieren Unmöglichkeit für uns, in diesen Tagen, ausreichende Mengen an Kräutern zu finden, erklärte er sich bereit, und überschüttete uns förmlich mit einer Masse an Kräutern der gewünschten Diversität, um selbst hierbei seine Macht zu demonstrieren.

Ragnos beschäftigte sich indes mit weiterer Ausbildung seiner Schützlinge. Ähnlich Rezzanjin, der allerdings seinem völlig erschöpftem Haufen eine kleine Pause gewähren musste. Irian beschäftigt sich bisher mit dem Sohn des Vorstehers von Kiesfurten, der ganz begeistert vom Rittertum, bereit ist jedes Training zu absolvieren. Möge der Übermut des Jungen ihn nicht das Schlachtenschicksaal ereilen lassen.

Am Nachmittag kam Bewegung in das Hauptlager im Effert. Ragnos und Teresa konnten eine kleine Gruppe ausmachen, die zum Lager hinzu stieß. Ein kleiner hagerer Mann in Roben. Gefolgt von einem kleinen Jungen , und sechs Kriegern in pechschwarzer Rüstung. Ich vermute, dass es sich um Xeraan und einen seiner Legionäre handelt. Ich saß zu frühen Abendstunden vor dem alten Turm und ergab mich schwermütigen Gedanken, als Daena zielstrebig vorbei kam. Ich fragte sie wohin sie eile, und sie berichtet mir, sie wolle ausziehen eine Alveranie zu finden. Temyr habe ihr den Geheimgang verraten, und sie wollte so eben gehen. Dass sie Temyr fragte, der ihr bis dato noch nie begegnet, und nicht mich, der versuchte sich offen und freundlich zu geben, schmerzt schon ein wenig, doch da ich nicht vermochte sie von diesem lebensgefährlichen Unterfang abzubringen, wünschte ich ihr alles Glück der Menschheit, und den Segen aller Götter. Ich versprach nach ihr zu suchen, so sie nicht binnen zwei Tagen wiederkehre, und ich werde an der Luke wachen für den Fall, dass sie käme. Möge Phex ihr gnädig sein.

\paragraph{23. Rahja}
Die Götter schenkten mir schreckliche Weisheit. Im Traume sah ich die eisigen Wölfe. Sie verfolgten die Alchimistin. Diese versuchte noch panisch zu fliehen, doch die Meute war bald über ihr. Ein lautes Pochen riss mich aus dem ohnehin unerquicklichen Schlaf. Das Heulen klang noch immer aus der Ferne. Leider war es nicht Daena, sondern vielmehr meine Gefährten Rezzanjin und Ragnos, die zusammen mit Irian, der noch zu fehlen scheint, zu nächtlicher Erkundung ausgezogen waren.

Meine Unruhe wächst von Stunde zu Stunde. Was ist aus Irian geworden?, den Vermutungen von Ragnos nach hat der Ritter die Wölfe im Lager aufgeschreckt. Vielleicht, haben sie aber auch tatsächlich die Alchimistin gefangen. Ich finde keinen Schlaf mehr, die Düsternis umfängt mich, wo immer ich mich befinde. Noch ein Hoffnungsschimmer und eine weite Enttäuschung. Abermals schreckte ich aus meinen lethargischen Gedanken. Ein pochen an der Luke. Dass es sich um Irian handelte, den ich schwer verwundet ins Lazarett bringe, lindert meine Sorgen nicht.

Während Temyr und Rezzanjin sich der Ausbildung des Haufens widmen und Toran die Kräuterkundigen instruiert, wage ich mich kaum von der Luke zu entfernen, könnte sie doch jeden Augenblick zurückkehren.

Aufgeregte Rufe auf den Mauern. Sie sammeln sich zum Angriff.

Irian instruierte mich über die gewonnenen Erfahrungen seines nächtlichen Ausfluges in aller Kürze. Sie beschwören einen Darai, noch an diesem Tag. Ich denke, dass sie mit dieser Bestie, die für ihre überderische Körperkraft bekannt ist, das Tor ein zuschlagen gedenken. Ein Blutopfer für die Beschwörung haben sie bereits. Wie ich mit Ragnos Helm erkennen konnte, einen jungen Mann. Sein Schicksaal ist bedauerlich, aber ich vermag ihm nicht zu helfen. Die schockierenste Erkenntnis allerdings ist, dass sie eine Opfer verhören, das sich ihnen bisher erfolgreich wiedersetze, offenbar ist der/ die Gefangene magisch gebildet, wenn auch nicht vollwertig. Bei allen Niederhöllen, ich befürchte gar, dass es sich um Deana handelt. Zur Sorge blieb mir indes kaum Zeit, ich begab mich schnellstmöglich auf das Torhaus eine Entschwörung vorzubereiten.

Zu später Abendstunde brach der Sturm über uns herein. Untote trugen eine Rampe an die Mauer. Die Söldner rückten unter großen Lederaufbauten vor. Trotz meiner guten Vorbereitung hatte ich den Darai offensichtlich unterschätzt. Er erreichte das Tor noch bevor mein Zauber zu wirken begann. Ich konnte ihn allerdings entschwören, bevor er größeren Schaden am Tor anzurichten vermochte.

Während die anderen, stark bedrängt, auf der Mauer sich ihres Lebens erwehrten. beschloss ich, die Gunst der Stunde zu nutzen und mich zum entblößten Lager der Söldner zu begeben um nach Deana zu suchen. Leider brachen sie den Angriff bereits zu dieser Zeit ab, und so blieb mir gerade noch das Lager auf magische Barrieren zu überprüfen, bevor ich mich zurück zur Burg teleportieren musste, um unentdeckt zu bleiben. Ich werde es heute Nacht noch einmal zu versuchen.

Der Kampf wurde unter geringen, aber deutlichen Verlusten gewonnen. Die Verteidigung bleibt standhaft.

\paragraph{24. Rahja}
Mein nächtlicher Ausfall war nicht vergebens, doch verlief er längst nicht wie gewollt. Entgegen meiner Annahmen entdeckte mich tatsächlich ein Wachdämon, er konnte mich allerdings nicht abhalten. In Form eines Blaufalken schaffte ich es in die Scheune, traf jedoch nur auf unheilig erwecktes ehemals totes Leben. Es gelang mir ihnen zu entgehen und durch den Schornstein des Haupthauses fiel ich beinahe in das feindliche Besprechungszimmer. Ich blieb unentdeckt, und konnte im Gegenzug einiges von Interesse belauschen. Mit der Entschwörung des Darai bin ich ihnen ein großer Dorn im Auge, und sie gedenken einen Alptraumdämon, einen Morkanen auf mich anzusetzen. Des weiteren planen ihre Kontakte in der Burg etwas größeres, müssen allerdings vorsichtig vorgehen um nicht aufzufallen. Als drittes plane der Waidmann etwas, dass in den nächsten Tagen zum Tragen kommen sollte. Und schlussendlich gedenken sie das Belagerungswaffenarsenal zu vergrößern.

Seit dem Morgenmahl greift eine Übelkeit unter den Burgbewohnern um sich. Wir haben noch keine Ahnung woher diese kommt. Ich werde Toran auf suchen und das weitere Vorgehen besprechen.

Das Essen konnte von der Vergiftung ausgeschlossen werden. Wir unterzogen das Brunnenwasser einer alchemistischen Probe, und wurden fündig. Toran hat sich zum Glück des Brunnenproblems angenommen und mit Hilfe der Göttin konnte das Gift gebannt werden. Den Betroffenen geht es zwar immer noch schlecht, doch der Schaden wird nur von temporärer Dauer sein. Wir suchen weiter nach den Übeltätern.

Ich habe einige Erkenntnisse gewonnen und konnte den Kreis der Verdächtigten engerschließen:
Lane, Irfinia und Larona habe ich geprüft und für harmlos befunden.

Allerdings gab es beunruhigende Vorkommnisse mit der Quatiermeisterin.
Ypolita informierte uns, dass jemand sich zutritt zu den Kellergewölben verschafft, und das dämonische Eisschwert gestohlen habe. Zu diesem Teil der Gewölbe haben nur ein kleiner Kreis an Personen zutritt, was die Suche der Schuldigen deutlich einschränkte. Tatsächlich zeigte sich, dass die Quatiermeisterin ... unter wahrlich meisterlicher Beherrschung stand. Es erforderte all mein Können ihren Geist zu befreien und selbst jetzt kann sie sich zwar an ihre Tat, nicht aber an die involvierten Personen erinnern. Entgegen Ypolitas Willen begab sich ... in selbstgewählte Gefangenschaft, ob ihres Verfehlens. Und die Schandtaten des Verräters setzen sich fort. Offensichtlich verfügt er über exzellente magische Künste.

Die Nacht werde ich wohl im Tempel verbringen zum Schutz vor dem Morkanen.

\paragraph{25. Rahja}
Was für ein Tag!

Der Feind, jener der es eigentlich sein sollte, verhielt sich bisher ruhig. Doch die Stimmung innerhalb der Burg drohte zu kippen. Eine Gruppe der Dörfler hatten sich unter einigen Rädelsführern zusammen gerottet und verlangte die Herausgabe eines größeren Essensanteils. Sie wetterten gegen die Amazonen, die ob der Verteidigungsaufgabe, reichlichere Portionen bekamen. Die Stimmung war mehr als gespannt, und die Amazonen waren kurz davor, die Angelegenheit auf ihre Art zu regeln, und die Wortführer und Unruhestifter zur Warnung zu hängen. Toran musste all seine Beredtheit, und die Macht seines Zeichens aufbringen, um die Fronten wieder anzunähern und eine Eskalation zu verhindern, auf das wir in der Gemeinschaft geeint, eine Chance haben dem Feind von Außerhalb entgegen zu treten.

Nach dem der Burgfrieden wieder hergestellt worden war, zogen Temyr und ich uns zurück in unser Gemach, wo wir, unbeachtet des abergläubigen Bauernpacks und der streng unmagisch ausgerichteten Amazonen, einen Dschin des Feuers beschworen, und in das Kaminfeuer bannten, bis er uns helfen kann, im nächsten Ansturm.

\paragraph{26. Rahja}
Wir haben vor der Morgenandacht ein kleines konspiratives Treffen abgehalten und Informationen über die Vorgänge der letzen Tage aus getauscht. Den Wissensstand nun wieder angeglichen werde ich mich im Laufe des Tages wieder der Suche nach dem Feind in unseren Reihen widmen, und verdächtige Personen prüfen.

Und der Verräter schlug abermals zu. Unter eindeutiger Beherrschung stehend erwischten wir einen kleinen Jungen namens Alvin wie er versuchte die Ställe anzuzünden. Ein Stall brannte nieder, doch das Feuer konnte in Schach gehalten und die Pferde gerettet werden.

Hinter der Scheune fanden wir einen Dolch, der neben nicht zu erwartenden Rußflecken, deutliche Reminiszenzen dämonisch, eisigen Ursprungs aufwies. Offenbar ein Dolch, mit welchem dem dunklen Herren des Eises geopfert wurde.
Ich habe mich zur Abendstunde unter das Volk gemischt und weitere Verdachtspersonen gefunden:

Alrike, eine Kräuterfrau aus Auenfuhrt weist eine leichte magische Begabung auf, möglicherweise verschleiert. Ich werde sie im Auge behalten. Ebenso einen alter Mann, aus Waldheim stammend.

Seit nun mehr zwei vollen Tagen ist Daena nicht wiedergekehrt, und der Verdacht, das sie jene magisch begabte Gefangene ist, wird mir beinahe zur Gewissheit. Ich werde mich heute Nacht auf die Suche nach ihr begeben, sinnvollster Weise wohl direkt im Feindeslager, dort wo ich sie vermute.

\paragraph{27. Rahja}
Es ist beginnt wohl gerade schon die zweite Stunde nach Mitternacht und ich finde immer noch keinen Schlaf. Der Gedanke an das Schicksaal dem ich sie überlassen musste, lässt mir keine Ruhe. Ich wünschte, ich könnte ihr mitteilen, dass ich es wenigstens versucht habe, und nicht aus Feigheit mein Versprechen nicht in letzter Konsequenz erfüllte. Erneut bin ich zum Lager ausgeflogen. Doch sie haben die Patrouillen verstärkt, im Limbus, wie in der dritten Sphäre haben sie Wachdämonen engagiert. Ein Großteil des Lagers ist zudem mit Hellsichtzaubern gesichert, die empfindlichst auf jegliches magisches Wirken zu reagieren scheinen. Ich schaffte es mit mehr Glück als Phex zur Verfügung stehen sollte, bis in den zweiten Lagerring, als ich, dem Waidmann ausweichend, direkt einer Patrouille in die Arme lief. Mir klingt jetzt noch das Bellen der dämonischen Hunde in den Ohren, doch Phex und Hesinde sei Dank ist mir keiner durch den Limbus gefolgt. Meinem raschen Entschwinden geschuldet, konnte ich ihnen entkommen, bevor sie die Chance bekamen meine Spur auf zu nehmen. Doch mein Befreiungsversuch ist restlos gescheitert. Sie müssen nach meiner Spähaktion vorsichtiger geworden sein und mein Missglücktes Eindringen diese Nacht, wird ihrer Vorsicht wohl kaum zur Minderung beitragen.

Trotz einer weiteren Übernachtung im Tempel, habe ich so schlecht geschlafen wie schon lange nicht mehr. Sicher hat auch der Frost, der über Nacht aufgezogen ist, dazu beigetragen. Es ist deutlich kälter geworden, und die Temperaturen haben etwas beängstigendes und lähmendes an sich.

Es ist keine natürliche Kälte, die unsere Glieder beschleicht. Dies muss das Ritual des Waidmannes sein. Der Burggraben hat bereits eine dünne Eisschicht angesetzt, und Temyr und ich hatten soeben einige Probleme damit, den Brunnen von seiner niederhöllischen Eisschicht zu befreien.

Der Feind baut seinen Vorteil aus. Ebnet sich den Weg und zermürbt uns Verteidiger. Nicht einmal Toran oder die Geweihten unter den Amazonen waren im Stande, das Fortschreiten der Kälte zu verhindern.

Außerdem zeigt sich, was die Feindeslager, die letzten Tage so intensiv beschäftigt hat. Sie haben nunmehr den Bau an zwei Belagerungstürmen vollendet. Lange wird es nicht mehr dauern, bis sie erneut zuschlagen, doch wir vermuten das uns noch Zeit bis zum ersten Namenlosen bleibt. Sie werden kommen, wenn unsere Götter am schwächsten sind. Welcher Feldherr würde daraus nicht seinen Vorteil ziehen wollen.

\paragraph{28. Rahja}
Die Nacht war zwar nicht so kurz wie die Vorherigen, dafür allerdings nur in Teilen zu genießen.

Zu mitternächtlicher Stunde weckte mich Ragnos, dem beim nächtlichen Wachgang, eine sonderbare Begebenheit aufgefallen war. Eine besonders zerlumpte ältere Frau aus den Reihen der Flüchtlinge, die sich zur späten Stunde mit der Amazone Roana traf. Er konnte zwar nicht verstehen worum es bei dieser Konversation ging, doch er war der Flüchtlingsfrau bis zu ihrem Nachtlager gefolgt, und führte mich nun dort hin. Sie ist, soweit ich das zu beurteilen vermag, wer sonst, absolut unmagisch. Wir sollten der Angelegenheit trotzdem auf den Grund gehen.

Wir haben die Flüchtlingsfrau heute Nachmittag aus dem Verkehr gezogen, und in einem der Ställe, unter nicht ganz orthodoxen magischen Methoden, die auf Zwang zur freiwilligen Antwort beruhen (der Zauberspruch Bannbaladin spielte hierbei eine nicht unerhebliche Rolle), zur Rede gestellt:

Ihr Name ist Alena, sie scheint in einem Konflikt mit Ypolita zu stehen und wünscht diese zu sprechen. Mit den Vorkommnissen auf der Burg scheint sie weiter nichts zu tun zu haben. Die Suche nach dem Verräter setzt sich erfolglos fort.

Von dem Treiben in der Burg, die kaum genug Platz für alle zu bieten vermag abgesehen, verhält sich alles ruhig im Tal. Äquivalent zur dämonischen Kälte, die das Gemäuer und alle Insassen der Burg befallen hat und langsam aber stetig voran kriecht und zermürbend für Leib und Seele dem allgemeinen Gemütszustand zusetzt, verhält es sich auch mit der Ruhe vor dem zu erwartenden Sturm. Noch rüstet sich ein jeder, und jedes Detail, im Feindeslager zu entdecken, entmutigt die Verteidiger ein Stückchen mehr. Dazu verdammt untätig zu sitzen und zu warten, bis das Unvermeidliche über uns herein bricht. Auch die Vorräte gehen langsam zu Neige. Trotz strenger Rationierung werden sie höchstens noch bis zum 30.Rahja halten. Die Aussichten sind düster und eine allgemeine Stimmung der Hoffnungslosigkeit macht sich breit unter den Bauern. Manch einer der unter Waffen steht, hat sich längst davon anstecken lassen. Und wahrlich wäre es töricht etwas anderes für Wahrscheinlich zu erklären. Doch brauchen wir einen starken Kampfgeist, wenn wir die Namenlosen Tage überstehen wollen.

In der Stunde dieser Not, als die Flamme des Wiederstandes weitgehend (ein Großteil der Amazonen ist hier natürlich auszuschließen) zu erlöschen drohte, bat Toran, Mutter Peraine um Hilfe, und sie gewährte uns ein Wunder. Sättigend wie nie zuvor war noch die kleinste Portion, und ein jeder der sich an dieser Mahlzeit beteiligte fühlte sich gestärkt in der Zuversicht auf die Zwölfe.
Überhaupt wäre ohne Torans stetiges Intervenieren der Zusammenhalt unter den Verteidigern längst zerbrochen.

In erfrischter Hoffnung auf die Götter, wagt ich auch wieder an meinen gescheiterten Befreiungsversuch, von vor zwei Abenden, zu denken. Ich vertraute mich Temyr an, der sich zwar betroffen gab, aber mein Urteil für berechtigt und jeglichen weiteren Versuch für aussichtslos hielt. So sie gefangen genommen worden war, weilte sie in zwischen wahrscheinlich nicht mehr unter den Lebenden und es wäre unklug, so kurz vor dem nächsten Ansturm, unsere Kräfte zu überstrapazieren, um ein Einzelschicksaal zu retten, und dann der Gemeinschaft gegenüber zu versagen. Das beruhigt mein Gewissen zwar nicht im Mindesten, aber es klingt Vernünftig. Wer weiß ob wir die nächsten Tage überleben, und wenn nicht, ob wir je in Borons Hallen gelangen werden.

\subsection{Der Fall Kurkums nach Firnen Wulfgrimm}
\paragraph{29. Rahja}
Die Flaute hält an, doch innenpolitische Ereignisse werfen ihre Schatten voraus.

Toran kam, als er sich in die Vorratskammer begab zur Vorbereitung der Essensausgabe, nicht umhin ein Gespräch zu belauschen, das offenbar in der angrenzenden Stallung geführt wurde:

Es handelte sich um Roana und Alena, ihre Mutter, wie sich im Laufe des Disputes ergab. Was immer Alena plante, sie wollte es diese Nacht versuchen. Roana bekräftigte mehrfach, dass sie mit dem Vorhaben nicht einverstanden sei. Sie sprach von Hochverrat und dass sie dafür ihren Kopf riskieren müsste.
Was auch immer heute Abend passieren soll. Wir werden versuchen zur Stelle zu sein.

Ragnos und ich erwarteten die beiden Frauen vom Tempel aus und folgten ihnen unauffällig zu einer der Nebentüren.
Wir waren wohl nicht die einzigen, denen die Beiden aufgefallen waren, denn hinter der Tür erwartete sie bereits eine größere Eskorte der Amazonen, bereit sie in Gewahrsam zu nehmen.

\paragraph{29.Raja}
Trotz Torans Bemühen ließ sich Ypolita nicht von ihrem Vorhaben eines öffentlichen Prozesses, mit wahrscheinlich angefügter Exekution abbringen und so begeben auch wir uns im Anschluss an das Morgenmahl in den Burghof zur improvisiert errichteten Richtstätte, gegenüber des Tempels.

Welch ein erstaunliches Ereignis:
Alena, welche in Wirklichkeit Ulissa, die Schwester Ypolitas ist, wurde des Versuches der Rache und des Verrates an ihrer Königin beschuldigt. Ihre Tochter Roana der Beihilfe.

Wie sich im Laufe der "Verteidigungsrede" ergab war Ulissas Ansinnen nie mehr gewesen, als die Königin um Vergebung zu bitten. Ihre Tochter hatte aus Mitleid mit ihrer Mutter gehandelt. In der Sekunde da Ypolita sich erhebt um bereitwillig ihrer Schwester die Vergebung zu gewähren, springt diese vor, und sinkt zu Boden, ein Eiszapfen ragt ihr aus der Brust. Sie hat ihr Leben geopfert um dieses Attentat zu verhindern und kann nun als Heldin ins Reich ihrer Göttin einziehen.

Man wartet so lange, und dann überrascht es einen doch.

Ein Ruf von der Mauer der zweite Sturm bricht los.

Es ist nun die vierte Stunde nach Mittag. Sie haben uns den Mittag über beinahe überrollt. Zwei Abschnitte der Mauer sind bereits gefallen und das Tor mussten wir improvisiert und in aller Eile wieder verstärken. Lange wird es nicht mehr halten. Wir haben einen hohen Preis bezahlt und fast die Hälfte der Verteidiger ließen bereits ihr Leben, doch auch wir haben einen hohen Tribut gefordert. Mit Hilfe des Feuerdschinnes konnten wir einen der Belagerungstürm noch weit in ihren eigenen Reihen in Brand stecken und mit ihm einige, der darauf und darum herum befindlichen, Soldaten. Dämonische Unterstützung blieb dieses Mal bisher aus. Und am Mauerabschnitt wo der zweite Turm ansetzte forderte ich mittels Flimm Flamm und Flammenschwert einen erheblichen Blutzoll. Rezzanjin und Irian eilten stets dorthin, wo die Verteidiger ins Wanken gerieten, Rezzanjins Schützlinge hielten sich tapfer, und Dort wo Ypolita mit einem Trupp ihrer Amazonen die Verteidigung /fast mehr den Angriff führte gerieten die Feinde stetig wieder ins Wanken. Toran versucht sich im Schutze der Mauer um Verwundete zu kümmern und schickt Peraines heilende Macht über uns und unsere Freunde. Temyr und Ragnos haben sich im Turm des Torhauses verschanzt zusammen mit der Gruppe Bogenschützen. Doch die Zahl unserer Gegner ist schier unermesslich trotz hoher Verluste, fallen unsere Männer/und Frauen wie die Fliegen ein jeder bis zum letzten kämpfend.

Die meisten Leitern vernichtet, auch der zweite Belagerungsturm fiel den alles verzehrenden Flammen zum Opfer, konzentriert sich der Ansturm nun auf Tor und Brücke. Wenn die schweren Eichentüren brechen, sind Mauern und Innenhof nicht mehr zu halten.

Die Feinde zu Fuße, dicht gedrängt, die Verteidigung wankend. Jetzt ist der Augenblick gekommen, da die Rittmeisterin ihre Schar zu einem Ausfall sammelt, einem Kampf ganz im Sinne Ihrer Göttin.

Der erste Überritt erfolgreich. Der Feind bläst zum Rückzug.

Doch der Feind ist zahlenmäßig weit überlegen. Die Fronten wechseln sich aus und somit auch das Kriegsglück der Amazonen. Mit Speerträgern und Schützen rücken sie der Reiterschar, bald vom Rückweg abgeschlossen, zu Leibe. Hörnerklang und Geschrei. Flammen schlagen aus dem hintersten Lager aus. Währe in dieser Stunde der Not, das nimmer erwartete Ersatzheer der Brilliantzwerge nicht aufgetaucht. Wir hätten keine der Reiterinnen je wiedergesehen. Der Feind für kurze Zeit zerschlagen. muss sich erst wieder Sammeln. Unter deutlichen Verlusten konnten sich Reiterinnen und Zwerge zurück zur Burg durchschlagen.

Mitternacht bricht an. Sie gönnen uns momentan eine kleine Pause. Aber die Mauer ist nicht mehr lange zu halten.Wir haben uns im Innenhof, die Flüchtlinge bereits im Hauptturm, verschanzt.
Doch der Sturm ist noch nicht vorbei. Noch lange nicht.

\paragraph{1. Namenloser}
Ich kann sie hören. Die Untoten kommen.

Der Angriff erfolgte heftiger, als je zuvor. Alles was ihnen noch an Truppen zur Verfügung stand wurde vom Feinde in die Waagschale geworfen. Langsam aber sicher, eher rasch und unaufhaltsam, brachen unsere Verteidigungslienen. Bald stand das Torhaus in Flammen. Temyr und Ragnos, eines gestürzten Balkens wegen, eingeschlossen in den Flammen. Die Untoten, ohne Rücksicht auf Verluste, wankten zum Tor. Von der Mauer darüber aus dirigiert, fuhr mein Flammenschwert eine Blutige Ernte ein. doch der Verlust weniger Körperteil mag diese unheilig belebten Leichen noch lange nicht zu stoppen. Während dessen beinahe ungeschützt dem Pfeilhagel ausgesetzt musste auch ich meinen Preis zahlen, und er hätte mich, um eine Nasenlänge, mit Gewissheit mein Leben gekostet. Nur Torans eiligem Handeln und der Gunst der Göttin, verdanke ich es wieder einmal, dass ich diese Zeilen noch zu schreiben vermag. Wir halfen Ragnos und Temyr, sich zu befreien und versuchten den Strom der Untoten zu stoppen, die durch das, in Zwischen restlos zerstörte Tor, in Scharen in den Burghof einfielen. Zur Seit der anderen Mauer hin, schlugen sich Rezzanjin, wie ein Blitz unter den Feinden, und Irian, eher ein Fels in der Brandung, zusammen mit der Mehrheit der Amazonen mit den Söldnern. Doch es waren und blieben zu viele. Die Mehrzahl der Verteidiger lag bereits erschlagen danieder. Unser Schicksaal schien uns längst gewiss, als ein weißer, geflügelter Hengst in der Tür einer brennenden Stallung auftauchte. Während wir uns zurück zogen, uns im Tempel zu verschanzen, bedeutete dieser Toran, auf zu sitzen und flog mit ihm von dannen. Unseres Heilers beraubt, sahen wir uns bald darauf mit dem Schlimmsten konfrontiert, was die Niederhöllen zu bieten scheinen. Ein Mauerteil wurde weg gesprengt, die Angreifer hielten inne und zogen sich ein gutes Stück zurück, und ein mehrfach gehörnter fliegender Dämon kam durch die Mauerlücke. Die Todesmutigen Amazonen die in seiner Nähe Standen fielen wie die Fliegen, als Ypolita, ein Gebet zur Göttin und das letzte Schlachtlied der Amazonen auf den Lippen sich ihm entgegen warf. Es war der beeindruckenste, wenn auch ein ungleicher Kampf, den die Geschichte wohl bisher zu bieten hat. Von Vornherein deutlich unterlegen hielt Ypolita lange Stand, länger als jeder andere es vermocht hätte. Und auch wenn es unwiederbringlich ihr Ende bedeutete, lange genug.

Ein Rauschen am Himmel, ein goldener Blitz und eine Flammenwalze die alles vernichtete. Das Tal verbrannt bis zur Höhe des Passes. Die Burg geschmolzen. Einsam erhebt sich nur noch der Tempel, zu Ehren Rondras, in mitten der kohlrabenschwarzen Landschaft. Die feindlichen Lager, die Söldner und Untoten, der Dämon, vernichtet, sowie alles andere auch, was sich außerhalb der heiligen Halle befand.

Es ist wohl um die Mittagsstunde, doch genau lässt es sich nicht bestimmen. Der Himmel ist noch immer mit dunklen Wolken verhangen.

Die Belagerung ist vorbei. und wir haben sie überstanden. Von Sieg kann an dieser Stelle kaum die Rede sein und erst die nächsten Stunden, Tage, vielleicht sogar Wochen werden endscheiden, ob wir überleben. Alle Überlebenden verwundet, die Mehrzahl schwer. Die Gegend verwüstet. Ungewiss, was wir an Proviant zusammen bekommen, doch bestimmt nicht sehr viel. Und hinter dem Tal und den Pässen ein weiterer Teil des Feindeslagers.

Bis wir zivilisierte Gegenden erreichen, haben wir noch nicht alles überstanden.

Der Kampf ums Überleben geht weiter.

\chapter{Der Winter des Wolfes}

\section{Geleitwort}


\begin{flushright}
Claas Völcker, Toronto, den 28.05.2025
\end{flushright}


\section{Die Tagebücher}

\subsection{Albträume, gesammelt von Iliricon Tannhaus}

\emph{
    Lieber Waldemar,}

\emph{
    enstchuldige dass ich hier den Bericht unterbreche. Die folgende Geschichte erzählte mir ein Ritter im Garethischen, und wenn sie auch wie eine wilde Fabel klingt, so habe ich Anlass ihr Glaube zu schenken. Ich habe beschlossen sie, trotz ihres unklaren Wahrheitsgehaltes in diese Chronik aufzunehmen, weil sie auch zeigt, wie die Gezeichneten zu mythischen Figuren wurden.
}

Ja, ich kannte die Gezeichneten. Ich war einmal, vor langer Zeit der Knappe des Edlen Irian von Rabemund, den man heute Stahlschwinge nennt. Einen gestrengeren und hartherzigeren Mann habe ich nie wieder kennengelernt, aber seine Treue gegenüber denen, die ihm ans Herz gewachsen waren, kannte keine Grenzen. Ich war ein einfacher Bauernjunge von einem Weiler im Tal von Kurkum. Ja, eben jenes Kurkum, welches in den furchtbaren ersten Schlachten dieses langen Krieges fiel. Ich wollte die Burg verteidigen, aber alles, was man mir erlaubte war, mich mit den Alten und Kindern im Rondratempel der Burg zu verstecken. Inmitten der Schlacht schlich ich mich aus dem Tempel, und klaubte zwischen den Verletzten in der Burg ein Schwert einer gefallenen Soldatin der Leuin auf. Ich war kaum kräftig genug, die schwere Waffe zu heben, und das Geschrei und Geheule der Verletzten ließn mir das Blut in den Adern gefrieren. Aber als ich einen dieser furchtbaren Kerle sah, wie er versuchte, den Edlen von Rabemund von hinten zu ermorden, da ergriff mich die gerechte Wut und ich trieb den blanken Stahl tief zwischen seine Rippe. Dann fiel ich in Ohnmacht.

Als ich wieder erwachte, waren nur noch rauchende Trümmer übrig von der einst so stolzen Burg, und von den tapferen Verteidigerinnen war kaum eine mehr am Leben. Nur noch etwa 30 Verwundete, schwache Alte und Kinder waren übrig vom Volke Yppolitas. Wir sahen die kläglichen Reste des Feindes in die Berge fliehen, aber das Radromtal war auch uns kein Schutz mehr. Zwei Wochen quälten wir uns über Pässe und durch Wälder, über nicht enden wollende Pfade und Straße, mit Hunger und der Angst vor dem Feind als unsere ständigen Begleiter.

Fast erschien es uns wie ein Wunder, als wir in der Mitte des Praiosmondes die stolzen Mauern Beilunks vor uns liegen sahen, damals noch sicher im Reich gelegen. Wie erfreut wir waren, dass man uns noch Einlass in die Stadt gewährte. Seinen Schwur meinem Vater gegenüber bedenkend, ließ von Rabemund mich und meine Schwester mit ihm ziehen. Ich sollte sein Diener und Knappe werden, für meine Schwester wollte er eine Anstellung am Hofe seiner Familie in Rommilys finden. Doch zuerst musste ein Krieg gewonnen werden. Zusammen mit einer Kompanie der Rondrakirche drängten die Gezeichneten die Fürstin Beilunks, Gwyduhenna von Falhadon, die Tore zu ihrer Stadt zu öffnen und ihre Truppen als Ersatz für die Prinzessin von Weiden und den Herzog von Tobrien zu senden, die einen Zug gen Eslamsbrück planten, um den Feind vom Überschreiten der Tobimora abzuhalten. Aber in ihrer Angst vor den nahenden Truppen des Bethaniers verfügte die Fürstin, das kein wehrhafter Mann und keine streitbare Frau die Stadt verlassen sollte, damit Beilunk geschützt bleibe. Mit wütenden Rufen der Rondrianer und weißen Worten des Pflegers des Landes, Toran Ostik, drängten alle versammelten darauf, Beilunk dennoch verlassen zu dürfen. Doch die Fürstin bleib wacker und ließ sich nur darauf ein, dass bis zum nächsten Sonnenaufgang alle die Stadt verlassen durften, die sich dem Heerzug der Weidner und Tobrier anschließen wollten.

Die Gezeichneten nahmen sich Pferde und verließen Beilunk noch am selben Abend. Sie ritten, soweit ihre Pferde sie tragen wollten, und kamen am Abend an einem Gasthaus an. Dort quartierten sie sich über Nacht ein und ein Albtraum sondergleichen begann für sie.

Man mag es kaum glauben, aber seit Beilunk hatte eine wahrhaft gefährliche Gruppe von Mördern sie verfolgt. Die geschmähte Hexe Savolina und ihr Geliebter Jasper hatten im Auftrage Azaril Scharlachkrautes viele alte Feinde der Gruppe versammelt, die es auf ihr Leben abgesehen hatten. So kam es, dass ein Paktierer, der sich selbst ``der Nachtmahr'' nannte, in das Gasthaus eindrang und die Gezeichneten in einen dämonischen Traum spann. Sie fanden sich in einem Boronkloster wieder, wo sie angeblich von ihrer Verwirrtheit geheilt worden sein sollten. Sie fragten, weshalb sie denn behandelt wurden und als Antwort kam, dass sie sich für die Gezeichneten gehalten hatten, eine wirre Sache, denn diese seien ja schon seit über 200 Jahren tot. Doch schnell kamen sie diesem falschen Spiel auf die Schliche, denn der Dämon, der dieses perfide Lügengespinst steuerte, zeigte sich erstaunlich unsicher, was verschiedenste Fakten über die Zwölfe anging. So bemerkten sie, dass der oberste Boronpriester des Traumklosters die Verkörperung dieses Dämons darstellte und Ragnos, der Vierte, erstoch ihn mit einem gestohlenen Küchenmesser.

Befreit aus dem Hirngespinst sahen die Gezeichneten die Mörder mit all ihrer Ausrüstung im Limbus verschwinden. Und mich und meine Schwester hatten sie mit im Schlepptau.

\subsection{In Warunk nach Rezzanjin al'Ahjan}

\paragraph{Nach der Schlacht von Kurkum}

Es war wahrlich eine der verrücktesten Nächte, die ich erlebt habe. Da versucht man die Borbaradianer, die einen in der Nacht überfallen nicht entkommen zu lassen, springt ihnen hinterher in den Limbus und dann entkommen sie einem doch und man findet sich alleine im grauen Nebel wieder. Ohne Zeit, ohne Orientierung und leider auch ohne Nahrung und keinem Tropfen Wasser zum Trinken. Verdammt zu sterben oder eventuell früher oder später von einem der Wesenheiten, die im Limbus herumschwirren, von denen Firnen immer erzählte, vernascht zu werden als sei man eine tulamidische Süßspeise. Glücklicherweise wurde das Warten auf den Tod jäh von Firnen unterbrochen, der nach einiger Zeit auftauchte, um mich zu retten. Zusammen hatten wir dann allerdings das Problem aus dem Limbus nicht mehr ohne weiteres heraus zu kommen, was daran liegt, dass man im Limbus nie weiß, wo man landet wenn man sich einen Ausgang bahnt. Also folgte ich ohne große Wahl Firnens Führung und er führte uns anhand von Kraftlinien zum nächsten Nodix, einem Knotenpunkt, der seiner Meinung nach in Warunk liegen müsste. Da kamen wir auch heraus, nur irgendwo in den Katakomben unter der Stadt. Nachdem wir uns unsere Wege durch übergroße Spinnenweben gebahnt hatten fanden wir auch schnell einen Ausgang.

Die nächsten zwei Tage warteten wir auf das Eintreffen unserer Gefährten. Anschließend wurden wir beim Marktgrafen von Warunk vorstellig, der uns auf ein paar seltsame Zeichen an einem Baum im Garten des Schlosses aufmerksam machte und auf einen verschwundenen Perainegeweihten.

Wir untersuchten die Zeichen im Garten und stellten fest, dass sie uns nicht viel sagten. Jedoch war beim Baum eine starke magische Aura auszumachen, was auf ein Ritual oder wenigstens auf einen starken Zauber zurückzuführen war. Auch Nachforschungen im örtlichen Archiv halfen uns nicht weiter, doch kamen wir zu dem Schluss, dass sowohl das Verschwinden des Perainegweihten, als auch die seltsamen Zeichen Werke borbaradianischer Umtriebe sein mussten. Wir versuchten also die borbaradianische Gruppe, die sich in der Stadt befand ausfindig zu machen, kamen nur nicht weit. Nachdem wir jedoch den Ort außerhalb der Stadt, an dem der Perainegeweihte verschwunden war, näher untersucht hatten, kamen wir zu der Erkenntnis, dass dieser in den Wald entführt worden sein musste. Die Spuren deuteten darauf hin, verloren sich jedoch nach kurzer Strecke. Da auch die Befragungen der Bewohner zu keinem Ergebnis führten waren wir überfragt, wie wir der Borbaradianer habhaft werden konnten.

Wir beschlossen uns an den letzten Bambustrinkhalm zu klammern, den wir sahen und warteten die Nacht und den frühen Morgen auf Leute, die die Stadt Richtung Wald verlassen würden und uns so zu dem Geweihten führen würden, so er denn tatsächlich noch im Wald war und nicht ein unrühmliches Ende gefunden hatte. Der Plan klappte und wir fanden die Höhle in der der Geweihte gefangen gehalten wurde. Überraschenderweise wurde er nicht überwacht und die Person, die wir verfolgten, brachte nur Essen und Trinken. Sie entdeckte uns nicht und so gelang es Ragnos sogar, sie in die Stadt zurück zu verfolgen und das Haus auszumachen in dem sie verschwand.

Wir retteten währenddessen den Geweihten, der den Borbaradianern wohl zu sehr auf die Schliche gerückt war. Er mahnte an, dass die Borbaradianer wohl ein Ritual unterhalb der Stadt planten. Schnell machten wir uns auf das Haus zu infiltrieren, in dem wir die verräterischen Bruderlosen vermuteten. Firnen übernahm dies nach langer Diskussion. Nach seiner Rückkehr war klar, dass die hochrangigen Borbaradianer um Azariel Scharlachkraut in dem Haus des Händlers untergekommen waren, jedoch jetzt von unserer Anwesenheit wussten. Wir sahen keine Möglichkeit das Haus gefahrenlos zu infiltrieren und so wollten wir bis zum nächsten Tag warten, um mit den Truppen des Grafen zusammen das Haus einzunehmen. Doch die Borbaradianer kamen uns zuvor. Von schrecklichen Träumen geplagt wachten wir am nächsten Tag auf und fanden uns gefangen in den Kellern des Händlerhauses wieder. Nachdem wir uns befreien konnten fanden wir es verlassen vor und unsere magischen Artefakte nicht mehr. Sie waren uns gestohlen worden. Mit viel Wut im Bauch rannten wir zum Grafenschloss und veranlassten eine sofortige Durchsuchung des Hauses, während wir mit einigen Gardisten des Grafen zum Eingang des Molochs von Warunk gingen, um das Ritual, welches sich nun andeutete, zu verhindern. Doch fanden wir die Gänge hastig verlassen vor und Anzeichen, dass das Ritual nicht in Gänze vollendet worden war. Ein kleiner Erfolg also war uns geglückt. Jedoch mussten war der Preis mit den verlorenen Artefakten, die Temyr und Firnen hergestellt hatten, groß. Bedrückt von dieser Niederlage reisten wir weiter nach Eslamsbrück um die kommende Schlacht mit unseren Kräften zu unserem Vorteil zu entscheiden.

\todo[inline]{Summarize the battle}

\subsection{Nachspiel von Eslamsbrück nach Irian von Rabemund}

\paragraph{30. Rondra 1020BF}
Immer noch geschockt von den Ereignissen in der Schlacht von Eslamsbrueck reiten wir, als waere der Daemonensultan persoenlich hinter uns her, Richtung Warunk. Schnell lassen wir das traurige Haeuflein unter Befehl Saldor Foslarins hinter uns. Die uns quaelenden Gedanken an diese verfluchten Schlacht treibt uns zu unmenschlichen Leistungen an. Und so erreichen wir die Stadt Warunk nach nur zwei Tagen.

\paragraph{2. Efferd 1020BF}
Wir erbeten beim Markgrafen von Warunk Unterkunft fuer eine Nacht, sowie Proviant und Pferde fuer die Weiterreise in Richtung Ysilia. Wir berichten dem Markgrafen von der bitteren Niederlage und dem Verlust Walpurga und Dietrads. Er ist bestuerzt von diesen Nachrichten, sieht sich aber machtlos den Folgen der Schlacht ausgesetzt. Frueh am naechsten Morgen brechen wir auf um Erbprinz Bernfried von Ehrenstein dye traurige Nachricht ueber den Verlust seines Bruders und seiner Schwaegerin zu bringen.

\paragraph{4. Efferd 1020BF}
Zwei Tage reisen wir schon Richtung Ysilia und wir haben immer noch das Gefuehl, dass uns etwas folgt. Am fruehen Nachmittag attackieren uns sechs boraradianische Soldaten und ein Bogenschuetze aus dem Hinterhalt. Wir hoeren auch niederhoellysche Woelfe, doch diese greifen uns nycht an. Wyr koennen uns den Angreyfern mehr schlecht als recht erwehren, dye Wunden von Eslamsbrueck sitzen einfach noch zu tief. Aber wir bekommen Unterstuetzung von zwei Magi aus Drakonia, einer Magierakademie aus dem Raschtulswall, die sich mit Elementarismus beschaeftigt. Eine luftelementarischen Adeptin namens Aria Kargosta und ein Erzelementarist, der uns schon einen Tag spaeter verlaesst um deren Auftrag im Sueden weiter zu verfolgen. Aria ist Temyrs zukuenftyge Geliebte/Lebenspartnerin, mal sehen was sich ergibt\footnote{Ich konnnte leider nicht erfahren woher Irian diese Voraussicht nahm.}.

Sie hat eine ganz schoen stuermisches Gemuet zum eher ruhigen Temyr, aber man sagt doch Gegensaetze ziehen sich an.

\paragraph{6.Efferd 1020BF}
Wir kommen in Ysilia an. Man hat das Gefuehl seit Eslamsbrueck fliesst die Zeit langsamer als gewoehnlich, so schnell haben wir Ysilia noch nie erreicht. Aria verschwindet ohne Abschied vor den Toren der Stadt. Rund um Ysilia lagern schon gute zwei tausend Fluechtlinge aus den verlorenen Baronien Tobriens und taeglich kommen mehr.

Wir werden, nachdem wir unser Anliegen genannt haben, direkt zum Erbprinz Bernfried von Ehrenstein, seiner Gemahlin Prinzessin Efferdane von Eberstamm-Ehrenstein und Kanzler Delo von Gernotsborn vorgelassen. Ihnen erzaehlen wir die bitteren Nachriten. Man bittet uns noch ein paar Tage auf der Burg zu verweilen. Man stelllt uns hervorragende Unterkuenfte.

\subsection{Summus Kate nach Firnen Wulfrgimm}

\paragraph{07. Efferd 1020 Bosparans Fall}
Wir sind noch immer in Ysilia. Seit der umstrittenen Ernennung Bernfrieds zum Herzog sind nun mehr zwei Tage vergangen. Ein Großteil der Fürsten des Landes sind abgereist und warten ab wen das Schicksaal in naher Zukunft zum wahren Herrscher küren wird. Die Stadt ist nach wie vor von Flüchtlingen überlaufen. Wir bewegen uns dazwischen wie Forellen in einem Fass Sardinen. Wir warten ab.

\paragraph{08.Efferd 1020 Bosparans Fall}
Der Tag war schrecklich ereignislos. Unsere Freundin, die Luftmagierin Aria, habe ich in zwischen aufgegeben. Wo auch immer sie jetzt ist, an unserer Gesellschaft scheint sie ganz offensichtlich nicht interessiert zu sein, auch wenn ihr Anliegen von weltbewegender und interessanter Art gewesen zu sein scheint.

Heute Abend jedoch genehmigte ich mir im Wirtshaus am Platz noch einen kleinen Krug. Dort traf ich einen Jägersmann, der erst vor kurzem in die Stadt gekommen war. Er war nicht besonders redselig, doch als ich anbot das Fass Koscher-Flussbreu, welches ich völlig überteuert beim Wirt erstanden hatte, mit ihm zu teilen, war er sogar zur Unterhaltung bereit.

Tatsächlich wusste er von einer Begegnung einen halben Tag vor Ysilia zu berichten, wo er einer Hexe, die erstaunlich präzise auf die Beschreibung Arias passte, bei sonderbaren Ritualhandlungen über den Weg lief. Der Fremde, der sich mir als Kunn Wolfhelm vorstellte, willigte ein mir und meinen Feunden, den Weg in den Wald zu weisen.

Temyr finde ich in persönlichen Studien vertieft auf seinem Gemach in den herzoglichen Räumen. Er war sofort bereit mich zu begleiten, war ihm doch keine Aufgabe eigen, deren weitere Verfolgung er für sinnvoller erachtet hätte. Ob uns Toran begleiten würde schien ungewiss, lag ihm die Hilfe in den Flüchtlingslagern der Stadt doch sehr am Herzen. Doch als wir ihm unseren Plan, die Stadt zu verlassen vortrugen, war er sofort bereit uns zu begleiten.

Wir werden uns beim 3. Hahnenschrei am Ost-Tor treffen.

\paragraph{09. Efferd 1012 Bosparans Fall}
Zur noch frühen Morgenstund brachen wir auf. Bald schon verfluchten wir die schlecht durchdachte Begleidung, die kaum für einen längeren Marsch geeignet war. Doch die, für diese späte Jahreszeit noch recht warme Sonne, beschien unseren Weg und hob meine Stimmung von Stunde zu Stunde ins unermessliche, währen Kunn seinen Mantel enger zog und Temr sich unentwegt über die erbarmungslose Kälte beschwerte. Ich unterhielt unseren Wildnisführer mit Gschichten aus thorwalschen Gewässern und Wüsten des Südens und maraskanischen Dschungeln. Zeit und Weg vergingen wie im Fluge. Etwa zur Mittagsstund erreichten wir eine Lichtung, bei der Kunn stehen blieb.

Das Laub war wie von einer großen Windhose komplett von der Lichtung gefegt. Mittels magischer Analyse konnte ich Reminiszenzen Elementarer Magie des Elementes Luft ausmachen. Die Indizien sprechen eindeutig dafür, dass unser Vögelchen ausgeflogen ist und zwar mit einem Luftdschinn.

Wir kehrten kurz vor Torschluss nach Ysilia zurück.

Mal sehen was der Morgen bringt.

\paragraph{10. Efferd 1020 Bosparans Fall}
Zur Mittagsstunde kam Kunn vorbei.

Sein Regiment, dem er sich anzuschließen gehabt hätte, war vor wenigen Tagen abgezogen worden. Da er nun keine Verplichtung dem Reiche gegenüber wahr zu nehmen hätte, wollte er sich anbieten, mit uns noch einmal in die Wälder Ysilias auf zu brechen und unsere suche fort zu setzen.

Sowohl Temyr als auch Toran willigten ein. Wir rüsten uns bis Morgen aus und brechen dann auf. Für längere Zeit.

\paragraph{11. Efferd 1020 Bosparans Fall}
Wir sind erneut zur Lichtung gewandert. Temyr hat mit meiner Unterstützung einen Dschinn der Luft beschworen und ihn befragt, wohin der Wind die Magierin gebracht habe. Folgendes waren die Worte des Wesens: "Vom Wind wurde sie getragen, doch der Ort an den sie zu gelangen versuchte ist ein Ort an dem kein Wind weht. So gelangte sie nichtdort hin. Der Wind brachte sie an einen Ortes des Windes und der Kälte, des kalten Windes. Nördlich von hier."

Nach der Ortskundingen Einschätzung Kunns handelt es sich vermutlich um die Sturmspitze, eines kleinen, doch wenig windgeschützten Berges, mit kahlem Kopfe, der als vorderer Ausläufer der Drachensteine, weit ins Land hinein ragt.
Wir sind noch einige Stunden gewandert und dürften besagten Berg morgen erreichen.

\paragraph{11. Efferd 1020 Bosparans Fall}
Wir habe die Sturmspitze erreicht, und einer Kraftlinie folgend einen Steinkreis gefunden. Doch war dieser verlassen. Wir fanden nur eine Mitteilung: "Auf dem Weg zu Sumus Kate (ein Alttobrischer Begriff, der etwa Hütte bedeutet). Bald wieder da"
Der einzige Anhaltspunkt dem wir irgendwie folgen können.
Den allgemeinen Legenden nach handelt es sich bei Sumus Kate um einen mystische Insel im Yslisee. Einem Ort der allein den Druiden bekannt und zugänglich sei.

\paragraph{14. Efferd 1020 Bosparans Fall}
Nach einem ganzen Tage Wanderung haben wir heute Mittag das kleine Dörfchen Seehaagen erreicht. Tatsächlich scheint unsere Freundin hier in Begleitung zweier Druiden hindurch gekommen zu sein. Sie liehen sich ein Boot und ruderten auf den Yslisee hinaus. Seit dem habe man sie nicht mehr gesehen. Das Boot haben sie nicht zurück gebracht. Es sei aber auch bereits knapp eine Woche vergangen seit dem, und es wäre ja allgemein bekannt das jene die im Nebel verschwanden nie wieder kehrten.

Entgegen der eindringlichen Warnungen eines der Fischer, der mir mittels Bannbaladin verfallen war, liehen wir uns sein Boot um den See zu befahren.

Die Nebelschwaden waren bald so dicht, dass man kaum die Hand vor Augen zu sehen vermochte und beständig flüsterten die Stimmen und rieten zur Umkehr. Je dichter der Nebel wurde, desto eindringlicher warnten die Stimmen vor der Weiterfahrt. Ich versuchte lange Zeit die Nerven zu behalten, bis ich schließlich neben meinen Gefährten, nicht mal mehr mich selbst von harmlosen Illusionen überzeugen konnte. Wir drehten ab, und hätte ich nicht mittels der magischen Kugel Madas Sternenkinder sehen könne, wir hätten nie aus dem Nebel heraus gefunden.

Wenn wir die Insel je erreichen wollen müssen wir uns wohl oder übel erst an einen Druiden wenden. Vielleicht können wir morgen im Dorf herausfinden in welche Richtung wir uns zu wenden haben. Ansonsten kenne ich einen Steinkreis wo vor einiger Zeit noch ein Druide lebte. Er kam im Kampf gegen Xeraan ums Leben. Möglicherweise wurde aber der Steinkreis erneut besetzt.

\paragraph{15. Efferd 1020 Bosparans Fall}
Wir haben die Hütte am Steinkreis im Nebelwald bei Viereichen aufgesucht und sie war bewohnt. Der Druide aber, der persönlich nicht in Erscheinung trat war über unseren Besuch nicht erfreut und sehr kurz angebunden. Ich versuchte ihm deutlich und freundlich unserer Anliegen vor zu tragen, und es ausreichend zu begründen, doch mit ihm war kein Reden.

Unverrichteter Dinge mussten wir uns auf den Rückweg begeben.

Vielleicht ist es nicht unserer Schicksaal, die Magierin zu finden. Vielleicht sollten wir dieser sonderbaren Begegnung einfach nicht so viel Bedeutung zu schreiben und nach Ysilia zurück kehren, wo zumindest Torans Qualifikationen stets benötigt werden.

Allein Temyr stellt in Frage, was denn geschehen soll, wenn wir uns schon in solchen Dingen dem vermeintlichem Schicksaal ergeben.

\paragraph{16. Efferd 1020 Bosparans Fall}
Zur späten Abendstund erreichten wir Seehaagen. Die Nacht lag bereits in den ersten Zügen und Nebelschwaden zogen vom Ufer des Yslisees her über das Land, und verfingen sich in den wenigen kleinen Gässchen zwischen der Häuseransammlung, die nur durch Fehlen einer geeigneteren Bezeichnung, Dorf genannt wird.

Der Wind des hatte gedreht und ein Hauch von Schicksaal lag in der Luft, als wir mitten auf der Gasse alten Mann stehen sahen. Das Haupthaar lange nicht geschoren, der Bart gleichsam lang und ungezähmt. Das Haar schlohweiß, kündend von der Weisheit eines langen Lebens. Die Augen, die schon zu viel gesehen haben müssen, nunmehr erblindet. Doch der Geist stark, der Wille der den Körper am Leben hält ungebrochen.

Es war Firnbart der Herr von Sumus Kate, und ältester der Druiden Tobriens. Wir wollten ihm unserer Anliegen gerade darbringen, als er uns bat, ihn zu seiner Insel zu begleiten. Es scheint als würden sowohl wir, als auch die Druiden auf gegenseitige Hilfe hoffen.

Das Boot das uns und den Alten auf den Yslisee brachte fuhr ganz ohne Hilfe. Die Magie dieses Ortes übersteigt vieles was ich bisher gesehen habe und ist von unbegreiflicher und uralter Macht.
Wollen wir sehen was die Götter uns dieses Mal zugedacht.

\subsection{Reise durch Tobrien im Auftrag der Druiden nach Toran Ostik}
16. Efferd
Nach unserer Anreise zum Yslisee und einem gescheiterten Versuch die Druideninsel auf eigene Faust zu betreten, wurden wir vom ``Herrn der Insel'' dem Druide Fyrnenbart eingeladen. In seinem Boot war es möglich die magischen Schutzvorrichtungen zu durchdringen und ``Sumus Kate'', wie ihre Bewohner sie nennen, zu betreten. Nachdem man uns klar machte, dass es normalerweise nur Druiden und den jeweiligen Landesherrschern erlaubt ist das Eiland zu betreten und eine ausdrückliche Warnung an die Magier aussprach, die alten magischen Gefüge der Insel zu respektieren, erklärte uns Fyrnenbart den Aufbau der Insel. An der Spitze des Hügels befindet sich ein druidischer Steinkreis, so wie wir auch einen in Klammsbrück haben. Dieser unterscheidet sich jedoch durch einen riesigen steinernen Thron, der einst Sitz des Riesenkönigs Skorbanor gewesen sein soll. Angeblich wirkt auch in ihm uralte und starke Magie, jedenfalls schienen Firnen und Temyr geneigt den Warnungen Fyrnenbarts Folge zu leisten. Uns wurde für die Dauer unseres Aufenthalts eins der komplett steinernen Häuser der Insel zugewiesen, die in einem Halbkreis angeordnet in einer Senke standen. Auf Nachfrage stellte sich heraus, dass weder die Häuser, noch die magischen Schutzvorrichtungen von den Druiden stammten, sondern von ihnen nur genutzt werden. Die ganze Kate scheint mir sehr mysteriös. Weder diese uralte Magie noch die Anbetung der Urmutter Sumu die hier offensichtlich stattfindet sind mir recht geheuer, obwohl ich zugeben muss, dass ich diese friedliche Naturanbetung doch vielen anderen Unglauben, die wir über die Jahren sahen vorziehe Nachdem wir uns eingerichtet hatten rief Fyrnenbart ein Versammlung der heimischen Druiden ein. Sie teilten uns mit, dass die südländische Magierin Aria, der wir auf Firnens Weisung hin schon seit einiger Zeit folgten wohl ebenfalls ohne Erlaubnis versucht hatte die Insel zu betreten. Nur indem Fyrnenbart sie in einen fast komatösen Schlafzustand versetzte war es ihm möglich ihr Leben vor der Magie der Kate zu beschützen. Bevor ich eine Gelegenheit sie zu untersuchen erfragen konnte, wendete sich die Versammlung bereits anderen Themen zu. Erfreulicherweise scheinen sich die Druiden der Gefahr durch den Dämonenmeister durchaus bewusst zu sein. Sie und die Magier fachsimpelten eine Weile über arkane Linien oder sumus Adern, wie die Inselhüter es nennen und baten uns dann zwei Aufgaben für sie zu erledigen, die dem Ziel dienen, es Borbarad unmöglich zu machen, die Linien zu seinen Zwecken zu missbrauchen. Es gilt die Kreissteine zu bergen, die unter jedem druidischen Steinkreis vergraben sind und den Kelch der Sumu, ein mächtiges magisches Artefakt zu finden. Für letzteres Verwies man uns an die Druidin Xindra, ein alte Schülerin Fyrnenbarts. Die Unterredung wurde aufgelöst und Grimmbart, der Heiler auf Sumus Kate, wies uns zu Aria. Äußerlich scheint ihr nichts zu fehlen. Sie liegt, wie von Fyrnenbart angekündigt, in einem tiefen Schlaf. Wäre das ein normaler Fall, so würde ich wohl auf eine schwere Kopfverletzung tippen, möglicherweise auch auf eine Beschädigung der Wirbelsäule. In beiden Fällen wohl nur mithilfe der Göttin heilbar. Firnen und Temyr bestätigten jedoch, dass ihr Schlaf magischer Natur ist und so ließen wir sie schweren Herzens zurück. Nach einiger Besprechung beschlossen wir dann uns zuerst um den Kelch zu kümmern. Hierfür wurden uns vom Zirkel die beiden jüngeren Druiden Cordovar und Stovar, die Aria auf die Insel begleiteten, zur Seite gestellt.

\paragraph{17. - 20. Efferd}
Reisen zurück zur Sturmhöhe. Bergen den dortigen Kreisstein.

\paragraph{21. - 23. Efferd}
Reisen zur Hütte der Druidin Xindra (etwa eine Meile von Perainefurten).

\paragraph{24. Efferd}
Wir erreichen gegen Mittag die Hütte der Druidin. Die deutlich aufgeweckter als ihre männlichen Kollegen wirkende Frau begrüßte uns freundlich und lud uns in ihre rustikal eingerichtete Hütte ein. Auf unsere Aussage Fyrnenbart hätte uns an sie verwiesen um den Kelch zu finden reagierte sie jedoch eher ungehalten. Es stellte sich heraus, dass sie über den genauen Aufenthaltsort des Kelches gar nicht Bescheid weiß und man uns nur an sie verwies, weil sie über ein weiteres magisches Artefakt: einen Druidenspiegel, mit dem es möglich ist ferne Orte zu beobachten, wacht. Unsicher darüber wie wir weiterhin verfahren sollten diskutierten wir eine Weile mit Xindra. Der Kelch der Sumu gilt wohl als Symbol der Herrschaft über Tobrien und hat deswegen mehr als nur magische Bedeutung. Deswegen schlug die Druidin vor, in der Bibliothek von Ysilia nach Informationen über seinen potentiellen Aufenthaltsort zu suchen. Außerdem kam die Idee auf den Drachen Apep den Ewigen nach dem Aufenthaltsort des Kelches zu fragen. Wie man Kontakt mit dem mächtigen Drachenfürsten aufnehmen könnte wusste uns Xindra auch nicht genau zu sagen und so beschlossen wir den Karfunkel, den uns Tecklador damals in der Gor anvertraute, zu befragen. Noch gegen Abend reisten wir gen Ysilia und Klammsbrück ab.

\paragraph{25. - 29. Efferd}
Reise nach Ysilia. Wir trafen die Stadt fast soldatenleer an. Die Stimmung an sich schien recht gedrückt. Während Firnen und Kunn gen Herzogenbibliothek aufbrachen und Temyr versuchte Einlass zur Akademie zu erlangen wendet ich meine Schritte zum Tempel der Göttin. Bruder Esmodan begrüßte mich herzlich und wusste mir auf Nachfrage zu berichten, warum die Stimmung hier in der Stadt sich so verändert hatte. Im Norden hatte ein Teil der borbaradianischen Armee Warunk genommen und Gerüchte gingen um ein großer schwarzer Skelettdrache hätte dabei geholfen. Normalerweise würde ich so etwas ja als Geschwätz abtun, aber die Beschreibung des Drachen erinnerte mich doch in unheimlichem Maße an Eslamsbrück. Nachdem der Rest der Gruppe zurückkehrte reisten wir weiter gen Klammsbrück.

\paragraph{29. Efferd - 2. Travia}
Reise nach Klammsbrück.

\paragraph{3. Travia}
Erreichen gegen Mittag Klammsbrück. Trotz der unruhigen Zeiten, die die Baronie Grunewaldt in letzter Zeit durchzumachen scheint, hat sich auf der Akademieburg nicht viel verändert. Illiricon begrüßte uns mit gewohnt geteilten Gefühlen und wir brachten ihn bei einem angenehmen Mittagsmahl auf den neusten Stand. Momentan bereiten sich die Magier auf die Analyse des Karfunkels vor. Ich gedenke heute noch einige Objekte zu weihen. Die kommende Zeit droht hart zu werden.

\paragraph{3. Travia}
Die göttliche Herrlichkeit ist überwältigend! Kaum vermag ich diese Zeilen zu schreiben. Selten fühle ich mich der Peraine so nah! Die Mauern des Burgschreins engen mich ein. Es treibt mich hinaus auf die Felder. Die geordnete Ruhe des Ackers bringt mir Freuden wie es sonst nur das Pflegen der Kranken tut. Sähe und du wirst ernten! So steht es geschrieben. Und doch versteckt sich hinter der Euphorie ein Drängen. Der Dienst an der Mutter ist nicht so einfach wie er es einst war. Der Feind der Götter drängt auf Dere. Immer schon war es mir klar: Dem Dämonenmeister muss Einhalt geboten werden! Allein kann ich es nicht bewältigen. Wie die Saat muss die Kunde ausgebracht werden, damit sie keimt. Doch wie? Und an wen?

\paragraph{4. Travia}
Beim Frühstück teilt man mir die Ergebnisse der Untersuchung mit. Nachdem die Magier mit ihren Analyse Zaubern nichts an dem Karfunkel finden konnten, benutze Temyr den Traumring um in die Gedanken des Drachen zu reisen. Wie es scheint lebt dieser trotz dem Fehlen seines Körpers weiter. Wie genau das Gespräch mit einer so alten und mächtigen Kreatur von statten ging habe ich nicht erfahren, doch folgendes fand Temyr über Apep heraus: Er kann nur über seine Vasallen kontaktiert werden. Um mit selbigen überhaupt sprechen zu dürfen muss man ihnen ein Geschenk überreichen. Apep selbst wird uns die Information dann auch nicht freiwillig geben. Er verlangt etwas ähnlich Wertvolles im Tausch. Die Beratung was diese Objekte sein sollten hielt den halben Tag an. Schnell konnte man sich einigen dem Vasallen 5 Stein des Midurits, das wir damals in der Gor Liskom von Fasars Mine entnahmen als Geschenk darzubieten. Das Tauschartefakt für Apep wurde jedoch Gegenstand heftiger Diskussionen. Vier Optionen boten sich uns: Das Karfunkel des Drachens, der Kelch, den die Magier vom Konvent der grauen Gilde erhalten hatten und der Möglicherweise ein Teil der legendären Waffen Sieben Streich ist, Firnens Augen und Temyrs Schlange. Alle diese Optionen sind aus dem einen oder anderen Grund verheerend. Der tote Drache scheint Temyr explizit darum gebeten zu haben seinen Seelenstein nicht an Apep zu verschenken. Die Lager sind gespalten. Wir werden sehen was der Morgen bringt.

\paragraph{5. Travia}
Die Entscheidung ist gefallen. Wir werden den Karfunkel an Apep verkaufen. Ich klammere mich immer noch an die Hoffnung, dass wir auf dem Weg zur Burg Drachenhaupt in den Drachensteinen eine bessere Option finden werden, doch tief in mir weiß ich, dass ich damit nur mein Gewissen beruhige. Boron möge uns allen verzeihen. Gegen Mittag werden wir aufbrechen.

\paragraph{5. - 10. Travia}
Reise zur Hütte der Druidin Xindra bei Perainfurten. Wir klären sie über unseren Plan auf, aber außer uns Glück zu wünschen kann sie nichts mehr für uns tun.

\paragraph{11. - 14. Travia}
Reise zur feste Drachenhaupt. Wir erreichten den Sommersitz der tobrischen Herzoge in den Drachensteinen noch recht früh am Morgen. Nähere Informationen über Apep fanden wir in der recht kleinen Büchersammlung der Feste nicht, doch wegen seiner Forschungen in der Akademie zu Ysilia machte uns Temyr auf die Hinterlassenschaften des Wulfgrimm von Dabonia aufmerksam. Dieser war wohl, wie so viele Ritter zu seiner Lebzeit, ausgezogen um den Kelch der Sumu zu finden und anders als die meisten schien ihm dies auch gelungen zu sein. Er schenkte dem Drachen seinen erstgeborenen Sohn und wurde daraufhin zum Fundort des Artefakts gewiesen. Dort angekommen beschloss er jedoch aus nicht näher genannten Gründen, dass es den Menschen nicht gestattet sei dieses wichtige Artefakt zu besitzen und beließ es dort. Er kehrte nach Hause zurück, nur um zu bemerken, dass seine Frau sich aus Kummer um ihren Sohn in den Tot gestürzt hatte. Eine wahrhaft tragische dunkle Geschichte, die zum Mysterium um diesen Kelch nur noch beiträgt. Ich hoffe das Opfer, das wir bringe wird es wert sein. Es schmerzt mich sehr, doch die Ziele des Dämonenmeisters sind schlimmer als alles andere.

Das Burgpersonal konnte uns an einen nahen Drachenhort verweisen. Da Apep über alle Drachen des Gebirges herrscht dürften wir so Kontakt mit ihm aufnehmen können. Ein paar Kletterhaken und Seile konnten sie uns auch noch mitgeben. Ich hoffe der Aufstieg wird nicht zu schwer. Wir haben in der Vergangenheit schlechte Erfahrungen damit gemacht.

Der Aufstieg war noch schlimmer als erwartet. Schon am ersten vereisten Abhang stürzte Cordovan ab, versuchte den Aufstieg dann mit einer leichten Verletzung nochmal und stürzte wieder. Firnen erklärte sich bereit sich um ihn zu kümmern und so konnte ich weiter nach oben klettern. Der zweite Anstieg war noch steiler und nur mit Hilfe der Göttin war es mir möglich ihn zu erklimmen. Stovar, unser druidischer Freund, schaffte es mir zu folgen, doch sowohl Kunn als auch Temyr rutschten am glatten Eis ab und fielen zurück auf den letzten Felsvorsprung. Schweren Herzens ließ ich sie zurück in der Hoffnung, dass die eisige Kälte ihre Wunden konservieren würde. Oben angekommen fanden wir tatsächlich eine riesige Höhle vor. Als wir darauf zugingen verdunkelte sich hinter uns die Paiosscheibe und ein gigantisches Biest schnitt uns den Weg zurück zum Rest der Gruppe ab. Erst einmal in meinem Leben war mir die Winzigkeit der Menschen so bewusst gewesen. Doch damals in der Höhle bei Kurkum blieb mir die Gewissheit im Hinterkopf, dass dieser Drache ein Wesen der Götter war und so wie ich nur ein Werkzeug ihres Willens. Nichts dieser Art konnte über die Riesenechse vor uns gesagt werden. Mit listigen und wütenden Augen starrte er uns an und nur unter heftigem Zittern konnte ich das Mindurit vorbringen um ihn zu besänftigen. Unser Angebot an Apep schien der Vasall zu meiner Überraschung über eine Art gedankliche Verbindung zu übermitteln. Wir übergaben ihm den Karfunkel und noch während sich das Biest in die Luft erhob rief es uns die gewünschte Information zu. Der Kelch befindet sich beim Grab des ältesten Drachengrafen an der Quelle des Tisamstroms. Etwas zittrig schafften ich und Stovar den Abstieg runter zu Temyr und Kunn. Beide hatten sich recht schwere Verletzungen an den Beinen zugezogen und weil wir keine Möglichkeit hatten sie zur Burg mitzunehmen schickte ich ein Gebet an die Göttin um sie zu heilen. Obwohl Stovar beim Abstieg dann doch nochmal stürzte schafften wir es alle zur Burg zurück.

\subsection{Der Fall von Sumus Kate nach den Berichten eines unbekannten Druidens}

Nachdem dem mächtigen Drachen Apep das Geheimnis des Entdeckungsortes des Kelches der Sumu mit mächtigen Opfergaben entlockt worden war, kehrten die Helden Toran Ostik, Geweihter der Peraine, Firnen Wulfgrimm, Magister von Klammsbrück, Temyr ibn Sahid, Spektabilität von Klammsbrück, Kunn Wolfhelm, Grenzjäger, Stover, ein Druide und noch ein Druide zurück zur Burg Drachenhaupt. Dort kurierten sie ihre durch unbedarfte Kletterfähigkeiten erlittenen Verletzungen aus und überlegten sich, wie sie zur Quelle des Tisam vordringen konnten, um das legendäre Artefakt, welches die Druiden so sehr benötigten, aus dem Grab des ersten Drachenfürsten zu erhalten.
Am nächsten Tag, es mag der 15. Travia gewesen sein brachen sie auf, um die Queste, an der schon so viele Ritter gescheitert waren zu vollenden. Es mag eine halbe Tagesreise gewesen sein, die sie geritten, bis die Straße an den Bachlauf des Tisam führte. Dort sahen sie sich direkt dem Problem gegenüber, dass der Weg entlang des Baches auf etwa 20 Schritt eingebrochen war. Doch für die Gezeichneten stellte das kein Problem dar. Der Magister teleportierte sich mit dem Grenzläufer rüber, während der Rest am Seil die kleine Schlucht überquerte. Doch später, als die Gruppe sich schon einige Zeit auf den vereisten Wegen am Bach entlang gehangelt hatte, passierte dem Grenzläufer ein Missgeschick, welches ihn in Richtung Bach beförderte, der an diese Stelle ein reißende Klamm bildete. Sofort eilt der Perainegeweihte Toran heran und kann ihn halten, doch auch er wird mitgerissen und gerade so von Druiden Stover gehalten. Nur von den beiden Magiern konnte diese missliche Situation gelöst werden. Der weitere Weg war weitestgehend ereignislos.

Doch nach ein paar Stunden schimmerte durch die schneebedeckten Wipfel der Nadelbäume eine schwarze Turmspitze durch. Immer mehr sah man von der kleinen Burg, bis sie sich von den Gefährten in voller Größe erstreckte. Wobei Größe in diesem Fall übertrieben war, da die Burg eine für die Lage zwar beachtliche Größe hatte, jedoch insgesamt relativ klein war. Das Erstaunen war groß, noch größer gar, als festgestellt wurde, dass das Tor der Burg verschlossen war. Nach einiger Zeit schickten sowohl Toran, als ein Druide seinen Geist aus, um die Burg zu ergründen. Nach getaner Arbeit bemühte die Spektabilität Temyr sein gesamtes magisches Potential und ließ die Heldengruppe über die Mauer schweben. Die Gefährten machten sich schnell daran den Tempel von Schutt zu befreien, denn dort vermuteten sie das Grab des Grafen. Doch sie bemerkten bald, dass in der Burg nicht alles mit rechten Dingen zuging, da geisterhafte Erscheinungen durch die Burg flogen und von den Anwesenden anscheinend unbeeindruckt ihrem Tagewerk nachgingen. Nach einiger Dauer war der Tempel freigelegt und die Erkundung konnte beginnen. Schnell interessierten sich die Magier für eine am Eingang der Krypta aufgestellte Rune, die anscheinend dem Drachen Famerlor gewidmet war. Die anderen durchsuchten währenddessen die Krypta und wurden nicht fündig. Als die Magier dazu stießen konnte festgestellt werden, dass sich der Gral aufgrund einer magischen Analyse in einem zur Krypta versetzten Raum befinden musste. Doch ein Eingang wurde nicht gefunden, einzig der Grenzjäger stellte fest, dass vor jedem Sarg ein etwa schwertbreiter Schlitz vorhanden war und, dass sich in einem der Särge eine winzige Fuge befand. Vor einem Rätsel stehend probierten die Gefährten erst ihre Schwerter aus, um festzustellen, dass das die vermutete geheime Türe nicht öffnete. Doch als sie anfingen die Burg nach den Schwertern durchsuchten wurden sie bald im Brunnen fündig. Sie machten sich sofort daran, die Schwerter auszuprobieren und tatsächlich öffnete sich die Türe beim hineinstecken der Schwerter sofort. Im freigelegten unterirdischen Raum fanden sie den sagenumwogenden Kelch der Sumu, von Wurzeln umrankt. Als sie sich aufmachten, um das Fundstück im Lichte des Tages genauer zu betrachten wurde ihnen der Weg von einer geisterhaften Erscheinung versperrt. Ein großer Ritter in voller Rüstung stellte sich ihnen auf sein Schwert gestützt entgegen. Dieser verlangte in getragener Stimme die Herausgabe des Kelches, der nicht umsonst von den Rittern dieser Burg gar von Anstürmen von Drachen beschützt worden war. Doch die Helden, wissend, dass der Kelch für die Zukunft des magischen Netzes in Tobrien einer entscheidende Rolle spielen könnte, gaben nicht nach und so stellte es den geisterhaften Ritter nach einer langen Diskussion zufrieden, die Schwüre der Helden zu hören, den Kelch mit ihrem Leben zu verteidigen und nur in die Hände jener zu geben, die Gutes im Sinn hatten.

In ebenjene Hände der Guten gaben die Helden den Krug auch etwa eine Woche später am 23. Travia, als sie ihn auf dem legendären Eilande Sumus Kate in die Hände von Firnbart dem Druiden gaben. Dieser bedankte sich ausführlich, doch bat sie direkt um einen weiteren Gefallen: Der Steinkreis an der Ostseite des Yslisees Septa Hegon sei von Schergen des Dämonenmeisters Borbarad eingenommen worden und jemand müsse herausfinden, was dort geschehe. Schnell wandten sich die Helden an Grimbart, der von dem Überfall auf den Steinkreis zu berichten wusste. Dieser war es auch, der die Luftdschinne anwies die Helden zum Steinkreis zu bringen. Doch die Dschinne, überwältigt von der Magie, die an diesem Ort wirkte und so setzten sie die Helden ein kleines Stück entfernt ab. Doch auch für die Helden war es unübersehbar, welche Kräfte an diesem Ort wirkten. Während der Magister aus irgendeiner Eingebung heraus die Beschwörung des Omegaterion vermutete, über gab sich der Perainegeweihte lautstark in die Büsche. Schnell beschlossen die Helden, dass hier nichts mehr zu machen sei und kehrten zu den Dschinnen zurück, die sie schließlich auch zurückbrachten. Zurück gekommen erzählten die Helden Firnbart von den Ereignissen. Dieser dachte kurz nach und verkündete dann, dass man versuchen werde die Wirkung des Rituals mithilfe eines Gegenrituals zu entkräften. Jedoch würden die Vorbereitungen Stunden brauchen und es sei mit einem heftigen Gegenangriff zu rechnen. Und so überließen die Gefährten den Druiden die Vorbereitungen für das Ritual, während sie selbst sich um den Ansturm kümmern würden. Temyr ging vorher noch in das Haus, in dem die Elementarmagierin Arya seit Wochen schlafend daniederlag. Mit einer sowohl herzzerreißenden, als auch schmalzigen Rede und einem zärtlichen Kuss erweckte er sie. Die anderen trafen jeweils ihre Vorbereitungen.

Als es dann Zeit wurde und die für das Ritual nötigen Vorbereitungen getroffen waren, versammelten sich die tapferen Streiter an der Ostseite der Insel zusammen mit ein paar Druiden den heftigen Angriff der Borbaradianer erwartend, während die Mächtigsten der versammelten Druiden im Steinkreis von Sumus Kate das Ritual mit der Hilfe des Kelches der Sumus und den zu diesem Zeitpunkt gesammelten Kreissteinen begannen. Schon kurze Zeit später begannen die Angriffe aus dem Osten. Dies zeigte sich zuerst anhand des plötzlichen Verschwindens des seit jeder um die Insel umherwabernden Nebels. Dann, offensichtlicher, anhand der schnell auf die Insel zufliegenden Kreaturen, die nach wenigen Augenblicken die Küste erreichten und sich in den Kampf mit den zur Verteidigung der Insel beschworenen Dschinne stürzten. Doch die Dschinne waren dem daimonischen Pack, welches aus mehreren drachenähnlichen schwarzhäutigen Kreaturen, den sogenannten Karakilim, mehreren überdimensionierten brennenden Salamandern und einer riesigen Flugschlange bestand zahlen- und kräftemäßig unterlegen und so mussten die Helden, nur von einer magischen Schutzkuppel geschützt sich den Wesen entgegenstellen. Zu allem Unglück hatten diese es zuvor noch geschafft einige borbaradianische Beschwörer und Söldner, gar eine Kämpferin mit einem daimonischen Enduriumschwerte, am Ufer abzusetzten. Tapfer schlugen sich die Helden und rangen einige der götterverdammten Wesenheiten nieder, doch zu groß war die Zahl der Gegner und zu schnell waren die ebenfalls zur Verteidigung der Insel abgestellten Druiden durch die daimonischen Klauen der Unwesen dahingerafft, als dass ein Sieg in diesem Kampf möglich gewesen wäre. Und so wandte sich der Geweihte Toran Ostik mit aller Inbrunst seiner Göttin zu und flehte sie an die Mächtigste der anwesenden niederhöllischen Kreaturen dorthin zu schicken, wo sie hervorgekrochen kam. Peraine in ihrer Güte erhörte die Worte des Geweihten und so entschwand die blutrünstige Kreatur. Doch auch diese Wundertat des Geweihten konnte die Überlegenheit der Daimonen nicht entscheidend verringern und so opferten sich die beiden Druiden Stover und der andere Druide in einem Druidenrache genannten Ritual der Sumu und setzten so für sich ungeahnte Kräfte frei, mit deren Hilfe unter anderen mehrere Daimonen und die Trägerin des Enduriumschwertes ihr Ende fanden, während die Helden, ihren Eid einhalten wollend den Kelch der Sumu zu retten sich gen Steinkreis begaben. Dort trafen sie auf Grimbart und einen weiteren Druiden, die sich darüber stritten, ob das Ritual mit dem Kelch beendet werden könne, oder ob der Kelch in Sicherheit gebracht werden müsste. Nach kurzer Überlegung, da die letzten verbliebenen Daimonen ihre Kräfte jetzt auf den Steinkreis konzentrierten, beschlossen die Helden den Kelch in Kreis zu lassen. Während Stover und der andere schon ihren letzten Lebensfunken ausgehaucht hatten, konzentrierte der Magister seine Kräfte in einem letzten mächtigen Spruch, mit dem er sich, Toran, Arya und Temyr von der Insel an Ufer zauberte. Der Verbleib von Kunn, dem Wildhüter ist bis aufs weitere ungeklärt. Und so kam es, dass nach Jahrhunderten der Suche nach dem Kelch der Sumu endlich jemand einen Erfolg vorweisen konnte, doch das Artefakt kurz nach seiner Wiederentdeckung wieder verloren gegangen war.

\subsection{Spionagemissionen im Auftrag des Herzogs von Tobrien nach Rezzanjin al'Ahjan}

\todo[inline]{Explain time split here, half the group did the previous thing, the other half this one}

\dots Im Zimmer des Kanzlers Delo von Gernotsborn bekamen wir dann in Anwesenheit vom tobrischen Herzog Bernfried und seiner Frau Efferdane von Eberstamm den gesamten offiziellen Auftrag. Und wenig war es nicht. Zum einen sollten wir durch die besetzten Lande reisen und dokumentieren inwieweit wo schon Städte besetzt worden sind, eventuelle Truppenbewegungen, sowie die Befindlichkeiten der Städte Eslamsbrück, Ilsur und Mendena herausfinden, sowie eventuell vorhandenen Wiederstand gegen die borbaradianischen Besatzter ausfindig machen. In Mendena selber sollten wir in der Burg des Grafen die tobrischen Hauer und das Schwert Schalljarß, die tobrischen Reichsinsignien an uns nehmen, sowie den Bruder des Herzogs Dietrant von seinem Leid und Schicksal als untoter Bardenspieler für den bruderlosen falschen Herzog Arngrimm erlösen. Von unserem Auftrag hängt also nichts weiter als ab, als das Schicksal und die Aufstellung Tobriens den nächsten Schlachten gegen Borbarad ab, da Bernfried ohne die Herrschaftsinsignien nicht die Legitimität hat um alle tobrischen Barone hinter sich zu halten.
Ich finde es immer noch verwunderlich, dass die tobrischen Barone die Gefahr die von Arngrimm und seinem Bündnis mir Borbarad ausgeht unterschätzen. Sie rennen sehenden Auges in ein Bündnis mit dem bruderlosen Bösen ohne zu erkennen, dass es ihr Verderben ist. Aber was kann man von Mittelreichern schon erwarten. Sie achten ohne Rücksicht auf das, was richtig ist auf ihren eigenen Vorteil. Zum Glück scheint Bernfried und ein zumindest nicht unerheblicher Teil der anderen wichtigen Persönlichkeiten des Mittelreichs anders zu denken und im Sinne Rur‘s zu handeln.
Der Auftrag birgt viel Risiko und wir müssen Kämpfe vermeiden, wenn es geht. Wäre auch besser, wenn wir nicht verletzt werden, denn Toran wird nicht dabei sein, sondern nur Irian, Ragnos und ich. Wir müssen also vorsichtig sein.
Wir rüsteten uns natürlich für die Reise aus. Alle unmagischen Ausrüstungsgegenstände durften wir aus dem Lager von Ysilia entnehmen, für die magischen Gebrauchsgegenstände wie Waffenbalsam und Waffengifte suchten wir die Akademie von Ysilia auf. Dort bekamen wir dann Rationen des Waffenbalsams jedoch nicht die von Irian gewünschten Waffengifte, da dieser mal wieder vergessen hatte, dass er sein KGIA-Siegel verloren hatte, sowie, dass eine weißmagische Akademie im Regelfall keine im Wehrheimer Index stehenden Substanzen auf Lager hat, geschweige denn die an Fremde herausgibt.

Bei der Verabschiedung konnten die hohen Herrschaften uns zwar keine weiteren Informationen zur Burg übergeben, bis auf die wenigen, die wir im ersten Treffen erhalten hatten, doch gaben sie uns einen Führer an die Hand, der sich in der Gegend um Mendena angeblich auskennen soll. Hoffentlich hält er uns nicht auf.
Der Weg aus der Burg führte uns erstmal durch das Lager der Flüchtlinge. Wenn ich mich nicht irre, ist es in den letzten Tagen gewachsen.
Die erste Nacht verbringen wir unter den Sternen. Mir macht das Fehlen der Magier ein wenig Sorgen, da wie so den Magiern der Gegenseite machtlos ausgeliefert sind. Aber sowohl Toran, als auch die Magier sind bei einem solchen Auftrag, in dem es um Schleichen und Verstecken geht wohl eher eine Fessel am Bein, als eine Hilfe.

\paragraph{Mehrere Tage später}
Endlich haben wir mal wieder ein Dach übern Kopf, nachdem Irian es geschafft hatte durch seine sympathische Umgangsart die Leute in der Schenke davon zu überzeugen, dass es nicht ratsam wäre sich seltsam verhaltene Fremde in der Schenke aufzunehmen. Ich frage mich langsam ernsthaft, wie er KGIA Agent geworden ist.

Wie auch immer, ich grübele immer noch darüber nach, was der Baron von Nevelung mit seinem Erscheinen bezwecken wollte und vor allem, warum er so schnell verschwand. Es ist wahrhaftig seltsam, dass er als Druide, das will er jedenfalls sein, mittelreichischer Baron werden konnte und dann auch noch ernsthaft in Erwägung zieht sich Borbarad anzuschließen. Er wirkte jedoch gar nicht wie Pedresco, der einzige Druide, den ich kenne. Wo Pedreso wohl gerade steckt?

Auf jeden Fall war der Baron von Nevelung für die Nebelwölfe verantwortlich, die wir die Nacht davor gesehen hatten. Und als er zu unserem Lager waren diese auch mit dabei.

Nun gut, ich schweife ab. Die Flussüberquerung am nächsten Tag war ein wenig brenzlig, vor allem weil Ragnos nicht gerade der geborene Flussschiffer ist, wobei der Fluss zugegebenermaßen nicht wirklich gut schiffbar war mit all dem Hochwasser. Die Pferde mussten wir bei der einzigen alten Frau im kleinen Weiler stehen lassen, worüber ich nicht wirklich traurig bin, denn schon am nächsten Tag erreichten wir Eslamsbrück. Dank Irians Paranoia und dem Unwillen der anderen kam es am Ende dann dazu, dass nur ich alleine Eslamsbrück betrat. Nach einigen Stunden, mehreren Verhören und vielen Verdächtigungen in meine Richtung verließ ich es dann auch wieder mit der Gewissheit, dass es nicht gefallen war, es wohl aber besser gefallen wäre, denn die Lage der Bewohner war miserabel. Ohne Nahrungsversorgung mitten im Feindesland ist es nicht einfach über die Runden zu kommen. Ich erfuhr, dass die Soldaten fünf Tage nach der Schlacht über Nacht verschwunden waren, vermutlich über den Fluss.

Wie reisten weiter und mussten in der Nacht feststellen, dass wir in der Nähe eines Söldnerlagers mit mindestens 20 Personen gerastet hatten. Wir suchten uns schnell einen anderen Lagerplatz ein gutes Stück entfernt.

Am nächsten Tag hatten wir es geschafft eine Söldnergruppe von fünf in gleicher Weise wie in Kurkum zu überfallen und schnell zu überwältigen. Wir beseitigten die Leichen, nahmen uns vom Proviant, was wir gebrauchen konnten und fanden in den Unterlagen der Söldnerführerin Briefe und Anweisungen unter anderem für Ragnos's Erzfeind. Man hatte tatsächlich bemerkt, dass wir in Ysilia fehlten und vermutlich in Richtung Mendena unterwegs waren. Das sind immer noch beunruhigende Nachrichten.

Auf jeden Fall gibt es ein paar Informanten der Borbaradianer in Ysilia.

An diesem Tag verpassten wir besagte Taverne und mussten in Gebüsch übernachten, wo wir in der Nacht von Wildschweinen überrascht wurden. Tatsächlich setzten sie uns viel stärker zu als ich erwartet hätte, sodass wir am nächsten Tag aufgrund unserer davongetragenen Wunden nur langsam vorankamen. Im Laufe des Tages trafen wir dann auf ein kleines Dorf, das wir auskundschaften wollten, zumindest, bis ich es schaffte einen Dämon direkt vor meine Nase zu rufen. Dieser verriet uns vermutlich sofort an seinen Beschwörer. Viel davon bekam ich nicht mit, denn ich rannte als wäre mir eine Maraskantarantel auf den Fersen. Wir schafften es die Verfolger gar nicht erst nahe kommen zu lassen und liefen bis tief in die Nacht zur Sicherheit bis zu einer Hütte eines Wildhüters, den Gerion kannte. Dieser begrüßte und mehr oder weniger freundlich und stellte sich als Ein-Mann Wiederstand gegen die borbaradianische Besatzung vor. Wichtig war vor allem, dass er den von Baron Yelnal von Dunkelstein und den Standort des von ihm unterhaltenen Wiederstands kannte, zu dem wir heute nach einem längeren Schlaf mittags aufbrachen und in dem wir uns jetzt befinden. Es ist zugegebenermaßen nur ein gut bewachtes Erdloch, aber sie haben Bier und morgen will uns der Baron von Dunkelstein, der zufälligerweise einen Geheimgang in die Burg hinein kennt zur Burg führen. Fast zu schön um wahr zu sein\dots

\paragraph{Nächster Tag}
Am nächsten Tag brachen wir früh auf. Ich geriet doch noch ein wenig ins Grübeln, ob wir dem Baron trauen konnten. Der Wildhüter meinte, dass der Wiederstand um den Baron in letzter Zeit schwere Verluste erlitten hatte und der Baron selber warnte uns niemanden leichtfertig zu vertrauen. Ich traue den Bruderlosen durchaus zu einen fiktiven Widerstand aufrecht zu erhalten, um die Widerstandskämpfer in einer Organisation zu haben und zu kontrollieren. Jedoch ist die Gelegenheit einfach zu günstig, sollte der Baron tatsächlich einen Geheimgang in die Burg kennen. Wir sind heute nur bis etwa eine Tagesreise vor Mendena gekommen. Wir ruhen uns jetzt aus, dass wir für morgen ausgeruht sind und die Nacht hindurch in die Burg kommen und unseren Auftrag erledigen.

\paragraph{2. Travia Eslamsbrück}
Ich hätte es vor sechs Tagen nicht gedacht, dass ich jetzt am Leben bin, geschweige denn, dass ich die Möglichkeit gehabt hätte Tagebuch zu schreiben, aber der Reihe nach:

Den Tag vor Mendena verbrachten wir wie immer in stillschweigender Wanderung abseits der Hauptstraße. Besondere Ereignisse gab es nicht wirklich, außer vielleicht, wenn man das wildgewordene Eichhörnchen mitzählt, welches Irian bei der mittäglichen kurzen Essenspause ein Stück Brot mit Trockenfleisch abluchste. Doch im Vergleich zu den letzten Tagen spürte ich, dass die Nervosität zunahm, immerhin wollten wir in die Höhle des Löwen gehen und die gut gehüteten tobrischen Reichsinsignien direkt aus dem Herzen von Arngrimm‘s neu erbauten Reich zu reißen und zu Bernfried bringen.

Gegen Nachmittag kamen wir am Waldrand an, hinter dem die gerodete Fläche vor der Stadt Mendena und der Feste Talbruck anfing. Schnell sahen wir, dass wir zum Glück nicht auch noch in die Stadt hineinmussten, da die Feste mit kleinem Abstand vor der Stadt stand. Wir beschlossen, dass wir die meisten Gegenstände unserer Ausrüstung wie beispielsweise die Verpflegung am Waldrand zurücklassen und machten uns dann auf, um die Burg herumzuschleichen, da der geheime Eingang auf der anderen Seite der Burg lag. Es dauerte ziemlich lange die Burg zu umqueren, sodass es fast schon dunkel war, als wir eine große Eiche erreichten, zwischen deren Wurzelwerk Jelnal von Dunkelstein eine massive Eichentür freilegte und sie öffnete. Vor unseren Augen zeigte sich ein auf wenige Schritt erleuchtete Treppe, die uns zu einem Gang führen würde, der neben dem Tempel des Boronsangers enden würde. Nachdem wir einer nach dem anderen in den Gang heruntergestiegen waren, schloss der Baron die Eichentür und wir fanden uns in völliger Finsternis wieder. Langsam ertasteten wir uns die Treppe hinunter und den glitschigen, kalten und klammen Gang entlang. Nach einer Zeit fanden wir vor unseren Händen über uns ein massives Hindernis vor. Mit einiger Kraft konnten wir es wegbewegen. Schnell stellten wir fest, dass es sich um die Sargplatte eines unbesetzten Sarges handelte, der in einer Krypta des Boronsangers der Burg lag. Schnell stiegen wir heraus und versuchten die Lage im Burghof zu überblicken. Wir endeckten, dass der Burghof für die vorangeschrittene Zeit ungewöhnlich bevölkert war. In der Mitte führte ein Magier Rituale durch. Vor dem Eingang des Haupthauses standen zwei Wachen, sowie vor der Kaserne in diesem Hof zwei, ebenso vor dem Tor, welches wir zur Erfüllung unseres Plans zu durchqueren gedachten. Blöderweise stand eine Wache auch vor dem Eingang des Boronsangers, sodass wir diesen nur schwer verlassen konnten. Nach reiflicher Überlegung entschlossen wir uns die Wache vor dem Boronsanger, die wie wir endeckten gerade schlief, bewusstlos zu schlagen. Dann würden wir uns in unseren erbeuteten Wappenröcken gewandt als Betrunkene ausgeben und versuchen mit der angeblichen Alkoholleiche in den anderen Zwischenhof zu gelangen. Die Ausführung klappte tatsächlich besser als gedacht und so kamen wir durchs Tor und nachdem wir den Bewusstlosen abgeladen hatten, schafften wir es unbemerkt aus der Kaserne zu laufen. Der Weg durch den Garten zum Küchengebäude wurde nur von einem aufmerksamen Soldaten, den wir abwimmeln konnten unterbrochen. Im Küchengebäude schafften wir es das Küchenpersonal davon zu überzeugen, dass wir in den Keller mussten. In diesem beobachteten wir dann den Gang, der durch das Verließ zum Hauptgebäude führte. Eine Wache stand direkt vor der Tür, eine weitere schob ihre Wache im Gang dahinter. Den vorderen lenkten wir ab und töteten ihm. Beim hinteren warteten wir bis er um die Ecke kam und brachten ihn um die Ecke. Die weiteren vier, die in den beiden Gängen dahinter Wache schoben schalteten wir auch aus, ohne, dass sie andere Wachen auf sich aufmerksam machen konnten.

Im Kerker fanden wir überraschenderweise Irians Knappen Bernfried. Dieser schien sehr erfreut und überrascht, dass wir ihn besuchten. Das Wiedersehen mussten wir leider unterbrechen, da wir eine andere Mission hatten, jedoch fassten wir den Entschluss, dass wir Bernfried mitnehmen wollten, so wir auf unserer Flucht wieder im Kerker vorbeikommen sollten. Wir schlichen vorsichtig weiter in die Eingangshalle und die Treppe hoch zum Thronsaal, der, wie wir feststellten noch mit Menschen gefüllt war. Wir schlichen daraufhin weiter die Treppe hoch und fanden glücklicherweise ein Loch in der Wand zum Thronsaal, durch welches wir die sich unten abspielende Szenerie beobachteten. Wir sahen viele borbaradianische Größen: Arngrimm von Ehrenstein, Azariel Scharlachkraut und weitere. Sie schienen sich bei Musik aus der Harfe des untoten Dietrant zu amüsieren. Nach einer Zeit wurde es hektisch als Jasper von Arsingen den Thronsaal betrat und sich bei Arngrimm beschwerte, dass er nicht die benötigten Informationen zur Verfolgung von uns bekam. Kein Wunder, hatten wir diese doch abgefangen. Die von Jasper zusätzlich angeforderten Truppen zu unserer Verfolgung wurden ihm von Arngrimm bewilligt. Nebenher zog Jasper noch über Scharlachkraut her, da diese uns nach unserer Gefangennahme wieder frei gelassen hatte. Danach erklärte Arngrimm leicht erregt, dass diese Veranstaltung aufgelöst sei und, dass sich ein jeder auf den morgen beginnenden Zug auf Ysilia vorbereiten sollte, der, wie wir erfuhren, hauptsächlich aus Söldnern zusammensetzten sollte, die sich schon in den Kämpfen im Süden mit Ruhm bekleckert hätten. Glücklicherweise blieben die Reichsinsignien Tobriens im Zuge der Aufbruchsvorbereitungen alleine mit zwei Wächtern im Raum zurück. Wir postierten Ragnos oben am Loch mit seinen Bogen und begaben uns nach unten, um die Wächter in ein Gespräch zu verwickeln. Dies gelang weniger gut, im Gegensatz zur schnellen Beseitigung der Wächter. Irian nahm die mendenschen Hauer in Leder gewickelt an sich, während ich das Schwert Schalljarß an mich nahm. Irian machte dann kurzen Prozess mit Dietrand und wir schlichen daraufhin wieder runter in den Kellerraum der Küche, wobei wir an einem Wächter vorbeischlichen und Bernfried mitnahmen. Im Kellerraum einigten wir uns nach einiger Diskussion den Stall in Brand zu stecken und dann über den Geheimgang über den wir die Burg betreten hatten, wieder verlassen. Auf dem Weg nach oben machten wir relativ viel Lärm und schafften es beinahe die Aufmerksamkeit der Dienstleute auf uns zu ziehen. Wir kamen gerade rechtzeitig um das Schauspiel mit anzusehen, in welchem die Jäger um Jasper beim Aufbrechen zu sehen waren, welches anscheinend in einem Nagrachritual vollzogen wurde, da es von bitterer Kälte begleitet wurde.

Wir warteten ein paar Minuten ab, dann schoss Ragnos den Brandpfeil aufs Stallgebäude ab und wir eilten die Treppe herab. Unten angekommen warteten wir kurz ab, bis wir sowohl unten als auch oben Geräusche hörten und rannten heraus, um in Richtung Krypta zu fliehen. Dies gelang uns ohne Probleme, da die Wachen mittlerweile glücklicherweise mit anderen Dingen beschäftigt waren. Schnell öffneten wir den Geheimgang und begaben uns in den dunklen Gang, der uns aus der Burg heraus zur alten Eiche führen sollte. Es dauerte gefühlt eine Ewigkeit, bis wir es geschafft hatten, an der Eichentür anzukommen. Wir stießen sie auf und erblickten sogleich das Gesicht von Yelnal von Dunkelstein, der. wie er sagte, erleichtert sei, dass wir es geschafft hatten. Schnell begaben wir uns im Schutze der Dunkelheit wieder in den Wald, in dem wir den Hauptteil unserer Ausrüstung und die Verpflegung gelassen hatten. Wir legten noch ein wenig Weg zwischen uns und die Stadt und machten dann kurz Pause um wenigstens noch ein klein wenig Erholung für die Nacht zu haben. Am Morgen wollten wir uns dann von Yelnal von Dunkelstein trennen. Er bestätigte noch einmal, dass Ilsur gefallen war und dann bedankten und verabschiedeten wir uns. Über den Tag reisten wir sehr schnell, sodass wir es schafften am späten Abend die Hütte des Gerion bekannten Wilderers zu erreichen. Leider war dieser nicht anwesend, sodass wir außerhalb der Hütte schlafen mussten. Doch wurden wir, im Nachhinein betrachtet vielleicht nicht ganz so unerwartet, in der Nacht von Wolfsgeheul geweckt. Und wurde schnell klar, dass dies die Eiswölfe waren, deren Heulen wir schon beim Aufbruch der Paktierer aus Talbruck gehört hatten. Schnell packten wir unsere Sachen und versuchten ihnen aus dem Weg zu gehen, indem wir flohen, doch wir wurden überrascht. Mehrere Eiswölfe griffen uns an und wir versuchten uns so gut es ging zu verteidigen. Doch die Dunkelheit erschwerte uns den Kampf deutlich, sodass wir schnell in die Defensive gedrängt wurden. Als die Wölfe dann plötzlich umdrehten und unter dem Fluchen eines Bruderlosen aus einem Gebüsch das Weite suchten entdeckten wir, dass wir gerade so glücklich davon gekommen waren. Gerion fanden wir tot mit einem Pfeil durch den Kopf auf dem Boden liegend. Irian war schwer verwundet und auch Ich hatte Wunden am Hals davon getragen, als ein Wolf mir an die Kehle sprang. Ragnos verband uns provisorisch und glücklicherweise war Bernfried, Irians Knappe, unverletzt geblieben. Schnell erwiesen wir Gerion die letzte Ehre und machten uns auf das zweite Mal in dieser Nacht vor den Ereignissen zu fliehen. Bis Bruder Praios`s Feuerball sich hinter dem Horizont hervorhob liefen wir irgendwie weiter und schlugen dann erschöpft unser Lager auf. Wir beschlossen noch, bevor wir uns zum Schlafen danieder legten, dass wir ab jetzt in der Nacht reisen und am Tag nächtigen würden, um so eine nächtliche Überraschung durch die Eiswölfe zu umgehen. Zwei Tage lang schafften wir es ohne größere Zwischenfälle zu reisen und wir unsere Wunden langsam auszukurieren. Nach eben diesen zwei Tagen fanden wir gerade als wir kaum zwei Stunden gereist waren ein Wirtshaus am Straßenrand vor. Unsere Vorräte waren knapp geworden und so erhofften wir uns, sie im Wirtshaus aufstocken zu können. Doch vor dem Wirtshaus bot sich uns ein schrecklicher Anblick: Eine Schankmaid war draußen vor dem Wirtshaus aufgeknüpft worden. Vielleicht war sie von dem bruderlosen Söldnergesocks dort als Mahnmal sich nicht gegen sie zu erheben platziert worden, viel eher vermute ich, dass die Söldner aus purer Lust am Töten über sie her gefallen waren. In der Küche fanden wir die Leiche des Wirtes und Bruder Phex sei Dank Verpflegung. Wir waren gerade dabei sowohl Schankmaid als auch Wirt in den Holzschuppen am hinteren Ende des Wirtshauses zu schaffen, als wir von einem Trupp bruderloser Söldner überrascht wurden. Glücklicherweise schafften wir es uns im Holzschuppen zu verstecken, doch mussten wir uns die Witze dieses bruderlosen Packs über das Ableben des Wirtes anhören. Nach kurzer Absprache war klar, dass wir den Holzschuppen anzünden würden, um uns anschließend davonzuschleichen. Dies getan setzten wir unsere Reise fort. Möge dieses Bruderlose Pack mitsamt Wirtshaus verbrannt sein.

Doch der nächste Tag wurde nicht besser. Wieder trafen wir kurz nach unserem Aufbruch aus unserem kleinen Lager auf Borbaradianer. Diesmal sahen wir von einem Waldrand aus, wie der Anführer eines Trupps von etwa zwei Dutzend Borbaradianern einen Bauern befragte, während dieser ausgepeitscht wurde. Immer wieder schlug die Peitsche auf den Bauern ein, während dieser unter Schmerzensschreien betonte nicht zu wissen wer den Herzog bestohlen hatte, geschweige denn wo die Diebe seien. Ich war kurz davor aus dem Wald zu stürmen um es mit diesen bruderlosen und zwölfgeschwisterverachtenden elendigen Bastarden, die die Bauern nur aus reiner Lust folterten, alleine aufzunehmen, als wir von dem Pack entdeckt wurden. Schnell rannten wir in den Wald hinein und suchten uns einen Ort, an dem wir es im Zweifelsfall mit dem ganzen Haufen aufnehmen konnten. Während Irian und Ragnos auf Bäume kletterten, blieb ich in einer Spalte zwischen zwei Felsen stehen und erwartete die Feiglinge. Lange dauerte es nicht bis drei dieser sich langsam ihren Weg durch den Wald bahnten. Sie entdeckten mich beim Vorbeilaufen nicht, erst als der erste mich in meinem Versteck erahnte, steckte diesem sogleich ein Pfeil im Kopf. Einen weiteren erledigte Ich und Irian schaltete den Letzten in einem sehenswerten Manöver aus, indem er von seinem Baum sprang und ihn sein Schwert in den Kopf hieb.

Die weiteren zwei Tage blieben ereignislos und so wähnten wir und schon fast in Sicherheit hinter den Mauern Eslamsbrücks, als wir, es müsste der 26. des Mondes Efferd gewesen sein, von Schnee überrascht wurden. Wie mir Ragnos und Irian versicherten, war das selbst für den Norden und Tobrien ein sehr früher Wintereinbruch und so stapften wir in der Nacht durch den Schnee. Doch Bruder Phex war uns nicht gewogen und so gerieten wir in einen Hinterhalt. Plötzlich sprach aus dem Wald hinter uns eine Stimme. Sie verlangte, dass wir die Throninsignien übergeben sollten, so uns unser Leben lieb sei. Eine unmöglich zu erfüllenden Forderung und so traten vier Männer hinter Bäumen hervor, bewaffnet mit Schwertern. Normalerweise würde ich in einer solchen Situation die Echse in mir erwecken, doch diesmal was es seltsam. Sobald ich in der Echsenform war, verlangsamten sich meine Bewegungen. Schnell wechselte ich zurück und initiierte mein Axxeleratus-Artefakt, das die Magier mir erstellt hatten. Doch in der Dunkelheit war dieser Kampf von Anfang an schon verloren. Schnell standen mir zwei groß gewachsene Söldner gegenüber die mich abwechselnd mit ihren Schlägen malträtierten. Schnell bekam ich zwei Wunden am Arm und am Bein und noch währenddessen hörte ich Ragnos's Schrei, der sich zwecks besserer Übersicht in einen Baum zurückgezogen hatte. Mit einem weiteren Streich fiel ich, unfähig mich größer und ohne heftigste Schmerzen zu bewegen. Doch Irian stand noch und auch Ragnos war noch nicht vom Baum gefallen. Ich hörte noch den letzten Schrei eines Söldners bevor mir die Augen zufielen.

Später wachte ich dann in Eslamsbrück auf und konnte mich ob meine Wunden kaum bewegen. Wie sich herausstellte hatte ein Wolf die Jäger aus der Stadt zu uns geführt. Man hatte länger gebraucht, um uns nach Eslamsbrück zu schaffen, da auch Irian und Ragnos ohnmächtig waren. Bernfried hatte sich anscheinend während des Kampfes versteckt und das Schwert Schalljarß und die mendenschen Hauer in Sicherheit gebracht. Viel ist sonst nicht passiert. Der tobrische Marschall Isenborn, dessen Truppen den Weg nach Eslamsbrück gefunden und gesichert hatten, besuchte uns nachdem wir aufgewacht waren und befragte uns zur Lage. Wir berichteten ihm von unseren Erfolg und dem heranrückenden borbaradianischen Heer. Wir erfuhren, dass Warunk am 16. Travia gefallen war.

Schnell wollten wir zurück nach Ysilia und nach einem Tag bescheinigte der Wundarzt uns Reisefähigkeit. Doch Irian und mich übermannte das Wundfieber und so konnten wir das Bett nicht verlassen. Es muss der erste Travia gewesen sein und ich kann mich nur noch schemenhaft daran erinnern, dass ein Perainegeweihter in unser Zimmer kam und mit uns zu Schwester Peraine betete. Heute geht es uns besser und morgen werden wir zusammen mit dem Marschall nach Ysilia aufbrechen, um dem heranrückenden Heer bruderloser Söldner in der Baronie Vierreichen Einhalt zu gebieten.

\paragraph{5. Travia}
Die Dynamik, die die Ereignisse der letzten Tage entwickelt haben ist schlicht gesagt beeindruckend. Doch wie die Zeiten nun mal sind, so ist auch die Dynamik nicht unbedingt als gut anzusehen.

Es begann damit, dass wir am 3. Travia von Eslamsbrück in Richtung Viereichen aufbrachen. Über den provisorisch auf den Trümmern der alten Brücke errichtete Holzsteg verließen wir die Stadt mitten im Feindgebiet zusammen mit dem Marschall Isenborn und dessen Truppen. Über die leicht verschneiten Feldwege traten wir die beschwerliche Reise an, die uns am Ende des Tages zu den Hügeln der Nachbarbaronie von Vierreichen führte, hinter denen wir Rauch erblickten. Neugierig erklommen wir die Hügel und entdeckten, dass Arnude oder wie die verdammte Stadt der Baronie heißt, brannte; vor der Stadt ein Lager der Borbaradianer in dem wir auch das Wappen von Arngrimm von Ehrenstein entdeckten. Er war mit seinem Heereszug anscheinend doch sehr schnell gereist. Beunruhigt von diesem Anblick setzten wir unseren Weg nach Viereichen fort, wo wir am fortgeschrittenen Abend auch angelangten.

Wir wurden auch direkt von einer Delegation um den tobrischen Herzog Bernfried, dessen Frau, seinem Kanzler und dem Schwert der Schwerter empfangen und begrüßt. Noch bevor wir und einen Platz zum Aufschlagen unserer Zelte aussuchen konnten, wurden wir in das Zelt von Bernfried geführt. Er bedankte sich bei uns und forderte uns auf ihm die Geschehnisse zu schildern. Wir taten dies schnell, jedoch ohne wichtige Details auszulassen. Besonders warnten wir ihn vor den mendenschen Hauern, da diese verzaubert seinen könnten. Im Nachhinein gesehen an sich sinnvoll, jedoch nutzlos. Nach der kleinen Erzählrunde wandte er sich uns zu und eröffnete uns, dass er vorhatte dich den Borbaradianern in einer offenen Feldschlacht entgegenzustellen, um die Schlacht nicht nach Ysilia zu tragen. Wenn das borbaradianische Heer so vorrücken würde wie erwartet, so wäre der Zeitpunkt der Schlacht auf den 5. Travia zu datieren. Eine ausführliche Schlachtbesprechung würde sich zu diesem Zeitpunkt anschließen und wir blieben, erfuhren jedoch nicht viel. Immerhin schienen wir eine realistische Möglichkeit zu haben, die Schlacht zu gewinnen, da die zwischen beiden Herzogen noch unentschiedenen Barone mit den Reichsinsignien wohl auf unsere Seite wechseln würden. Mit einem Banner mehr auf unserer Seite standen wir zwar, was Ausrüstung und Ausbildung der Truppen anging schlechter da, jedoch war der Vorteil der Masse nicht zu verachten. Am Ende der Besprechung vertraute uns Bernfried mit einem zufrieden erscheinenden Gesichtsausdruck noch an, dass am 12. Boron eine Heerschau des Reiches in Praske stattfinden würde. Die guten Nachrichten am Ende des Tages machten den Aufbau des Zeltes doch gleich kurzweiliger. Endlich hatte man im arroganten Gareth erkannt, dass Borbarad kein Laienzauberer mit Größenwahn, sondern eine sehr ernst zu nehmenden Gefahr war. Mit den Truppen aus dem gesamten Mittelreich, so sehr sie den maraskanischen auch in Bezug auf Kampfkraft, Moral und Stil unterlegen waren, könnte sich der Krieg schnell zugunsten der unsrigen, richtigen Seite entscheiden.

Der 4. Travia war geprägt von allgemeiner Geschäftigkeit, die mit den Vorbereitungen auf eine Schlacht einhergeht. Wir selbst hatten nicht sonderlich viel zu tun und so nahmen wir uns viel Zeit, um uns um die überraschend vor unserem Zelt aufgetauchten Leute zu kümmern, die unsere Geschichten hören wollten, unseren Segen haben wollten oder sich allgemein der Präsenz zweier der Gezeichneten in ihrem Lager versichern wollten. Am Abend gab es nach dem Abendessen in Bernfrieds Zelt noch eine kurze Besprechung, die zum großen Teil das am Vorabend Besprochene bestätigte. Wir, zu schade um uns an der Front zu zerreiben sollten beim Herzog bleiben und flexibel auf die Begebenheiten und seine Befehle reagieren. Die Barone, die am Vortag noch unentschieden zwischen den Lagern der Schlachtparteien gelagert hatten, waren, nachdem Bernfried die herzoglichen Insignien hervorzeigen konnte, auf unsere Seite gewechselt. Des Weiteren teilte der Herzog uns mit, dass der mit der Analyse der Hauer beauftragte Magier bei den Hauern nur den Träger positiv beeinflussende Zauber festgestellt hatte.

So gingen wir mit den besten Voraussetzungen in den entscheidenden Tag, dem Tag der Schlacht. Gegen Mittag standen die Formationen der beiden Heere bereit und die Schlacht konnte beginnen. Der Herzog ließ sich von Ragnos die Hauer umlegen und so zogen unsere Truppen von Rondrageweihten angeführt, Schlachtchoräle der Rondra auf den Lippen in die Schlacht. Schnell trafen die Truppen aufeinander und die Schlacht entbrannte. Schnell wurden wir losgeschickt, um die sich auf den Hügeln zur Tobimora hin befindlichen feindlichen Geschützte zu zerstören, die unter unseren Truppen eine blutige Ernte einfuhren. Schnell jedoch wurde ich als einziger unserer kleinen Truppe von einem dämlichen gegnerischen Pikenier von meinem dämlichen Pferd geholt. Nie wieder werde ich mit so einem Vieh in die Schlacht reiten. Der Pikenier musste für seine Tat natürlich sterben und leider schaffte ich so den Anschluss zu Ragnos und Irian nicht, die auf ihren Pferden in Richtung der Katapulte davongallopierten. Kurze Zeit versuchte ich ihnen zu folgen, jedoch war ein schnelles Vorankommen durch die Schlachtreihen unmöglich und so wendete ich mich danach wieder in die Richtung des Herzogs, versucht auf dem Weg viele der Gegner aus der Schlacht zu nehmen. Nach einiger Zeit, ich war noch nicht beim Herzog wieder angekommen, hörte der Beschuss auf unsere Truppen auf und Flammenkugeln erleuchteten die den Katapulten zugewandte Flanke der Borbaradianer stark. Doch als ich kurz vor den anderen wieder beim Herzog angekommen war, wendete sich der Verlauf der Schlacht plötzlich. Es drang Nachricht an uns heran, dass wir vom Lazarett abgeschnitten worden waren. Von Wölfen, die uns von hinten angriffen. Schnell wurde der Rückzug befohlen, doch konnte die gegnerische Kavallerie tief in unsere Reihen vordringen und viele unserer Truppen niedermetzeln. Auf einmal, mitten im Rückzug begriffen, sahen wir einen pfeilförmigen weißen Blitz in mitten der Runde des Herzogs eintreffen. Die Magierin des Herzogs fiel tot zu Boden und mit ihr der Herzog, tot, ohne erkennbare Wunde oder ersichtlichen Grund. Pfeilschnell machte die Nachricht die Runde in unserem Heer und der vormals halbwegs geordnete Rückzug wurde chaotisch. Als der Kanzler gerade den Herzog aufheben wollte, um ihn in Richtung Lazarett zu bringen, wurden wir von Jasper von Arsingen, Ragnos Erzfeind, überrascht, der auf einem 10 Schritt großen drachenähnlichem Dämon reitend, unseren Weg mithilfe des Unwesens versperrte. Sofort wütete der Dämon zusammen mit Jasper unter den Anwesenden. Ewig schien der Kampf zu dauern, die von hinten heranrückenden Gegner bedrohlich nahe, als Irian es endlich schaffte den toten Herzog auf seiner Schulter zum Wald in Sicherheit zu tragen. Ich schaffte es mich aus dem Kampf mit dem Dämon und Jasper zu befreien und rannte hinter Irian her. Ragnos schaffte selbiges und so gelangten wir knapp nach Irian in den Wald.

Mittlerweile hatte leichter Schneefall eingesetzt und wir hörten durch die dicht aneinander gewachsenen Bäume die Geräusche der Schlacht nur noch sehr gedämpft. Kurz später verstummten sie und wir fanden uns in einer seltsamen Atmosphäre wieder. Die Schlacht verloren mit dem toten Herzog im Wald, abgeschnitten vom Kampf, krampfhaft an der letzten Hoffnung festhaltend, doch verloren zwischen den schneebedeckten Bäumen. Doch plötzlich hörten wir ein scharfes Halt und aus den Schneegestöber rechts von uns schälte sich die Gestalt, die man am wenigsten sehen wollte: Arngrimm von Ehrenstein. Ihn begleitend ein Dutzend Söldner, die Hälfte mit auf uns gereichten, gespannten Bögen. Mit einem höhnischen Lächeln wies er uns an den Herzog fallen zu lassen, was Irian kurz zögernd auch tat. Dann stieg er in genüsslicher Langsamkeit von seinem Ross ab und ging zur Leiche des Herzogs, nahm ihn die Hauer ab und hängte sie sich um. Doch als er sich zum Gehen wandte, befahl ihm eine weitere im Schneegestöber auftauchende Gestalt anzuhalten. Ein alter, nein uralter Mann ging festen Schrittes auf Arngrimm hinzu und als dieser auf Arngrimms Frage wer er sei antwortete er sei Jarlach von Ehrenstein, erster der tobrischen Herzoge, Heiliger des Firunsglaubens, erbleichte Arngrimm, ließ die Hauer fallen und ritt mitsamt seinen Mannen, wie von Dämonen verfolgt davon. Seelenruhig nahm sich der alte Mann die Hauer, während wir vor Erstaunen unfähig jeder Handlung danebenstanden, betrachtete sie kurz, packte sie dann fester, während sie sich in seinen Händen in zwei Kristalle verwandelten. Diese verflüssigten sich dann, als der alte Mann sie über Bernfrieds Leiche hielt und die Flüssigkeit tropfte auf Bernfrieds toten Körper und dieser erwachte. Danach wandte sich Jarlach zu uns. Er erzählte uns, dass Arngrimm Bruder Firun erzürnt hätte. Firun habe deshalb in seiner Wut Kälte auf Tobrien hinabgesandt und ein Winter des Wolfes stehe an, der kälter als alle Winter bisher werde. Diese Worte gesprochen wandte er sich Irian zu und deutete ihm in den Wald zu folgen. Bernfried hingegen stand auf und eilte aus dem Wald heraus zum Lazarett. Ragnos uns ich folgten ihm und zusammen trafen wir Efferdane von Eberstamm, Bernfrieds Frau, blutüberströmt aber am Leben an. Nach einer kurzen innigen Umarmung forderte Bernfried einen Soldaten an, der ihm Nachricht bringen sollte und für ihn schreiben etwas dokumentieren sollte. Dann wendete er sich zu uns um und sprach, während der Soldat eifrig das Papier systematisch mit Tinte beschrieb. Ragnos, dem nicht anwesenden Irian und nicht zuletzt mir gebühre der Dank des Hauses Ehrenstein, doch für Ragnos gab es mehr als das. Der Herzog fragte ihn, ob er die seit einiger Zeit nicht besetzte Stelle des Barons von Grunewald annehmen wolle. Ragnos bejahte. Und so nahm der Herzog Tobriens Bernfried von Ehrenstein meinem guten Freund Ragnos dem Jäger am Waldesrand, nach der blutigen Schlacht von Viereichen den Lehensschwur ab.

\subsection{Heerschau in Praske nac Ragnos von Grunewaldt}

\paragraph{16 Boron 1020 B.F. Praiosstunde}
Am Horizont erscheinen die ersten Zinnen der Feste von Praske, in der die Heerschau vom Reichsbehüter höchst selbst einberufen, stattfinden soll. Die Kälte lässt kleine Dampfwolken entstehen bei jedem Atemzug und ein wohliges Schaudern durchfährt mich an den Gedanken eines wärmenden Lagerfeuers, während die Stadt immer näher rückt.

Nach nicht all zu langer Zeit erreiche ich mit meinem Gefolge und meinen Gefährten Praske. Eine beachtliche Zeltstadt hat sich bereits durch die vor uns eingetroffenen Adligen , Barone, Grafen und deren Gefolge aufgebaut. Die verschiedensten Wappen und Banner wehen im kalten Schneegestöber. Irian könnte mir sicher mehr erzählen, hat er sich doch als sehr bewandert in Sachen Heraldik, Staatskunde und Politik herausgestellt. Ich denke zusammen mit Toran werde ich den mir jüngst anerkannten Baronstitel gerecht werden, hoffe ich zumindest.

Unser Tross hält nun auf das Lager von Herzog Bernfried zu, um dort unser Lager aufzuschlagen.

Wenige Stunden später sitzen wir am Lagerfeuer um uns bei einem heißen Becher aufzuwärmen und uns anschießend bei Bernfried vorstellig zu machen.

Nach einem recht kurzen, doch sehr nettem Gespräch mit Bernfried meinem Herzog, weiß ich nun das die Heerschau zwei bis drei Wochen dauern wird und alles im allen für unsere derzeitige Lage sehr verschwenderisch daher kommt. Weshalb es auch gilt die nicht vom Krieg betroffen Barone und Adligen von der durchaus kritischen Situation Tobriens zu überzeugen.

Toran wird dabei sicherlich hilfreich sein. Nun suche ich im Lager erst mal nach meinen restlichen Gefährten, die sich über das Lager verstreut haben, um sie über ihre Einladung zum Abendessen mit dem Herzog Bernfried zu informieren. Und tatsächlich am Abend betreten wir fast vollzählig, das Zählt des Herzogs, nur Firnen fehlt, allerdings mit einer recht angenehmen Vertretung. Aria hat sich zu unserer Gruppe hinzugesellt nachdem Temyr mich leicht beschwipst von einer der Lagerfeuergruppen und deren Gebräu, wie ich mir von Iran hab sagen lassen, fragte ob sie mitkommen dürfe.

Nun speisen wir in hoher Gesellschaft oder zu großen Teilen auch alten Freunden zu Abend. Tatsächlich bin ich recht erfreut Oleg mal wieder zu sehen, welcher mit dem Schwert der Schwerter an der Tafel sitzt. Des weiteren wohnen noch Walpurga von Weiden, Rondragan von Streitzig und natürlich Bernfried mit seiner Gemahlin dem Abendmahl bei. Ich unterhalte mich etwas und erfahre Recht schnell das die anderen Barone und Grafen für den Krieg so ziemlich alles einziehen, was ein Schwert oder Schleuder halten kann. Grausam, zurück werden elternlose Kinder und altersschwache Greise bleiben, die ihre Kinder nicht einmal mehr begraben können. Mir graut vor dem Tag da ich das selbe tun muss und die Schreie der Sterbenden mir in den Schlaf folgen werden.

Die Zwölfe mögen uns beistehen.

\paragraph{17 Boron}
Gerade als wir unser Frühstück beenden wollen, erscheint ein edel gewandeter Mann in garether Farben. Als auch schon mit lauter Stimme verkündet, dass wir für den 20 Boron am Mittag zum Reichsbehüter geladen sind. Alles schön und gut für meinen Geschmack hätte er allerdings nicht so schreien müssen, hab ich doch meine gestrigen Sorgen am Abend im Alkohol ertränkt. Was mir im nach hinein nicht wirklich geholfen hat, aber diesen Fehler werde ich wohl noch öfter machen.

Am Mittag mach ich mich auf den Weg zu einem alten Bekannten, dem Bären von Weiden. Im Zelt treffe ich zu meiner Verwunderung Rezzanjin an, welcher bereits bei einem großen Krug Bier am Tisch sitzt, kurze Zeit später bin ich mit von der Runde. Ich halte mich allerdings wegen des gestrigem Abends etwas zurück. Anschießend gehe ich mit Rezzanjin noch ins Lager, wo er sich noch weiter einen Namen als bester Kämpfer des Lagers macht, kurzerhand setze ich noch einige Silberlinge auf seinen Sieg und welch Wunder wenige Minuten bin ich einige Silberling reicher.

Anschließend schlendern wir zurück zum Lager um den Abend ausklingen zu lassen.


\paragraph{18 Boron}
Iran wirkt heute morgen sehr schlecht gelaunt, aber wann ist er das mal nicht. Ich werde ihn einfach in Ruhe lassen und nicht drauf ansprechen, das hat sich bisher ganz gut bewehrt denke ich. Am Mittag überbringen mehrere Vertreter von Baronen und eines Grafen ihre Einladungen mit ihnen zu speisen. Unter ihnen auch zwei Baroness persönlich, deren Einladungen ich aus Höflichkeit natürlich persönlich nachgehen werde.

Auch dem Vertreter des Grafen gebe ich die Zusage, dass ich persönlich erscheinen werde, ich denke ich werde Toran fragen ob er mir in diesen Fall behilflich sein kann.

\paragraph{18/19 Boron}
Die nächsten beiden Tage verlaufen weitgehend ereignislos. Ich bin noch auf diversen Essen eingeladen und versuche die Grafen und Barone von der Situation Tobriens zu überzeugen, was mir teilweise so darf ich sagen gelingt.

\paragraph{20 Boron}
Heute steht der Besuch bei seiner kaiserlichen Majestät, dem Reichsbehüter an. Wir gehen frühzeitig los, um keinesfalls zu spät zu kommen und so müssen wir tatsächlich noch einige Minuten warten. Wir werden dann schließlich vom Schwert der Schwerter persönlich hineingeleitet zusammen mit Oleg. Am Eingang legen wir unsere Waffen ab, wie es die Höflichkeit verlangt. Nun werden wir endlich vorgelassen, bringen die Standesgemäßen Verbeugungen hinter uns und beginnen auf die Aufforderung des Reichsbehüters unsere Erlebnisse seit der Gor bis heute zu schildern. Besonderes Augenmerk wird natürlich auf die letzten Schlachten bei Kurkum uns Eslamsbrück gelegt und welche wichtigen Persönlichkeiten der Feind in seinem Heer mitführt. Nach einigen Stundens des Erzählens dankt uns der Reichsbehüter für den ausführlichen Bericht und entlässt uns. Erschöpft kehren wir in unser Lager zurück mit dem Ausblick auf schwere Zeiten.


\subsection{Der Fall von Ysilia nach Irian von Rabemund}

\paragraph{1. TSA im Jahr 5 nach Borbarads Erscheinen}
An den ersten Tag des neuen Monats sieht man Temyr und Firnen im Beisein der Spektabilitaet von Ysilia und dem ersten Hofmagus, wo sie über den Einsatz von Seiten der Borbaradianer und des Reiches im bisherigen Krieg diskutieren.

Am Nachmittag nehmen Rezzanjin, Toran und Ragnos an einer Stabsbesprechung mit Reichsbehueter Brin und Prinzessin Emer, Herzogin Efferdane, dem Schwert der Schwerter, Prinz Pelmen, Marschall Iseborn. Sie sprechen über die Situation in Tobrien, wenn der Winter endet, denn dann können die Borbaradianer rund 3000 Soeldlinge gegen das Reich werfen. Wichtige Punkte sind, dass Fürst-Protektor Haffax vom Eilande Maraskan zurück gerufen wurde. Ausserdem zieht der Schwertzug der Rondrakirche Richtung tobrischer Ostkirche und verheert das feindliche Hinterland. Es wird schlussendlich beschlossen, Ysilia auf eine Belagerung vorzubereiten und die Fluechtlinge moeglichst schnell weiter Richtung Norden zu bringen, eventuell sogar ueber den Sichelpass nach Weiden. Efferdane uebernimmt die Aufsicht ueber die Belagerungsvorbereitungen, Marschall Iseborn trainiert die Landwehr weiter und Baron Ragnos spricht mit der Akademie ueber ihren Beitrag zur Verteidigung.

\paragraph{2. TSA}
Die ersten Fluechtlinge ziehen Richtung Norden

\paragraph{6.TSA}
In der Nacht vom sechsten auf den siebten Tsa traeumen wir alle und Herzog Bernfried von einem brennenden Boron's Rad vor Ysilia, dem Markenzeichen Haffax. Uns wird aber nicht geglaubt, dass Haffax die Seiten gewechselt hat.

\paragraph{13. TSA}
Der Herzog haelt eine Adelsversammlung ab, wobei ein Bote vom Baron von Ebelried vov 1000 Soeldlingen vor der Burg Ebelried berichtet und im Namen seines Herren, um Entsatz bittet. Der Reichsbehueter will mit vier Regimentern (2000 Soldaten) Richtung Ebelried ziehen, obwohl wir Ihm sagen, dass es eine Finte des Feindes ist und der Hauptschlag in Ysilia fallen wird.

\paragraph{14. TSA}
Baron Rangolf von Liliengrund berichtet von gut 2 Regimenten Soedlingen, welche durch seine Baronie Richtung Ysilia ziehen, dass Reichsheer ist leider schon auf dem Weg nach Ebelried.
Efferdane erzaehlt uns von ihrem ersten Treffen mit Bernfried.

\paragraph{15. TSA}
Das feindliche Heer ist da!
Der Baron von Südwall bittet uns die Stadt kampflos zu übergeben, er sichert uns freies Geleit zu. Der Herzog verneint dies. Haffax persoenlich erscheint auf dem Schlachtfeld und prophezeit den Fall Ysilia sei nur eine Frage von Tagen.
Ich verstehe nicht, warum mein Lehrmeister dem Reich den Rücken gekehrt hat, obwohl die Herrschaft über Maraskan eine ganz schoene Schmach war. Das Schlimmste aber ist, wir wissen, dass er die Wahrheit spricht, noch nie hat jemand gegen Haffax bestanden, und er weiss es auch.

An diesen Abend lassen wir uns von Rezzanjin zu eine Spaehmission verleiten. Wir ueberfallen einen kleinen Aussenposten mit 20 Gegnern, und obwohl wir diese besiegen, bekommen wir keinen wertvollen Informationen, nehmen aber schwere Wunden hin, verflucht seiest Du, Rondra.

\paragraph{16.TSA}
Schon am naechsten Tag greift das feindliche Heer an, doch es hat nur einen Rammbock und ein paar Leitern, zu wenig um eine befestigte Stadt zu nehmen.

Doch es erhebt etwas gar Abscheuliches aus den Fluten des Ysili Sees, ein Mahaytam, eine Daemonenarche. Gut 80 Meter lang, überall fließt schwarzes Wasser hervor, hebt sich die Daemonenarche auf ihren sechs Beinen Richtung Stadtmauer. Auf ihr stehen dutzende Krabbenwesen, sogenannte Hummerier.

Es laesst sich immer wieder auf die Mauer fallen und schlaegt so eine weite Bresche in die Mauer über die Soeldlinge und Hummerier in die Stadt eindringen.

Wir bilden mit Marschall Iseborn und ein paar Wolfengardisten einen Stosstrupp.

Über die Mauer dringen wir in den Mahaytam ein. Das Innere besteht aus verfaulten Holz und dicken Adern mit schwarzem Wasser. Als ein Gardist eine Ader zerschlaegt, faengt es an furchtbar zu stinken und der Gang verengt. Wir rennen tiefer in die Daemonenarche hinein und in der ersten Kammer toeten wir ein paar Hummerier. In der zweiten Kammer finden wir eine Art Herz. Dies exorziert Toran, whaerend wir ihn gegen immer mehr Hummerier verteidigen. Nachdem Exorzismus opfert sich der Marschall und die Gardisten, um uns die Flucht zu ermöglichen.

Die Daemonarche zieht sich zwar zurück, aber der Schaden ist schon angerichtet. Die Mauer ist zerstoert, der Marschall gefallen und die eindringenden Hummerier richten ein Massaker unter den Verteidigern an.

Ich bringe Rezzanjin zur Heilung auf die Burg und berichte Efferdane von der Lage der Stadt. Gleichzeit muss Firnen ansehen wie ein Daemon durch seinen Gardianum schreitet und die Spektabilitaet zerreisst. Toran muss Bernfried abhalten sich sinnlos zu opfern, als Efferdane von der Burg Verstaerkung bringt. Als sich die Nacht ueber Ysilia senkt, ist die Stadt verloren und man zieht sich kaempfend auf die Burg zurueck. Es verbleiben knapp 200 Streiter.

Haffax versucht Bernfried zu einer Uebergabe der Burg zu bekoomen, dieser lehnt wieder ab.

Kurz darauf greifen Galottas Anhaenger und Maraskaner die Akademie an.Wo schnell maechtige Magierduelle entbrennen. Temyr und Toran unterstützen die Akademiemagier bei ihrem verzweifelten Abwehrkampf. Firnen versucht das unersetzliche Wissen der Akademie zu retten, wird aber von einer Rohalswächterin dabei gehindert, da sie ihn als Verraeter sieht.

Die Akademie faellt schliesslich und parallel startet der Angriff auf die Herzogsburg durch besessene Soldaten, welche die Tore oeffnen, und weiße Daemonenwoelfe, die dann in die Burg eindringen. Rezzanjin greift die Daemonwoelfe an und ich die Besessenen, aber schnell ist der Kampf verloren und wir ziehen uns in den Bergfried zurück. Dort treffen Rezzanjin und ich im obersten Stockwerk mit Bernfried, Efferdane und Pelmen zusammen, um die Flucht aus der Burg zu planen.

Als ploetzlich die am Fenster stehende Efferdane umkippt.

Aus ihrer Brust ragt der Sechste der verfluchten Freipfeilen, die grausam unter den Ehrensteins Ernte hielten. Statt Bernfried entgueltig zu berechenen, steht er das erste Mal in deiesem Krieg gerade und man sieht, dass erste Mal Trotz und haelt eine bewegende rede zu den letzten Ueberlebenden Ysilias, bevor wir durch einen geheimgang Richtung Ebelried aufbrechen.

\paragraph{20. TSA}
Der Reichsbehüter gibt zu auf eine Illusion reingefallen zu sein, und bittet den Herzog um Verzeihung und die beiden versöhnen sich. Wir werden zu Reichsberater ernannt, damit unsere Warnung immer das Ohr des Reichsbehueters erreichen.


\chapter{Schatten über Grunewaldt}

\section{Geleitwort}

\begin{flushright}
Claas Völcker, Toronto, den 28.05.2025
\end{flushright}

\section{Tagebücher}

{\itshape
Lieber Waldemar, 

Vergib mir, dass ich hier von der strikten chronologischen Reihenfolge der Ereignisse abgewichen bin. Von allen Ereignissen dieses Winters verlangt wohl vor allem eines ein eigenes Kapitel.

Mit schwerem Herzen, Iliricon
}

\subsection{Die Vorbereitungen in Grunewaldt nach Irian von Rabenmund}

\paragraph{28. Boron 1020BF}
Am 28. Boron treffen wir im verschneiten Burghof von Grunewaldt ein. Ragnos ruft als Baron die Vertreter der Baronie zusammen, um das weitere Vorgehen abzustimmen. Der Verwalter gibt bekannt, man habe die Ernte eingebracht und habe somit genuegend Vorraete fuer den Winter. Desweiteren werden die Rekrutierungsmassnahmen vorgestellt. Hauptmann Eichinger moechte die Buetteln auf 20 Mann aufstocken, um auch waehrend der Belagerung Recht und Ordnung aufrecht erhalten zu koennen. Eine Adeptin soll als Repraesentatin nach Ysilia geschickt werden und es soll dauerhafte Kommunikation zwischen Grunewaldt-Klammsbrueck byw. Grunewaldt-Kornheim ermoeglicht werden. Auch sollen Spaeher den Weg nach Ysilia und nach Praske bewachen. Ragnos fordert die Bevoelkerungslisten der Baronie an und es sollen Fremde o. Verdaechtige gemeldet werden. Fuer eine Belagerung nuetzliche Personen sollen nach Grunewaldt beordert werden. Rezzanjin reist nach Kornheim, um die dortige Bevoelkerung auszubilden.

\paragraph{30. Boron 1020BF}
Ein Inquisitionsrat und sein Gefolge erbitten eine Audienz bei Ragnos. Diese wird im Beisein von Toran und meiner Wenigkeit abgehalten. Inquisitor von Falkenstein war auf dem Weg nach Ysilia, als ihn das schlechte Wetter in Grunewaldt festsetzte. Er erbittet Unterkunft, bis er weiterreisen kann. Ausserdem will er nach unheilygen Treiben in der Baronie suchen und ausbrennen, es wird erlaubt, aber nur in Anwesenheit des Pleger des Landes Toran Ostik.

\paragraph{1. Hesinde 1020BF}
Abreise der Magier

\paragraph{2. Hesinde 1020BF}
Grunewaldt:
Die neuen Buettel werden vorgestellt und magische analysiert. Fuerderhin verhaftet der Inquisitionsrat den Alchimisten wegen Daemonenbuendelei. Toran fuehrt eine zweite Untersuchung der Laboratorien durch, wo er keinen konkreten Beweis findet, waehrenddessen entzuenden sich Experimente, da Alchimist mitten drinnen verhaftet worden ist, und versuchen einen Brand, wobei nur ein paar Ingredienzien und die Laboreinrichtung beschaedygt wurden.

Klammsbrueck:
Temyr, Aria und Firnen treffen am spaeten Abend in Klammsbrueck ein. Iliricon erzaehlt, dass er sich mit Daemonenbeschwoerung (Invocatio Minor), seine Freunde eroeffnen ihm, dass ein Inquisitor zu Besuch kommen wird.

\paragraph{7. HesInde 1020BF}
Der Inquisitionsrat besucht Klammsbrueck.

\paragraph{8. Hesinde 1020BF}
Befragung der Adepten und Novizen.

\paragraph{14. Hesinde 1020BF}
Kornheim:
Am 14. im Monat Hesinde kommen drei Reiter in Kornheim an. Firngrimm von Burmisch, seine Knappin und ein Torwaechter aus Praske. Sie erzaehlen vom Fall der Stadt Praske am 10. Hesinde. 4-6 Banner unter Travian von Rabenmund waren mit Untoten, sowie dem Jagdmeister mit niederhoellischen Woelfen, welche in der Stadt nachts erschienen,aus Viereichen gekommen. Die Tore wurden durch den Verrat von Rondradan's restlichen Getreuen geoeffnet.

Grunewaldt:
Ich suche am spaeten Abend Toran auf. Ich erzaehle ihm meinem Pakt mit Jarlak, von Eichenbruder und den Wolfstraeumen. Er ist sichtlich schockiert und unterzieht mich einer Seelenpruefung.

16. Hesinde 1020BF
Rezzanjin kommt den Boten aus Praske an. Ragnos entsendet aufgrund der Nachrichtenlage Boten nach Klammsbrueck und Ysilia. Alle des Rudels haben an diesem Abend Traeume.

\paragraph{17.Hesinde}
Grunewaldt:
Rezzanjin reist wieder nach Kornheim.

Klammsbrueck:
Temyr ruft eine Krisensitzung ein, um ueber die Evakuierung der Novzien und der gefaehrlichen Gegenstaende zu beratschlagen. Man beschliesst den Kelch und die Buecher per Dschinn nach Festum zu schicken, waehrend die Novzien nach Kleinwardstein reisen.

Die Abschlusspruefung von Waidgard Fredor wird auf den 24. Hesinde in Grunewaldt vorverlegt.

Man bereitet sich darauf vor, die Tobimorabruecke von Klammsbrueck zu zerstoeren, wenn die letzten Fluechtlinge Richtung Stahlheim sie ueberquert haben.

\paragraph{19. Hesinde}
Rezzanjin erreicht wieder Kornheim und befiehlt der Bevoelkerung sich abreisebereit zu machen. Die Spaeher berichten, dass die Strasse nach Sueden menschenleer sei.

\paragraph{20. Hesinde}
200 Alte und Kinder brechen Richtung Stahlheim auf, waehrend 200 Wehrfaehige Richtung Grunewaldt ziehen. Ein Spaeher berichtet blutend, dass ein Heer von Sueden kommt.

Am Abend sieht Rezzanjin Kornheim in Flammen aufgehen und hoerte das siegesgewisse Johlen der feindlichen Soldaten, als er mit dem letzten Hundert nach Grunewaldt entschwindet.

Der Sueden der Baronie ist mit rund 700 Menschen, welche weit verstreut auf Bauernhoefen leben, an den Feind gefallen.

\subsection{Die Ereignisse vor der Belagerung auf Klammsbrück nach Firnen Wulfgrimm}

\paragraph{17. Hesinde 1020 Bosparans Fall}
Gestern Abend ist die schreckliche Kunde aus Grunewald übermittelt worden: Braske ist gefallen durch Verrat und Hinterlist und der Feind marschiere nun gen unserer kleinen Baronie. Geschlafen habe ich so schlecht wie seit der Heerschau nicht mehr. Alpträume plagten meine Ruh. An einen erinnere ich mich noch genau und kaum vermag ich mit Gewissheit zu sagen, dass es wirklich Traum und keinen Realität war. Ich saß noch bis spät im Turmzimmer. Die Burg schlief bereits. Plötzlich kam eine Eule durch das Fenster geflogen. Sie verwandelte sich in eine jung anmutende Frau, die mich bat mit ihr zu kommen, wie in früheren Zeiten. Sie schlug mir vor zu ihrem Meister zu gehen, der mich wohl retten können von meinem Schicksaal. Ich wachte schweißgebadet auf und nutzte die frühen Morgenstunden um übermeinen Traum nach zu denken. Wie ich so das Klammstal durchstreifte erreichte ich mit den ersten Strahlen der Prajosscheibe die Lichtung von der damals Erwen verschwand. Und mich beschlich der Verdacht, das die Frau in meinem Traume nur Azariel Scharlachkraut, die elende Verräterin, gewesen sein kann.

Heute Mittag haben wir zusammen mit den Lehrkörpern der Akademie Ratschluss gehalten.

Wir werden die Prüfung der Schülerin Weidgard Fredor nicht wie üblich am 30. Hesinde, sondern vorgezogen bereits in einer Woche abhalten. Also am 24. Hesinde. Sobald die Prüfung abgeschlossen ist werden wir die Schülerschaft unter Aufsicht 2 Bauerskinder und der frischen Adeptin nach Kleinwardstein schicken und ausser Gefahr bringen.

Die beiden Adepten Arn Wulfgard und Tanja Winterkalt werden nach Stahlheim gehen und dort als Vertreter unserer Magierschaft die Stellung halten, und gegebenen Falls Kontakt mit uns aufnehmen können.

Iliricon,Temyr und ich haben uns im Anschluss noch über die Besonderen Besitztümer der Akademie beratschlagt. Die kostbarsten und seltensten Werke werden wir mit Dschinnes Kraft, nach Festum schicken, zu Rakorium Muntergonis. Den Kelch sowie das Akademiesiegel behalten wir bei uns und bringen es zusammen mit uns in Sicherheit, wenn der Fall aller Fälle eintritt.

Morgen werden wir mit den Schülern und Lehrkörpern nach Grunewald aufbrechen um der Stadt im Kampfe bei zustehen.

\paragraph{20. Hesinde 1020 Bosparans Fall}
Wir haben heute Grunewald erreicht, und sind alle noch mächtig erschöpft von der anstrengenden Reise durch den ersten Schnee. Ragnos bat uns eine Tagesreise gen Stahlheim hinauf zu ziehen um die dortige Brücke über die Tobimora zu sprengen sobald der letzte Flüchtling hinüber ist. Temyr und ich werden heute noch aufbrechen.

Die Prüfung haben wir aufgrund der Umstände um weitere zwei Tage nach hinten verschoben.

Die Baronie, scheint mir, wurde gut vorbereitet auf diesen Krieg. Rezzanjin befindet sich noch oben in Kornheim und koordiniert die Flucht der Bauern. Die Adeptin Thargrim Eichenmoor hatte bereits Nachricht nach Ysilia geschickt.

\paragraph{21. Hesinde 1020 Bosparans Fall}
Heute Morgen ist ein Flüchtlingstross vorbei gekommen mit etwa 200 Alten, Kranken und Kindern, die nach Stahlheim ziehen sollten. Sie wussten nicht wie viele noch folgen würden, aber die meisten anderen würden gen Grunewald gehen.
Spätestens morgen Mittag reißen wir die Brücke ein.

\paragraph{22. Hesinde 1020 Bosparans Fall}
Es sind keine weiteren Flüchtlinge gekommen. Das heißt 300 weniger als angekündigt. Vielleicht haben sie es nicht mehr rechtzeitig geschaft.

Es ist bereits spät am Abend und wir haben vor einer halben Stunde die Baronsstadt erreicht. Der Himmel ist von Süden her zugezogen und der Schnee fällt stetig weiter. Wenn das so weiter geht haben wir hier bald einen Schneesturm, der sich gewaschen hat.

Rezzandjin ist gestern bereits mit allen die noch aus Kornheim fliehen konnten eingetroffen.

Die Zollbrücke über die Tobimora hatte Ragnos ebenfalls einreißen lassen. Kornheim stand nach letzten Berichten gestern bereits in Flammen. Wir habe an der Brücke folglich richtig entschieden.

Ich wollte mich gerade zur Ruhe betten, als ich, wie auch viele andere herbei gerufen wurde. Eine düstere Gestalt, unter einer Kapuze verborgen glitt in den Ratssaal. Mit dröhnender Stimme unterbreitete es uns das Ultimatum des Rabenherrn und des Spiegelherren.

Der Spiegelherr ist ein Titel der dem unheiligen Gegenspieler der Hesinde zugeschrieben wird, und sich wohl auf Borbarad bezieht. Hinter dem Rabenherrr vermutet Irian seinen Vetter Travian von Rabemund, der vielleicht als Heerführer vor Grunewald steht.

Den Boten konnte ich zwar als Ilusion entlarven, doch gar so mächtig gewirkt das ich nicht dagegen zu handeln vermochte.

\paragraph{23. Hesinde 1020 Bosparans Fall}
Heute Morgen haben wir die aktuelle Situation den Bewohnern mitgeteilt. Auch die magische Abriegelung der Stadt wurde verlautbart. Die Bürger werden im Laufe des Tages schlecht ausgerüstet und in Banner eingeteilt. Wir Magier werden uns heute Mittag noch um die Beschwörung dreier Luftdschinne kümmern, die jegliche Überschreitungen der Stadtgrenzen beiderseits zu unterbinden haben.

Heute Abend gab es großen Tumult in der Stadt. Mit Einbruch der Dunkelheit sind Eiswölfe aus dem Schatten gekrochen und begannen durch die Sraßen zu hetzen. Wir erschlugen die Brut samt und sonders. Einer der Lufdschinne fischte eine Magierin aus der Luft und schlug sie in die Flucht. Ich konnte sie deutlich der hexischen Repräsentation zuordnen.

Die Auswirkungen sind erschreckend. 30 bis 40 Bürger sind in Borons Hallen ein gezogen. Etwa 15 sind verletzt und wurden in die Burg geborgen.

\paragraph{24. Hesinde 1020 Bosparans Fall}
Unter Torans Aufsicht wurden die Toten verbrannt. Temyr hat den Lufdschinn befragt und glaubt das die Hexe zuvor Gelegenheit hatte zu zaubern. Ich werde im Laufe des Tages alle antretenden Banner analysieren.

Ich habe nichts beunruhigendes gefunden, doch die Wunden der Verletzten haben Fäulnis angesetzt 3 sind bereits verschieden und einer ist in tollwutartige Raserei verfallen.

Die Götter stehen uns bei gegen diesen Feind.

\paragraph{25. Hesinde 1020 Bosparans Fall}
Es ist heute nicht viel geschehen doch Toran, dem die Zuversicht der Zwöfe normaler weise zu Gesichte steht, lief umher wie sein eigener Schatten. Die Götter scheinen ferner denn je. Selbst im Tempel der Peraine konnte er keinen Kontakt mehr zu seiner Göttin finden.

Die Zeiten schein ärger als erwartet.

\paragraph{26. Hesinde 1020 Bosparans Fall}
Die Zeiten sind mehr als arg. Ich sitze auf den Zinnen des Bergfrieds zu Grunewald und lasse die erschreckenden Bildnisse der Nacht noch einmal revue passieren. Viele von uns, zumindest jene die in den letzten Jahren gemeinsam durch die Gegend reisten, hatten den selben beunruhigenden Traum gehabt, und bei allen war das Wasser im Zimmer gefroren.

Im Traum sah ich eine Lichtung, umrahmt von schwarzen Bäumen die wie Hände des Todes in den Himmel ragten. Aus dem Schatten der Bäume traten sieben schwarz gewandete hervor, die je einen zerlumpten Bauern mit sich führten. Dann traten drei Gestallten in ihre Mitte. Ein Mann der einen Bogen mit sich führte, die Kälte die er zu verströmen schien, griff spürbar über auf mich, eine Frau mit langem schwarzen Haar und ungewöhnlich blasser Haut, und eine Gestalt verborgen hinter einer dunklen Kapuze, der zwei Magierstäbe bei sich führte. Die Drei hoben zu einer Beschwörung an, was sie sagten konnte ich nicht erahnen. im Höhepunkt der Zeremonie schnitten alle sieben Beiwohnenden, den gefangenen Bauern die Kehle durch, wie auf ein geheimes Zeichen hin. Im selben Moment zerbrach das Wesen hinter der Kapuze die beiden Magierstäbe wie Reisig über dem Knie. Ein stechender Schmerz fuhr mir in die Brust und schweißgebadet wachte ich auf.

Kurze Zeit später kam Rezzanjin vorbei. Auch er hatte jenen Traum gehabt und festgestellt das außerhalb unserer Zimmer kein Wasser gefroren war, und schon gar nicht bis zum Grund.

Der Mann mit dem Bogen war ohne Zweifel Jasper von Arsingen (Ragnos Freund), die blasse Frau ist die Hexe, die der Dschinn abfing, die Bschreibung passt genau. Der Unbekannte hinter der Kapuze ist jener Magus der die Illusion zu uns schickte, einer dem Toran bei Eslamsbrück begegnete. Was immer sie beschworen haben war mächtig genug sich die Magie zweier nahezu unzerstörbarer Artefakte zu nutzen zu machen. Mein Stab, und ich vermute Temyr erging es ebenso, wurde zerstört.

Beim Frühstück zeigte es sich, das Rezzanjin mit seiner Vermutung ins Schwarze getroffen hatte. Die Anderen stimmten auch meinen Überlegungen bezüglich der Deutung zu.

Der Schneesturm, der die letzten Tage gewütet hatte, verzog sich heute Mittag vollständig. Über Kornheim sah man noch immer Rauch aufsteigen und an der Forstgrenze auf der anderen Seite der Tobimora hat der Feind bereits sein Lager aufgeschlagen. Über der gezogenen Palisade stehen wehende Banner. Sie zeigen die siebenstrahlige Dämonenkrone und darin einen blutroten Raben.

Frisch verbunden und mit den Kräften am Ende liege ich danieder. Der heutige Abend hat teuren Tribut gefordert. Irian, der seit er sich Toran anvertraute, seine Kammer kaum mehr verlies und jeglichen Kontakt scheute, suchte Temyr und mich auf mit interessanten Fragen über Borbarad, seine Person und sein Ziel, als ich bemerkte wie sich die Stimmung schlagartig zu verschlechtern schien. Meine Analyse des Raumes verriet mir die zuvorrige Anwesenheit eines Dämons des Herrn der ewigen Lüge, der wohl zwietracht zu säen versucht hatte, mit Erfolg. Unfähig gegen den Zauber anzukommen zerfleischten wir uns bald verbal, auch wenn ich zumindest meine Erkenntniss deutlich machen konnte. Als es zu handgreiflichen Auseinandersetzungen kam versteinerte Temyr uns alle samt. Als die Versteinerung brach wirkte der Zauber zwar immernoch, aber die anderen waren bereits anwesend und wir gemeinsamm schaften es Iliricon und ich den dämonischen Zauber zu brechen.

\paragraph{27. Hesinde 1020 Bosparans Fall}
Der Feind hat den Tumult auf der Burg genutzt um einen Teil des Heeres über den Fluss zu bringen. Die Stadt ist nun vollständig umstellt. Alle Wege sind abgeschnitten.

Noch beim Frühstück kam der Verweser der Stadt aufgebracht herein. 2 Kornkammern seien von Fäulnis und Ungeziefer befallen. Jetzt wissen wir was es war, das die Hexe gezaubert. Wahrscheinlich gehen sowohl der Dämon, als auch die Kornfäule auf ihre Rechnung.

Mittels der Variation Welle des Schmerzs, des Kantus Fulminiktus, war es mir möglich das Ungeziefer in den Scheuern zu vernichten. Das Korn zu sortieren wird allerdings noch einige Zeit in Anspruch nehmen.

Magisch ausgezehrt und Körperlich noch stark am Ende bettet Toran mich im Tempel der Perain zu besonderer Ruh. Die Prüfung wird Iliricon für mich übernehmen.

(Ich erfuhr am folgenden Tag: Kaum war die Prüfung beendet, die Adepta hatte mit cum laudum bestanden, da kam der Schrei von den Mauern. Der Feind stellte schweres Geschütz auf. Die Belagerung hatte begonnen)

\subsection{Erste Belagerung und Ausfall der Dämonen nach Toran Ostik}

\paragraph{28. Hesinde}
Der Morgen begann mit einem Katapultangriff an zwei Fronten. Untote wurden aus Richtung des Flusses auf die Stadt geschleudert, während ein Söldnerlager die Burg aus dem Firun mit Steinen beschoss. Rezzanjin konnte sich der Untoten allein entledigen, doch um die Katapulte los zu werden, mussten die Magier zwei Feuer- und zwei Erzdschinne nutzen. Während ich dies schreibe überprüfen die anderen die Schäden. Die Mauer wurde im Praios und im Efferd leicht beschädigt. Ungefähr zwei Dutzend der Bewohner konnten wir nicht mehr retten. Eine der Rüstkammern am Rand der Burghofs ist wohl dauerhaft nicht mehr betretbar.

\paragraph{29. Hesinde}
In den ersten Schimmern der Praiosscheibe weckte uns eine der Torwachen aus unruhigem Schlaf. Aus Richtung Kornheims (zwischen Praios und Efferd) näherte sich uns ein Zug von Flüchtlingen. Um sie nicht ungeschützt dem Feinde auszusetzen, schickten wir einen kleinen Trupp inklusive Rezzanjin, Firnen und meiner Wenigkeit über den Fluss. Trotz starkem Beschuss erlitten wir beim Rückzug keine Verluste. Unter den Neuankömmlingen befanden sich auch zwei selbsternannte ``Händler'' (einer tulamidischer Abstammung und ein Angroschim), die bei näherer Befragung über ihre Tätigkeit eindeutig logen. Da wir im Moment wirklich schlimmeres am Hals haben und Firnen sie als eindeutig nicht-magisch klassifiziert hat habe ich beschlossen der Sache nicht weiter nach zu gehen.

Noch am selben Nachmittag trafen zwei weit schlimmere Zeitgenossen am Stadttor ein. Die beiden stellten sich als geflohener Adepten der Klammsbrücker Akademie, die in Stahlheim postiert gewesen waren vor. Temyr kamen ihre Gesichter zwar unbekannt vor, doch als bei einer schnellen Überprüfung nichts weiter verdächtiges festzustellen war, ließen wir sie gewähren. Erst nachdem sie sich schon einige Zeit übers Gelände bewegt hatten kam Firnen die Idee sie genauer zu analysieren. Die Adepten waren in Wirklichkeit sogenannte Quezlinger, gehörnte Dämonen des Herren des verbotenen Wissens. Leider konnten sie entkommen nachdem sie sich in winzige Insekten verwandelten. Der Abend neigt sich dem Ende zu und wir können von den Mauern aus beobachten wie im Feindeslager ein neues Katapult gebaut wird. Unsere Dschinne gehen bald zur Neige.

\paragraph{30. Hesinde}
Nach einer viel zu kurzen Nacht schickten wir einen Wasserdschinn los um das inzwischen fertiggestellte Katapult zu zerstören. Schnell jedoch mussten wir uns Anderem zuwenden, als einer der zur Wache aufgestellten Luft-Dschinne uns emotionslos darauf hin wies, dass Dämonen es in die Stadt geschafft hätten. Der Kampf war kurz aber brutal. Die Dämonen ergriffen Besitz von vielen der Wachen und sogar von Irian und zwangen sie sich gegen uns zu wenden. Die Verwundeten beschäftigten uns noch bis zum nächsten Morgen\dots

\subsection{Der Ansturm der Untoten nach Rezzanjin Al'Ahjan}

\paragraph{1. Firun}
Wir brauchten bis spät in die Nacht um die Unwesen zu entschwören und die Leichen auf die Scheiterhaufen zu stapeln, während Toran die Verwundeten behandelte. Die Nacht war kurz und der nächste Tag begann nicht besser. Ich wachte unter dem sanften Beben der Erde und dem lauten Getöse von aufeinandertreffenden Steinen ausgelöst durch den neuerlichen Katapultbeschuss auf. Verschlafen versuchte ich in aller Eile die Situation zu überblicken. Vier große Katapulte griffen uns von Süden an, vermutlich aufgestellt durch das südliche Lager. Doch auch eines im Osten katapultierte Steine auf die mittlerweile löchrigen Mauern von Grunewald. 

Wie geplant wurde nicht die gesamte Mauer zerstört, sondern nur der Teil im Osten, der schon den meisten Beschuss abbekommen hatte. So würden wir trotz eines Angriffes noch strategisch gute Verteidigungsstellungen haben. Wie Irian berichtete waren schon zwei große Breschen ersichtlich, bei denen die Mauer so gut wie dem Erdboden gleichgemacht wurde.

Doch kaum hatten wir die Schäden begutachtet und angefangen über die weitere Verteidigung nachzudenken, sammelten sich im südlichen Lager feindliche Truppen und machten sich zum Angriff bereit. Gerade so hatten wir Zeit unsere Truppen zu sammeln und an den Breschen aufzustellen, schon sahen wir uns mit der geballten Kraft der aus Dämonen bestehenden Speerspitze des borbaradianischen Angriffs konfrontiert. Doch auch die Dschinne, die zur Überwachung des Luftraumes über Grunewald abgestellt wurden stürzten sich wie wild in den Kampf. Dieser wurde kurz und heftig geführt und die Dämonen, bei denen sich es wohl um Karakilim gehandelt hat, dabei nur um Haaresbreite zurückgeschlagen. 

Doch der richtige Angriff begann erst. Oder auch nicht, da der von den Magiern im Flussbett platzierte Wasserdschinn unter den verzweifelten Schreien der gegnerischen Söldner das Eis brach und so eine Passage vor der der Burg näheren Bresche unmöglich machte und zugleich mehrere borbaradianische Söldner in den Tod riss. Dennoch schafften es einige bruderlose Söldner zur Mauer zu gelangen, doch ich schaffte es alle die durchgekommen waren zu töten. Während ich im gegnerischen Pfeilhagel zur nächsten Bresche eilte sah ich, wie sich einige Söldner auf den Mauern mit unserer Landwehr Gefechte lieferte, die die Landwehr haushoch zu verlieren drohte. Doch als ich gerade, nachdem ich einem weiteren Pfeilhagel ausgewichen war der Landwehr helfen wollte, zogen sich die Söldner offensichtlich unter schweren Verlusten schon wieder zurück. Einzig ihre Toten und Verwundeten, sowie viele Tote und Verwundete auf unserer Seite ließen sie zurück.

Nachdem auch die Bogenschützen der Feinde sich weit genug entfernt hatten, bargen wir die Verwundeten und sammelten die Toten. Toran ging also weiterhin seinem nie endenden Tagewerk, der Versorgung der Verwundeten nach. Die Magier zogen sich währenddessen auf die Burg zurück um mit vereinter Kraft weitere Dschinne zur Verteidigung der Burg zu beschwören. Dass diese bitter nötig waren sollte sich am Abend zeigen. Der Nachmittag hingegen verlief recht ruhig. Ich kümmerte mich um die Neueinordnung der Banner, sowie um die Bergung des Korns aus den von den Katapulten an den Rand des Einsturzes gebrachten Häusern. Gegen Abend hatten wir soweit alles sortiert, doch leider gab es neue schlechte Nachrichten. In der südlichen Stadtmauern, nahe der Burg, war eine weitere große Bresche geschlagen worden. Diese würde ungleich schwerer zu verteidigen werden. 

Mit Einbruch der Dunkelheit schickten die Magier jedoch einen Erzdschinn, um das westliche Lager zu verwüsten, in dem sich die Borbaradianer gerade dazu bereit machten ein weiteres Katapult fertigzustellen. Doch auch das östliche Lager ließen wir nicht verschont. Ein Feuerdschinn wütete hier unter den Katapulten und allen, die sie verteidigten. Doch die Rache kam promt. Etwa eine Stunde später sahen die Wachen auf den Resten der Mauer einen Heerwurm von Süden aus zur Burg schleichen. Doch die Zahlen, die dieser führte waren zu groß, um rein aus Söldnern zu bestehen, die im südlichen Lager verweilten. Sollten sich Söldner im Wald versteckt gehalten haben, um uns zu täuschen?

Es war schlimmer! Als der Wurm näher kam, erkannten wir bleiche Knochen unter den rostigen Kettenhemden: Es waren Untote gefolgt von zwei Bannern menschlicher Kavallerie. Als die Untoten in die Breschen stießen, waren wir vorbereitet. Wir hatten nur die burgnahe Bresche stärker besetzt, in der Hoffnung, dass sie hier mit der größten Kraft zuschlagen würden. Einzig die Praioten und ein Banner Landwehr hielten die östliche Bresche. Zu unserem Schrecken jedoch war die Landwehr nicht auf Untote vorbereitet und so fielen viele beim ersten Kontakt mit den Untoten und der Rest floh. 

Irian, der, von wo auch immer her, etwas Erfahrung mit den Untoten hatte und ich, den bekanntlich nichts aufhalten kann, stellten sich den Untoten. Schnell gewannen wir die Überhand über die kleine Gruppe die wir bekämpften, denn auch wenn Untote von einem Treffer nicht zu Boden gehen, die selbst können nicht annähernd so gut mit dem Schwerte umgehen, wie ein erfahrener und guter Kämpfer. Auch die Landwehr schien diese Tatsache zu beflügeln und so wandten sie sich in der Flucht begriffen wieder ihrem knöchernen Feind zu. Die Reiterbanner schienen nicht in den Kampf eingreifen zu wollen und so schafften wir es die Untoten in ihre Einzelteile zu zerlegen bis zum letzten Skelett aufzureiben, ohne das die Reiterei eine Möglichkeit sah anzugreifen. 

Der Kampf an der anderen Bresche schien ähnlich verlaufen zu sein. Hier waren die Praioten ein Fanal zwölfgeschwisterliche Ordnung und hielten die Landwehr bei der Stange. Doch als wir am späten Abend in der Burg saßen und die Lage überblicken konnten, sahen wir, dass die Verluste hoch waren. Zwar hatten wir die wohl insgesamt vier Banner starken Untoten vernichtet, doch selber hatten wir auch nur noch knapp vier Banner Truppen übrig. Vier von elf. Wenn man es so bedenkt sind viel zu viele Männer und Frauen gestorben, um die Hauptstadt der Baronie zu erhalten. Die Stimmung war schlecht, auch wenn die Borbaradianer mittlerweile ihr westliches Lager komplett aufgeben hatten und uns so der Weg nach Klammsbrück offenstand. Doch am Abend wurde uns klar, dass wir die Kinder und die Alten nicht länger in der Stadt lassen konnten. 

Das westliche Lager war uns hier ein besonderer Dorn im Auge, da wir die Leute unmöglich nach Klammsbrück bringen konnten, allein schon weil die Vorräte für so viele Leute nicht zu transportieren waren. Einzig nach Ysilia konnten wir die übrig gebliebenen kampfunfähigen Leute der Stadt bringen. Und so musste das westliche Lager verschwinden. Ein Frontalangriff war keine Option und so mussten wir uns auf andere Möglichleiten verlegen. Im Klammsbrück so berichteten die Magier, wären noch Dschinne gebunden, die man nutzten könne. Und so begann eine lange Diskussion, in der wir nach und nach alle unehrenhaften Möglichkeiten die Dschinne zu benutzten ausschlossen und uns letztendlich auf einen guten Plan einigten: Wir würden in der Nacht drei Dschinne auf das Lager loslassen, die Chaos stiften, die Zelte und Klamotten der Söldner verbrennen, sowie deren Waffen verbiegen sollten. Dann würden wir sie mit zwei der vier verbleibenden Banner umkreisen, sie zum Überlassen der Rüstungen auffordern und danach alle, die gehen wollten gehen lassen. Ob mit der Schmach der schandvollen Niederlage in ihr Hauptlager zurück oder wohin sie wollten. Allerdings erst nachdem der Flüchtlingstross mit einigen Jägern und der Adepta am Lager vorbei in Richtung Ysilia aufgebrochen waren. Auf Ragnos kam die wichtige Aufgabe zu eventuelle Reiter, die das südliche Heerlager benachrichtigen wollten an ihrer Aufgabe nachhaltig zu hindern. Noch in dieser Nacht wollten wir den Plan in die Tat umsetzten.

\subsection{Der Fall von Grunewaldt nach Rezzanjin Al'Ahjan}

Und so standen wir vor dem Fallgitter des Burgtores und warteten darauf, dass es herausgezogen werden würde, sodass wir ausschwärmen konnten um die uns zahlenmäßig überlegenen Feinde in ein letztes entscheidendes Gefecht zu verwickeln. Stundenlang hatten wir vorher im Kaminzimmer der Burg darüber beraten, diskutiert und gestritten, wie wir uns und die letzten Mannen der Stadt aus den Händen der Borbaradianer entziehen konnten, doch kein noch so elaborierter Plan hatte sich durchsetzen können; zu groß die Möglichkeiten der Borbaradianer mit ihrem gezähmten Eisdrachen und den Magiern, die immer noch viel auszurichten vermochten. 

Und so sahen wir uns gezwungen die letzte aller Möglichkeiten in Betracht zu ziehen: Den Kampf, geführt entgegen aller Vernunft, entgegen einer doppelt so großen Streitmacht, mit besser ausgebildeten Soldaten, gegen mächtige Krieger, Bogenschützten und Magier, die in den Reihen der Gegner nur darauf warteten unser Blut den knöcheltief liegenden Schnee färben zu sehen und nicht zuletzt gegen einen Drachen, geknechtet durch die Rituale der feindlichen Magier.

Doch wir arbeiteten einen Plan aus, riskant und mit großer Gefahr für jeden einzelnen, um den Gegner zu überraschen und an seinen Schwachstellen zu erwischen. Jeder einzelne unserer kleinen Heldengruppe musste sich seinen persönlichen Gegner in den Reihen der Borbaradianer stellen: Temyr der Hexe Lorana, Firnen Azariel Scharlachkraut, Toran bekam es mit dem geheimnisvollen Magier zu tun, Irian mit dem Feldherr von Rabenmund, Ich würde Dorkstein ausschalten müssen. Doch Ragnos hatte die schwerste Aufgabe: Er musste es mit Jasper von Arsingen und dem Drachen gleichzeitig aufnehmen müssen. 

Und so war er schon aus der Burg geschlichen, um sie von Süden durch eines der Löcher in der Mauer kommen zu überraschen, während wir noch darauf warteten, dass sich das Fallgitter in Bewegung setzten würde. Und als das Gitter sich langsam und unregelmäßig zu heben begann kamen sie wieder: Die Gedanken die jeder zu Beginn einer Schlacht hat. Gedanken an Verwandte, Freunde, die Eltern. An jene Leute, die man gerne noch einmal in den Arm nehmen würde, bevor man aus dem Leben aus- und in den Kreislauf von Leben und Tod wieder neu eintritt. Gedanken an die Orte die man besucht hat, an die Leute, die man getroffen hat, an die Abenteuer, die man erlebt hat, an die Freunde, die man über die Jahre gewonnen hat. Gedanken, die einem zeigen, wer einem wichtig ist. Ich realisierte in diesem Moment, dass ich mit niemanden lieber an meiner Seite in die Schlacht rennen wollte, als mit meinen über die Jahre lieb gewonnen Gefährten, mit denen ich meine ganzen Abenteuer durchlebt hatte.

Naja, bis auf Irian vielleicht, der ist und bleibt ein Fremdji.

Und als das Gitter des Burgtores oben angekommen war, waren die Gedanken verschwunden, jetzt zählte nur die Schlacht, das Blut des Gegners im Schnee.

Ich zog mein Schwert richtete es in die Richtung der Gegner und lief auf leisen Sohlen in Richtung der halbzerstörten Stadt davon, den Rest der Burgbesatzung mir dicht auf den Fersen. Wir vermuteten, dass die Gegner sich hauptsächlich im hinteren Teil der Stadt und dem Marktplatz aufhalten würden. Und dorthin gingen wir, wir mussten sie frontal überraschen. Tatsächlich begegneten wir keiner Menschenseele auf dem Weg dorthin, doch mein Instinkt sagte mir, dass wir auf dem Marktplatz fündig werden würden. Und so kam es auch. Kaum hatte ich ein paar Schritte auf dem Marktplatz zurückgelegt, schon schoß ein Armbrustbolzen links neben mir in die Häuserwand und die ersten borbaradianischen Söldner stürmten aus den Häusern. Ich gab den wenigen Landwehreinheiten hinter mir ein Signal, die ersten Häuser zu durchsuchen und stürzte mich dann in den Kampf. 

Die ersten Reihen bestanden aus relativ lausigen Kämpfern, die ich und Irian, der neben mir stand relativ schnell abfertigten. Doch die Landwehr hatte schon so ihre Probleme. Doch die Zeit, die ich brauchte, um meinen zweiten Söldner zu Boden zu schicken hatte Dorkstein anscheinend genutzt, um sich, oder besser gesagt die vier Kämpfer vor sich in Position zu bringen. Mir war fast klar, dass er zu feige war sofort das direkte Duell zu suchen, nachdem er das letzte Mal so schlechte Erfahrungen gemacht hatte. Doch die Kämpfer vor ihm erwiesen sich als echte Herausforderungen, zwei griffen mich an, zwei Irian. Während ich die Schläge noch gut abfangen konnte, ohne selbst entscheidende Treffer zu landen, ging Irian schnell unter. Dem Ansturm der zwei konnte er schwerlich standhalten. Während ich einen der meinigen zu Boden schickte, fiel auch Irian. 

Schnell versuchte ich mich meines Zweiten zu entledigen, während einer von Irians Gegner Anstalten machte, mich anzugreifen und der andere versuchte Toran zu erledigen, der anscheinend, während er Irian helfen wollte in den Kampf hinein geraten war. So schnell wie es ging befreite ich mir von meinen Gegnern, und befreite Toran von seinem lästigen Gegner. Doch in der Zwischenzeit hatte auch Dorkstein sich von seinen Gegnern befreit und so kam es zum Unvermeidlichen. Um uns herum bildete sich ein kleiner Kreis und die Kämpfe um uns herum schienen auf einmal mit weniger Biss und Leidenschaft geführt. Niemand wollte es verpassen, wenn zwei Meister ihres Faches sich in der Kunst des Schwertkampfes messen würden. 

Dorkstein sprach ein paar Worte, die ich aufgrund eines gewaltigen Drachengebrülls nicht so recht mitbekam, doch als er sich verbeugte, tat ich es ihm gleich, verwundert wieso der Borbaradianer mitten in der Schlacht auf einmal auf Duelletikette bestand. Danach hob er sein nachtschwarzes Schwert in meine Richtung und ich sah ihm für einen kurzen Augenblick tief und die Augen, während meine linke Hand über meine rechte Armschiene fuhr. Danach schickte ich mein Schwert in einem Bogen Richtung seines Körpers. Er parierte wie erwartet und schickte einen Schlag zurück, den ich wiederum parierte. Ich versuchte es mit einem fintierten Schlag und hatte Erfolg, er im Gegenzug auch. Hätte ich nicht sofort die heilende Wirkung von Torans Gebeten gespürt, so hätte ich mich versucht ins N`Churr zu begeben. Doch so machte ich weiter, setzte Finte nach Finte, während mein Gegner auch weiterhin Schlag auf Schlag setzte. 

Doch ich traf ihn mehrfach am Kopf, während er nicht traf, zu Beginn, weil ich zu gut war, später dann, weil er durch die größer werdende Wunde an seinem Hals langsamer wurde. Und mit einem letzten brillanten Schlag, dem er nichts mehr entgegenzusetzen hatte sendete ich ihn zu Boden. Sofort trennte ich den waffenführenden Arm von seinem Torso und nahm das Schwert so aus seiner Kontrolle. Ich nahm es an mich, betrachtete es kurz und rekte es mit meiner linken in die Höhe eine Schrei ausstoßend, den meine Landwehrsoldaten ein Dutzend Male erwiderten. Wir hatten die Oberhand, nicht mehr lange und der Kampf würde zu Ende sein.



\chapter{Rohals Versprechen}

\section{Geleitwort}


\begin{flushright}
Claas Völcker, Toronto, den 28.05.2025
\end{flushright}


\section{Die Tagebücher}

\subsection{Der Auftakt des Conventes nach Firnen Wulfgrimm}
\paragraph{01. Ingrimm 1020 Bosparans Fall}
Heute zur Mittagsstunde haben ich und meine Freunde in der yaquiertaler Hof Unterkunft bezogen. Abgesehen, von einem thorwalsschen Magus, der sich mit einem tulamidischem Magierer über den Preis, eines von dessen Seite aus unverkäuflichem nur wenig magischem Amulett stritt, verlief die Reise problem und ereignislos. Als Begleiter seiner Spektabilität iben Sahid haben wir den überzogensten Luxusflügel bekommen. Wir erwarten noch die Ankunft Torans in den nächsten Sunden, bis spätestens morgen, dann sind wir wieder alle vereint.

Es gild diesen Konvent ausreichend zu nutzen. Meine Verteidigung für die Verhandlung meiner Anklage sollte ich ausreichend untermauert haben bis dahin. Und viele Hypothesen habe ich in den letzten 3 Wochen ausgearbeitet, bezüglich unseres Feindes, und die werde ich nun, in Rücksprache mit den Forschergeistern, die die letzten 6 Jahre sicherlich so manches erdacht haben, zu bestätigen oder Verwerfen suchen.

Viel zu Denken gab mir vorallem das Gespräch mit Talesin von Borbra, den ich jüngst besuchte. Talesin war mit einer Expeddition in die Gor aufgebrochen, und war nahe der schwarzen Festung vom untoten Drachen Razazor angegriffen worden.
Er meinte al Borbarad seinen Körper in Besitz genommen hatte habe er ein paar Gedanken des Sphärenschänders gesehen. Borbarad suche demnach nach einem irgendwie geartetem Desiderat, eim Schlüssel, der ihm die Sphären öffne. Und wie sich auch jetzt am Drachen zeigt, hat Borbarad trotz seiner Umtriebe im tobrischen immer noch schtarkes Interesse am alten Zentrum seiner Macht.

Wir unterhielten uns auch über die letzte Schlacht gegen den Bethanier, geführt von Rohal dem Weisen. Wir welzten ein paar alte Schrifften und fanden schließlich eine Abschrift, die den Bannfluch des Los zitierte. Seinerseits gesprochen von Rohal dem Weisen, verschwanden er und der Bethanier darauf hin aus der derischen Sphäre. Rohal riss demnach den Geiste Borbarads mit sich in eine Globule. Zeitlich und räumlich von Dere getrennt sollte niemals ein lebendes Geschöpf, den bösen Geist je wieder beschwören können. Zu blöd das es durch einen Toten geschehen konnte.

Zur borbaradschen Beherrschung bei der Verwendung borbaradianischer Canti ist wohl Cayonon Silberbraue zu befragen. Über die Globulenabschnitte unter Borbra sollte ich mich vielleicht einmal mit Aleya Ambaret unterhalten. Vielleicht schaffe ich es sogar heraus zu finden, wie der Dämon vor Ysilia meinen Schutzzauber umgehen konnte.
Heute Abend ist bereits erstes gemeinsames Bankett in der Akademie. Mal sehen, wen ich dort so antreffe.

Welch ein Disaster. Den Göttern sei Dank dass ich trotzdem jetzt hier sitzen und diese Zeilen schreiben kann.
Meine Reputation ist momentan nicht die Beste vor der weißen Gilde und auch die Graue schaut gespalten auf mich herab. Ich hate gerade erst problemlos alle Sicherheitskontrollen durchlaufen, als ein Rohalswächter auftauchte und behauptete ich sei bereits gekommen. In einm fast einstündigem Verhör unter allerlei magischen Überprüfungen konnt ich schließlich beweisen, das ich der Richtige bin, aber offensichtlich läuft auf dem Konvent ein Doppelgänger umher. Bzw. jetzt da er die Kontrollen passiert hat, ist er in wechselnder oder vielleicht sogar ursprünglicher Gestalt nicht mehr zu finden.

Mit Aleya Ambaret konnte ich bereits srechen und er meinte, dass die derische Sphäre um sich zu schützen. schadhafte Orte einfach abstoße. Als Globule in sich geschlossen liegen sie dort wo sie waren im Limbus verbannt. Wo sich heute das kleine Dorf Borbra befindet, erhob sich in früheren Zeiten, sowohl die Prachtvolle Residenz Assarabads, alsauch die Hallen des Tharsonius von Bethana, der sich Borbarad nennt. Offensichtlich hat Borbarad in früheren Zeiten ebenso an diesem Ort Sphärenrisse entstehen lassen um dämonischen Wessen Einzug zu ermöglichen.

Cayonon Silberbraue war leider nicht Anwesend, aber er hält morgen einen Vortrag, bei dem ich ihn eigendlich erwischen sollte.
Es war ein langer Tag und vor den Anstrengungen von mehreren Tagen Essens, unterbrochen durch mehrstündige Vorträge komplizierter wissenschaftlicher Sprechweise, mindestens in Bosparano gehalten, können wir alle eine gute mütze Schlaf durchaus vertragen.

\subsection{Der erste Vortrag nach Rezzanjin Al'Ahjan}

\paragraph{15.Ingerimm}
Da saßen wir also im großen Puniner Hörsaal als einige von vielen, denn der Hörsaal war bis zum Rand gefüllt und warteten erwartungsfroh auf den ersten Vortrag des Magierkonvents, welches an ebenjenem Tag begonnen hatte.

Doch dieser sollte auf sich warten lassen, da mehrere Weißmagier versuchten den vortragenden, Karjunon Silberbraue, ein anscheinend sehr umstrittener Magier vor allem bei der weißen Gilde, am Betreten des Hörsaals zu hindern. Es ging erst weiter als Firnen Nostrianus Eisenkolber, den Ordensanführer der Pfeile des Lichts mit seiner Anwesenheit ablenkte, sodass dieser seine ganze Wehemenz nicht mehr uneingeschränkt dem Verhindern der Vorlesung schenken konnte. Es ist ganz und gar nicht gut, dass der Kerl, der Firnen offensichtlich nicht leiden kann im Gericht über sein Wohl und Wehe entscheiden wird.

Als der Vortrag dann endlich begann, wurde schnell klar, dass ich unerwünscht bin, da der Vortrag auf Bosparano gehalten wurde. Ich ging also und beschloss den geheimnisvollen Treppenmord mal genauer zu untersuchen. Hierzu musste ich erst einmal herausfinden, ob der ermordete noch im Besitz des Amuletts war, um welches er mit den Thorwaller vor Punin gestritten hatte. Ich fragte also im Borontempel nach dem Verstorbenen Elenviner Magier. Diese hatten die Leiche tatsächlich aufgenommen, die weltlichen Besitztümer jedoch nicht bekommen. Also fragte ich bei Koosmar, dem Organisator des Konvents nach und tatsächlich hatte dieser die Gegenstände aufbewahrt. Da seine Assistentin, mit der ich sprechen konnte, aber nicht herausrücken wollte, ob das Amulett noch in seinem Besitz war, musste ich warten bis sich er selbst der Aufgabe annahm.

Da wir bis zur Mittagsstunden nichts mehr zu tun hatten, beschlossen wir unsere Zeit in der Teestube zu verbringen, wo wir auch Tarlisin von Borbra trafen, welcher gerade einer kleinen Truppe Schaulustiger von seinen Abenteuern berichtete. Anscheinend genervt von diesen übergab er uns die Aufgabe des Erzählers gewandt und machte sich aus dem Staub. Lang und breit erzählte ich daraufhin den versammelten Jungmagiern von den Abenteuern der Gezeichneten. Gegen Mittag wollte ich dann mit den anderen, die beim Vortrag zugehört hatten in der Stadt etwas essen gehen. Doch kaum waren wir auf dem Hof, schon hörten wir einen Schrei aus dem Garten. Schnell eilten wir zum Ort des Geschehens und fanden dort eine panische Elfe, sowie Elcarna von Hohenstein, der über den leblosen Körper eines horasischen Magiers gebeugt war. Neben diesem war ein Weinglas, der Inhalt zum größten Teil auf dem Boden verteilt und der Magier hatte keine sichtbaren Wunden. Man musste also davon ausgehen, dass der Wein vergiftet war. Schon kurze Zeit später kam auch Sirdon Koosmar angerannt und begutachtete den Toten. Ich fragte ihn nach dem Amulett des am Vortag verstorbenen Elenviner Magiers, doch er wusste noch von nichts. Wir überließen den toten Horasier den Magiern zum Untersuchen und gingen essen. Vor der nächsten Vorlesung ging ich nochmal zu Koosmars Assistentin und ließ mir erzählen, dass das Amulett nicht mehr im Besitz des Toten war. Das machte es umso dringlicher den Thorwaller aufzusuchen mit dem er gestritten hatte. Diesen fand ich zugleich in der nächsten Vorlesung und stellte ihn danach zur Rede. Nachdem ich Aleya Ambareth abgewimmelt hatte, der ihn begleitete, stellte ich den Thorwaller zur Rede. Es stellte sich heraus, dass er wohl nichts mit dem Mord zu tun hatte. Darüber hinweg wollte er auch nicht herausrücken, wieso er die Amulette brauchte, da er sie für seine Spektabilität Cellyana von Kunchom sammelte. Darüber hinaus gab er mir zu verstehen, dass diese einen Vortrag über ebenjenes Thema halten würde. Wieder waren wir nicht weiter gekommen. Aber es blieb uns wenig Zeit darüber zu grübeln, da wir uns für den am Abend stattfindenden Ball fertig machen mussten. Er war ein lustiger Abend, wir trafen viele bekannte Leute und feuerten ausgelassen. Doch gegen Mitternacht verschwand Temyr auf einmal und ward bis zum nächsten Morgen nicht mehr gesehen\dots

\subsection{Firnens Urteil nach Irian von Rabenmund}
\paragraph{16. INGERIMM im Jahr 5 nach Borbarads Erscheinen}
An diesem Nachmittag erfolgte die Analyse des Zweiten Zeichens, von der ich nur aus zweiter Hand berichten kann, da ich auf Grund meiner nicht zugehoerig zur magischen Zunft noch den gezeichneten ausgeschlossen war. So wurde anscheinend die Geschichte, wie Toran zu dem Zeichen kam und dessen Faehigkeiten, sowie die passenden Prophezeiungen besprochen.

\paragraph{17. INGERIMM}
Zur Morgenstund hoere ich mir den Vortrag zur Blutigen See an, waehrend die anderen im Vortrag zu den Untersuchungen ueber die Zerstoerung Altaias sitzen.

Am Vormittag finden die Analyse des Dritten Zeichens statt, an der aus oben genannten Gruenden auch nicht teilnehme.

Die wichtigsten Punkte waren, so wurde mir zumindest berichtet, Frage ueber einen Asfaloth-Pakt (Goetter und Daemonen, sind auch nur zwei Seiten derselben Medaille) durch Carolan Schlangenstab, ob es Einblicke in die echsische Kulur gaebe durch den Spinner Muntagonus, sowie Theamtisierung von Rezzanjin's Empathielosigkeit und der Verfolgung der Inkarnations-Theorie.

Gegen Mittag erhalten wir die Einladung zum Zwölfgötterjosten im Rondra in Perricum.

Zum Mittagessen werden wir von Rohezal vom Amboss in einer abgelegen Taverne eingeladen, der Firnen immer wieder besorgten Blicke zu wirft.

Als wir gerade wieder zur Akademie kommen, taucht plötzlich ein Karakilim über Punin auf und laesst den abgeschlagenen mit einer B-Zhayad-Rune gebranntmarkten Kopf von Atavar Friedenslicht fallen, aus dessen toten Mund die bedeutenden Worte kommen:

``Euer Zeitalter ist zu Ende, Kehrt zurueck ins Licht, Fabelwesen!''
Am Abend treffen wir auf Bitten Eternenwachts mit Rumina von Vinsalt und diskutieren mit ihr ueber die Moeglichkeit Borbarad als Sohn Nandus zu verdammen, da diese wehemmt dagegen ist, ein goettliches Wesen seine eigene Goettlichkeit abzuerkennen. Sie leidet gerade als Nandusgeweihte unter den taten den Nandussohns und hat eine Sinneskrise.
Bei unserer Rueckkehr wird Firnen beschuldigt den Magier Veranesco ermordet zu haben.

\paragraph{18. INGERIMM}
Am Morgen des 18. Ingerimms erleben wir, wie der Großmeister der Grauen Staebe, Tarlisin von Borbra, vom Hochmeister der Rohalswaechter Nostrianus Eisenkolber wegen Daemonenbuendelei angeklagt wird. Ein wahrhaft schwarzer Tag für die Einigkeit der Magier!

Wir wohnen Eternenwacht's Vortrag ueber die wachsende Borbarad und Daemonenverehrung in den gefallenen Landen bei, die sorgniserregend beobachtet wird, wozu aber noch keine Problembekaempfung gefunden wurde.
Am Ende des Vortrags erheben die Ucuriaten das Wort und verkuenden für die Welt, dass die Einigkeit der Kirchen für die Kampf wider dem gefallenen Alveraniar oberste Prioritaet hat, so nehmen die Ucuriaten nun Mitglieder aller Kirchen auf und wollen diesen als sacrosankte Botschafter dienen.

Am Nachmittag versammelt sich das Konvent, um über die Zusammensetzung von Rohals's Stein des Weisen zu entscheiden. Zur Dikussion steht, ob man die Onyx-Splitter wieder vereinen soll oder ob man auf Niobara's Warnung hoeren soll. Nach langen Ringen, gerade unter den Anhaengern von Rohals's Lehren, entschließt man sich den Stein des Weisen neu entstehen zu lassen.

Am diesen Abend finden wir uns ein,um dem Urteil ueber die Verfehlungen des Firnen Wulfgrimms zu lauschen.
Das Gericht setzt wie folgt zusammen:

Kläger: Spectabilitus Saldor Foslarin, vertreten durch Hauptfrau Lanzelind Heilenhorst, Pfeile des Lichts

Collegium Iustitium

Hohe Richterin: Spectabilita Racalla von Horsen-Rabemund, Convocata Secunda\\
Erster Gerichtsdiener: Spectabilitus Olorand von Gareth-Rothenfels zu Perricum\\
Zweiter Gerichtsdiener: Archomagus Rohezal vom Amboss\\
Erster Beisitzender: Spectabilitus Nostrianus Eisenkolber, Ordo Defensores Lecturia\\
Zweite Beisitzende: Spectabilita Prishya von Garlischgrötz, Convocata Prima der Grauen\\
Dritter Beisitzender: Archomagus Elcarna Erillion von Hohenstein zu Lowangen\\
Protokollant: Archomagus Carolan Schlangenstab zu Kuslik

Wächter des Argelionsrechts: Inquisitor Parinor von Oppstein\\
Weiser des Argelionsrechts: Magister Erechton, Sacer Ordo Draconis

Die Anklagepunkte lauten:

\begin{enumerate}
\item Besitz und Gebrauch verbotenen Schrifttums nach Codex Band V § 16.1
\item Täuschung der Obrigkeit durch Anwendung der Magie nach Codex Band V § 3.2
\item Kenntnis und Anwendung von untersagten Formeln und Canti nach Codex Band III § 14
\item Falschaussage vor einem Gildentribunal nach Codex Band V § 24.3
\item Flucht vor einem Gildentribunal nach Codex Band VI § 24.2
\item Heimtückischer Mord in zweifacher Ausführung durch Anwendung von Magie nach Codex Band V § 2.1
\item \begin{enumerate}
\item Der Mord an Wachmann Ernst Travian Rondratreu, Stadtwache zu Ysilia
\item Der Mord an Magister Minorum Dexter Hufstädter, Agent der Pfeile des Lichtes
\end{enumerate}
\item Paktierei mit dem Allerunheiligsten Herren der Rache in willentlicher Verleumdung der Zwölfgötter und aus niedersten Beweggründen Codex Band I § 14.1
\item Entzug als angeklagter Seelenpaktierer aus der Gerichtsbarkeit der Gilden durch den unter 6.1 genannten Mord nach Codex Band V § 2.3
\end{enumerate}


Im ersten Punkt kommt es zu einem Freispruch aufgrund eines Formfehlers der Anklage, er muss aber ein Bußgeld von 10 Dukaten bezahlen.

Unter Punkt 2 wird Firnen im Punkto: Anmaßung eines anderes Standes, als nicht schuldig angesehen, aber schuldig gesprochen für die Aneignung von mittelreichischem Staatseigentum, was an ein weltliches Gericht verwiesen wird.

Firnen gibt die Kenntnis von Tempus Stasis und Verbotenen Pforten zu und muss deshalb 10 Dukaten Bußgeld zahlen. Die Anklage zur Beherrschung der Blutmagie und der heptasphaerischen Invocation wird allerdings zurück gezogen.

Punkt 4 wird von der Anklage zurückgezogen.

Bei Punkt 5 müssen die besondern Umstaende beachtet werden unter denen sich Firnen für schuldig erklaert.

Bei beiden Morden wird Firnen wegen Mangel an Beweisen fuer unschuldig gehalten. Ihm wird aber vertretbarer Totschlag vorgehalten und er hat schon wieder ein Bußgeld von 10 Dukaten zu zahlen.

Firnen leugnet den Daemonenpakt nicht, haelt aber seinen Kampf gegen Borbarad und Seine Schergen dagegen.

Schlußendlich wird der Magier Firnen Wulfgrimm als schuldig in den Hauptanklagepunkt gesehen. Er wird aber aufgrund seiner Verdienste wider dem Nanduszwilling nicht der Purgation unterzogen, sonder aus der Großen Grauen Gilde des Geistes verbannt.

\subsection{Das Auffinden des Mörders nach Temyr ibn Sahid}

\paragraph{19. Ingerimm im Jahre 1020 nach dem Falle Bosparans}
Am nächsten Morgen begaben wir uns direkt nach dem Frühstück in die Hallen der Akademie zurück, um dem traditionellen Götterdienst oder der Konkurrenzveranstaltung des Tomeg Aterion -- ``Warum wir nicht im Götterdienst sind'' -- beizuwohnen. Die Inhalte der letzteren lassen sich freilich in einem äußerst populären tulamidischen Sprichwort zusammenfassen, dessen Erklärung hier aber zu weit gehen und die leutseligen Gemüter meiner Gefährten unnötig belasten würde\dots Kaum hatten wir uns jedoch hernach wieder zusammengefunden, als Sidor Kosmaar mit ahnungsvollen Schritten herangeeilt kam und unsere Sorgenliste um einen weiteren Punkt ergänzte. Der Mörder, dessen verachtenswertes Geschick schon die letzten Tage den Konvent in Atem gehalten, hatte ein weiteres Opfer gefordert: eine junge Magistra namens Yaneska Yanolof, die auf dem Weg zu ihrer Herberge erdrosselt aufgefunden worden war.

Was war das Ziel des Attentäters? Wie stand er in Verbindung mit den Plänen des Thorwaler Magus und seiner Vorgesetzten, Seliana von Khunchom, in welche Rezzanjin Einblick zu gewinnen versucht hatte? Und war es eben dieser, der bereits am ersten Tage in Gestalt Firnens den Konvent aufgesucht hatte? Über all diese Fragen diskutierten wir reiflich und hitzig, und griffen dabei einen Gedanken auf, den Firnen als erster gefasst hatte, nämlich die Identität des Assassinen betreffend. Handelte es sich bei dem Schurken, der ja in der Kunst der Magie durchaus bewandert schien, um einen der abtrünnigen Tuzaker Magier, die so früh schon unsere Pläne gestört hatten? Wenn ja, so würde er mit Sicherheit auch hinter uns selbst her sein, und seiner Chance unter allen Umständen harren. Die Überlegung schien naheliegend, und so fassten wir den Plan, dem Täter eine falsche Fährte auszulegen und seiner in persona habhaft zu werden. Als Ausleger der Falle wählten wir meine feierliche Ernennung zum Erzmagus, der sich nach guter tulamidischer Tradition eine weniger offizielle, aber dafür weitaus ausschweifendere Festivität anschließen musste. Mit meinem Kopf in Reichweite, so folgerten wir, würde es dem Mörder eine unwiderstehliche Verlockung präsentieren, erneut zuzuschlagen. Nur dass in diesem Augenblick, selbstverständlich, wir selbst zuschlagen würden\dots

Wir besprachen den Plan in aller Eile, aber gebotener Vorsicht mit der Magistra Prima von Garlischgrötz, die unserem Vorhaben zustimmte und erste Vorbereitungen in die Wege leitete. Als wir uns zurück in das Gedränge des Konvents begaben erreichte uns eine Depesche des Hesindetempels, welche zur Audienz bei der ehrenwerten Haldana von Ilmenstein lud. Nach dem letzten Vortrag des Tages, in welchem Dscheleff in einer flammenden Rede den Zusammenhalt der Magierschaft beschwor und zur Unterstützung der Gezeichneten aufrief, verließen wir die Hallen der Akademie am Nachmittag und wandten unsere Schritte zum Haus der weisen Göttin. Die Magistra der Magister empfing uns freundlich und dankte für unsere Bemühungen im Kampf gegen Borbarad. Ebenso lag es ihr jedoch daran, Erkundigungen einzuziehen, worin wir ihr mittels eines ausführlichen Berichts aushelfen konnten. Über unser Auskünfte war es schon Abend geworden, und die Geweihte hieß uns zu bleiben für ein Treffen von Hesinde- und Magiergeweihten. Uns so trafen dann im Herzen der Nacht, zur Hesindestund, die Geweihten der Göttin ein, um die allgemeine Verdammung des Borbarad vor den Augen von Göttern und Menschen zu beschließen. Die höchsten Geweihten des Phex, des Nandus' und der Hesinde traten um den Altar, und mit donnernden Stimmen, die nicht Menschen, sondern Göttern gehörig waren, verkündeten sie das Wort vom einigen Ratschluss. Ein würdiges und gerechtes Zeichen im Angesicht eines Götter und Menschen verachtenden Gegners!

\paragraph{20. Ingerimm}
Der folgende Tag begann mit einer gewaltigen Überraschung, der sich noch weitere anschließen sollten -- gute wie schlechte. Als wir nämlich wie üblich unsere Schritte zum Pentagrammaton lenkten, versperrten uns dort die Rohalswächter unter Nostriamus Eisenkolber den Weg. Selbiger trat mit gequälter Miene vor und übergab Toran mit zitternden Händen seinen Onyxstab, den er zuvor unter keinen Umständen herzugeben bereit war. So erstaunt waren wir über dieses kleine Wunder, dass wir kein Wort hervor brachten -- im Gegensatz zu Eisenkolber, der uns mürrisch beschimpfte und auch sonst wieder ganz der Alte zu sein schien. Mit diesem Schatz in den Händen begaben wir uns ohne Umschweife zu Seliana von Khunchom, die über alle Maße erfreut über einen Stein dieser Größe war. Danach widmeten wir uns weiterhin der Vorbereitung unserer Falle, bis die ersten Vorträge des Abends unsere Aufmerksamkeit verlangten. Abermals teilten wir uns unserer Interessen gemäß auf und besetzten verschiedene Hörsäle, um das Angebot des Konventsplans auszuschöpfen. Der Vortrag des Khadil Okharim beleuchtete die Fortschritte bei der Erstellung des bastrabunschen Bannschildes, an welchem wir selbst tatkräftig mitgewirkt hatten. Die Rekonstruktion der Bannformeln schien gut voranzugehen, doch hatten sich tiefere Probleme bei der Entschlüsselung der materiellen Komponenten der Spruchform aufgetan. Magister Okharim war gerade dabei, die feineren Züge seiner Analysen zu erläutern, als auf dem Flure ein unglaublicher Tumult ausbrach und die versammelte Magierschaft in heilloses Durcheinander stürzte: Während des parallelen Vortrages der Magistra von Garlischgrötz hatte sich, wie Firnen später atemlos berichtete, ein Riss im Sphärengefüge aufgetan, aus welchem sich unaufhörlich Zantim ergossen und die Zuhörer angegriffen hätten. Irgendwie gelang es zwar, den Dämonen Herr zu werden, aber unter welchen Verlusten! Viele hatte die Dämonenbrut hingemetztelt oder schwer verletzt; das eigentliche Unglück wurde aber erst später offenbar. Im Schutze der Verwirrung und Panik waren nämlich die gesammelten Onyxsplitter aus den Räumen der Seliana von Khunchom entwendet worden, nicht ohne ihrem treuen Untergebenen Ilachim die Kehle durchzuschneiden.

Nun hieß es, auf das Gelingen unserer Falle zu bauen und den Mörder, wenigstens jedoch die Steine, zurückzubringen. Nachdem der feucht-fröhliche Abend im besten Gasthaus Punins ausgeklungen war, begab ich mich allein durch die nächtliche Gassen der Stadt zurück zu unserer Unterkunft -- stets gedeckt von den unsichtbaren Schatten meiner Freunde, die mir unauffällig folgten. Und tatsächlich dauerte es gar nicht lange, bis sich der Unhold aus seinem Versteck begab. ``Paralysis!'' hörte ich hinter mir eine Stimme schreien und eine Welle magischer Energie brandete über mich hinweg. Da ich aber mit einem solchen Zug gerechnet hatte, gelang es mir den Zauber abzuschütteln und dem Magus -- es handelte sich tatsächlich um einen Maraskaner -- einen saftigen Ignifaxius entgegen zu schleudern. Sogleich sprangen meine Gefährten hervor und warfen sich in das Kampfgetümmel. Der Magier hatte mit so viel Gegenwehr freilich nicht gerechnet und versuchte sofort, dem Kampf zu entgehen: Sein ganzer Körper schien in sich zusammenzufallen, zu schrumpfen, da wuchsen schwarz-ledrige Schwingen daraus und mit raschem Flügelschlag verschwand der Blutsauger in der Nacht. Rezzanjin jedoch rannte, von der Kraft eines Zaubers erfüllt, hintendrein, und konnte den Maraskaner bis zu einem verfallenen Turm verfolgen, in dessen labyrinthischen Eingeweiden er den Kerl aber verlor. Als der Rest von uns endlich mit menschlicher Geschwindigkeit dort eintraf, hatte Rezzanjin bereits das ganze Versteck auf den Kopf gestellt und neben den Steinen auch einen Brief entdeckt, der an einen gewissen Torben Dergeler adressiert und von unserer alten ``Freundin'' Azariel Scharlachkraut unterschrieben war. Offfenbar hatte sich die Hexe noch bis gestern selbst in Punin aufgehalten und hatte das Wirken des Tuzakers überwacht. Weiterhin enthielt das Schreiben Anweisung, eine gewisse Hexe aus Maraskan namens Lavinia aufzusuchen, sobald der Auftrag erledigt wäre. Wenigstens dem hatten wir einen Stich versetzen können. Mit den Steinen im Gepäck begaben wir uns zur Akademie.

\paragraph{21. Ingerimm}
Das Arbeitskabinett der Convocata Prima war schon bis auf den letzten Platz ausgefüllt, als wir tags darauf auf dem Konvent eintrafen. Seliana von Khunchom hatte bereits die Fakten rund um die Onyxsteine und ihre Arbeit resümiert, und erging sich dann in weiteren Ausführungen über die nun zu beschließende Vorgehensweise. Zunächst einmal galt es, für den Stein der Weisen die fähigsten Artefaktmagier, Analysten und Handwerker auszuwählen, die diesem delikaten Gegenstand Form verleihen sollten. Die anwesenden Magier diskutierten wohl manche lange Stunde, bis endlich die fünf Betreffenden zusammengestellt waren. Der Ratschluss lautete auf: Salpicon Savertin, Firnen Wulffgrimm, Salandrion Farnion Finkenfarn, Ragnos vom Svelltal und mich, Temyr ibn Sahid. Das Wunder konnte vollzogen werden.

\subsection{Der Stein der Weisen nach Firnen Wulfgrimm}

\paragraph{21 Ingrimm 1020 Bosparans Fall}
Der Stein des Weisen soll nun beschlossener Maßen wieder zusammen gesetzt werden. Sowohl Temyr als fähiger Artefaktmagierer, als auch ich, mit meinen überdurchschnittlichen Analysefähigkeiten, werden dabei mithelfen dürfen. Da Temyr sich bereits ein paar Gedanken machen wollte, über eine mögliche Artefakt Thesis, zur Reaktivierung der vorhandenen Strukturen, analysierten wir noch einmal einen Großteil, der Splitter. Dabei erregte Zulhamid in mir einen schrecklichen Verdacht. Ich wollte soeben einen Analys zaubern, da offenbarte mir der Magiermogul bereits den gewünschten Anblick. Er war völlig aus dem Häuschen und wollte keine Sekunde länger mit dem Zusammensetzen warten. Die magischen Strukturen trügen SEINE Signatur, die Art und Weise, des Assarabat, beziehungsweise des Bethaniers, und das Artefakt brächte uns einen Weg IHN zu finden. Was ist nun, wenn wir weder Kontakt zu Rohal erreichen, noch ihn beschwören können, sondern den Bethanier rufen? Ich vermag mir dieses Schreckensszenario kaum auszumalen. Das Artefakt ist unsere einzige Handlungsmöglichkeit und ich bete zu den Göttern, dass sie uns etwas Gutes bringt.

Toran kam kaum später herein. Er hatte im Tempel meditiert und eine beunruhigende Ahnung hatte ihn erreicht. In Gedanken an die Göttin, sah er einen Turm wie aus Glas und goldene Schuppen, und ein ungutes Gefühl beschlich ihn, dass ihn jetzt noch frösteln lässt, wenn er daran zurückdenkt. Der einzige unsichtbare Turm, er wird eigentlich nur durchscheinend, den ich kenne, ist der Hofmagierturm im Ambossgebirge. Derzeitig verwahrt und bewohnt von Rohezaal vom Amboss.

Wir wollten Rohezaal von unseren Theorien und Befürchtungen in Kenntnis setzen, doch er bat uns ihn erst am frühen Abend in der kleinen Bibliothek zu treffen. Torans Vision könnte eine seiner Theorien bestätigen.

Rezzanjinn beschloss uns zur Unterredung zu begleiten, Irian hatte schon am Mittag die Stadt verlassen, warum wusste keiner. Tatsächlich erwartete uns Rohezaal bereits in der kleine Bibliothek. Er hob gerade zu sprechen an, uns eine Mitteilung immenser Bedeutung machen zu müssen, als er nach vorne sackte. Ein kleiner Borndorn ragte aus seinem Nacken. Während Toran, Temyr und ich uns erfolgreich bemühten den Erzmagier im Leben zu halten, bemühte sich Rezzanjinn den bereits bekannten Maraskaner, der hinter einem Vorhang verborgen gestanden hatte, dingfest zu machen. Trotz eines abgeschlagenen Armes, entkam der Asfalothpaktierer als Vogel durch das Fenster. der Arm wuchs nach. Rohezaal erklärte uns von einem Hexagramm aus Orten, die mit Rohals Wirken zu tun gehabt haben. In der Mitte des Hexagramms der Ort Wagenhalt, in dem Rohezaal der Legende nach erschien. Rohezaal glaubte, das der weise Magier im Falle einer Beschwörung, über diese Orte seinen Weg gehen würde von einer geistigen Gegenwart, bis zur Verkörperung. Sobald sich die Theorie bestätigen würde, wolle er in den Amboss reisen, und versuchen, eine Verkörperung zu verhindern, den usn allen klingt noch immer die Warnung Nihobaras von Anchopal in den Ohren: "Niemals vollendet sei der Ruf nach den Beiden, um göttliche Allmacht auf Deren zu meiden." Der Plan Rohezaals klingt verrückt, aber er könnte klappen.

Eine geraume Zeit schon von unseren Stäben, durch ihre Zerstörung entbunden, beschlossen Temyr und ich ihre Spektabilität zu Punin zu bitten, uns Material aus den Akademiegewölben nehmen zu dürfen, um uns neue Stäbe zu erschaffen. Die Bitte schlug sie uns nicht ab, und nun wird sich zeigen, ob wir in einer Nacht zu vollbringen vermögen, zu dem der Eleve knapp eine Woche benötigt.

Ich fand einen knorrigen Stab von etwa eineinhalb Schritt Länge, aus dem Holz der Steineiche, das ich nicht im Besitz einer Punier Akademie vermutet hätte. Vielleicht war es ein Experiment gewesen, das die Akademie einst beschlagnahmt hatte, denn es befand sich in die Verästelung am oberen Ende des Stabes geschnitzt, eine ebenso ungewöhnliche Walrune, wie bei den Thorwalern üblich. Die Verästelung beinhaltete bereits eine Fassung, nur der Stein schien abhanden gekommen zu sein. Ich fand passend einen bordeaux farbenen grobbehauenen Topas, den eine Silberne Mondsichel zierte.

Das Holz scheint eine ungewöhnlich große Affinität zu Blitzen und Feuer zu haben. Der Stein hat eine eigene antimagische Kraft. Die Walrune hat sich magisch mit der silbernen Mondsichel verbunden. Es muss eine Art thorwalsches Zauberzeichen sein, es wirkt der Verständigung zugehörig. Kaum hatte ich den Stein in den Stab eingesetzt, erschienen magische Zeichen, des Feuers und der Antimagie in silbrigem Arkanum, und ebenso Flammen die, sich um den Stab rankend, diesen schmücken in unvergleichlicher Weise. Ich denke der Stab ist passend wie kein zweiter. Und ich danke den Götter für die merkwürdige Vielfalt der Sammlung einer so alten Akademie wie der in Punin.

Etwa zwei Stunden vorm Weckruf der Kirchenglocken traf ich Temyr in der großen Halle der Akademie. Er war ebenso fertig geworden. Es war ihm gelungen einen etwa zwei Schritt langen Stab aus Rashtulszeder zu bergen. Nun schmückte ihn zu dem ein Amethyst in Form einer stilisierten Rose.

Müde und zufrieden kehrten wir in unser Gasthaus zurück. Wenigstens die letzte Stunde werde ich noch zu schlafen versuchen, bevor vielleicht der Größte Tag seit langer Zeit anbricht.

\paragraph{22. Ingrimm 1020 Bosparans Fall}
Es bereits Abend ich bin gleichermaßen erschöpft und verstört. Die Ereignisse des Tages übertreffen fast alles bisher erlebt an Unvorstellbarkeit. Doch bei allem Schrecken bleibt ein Schimmer der Hoffnung.

Also von Vorn: Wir begannen bereits zu früher Stunde mit der Rekonstruktion des Steins des Weisen. Die bereits angefertigten detaillierten Analysen erleichterten das Vorankommen erheblich. Stück für Stück setzte sich das Artefakt zusammen, und obgleich es ungemein Kräfte zehrend war, dauerte es nur wenig Stunden. Das ungeheuerlichste Artefakt, das ich je gesehen hatte war zusammen gesetzt worden. Ein Schwarzes Auge, in Form einer etwa Kopfgroßen Kugel, das nicht die Zeit, sondern die Spähren zu durchblicken scheint, und eine Invokation herbeiführen konnte die nicht der siebten Sphäre zugeordnet werden kann.

Die drei Konvokati Primae der Gilden legten ihre Hände an das Artefakt und schienen in Trance zu versinken. Bilder zogen durch den Raum, Bilder jener Orte, die Rohezaal dem Hexagramm zugeordnet hatte. Er schien also in seinen ersten Punkten Recht zu behalten. Er bat uns ihn zu seinem Turm zu begleiten, und mit ihm flogen wir auf seinem Freund, Faldegorn, einem Kaiserdrachen, ins Ambossgebirge.

Tatsächlich offenbarte sich uns dort das gelegentliche aber seltene Spektakel, den unsichtbaren Turm zu sehen, in der durchscheinenden spiegelnden Art, die ihm im Volksmund diesen Namen einbrachte. Rohezaals Tochter eilte uns entgegen. Sie sagte alle Tiere habe die Gegend verlassen, beziehungsweise, haben sich auf einer Bergwiese, ein Stück unterhalb des Turmes gelegen, versammelt. Auch wir eilten schnellen Schrittes zu dieser Wiese und tatsächlich: Als wir dort ankamen bot sich uns ein Anblick, den ich auf immer in meinem Herzen tragen werde. Aus den leichten Schwaden eines Bergnebelfeldes heraus kam ein Mann geschritten. Seine Haltung edel, sein Gang bewusst und rechtschaffend, sein Blick voller Güte und seine Worte voll Weisheit. Sein Wesen voller Freundlichkeit. Rohal war zurückgekehrt. Als wir mit seiner Hilfe unsere Sprache wieder fanden, bedrängten wir ihn schließlich doch mit den dringlichsten Fragen, die uns seit geraumer Zeit nun schon, so arg beschäftigten. Was sollten wir unternehmen um Borbarad aufzuhalten? Rohal sah uns unsere Zielstrebigkeit nach, auch wenn er uns ermahnte stets auch die Welt im Auge zu behalten, wenn wir unseren Weg zu sehen versuchen. Er verriet uns die Schwachstelle Borbarads: Der Bethanier hat Satinav um seine Zeit gebracht, als er sich zurück holen ließ, und mehr denn je hat er seine Zeit verkauft. Wenn wir den Bethanier nun zu hindern gedächten, so sollten wir seine Zeit finden und hüten. Diese Aufgabe übertrug er uns und im besonderen Temyr, dem er zum Zeichen seines Auftrages, die Kappe des Weisen auf sein Haupt setzte. Der fünfte Gezeichnete ist erschienen. Er trägt das Zeichen der stählernen Stirn und das Wissen um seinen Frevel.

So erleichtert über die Einsicht in die Tiefen der Welt und die begründete Hoffnung auf ein Morgen, standen wir ahnungslos auf der Bergwiese und redeten noch von Angesicht zu Angesicht mit dem Weisen persönlich, als das Unheil heraufzog. Der Himmel verfinstert sich und die Vögel hatten längst das Weite gesucht, als sich die Sphären öffneten und Borbarad hervor trat. Auf seinem Haupt trug er, und noch jetzt schaudert mich die bloße Erinnerung dieses Anblicks, die siebenstrahlige Dämonenkrone. Sieben Erzdämonen hatte er seine Seele versprochen in der Gewissheit, dass sie sie nie würden fordern können. Sie sollten sich selber gegenseitig zerfleischen, um den größten Schatz der Niederhöllen, seit der Versuchung des Namenlosen, die Seele des Alveraniers des verbotenen Wissens.

Borbarad lachte Rohal aus, ob der Schwäche ihn nicht vernichtet zu haben, er beschämte die Liebe die ihm der weise Zwillingsbruder entgegenbrachte, überhörte die Warnungen und den Rat. Er lästerte seines Unvermögens auf die Hilfe des Los angewiesen zu sein, und sprach den ebenso schrecklichen wie verächtlich kurzen "Bannspruch des Borbarad": "Rohal sei nicht mehr!" wir wollten fliehen, wie Rohal uns als letzte Worte riet, doch der Bethanier hielt die Zeit, und ließ sie nach seinem Willen fließen. Wir mussten mit ansehen, wie Rohal langsam verblasste, bis er schließlich verschwand. Unsere Freiheit erlangten wir erst wieder, als Borbarad sich abwendete. "Zerreißt sie!", sprach er und überließ einem Dutzend Zants das Feld. Wir kämpften einen ungleichen Kampf. Ich sah noch wie Irian von zwei Zants in die Zange genommen, ein Bein ausgerissen wurde, als mir selbst, durch die erlittenen Wunden die Sinne schwanden. Rezzanjinn hat sich gut geschlagen, und Temyr hatte, Phex muss ihm geholfen haben, kaum ein Kratzer abbekommen. Toran muss beinahe mit seinem Leben für unser Überleben bezahlt haben, doch die Göttin gewährte ihm ihre Gunst. Und ohne die Unterstützung Rohezaals, und vor allem Faldegorns, hätte wahrscheinlich keiner von uns diesen Massaker überlebt, aber Borbarad hatte uns unterschätzt, unser Freunde die uns helfen, und die Götter die uns schützen.

Er hat uns den Weisen Anführer genommen, doch die Saat war längst gelegt. Der Auftrag längst erteilt, sein Zeichen längst gegeben. Auf dass er verschwinden musste, nicht aber konnte genommen werden, jenen die nun nach seinem Rate handeln.

\subsection{Borbarads Zeit rekonstruiert durch Iliricon}

{\itshape

}
\todo[inline]{Fragment}

\subsection{Die Reise nach Drakonia nach Rezzanjin al'Ahjan}

\paragraph{Mitte Rahja}
Nach dem Zwischenfall in dem kleinen beschaulichen Jassafheim, bei dem Irian fast umgekommen wäre setzten wir unsere Reise fort. Entgegen unserer Erwartungen und zu unserer Freude ließen sich unsere Verfolger vorerst nicht mehr blicken. Keine lächerlichen Hinterhalte und auch keine als Dorffrauen verkleideten Hexen, die versuchten uns umzubringen. Wir hatten also ein paar schöne Tage, in der ich die Schönheit des Yaquirtals im Sommer genießen konnte.

Am frühen Nachmittag dann kamen wir in Tenn an, einem beschaulichen Dorf an der Mündung des Bosquirs in den Yaquir. Eigentlich mussten wir hier, laut Arya, ab, um den Bosquir bis zur Quelle zu folgen, um von dort aus in Hochgebirge vorzustoßen. Doch vorher wollten wir noch Ausrüstung besorgen, um in den auch im Sommer schneebedeckten Bergen des Raschtulswalls nicht zu erfrieren. Wir beschlossen nach kurzer Diskussion, dass diese in Punin wohl am besten zu holen sein würde, weshalb wir den halbtägigen Eilritt zu unserer Villa vor Punin auf uns nahmen. Unser Hausverwalter war ziemlich überrascht, als wir am späten Abend in der Villa ankamen. Doch schon am nächsten Morgen brachen wir nach Punin auf. Es dauerte eine ganze Weile, bis wir die Winterkleidung zusammenhatten, im Sommer ist sie nur schwer zu bekommen, und noch eine ganze Weile länger, bis Temyr von seinem eigenen kleinen Einkaufsrundgang zurück war. Danach machten brachen wir wieder auf und erreichten am früher Abend, gerade noch rechtzeitig für die Feierlichkeiten anlässlich des Rahjamondes, das kleine Örtchen Schlangentod. 

Nachdem wir uns im örtlichen Gasthaus einquartiert hatten, mischten wir uns, bis auf Irian und Ragnos unter die Feiernden. Doch plötzlich, die Praiosscheibe war noch nicht hinter dem Horizont versunken, schoss ein Feuerpfeil über die angrenzenden Häuser und schlug in das große Festzelt ein, welches sofort Feuer fing. In dem nun ausbrechenden Tumult hörte ich Ragnos auf einem Dach noch schreiend auf Angreifer aus dem Wald hindeutend, bevor er von seinem Dach verschwand und ich, mein Tuzakmesser ziehend, in Richtung Wald lief. Dort sah ich Ragnos, vermutlich einem Pfeil ausweichend hinter einem Baum verschwinden und einen bulligen Mann mit einem Tuzakmesser in der Hand auf mich zu rennen. Schnell entwickelte sich ein Kampf, in dem ich plötzlich mit fürchterlichen Schmerzen flachlag, als der Bruderlose vor mir mich berührte. Den Treffer, den er landete, heilte Toran, der wohl auch gekommen war. Doch kaum hatte ich mich erholt, waren auch schon fast alle Gegner wieder verschwunden. Nur Firnen hatte mit der Hilfe von Temyr einen Zwerg erledigt, der, wie Firnen erklärte, wohl mit dem Praios entgegenstehenden Erzdämonen ein Pakt geschlossen hatte und sich deshalb Firnen entledigen wollte. Irian war mal wieder, wie immer, von Travian von Rabenmund mit einem Schlag erledigt worden, während Ragnos irgendwie drei Gegner beschäftigt hatte, ohne mehr als ein paar Kratzer einzustecken. Insgesamt konnten wir sechs Gegner ausmachen. Travian von Rabenmund, der Kerl mit dem Tuzakmesser, Savolina die Hexe, mit der Temyr schon in Grunewald eine Auseinandersetzung hatte, eine brabaker Magierin, eine weitere Hexe und der nun tote Zwerg. Hierzu würde wohl noch Torben Dergeler hinzukommen, der im Kampf, wohl aufgrund seiner hierbei unzureichenden Fähigkeiten keine Erscheinung gemacht hat, aber uns schon zuvor angegriffen hatte.

In der Nacht erholten wir uns von unseren Verletzungen, immer darauf gefasst, dass ein weiterer Angriff erfolgen könnte, doch es blieb ruhig. Nach eineinhalb Tagen Reise erreichten wir Wildenfest und befanden uns auch endlich wieder am Bosquir. Nicht mehr lange und wir würden unsere Pferde hinter uns lassen müssen. Doch vorerst behielten wir diese und gaben sie kurz bevor wir die Klamm des Bosquirs erreichten an einem kleinen Gehöft ab. Nachdem wir auch die Klamm hinter uns gelassen hatten führte uns Arya einem kleinen Bergpfad in das Gebirge hinein. Erst hier, als ich den Djer Tulam, den höchsten Berg des Raschtulswalls an Horizont aufragen sah, wurde mir bewusst, in was für ein Gebirge wir uns begeben wollten. Anstrengende Tage lagen vor uns. Doch schon vor dem Sonnenuntergang sollte sich uns das erste Wunder dieser Berge zeigen. In einem Tal hörten wir plötzlich einen Gong, der von einer Felswand her zu kommen schien. Diese lag auf unserem Weg und so kamen wir näher. Aus der Nähe betrachtet zeigte sich und ein Gebäude in die Felswand hineingebaut, welches unten ein Loch hatte. 

Als wir uns bemerkbar gemacht hatten, wurde aus dem Loch ein Korb hinabgelassen, in dem wir einer nach dem anderen hochgezogen wurden. Wie sich herausstellte, war das Gebäude ein Boronkloster, namens Rabennest, in dem wir für die Nacht gerne beherbergt wurden. Auch hier stellten wir vorsichtshalber Wachen auf, was sich hätte auszahlen können, hätte nicht ich die Wache gehabt. Ich sah noch eine Eule und schon hörte ich die ersten Schritte hinter mir. Dann entbrannte der Kampf. Da ich in einem anderen Raum war, konnte ich meine Freunde nicht warnen. Allerdings versperrte mir auch der Kerl mit dem Tuzakmesser den Weg. Er war gewiss ein guter Kämpfer und so brauchte ich etwas um seiner Herr zu werden, während sich hinter meinem Rücken ein weiterer Bruderloser mittels Tiergestalt begeben hatte. Ich hörte nur das Schlagen von Flügeln. Doch endlich brachte ich den Tuzakmesserkämpfer mit einem Schlag zum Kopf danieder und drehte mich sofort um, um dem anderen Bruderlosen Verräter, welcher Torben Dergeler war, auch eine mitzugeben, was mir auch gelang. Dieser fiel sofort tot um, weshalb ich nach oben eilte um meinen Freunden zu helfen. 

Wie erwartet war es ziemlich chaotisch. Irian lag verletzt am Boden, Travian von Rabenmund stand über einem blutenden Toran und Firnen und Temyr versuchten sich gemeinsam mit Ragnos der Magierin, Savolina und der anderen Hexe zu erwehren. Sofort forderte ich den Rabenmund zum Duell, doch während wir noch aufeinander zuschritten hatten sich ein paar Boronpriester auf die Hexe gestürzt und Firnen war drauf und dran mit seinem Schwert die Magierin anzugreifen, weshalb Travian, seine Niederlage eingestehend, wohl mithilfe eines Tranversalis Ringes entschwand. Sofort nahm ich mich der Brabakerin an, welche unter meinem unwiderstehlichen Schlag fiel. Doch Savolina schaffte es sich ihren Besen zu schnappen und flog aus dem Gebäude, nur um fünf Schritt davor, in der Luft schwebend stehen zu bleiben und ihren Bogen mit einem ihrer immer treffenden Pfeile auf Ragnos zu richten. Ihre letzten Worte deuteten auf den Kummer hin, den ihr der durch Ragnos verursachte Tod ihres Geliebten wohl gemacht hatte, weshalb sie ihn rächen wollte. Dann schoss sie den Pfeil ab. Doch er traf nicht Ragnos, sondern sie selbst. Nagrach hatte wohl was dagegen, dass Ragnos stirbt. Ob man sich deshalb Sorgen machen sollte? Wohl nicht.

Die Boronpriester kümmerten sich schnell um die noch sterbende Brabakerin und versuchten ihre Seele zu retten. Ohne Erfolg, wie ich hoffe, denn die Schandtaten, die sie sterbend gestand waren grausam. Ich schaute derweil nach den beiden Toten ein Stockwerk tiefer. Dergeler war verschwunden, schon wieder, aber er war wohl oder übel tot, ich hatte ihn gut getroffen. Die andere Paktiererleiche brachte ich zu den Boronis hoch.

Am nächsten Tag setzten wir unseren Weg in die Berge fort. Die folgenden Tage waren anstrengend und begleitet von stetigen Auf- und Abstiegen tiefer ins Gebirge hinein. Irgendwann wurde es so kalt, dass wir unsere Winterkleidung anziehen und unsere Schritte durch den Schnee setzten mussten. Am 30. Rahja wurde uns gen Abend auf einmal der Weg durch einen Haufen Steine versperrt. Glücklicherweise stand darauf ein schwer verständlicher Ferkinakrieger, der uns in seinem seltsamen Tulamidya erklären wollte, dass wir das Gebiet hinter ihm nicht betreten durften. Nach ein paar Verständnisproblemen erklärte er uns, dass dies nur Ferkina Krieger dürften. Man müsste schon einen im Kampf besiegen, um vom Schamanen für würdig erachtet zu werden. Glücklicherweise kamen auf ein Handzeichen des schon etwas in die Jahre gekommenen Kriegers gleich mehrere junge Ferkinas hinter den Felsen hervor. 

Um das Ganze abzukürzen zog ich mein Schwert, doch ihre erschreckten Gesichter verrieten mir, dass sie wohl einen Faustkampf meinten. Folglich griff ich den nächstbesten mit meinen Fäusten an. Doch dieser kannte sein Handwerk gut und so brauchte ich eine ganze Weile, bis ich ihm einen finalen Tritt verpassen konnte. Daraufhin willigte der Ferkinakrieger ein uns zu ihrem Schamanen zu bringen. Er führte uns eine Weile in ein kleines Seitental, in dem eine paar Fellzelte und Hütten standen. Er brachte uns direkt zum größten und am meisten mit Knochen verzierten Zelt, doch die größte Überraschung saß am Lagerfeuer. Es war Raidri Conchobair, der größte Held Aventuriens, der Bezwinger der Blutzwillinge, klar zu erkennen an dem Schwerterpaar, welches neben ihm auf dem Boden lag. Doch zuerst verlangte der Schamane unsere Aufmerksamkeit. Er begutachtete mich eine längere Zeit und meinte dann, nachdem er mir mit einem Messer ein wenig Blut abgezapft hatte, dass mein Blut gut genug sei und, dass wir das Gebirge durchqueren dürften, jedoch erst nachdem wir eine Nacht bei den Ferkinas verbrachte hatten. Wir gesellten uns danach zu Conchonair an Feuer und merkten schnell, dass er uns zumindest von Hören her kannte. Wir verbrachten dann noch einen geselligen Abend am Feuer, nur gestört von den Ferkinafrauen, die und abwechselnd Fleisch und vergorene Ziegenmilch andrehen wollten, die wir nach einiger Zeit ablehnten.

Am darauffolgenden Tag ging es weiter in die Berge hinein. Die Ferkinas gaben uns noch stinkendes Ziegenfett mit, welches wir auf unserer Haut verteilen sollten. Conchobair meinte, es helfe gegen die unerbittliche Höhensonne. Gegen Mittag sahen wir auf unserem Weg eine große Statue. Als wir näher kamen stellte sich heraus, dass es sich um eine stark verletzten Troll handelte. Firnen meinte sogar, dass dieser sehr zauberkräftig sei und keine zwei Tage lang versteinert war. Plötzlich fing der Troll an sich langsam und dann immer schneller zu bewegen. Als wir schließlich mit ihm redeten, stellte sich heraus, dass der Troll ein Schamane, auf dem Weg zum Konzil war. Firnen heilte ihn zum Teil, doch als wir mit ihm weiterreisen wollten, stellten wir fest, dass die Reisegeschwindigkeit des Trolls weit über der unsrigen lag. Am Abend bekamen wir noch einmal einen atemberaubenden Blick auf dem Djer Tulam, der hinter zwei hohen Berggipfeln aufragte. Am nun folgenden zweiten Tag des Bruderlosen sollten wir laut Arya das Hohe Tal erreichen. Ein grünes Wunder inmitten der weißen Gebirgslandschaft. Etwa am Mittag erblickten wir es dann etwa 300 Schritt unter uns. Tatsächlich war das etwa eine Meile lange begrünte Tal eine wahre Pracht. Ein Ort des Lebens inmitten einer lebensfeindlichen Umgebung. Wie wir schon einmal ein Tal in der Gor gesehen hatten.

Doch stellte sich jetzt das Problem, wie wir dort hinunterkamen. Arya meinte es gebe einen Pfad hinunter, der würde jedoch lange dauern und sei gefährlich. Wir entschieden uns dazu mittels Nihilogravo hinunter zu schweben. Ein wahrhaftiges Erlebnis, auch wenn es von kurzer Dauer war, da Temyr den Spruch nicht lange aufrechterhalten konnte. Unten angekommen suchten wir uns erstmal einen Weg aus dem Dschungel zum nächsten Wunder dieses Gebirges, der Himmelstreppe, die uns bis zum Dach der Welt führen sollte. Auf dem Weg durch das Dickicht sahen wir immer wieder kleinere Tiere, vor allem Zeigen, die in dem grünen Paradies lebten. Am Ende des Tals wich der dichte Wald einer kahlen Wiese und dort sahen wir sich den nächsten Steilhang hinaufwindend die Himmelstreppe. Bis zum Ende konnten wir gar nicht sehen, so hoch war sie, und so beschlossen wir für die Nacht hier zu rasten, um den anstrengenden Aufstieg mit 6000 Stufen, so sagte Arya am nächsten Tag zu beginnen.

Das taten wir dann auch und es nahm den größten Teil des Tages ein. Immer wieder legten wir Pausen ein, doch am Ende der Treppe brannten bei allen die Beine und wir waren froh sie hinter uns gelassen zu haben. Doch wir waren jetzt auf dem Dach der Welt, welches eine weitläufige Hochebene war, welche von drei weißen Gipfeln und drei Vulkanen begrenzt wurde. Über die weite mit Gras und weißen Blumen bewachsene Ebene hinweg sah man schon Drakonia, eine riesige Festung, die je näher wir ihr kamen nur noch größer wurde. Auch Zitadelle der Elemente genannt, wie Arya einwarf, war dieses mit hoch emporragenden Türmen gespickte Sechseck wahrhaftig imposant. Doch erst am Tor dämmerte mir, dass diese Festung nicht von und nicht für Menschen geschaffen war. Durch das Tor hätte Faldegorn, Rohezals Kaiserdrache, spazieren können, ohne den Kopf einziehen zu müssen. Geziert wurde es von einem gigantischen Vogelbild, welches wohl den Lichtvogel, dessen Erscheinen wir beizuwohnen gedachten, zeigte. 

Doch einfaches Anklopfen öffnete das Tor nicht. Zuerst erschienen sechs Elementarwesen, welche uns nach dem Grund unseres Besuches fragten. Wir wollten zuerst die Suche nach dem Kind Yasinthe von Tuzaks unerwähnt lassen, doch da uns der Eintritt so verwehrt blieb, erwähnten wir es schließlich. Die sich dem Tor anschließende Halle war beeindruckend. Wir wurden von einer Adepta in Empfang genommen und schließlich auch von Pyriander di Archos, welcher der Akademieleiter war. Wir bekamen Quartiere zugewiesen und uns wurde erklärt, dass wir die Akademie gerne besichtigen könnten, jedoch besser einen Führer mitnehmen sollten, um uns in dem riesigen Bau nicht zu verlaufen. Auf die Frage, ob denn Yasinthe von Tuzak mit ihrem Kind da seien, bekamen wir die nur mäßig zufriedenstellende Antwort, dass sie zwar da seien, aber in ihrer Meditation nicht gestört werden sollten. So wollten wir das auch nicht tun. Wir trafen vor unserem Zimmer Morena vom Blautann zusammen mit Ruban dem Rießlandfahrer, die beide auch eingeladen worden waren. Des Weiteren waren noch zwei hohe Vertreter der Zwerge, der rote Pfeil, auch bekannt als Tenobaal Totempfeil, sowie Xenos von den Flammen anwesend, welchen wir schon in Punin getroffen hatten. Ich fand auch noch zwei Magier aus Maraskan, welche meine schöne Insel schon seit ihrer Kindheit nicht mehr gesehen hatten, weshalb ich ihnen gerne davon erzählte.

Insgesamt hatten wir bis zum 1. Praios nicht viel zu tun. Temyr, Firnen und Irian besichtigten deshalb intensiv die Akademie zu besichtigen, während ich lieber mit den anwesenden Leuten sprach. Als wir uns nach dem Kind umhörten mussten wir feststellen, dass es weder Junge noch Mädchen war und, dass es allgemein als seltsam beschrieben wurde.

Am ersten Praios dann fanden wir uns vor dem Erheben der Praiosscheibe in der großen Halle der Akademie ein. Dort hatten sich schon die meisten Leute eingefunden, die dem Ritual um den Lichtvogel beiwohnen wollten. Die sechs Elemantarzweige hatten sich in ihren Gruppen zusammengefunden, um ihre vorbereiteten Rituale zum Raschtulkanscharot, dem Vulkan, in den sich der Lichtvogel hineinstürzen würde, hinzutragen. Die eingeladenen Gäste sammelten sich auch und wurden in die Mitte der nun losgehenden Prozession genommen. Erfreut stellten wir fest, dass Yasinthe von Tuzak und ihr Kind auch bei der Prozession dabei waren. Am Raschtulkanscharoth angekommen verteilten sich die sechs Elementzweige auf sechs kleine Tempel oder Erhebungen ihres Elementes und bereiteten ihre Rituale vor. 

Kurz bevor sich die Praiosscheibe erhob setzte Pyriander de Archos zum Sprechen an und eröffnete feierlich die Rituale. Schon erhoben sich die Stimmen der Elementarmagier und von den verschiedenen Tempeln aus ertönte Musik oder Gesang. Mit dem Aufgehen der Praiosscheibe vereinigten sich diese und von ganz weit oben herab flog ein Vogel aus purem Licht, der aus allen sechs Elementen zu Bestehen schien. Er kreiste kurz über unseren Köpfen, um dann schließlich langsam herabzusteigen und sich in den Krater zu stürzen. Der Krater selbst leuchtete kurz auf, als der Lichtvogel ihn berührte. Dann nichts. Eine kurze Weile später stieg aus dem Krater ein Ei aus Licht auf und schwebte vor dem Tempel des Feuers. Doch plötzlich verdunkelte sich der Himmel aus dem Nichts und ein Vogel schwarz wie die Nacht, groß wie der Lichtvogel selbst, stürzte sich aus dem Himmel herab in Richtung Ei. Unvermittelt tauchten in allen Tempeln Massen an Dämonen auf, die die Magier überraschten. Während in den Tempeln ein Kampf entbrannte, an dem wir uns auch beteiligten, verwandelte sich der schwarze Vogel in eine Raubkatze und umklammerte das Ei. Flammenlanzen, die jetzt vereinzelnd von den Tempeln auf das Unwesen schossen, schienen an seinem Fell einfach verschluckt zu werden. Schließlich verwandelte sich der Dämon in einen Vogel und flog mit dem Ei weg. Der Dämonen wurden wir schnell Herr, doch das Ei war weg. Schnell fand sich eine Gruppe zusammen, angeführt von Raidri Conchonair, mit uns, Morena, Ruban dem Rießlandfahrer, dem roten Pfeil, Pyriander di Archos und weiteren Leuten, die auf Elementaren dem Nachtschattendämon, wie Firnen erläuterte, zu folgen, das Ei zurückzuholen und schließlich Los ermöglichen wieder ein Auge auf Dere zu werfen.

\subsection{Fragmente der Drakonia- und Lichtvogelexpedition}

{\itshape
Über die legendäre Lichtvogelexpedition konnte ich in den Schriften der Gezeichneten selbst leider nur weniges entdecken. Es stellt sich aber heraus, dass Teile der Tagebücher Firnens wohl in den Wirren der Schlacht an der Trollpforte verschollen gegangen sind. Diese Fragment wurde aus den Ruinen von Burg Hageltrog im Finsterkamm geborgen.
Es wurde scheinbar von einem gewissen Kunibald van der Vaag zusammngestellt, über den ich jedoch nichts nöheres in ERfahrung bringen konnte.

Die folgenden Forschungstagebucheinträge und Schrifenzusammenführungen wurden vom Rohdensteiner Galan von Blaufelden angefertigt, der die schwer beschädigten Fragmente teilweise zu rekonstruieren vermochte.}

In den folgenden Schriften ist eine Geschichte zusammen getragen, die so großartig wie unglaublich ist. Eine Geschichte die Einblick gewährt in die Spielregeln der Götter. Eine Geschichte die vom Weltenende erzählt, und wie in Aventurien ein neues Zeitalter anbricht. Es gibt verschiedene Balladen, die von Heldentaten und Abenteuern berichten. Aber die Zusammenhänge und Historischen Fixpunkte, die der Geweihte Galan von Blaufelden, in der folgenden Sammlung offenbart sind, wie von Hesinde persönlich gezeichnet.

Es ist fast zuviel was mir die Götter offenbarten. Fast 7 Jahre habe ich gebraucht um die Puzzelstücke der Historie zusammen zutragen. Aufmerksam wurde ich durch eine Schrift die sich auf den Magus Firnen Wulfgrimm zurückführen lässt. Meine Recherche führte mich durch Sagen und Legenden doch ich glaube erkannt zu haben, dass die meisten Lieder und Geschichten, weit wahrer sind als sie sich je gewünscht hätten.

Die Schrift Firnen Wulfgrimms, der Träger des ersten Zeichens war, gehen auf den 29. Rahja des Jahres 2019 nach Bosparans Fall zurück. Die Mehrzahl der Wörter sind stark verblasst, oder vom Wind und Wasser verunstaltet, doch es lässt sich etwa folgende Geschichte daraus ablesen:

Sechs Tage ist es her seit wir die Horasische Grenze passierten. Die Zwischenfälle mit Travian von Rabenmund und dem Rest seiner Schergen verblasst schon fast zu einem Schatten meiner Erinnerung, so viel hat dieser Tage meine Konzentration beansprucht. Ich weiß schon gar nicht mehr wie es dazu kam, dass sie uns des Nachts in einem Bergkloster der Boroniten überfallen konnten, doch wie ich mich erinnere hat Azzariell Scharlachkraut den Tot gefunden, durch einen Freipfeil von Ragnos.

Vor knapp einer halben Woche haben wir Unterwegs Raidri Conchobair getroffen und ich bin froh, das Rezzanjinn und Raidri sich nun endlich aus dem Weg gehen können. Aus den Umfangreichen Erzählungen von und um den Schwertkönig geht hervor, dass er die Gezeichneten tatsächlich etwa am 25. Rahja \dots

Die abwechselnden Pralereien und überzogenen Sticheleien der Beiden gingen mir allmählich gehörig auf den Geist. Heute Mittag ist uns auch der Troll-Schamane wieder begegnet, dem wir vor zwei Tagen geholfen haben. Tatsächlich erkannte er uns wieder und sprach uns in gebrochenem Tulamidia an. Er dankte für die Unterstützung die wir ihm Gewährt hatten. Er sagt uns, wir seinen ``Ro-Shott-door'', vermutlich etwas wie ``Ich stehe in eurer Schuld'' oder ein Äquivalent der Trolle. Der Schamane blieb nicht bis zur Jahreswende, sondern zog wohl gleich wieder seiner unbekannten Pfade.

Die folgenden Abschnitte beschäftigen sich mit der Ankunft in den Hallen des Konzils. Sie sind kaum zu rekonstruieren, aber aus den verwendeten Adjektiven geht Staunen und Ehrfurcht hervor. Das Mächtige Tor der Festung, gehütet von einem Elementaremn Meister und von Sterblichen nicht zu öffnen, ist dabei, obgleich häufig beschrieben, nicht das eindrucksvollste an dieser Anlage. Auch aus anderen Schriften ist seit der Offenbarung des Konzils viel über die beeindruckenden Hallen und uralten Gänge bekannt geworden. Die Magistrae Ignazia von der Wehl, Absolventin zu Punin, reiste einst zum Konzil der Elemente und beschrieb es etwa wie folgt:

``Tief in den Höhen des Rashtulswalls gelegen, hinter sagenumwobenen Tälern und Legendenumrankten Gipfeln, erhebt sich eine Festung gleich einem Berg. Geschaffen vor Äonen, bewohnt von Wesen aller Zeitalter, und von einem Ausmaß das kein Sterblicher je erfasste. Zwischen verwunschenen Orten voller Magie, ein Quell den viele magische Zünfte unserer Zeit besuchen, finden sich in den alten Gängen, Hölen und Hallen, Reliefs und Bibliotheken aus längst vergessener Zeit, Was dort geschrieben steht hat noch keiner je erkannt, und je weiter man durch die Anlage streift, desto mehr überkommt einen Ehrfurcht und Staunen vor diesem Abriss der Äonen, vielleicht seit Leben auf Aventurien existiert.''

Ich denke die Eindrücke der werten Magistrae, decken sich in etwa mit denen die auch der Gezeichnete Magus mitteilen wollte, von seinem ersten Besuch im Konzil der Elemente. In den Berichten Wulfgrimms findet der Name Yasinthe Erwähnung. Möglicherweise bezieht es sich auf Yasinthe von Tuzak. Sie war in den Jahren zuvor Geliebte der Göttin gewesen, bis sie ein Kind bekam und bald darauf verschwand. Es ist Überliefert, das die Gezeichneten zu einem Späteren Zeitpunkt Kontakt zu ihr hatten und auch das Kind eine wichtige Rolle spielte im Kampf gegen den Dämonenmeister. Ob sich ihr erstes zusammentreffen allerdings bereits auf diese Tage datieren lässt, bleibt uns vorenthalten.

Die Aufzeichnungen zu den nächsten Tagen sind zum Glück umfassender erhalten und ließen sich vollständig rekonstruieren:

Es ist nun der 30. Rahja und der letzte Tag vor der Wende des Jahres. Ich habe mir ein paar Stunden ruhe gegönnt und mich mit diesen meinen Aufzeichnungen zurück gezogen. In den Hallen des Konzils herrscht ein Trubel, wie ich ihn in Kunchom auf dem Basar nicht erlebt habe. Alle sind in heller Aufregung. In der Mitte der Nacht so erzählen die Magier und Druiden, wird der Lichtvogel, ein Geschöpf des Los, herabsteigen in die Gluten des Vulkans. Wenn er vergeht, so steigt sein Ei herauf, und mit den ersten Strahlen der Sonne wird sich der Vogel erneut daraus erheben. Seit die Bewohner des Konzils ihre Chronik kennen, bringen alle Zweige der dort ansässigen Magier mächtige und kunstvolle Geschenke zu Ehren des Lichtvogels dar. Die Elementaristen haben sich, nach allem was ich bisher gesehen habe, bei weitem Übertroffen. Kunstvolle Gebilde, geschaffen aus einem reinem Element, gepaart mit Darbietungen der Magier selbst. Die sonst recht schweigsamen Magi des Eises haben sogar einen alten Choral eingeübt. Auch viele Fremde sind schon zusammen gekommen. Heldenfiguren, die bereits zu Lebzeiten in die Legenden eingegangen sind, wie Raidri Conchobaire oder Sintbart der Entdecker. Auch einige Diener und Dienerinnen Sumus, sowie Magier der näherliegenden Südlande sind zu diesem Spektakel angereist. Ich bin gespannt welch Wirken der Götter und der Welt uns heute Nacht offenbar werden wird.


In unserer Zeitrechnung schreiben wir heute den ersten Namenlosen und die Tage scheinen mir düsterer als jemals zuvor. Auf den sanften Lüften eines Dschinns schwebe ich weit über dem Land und erholte mich noch einmal von den Schrecken der Nacht, die noch immer anhält und nicht weichen möchte. So erhebend es hier oben auch sein könnte, so niedergeschlagen und bedrückt wie heute waren wir höchstens als der Dämonmeister höchst selbst erschien und uns Rohal nahm, in der Stunde, da alle Hoffnung auf ihm Ruhte. Doch damals erhielten wir die Kappe des Weisen und seit her ziert sie Temyrs Haupt. Heute jedoch erhielten wir nichts und verloren alles. Ich will es euch von vorne berichten.

Zur Feier der Wiedergeburt des Lichtvogels hatten sich alle in oder kurz vor der Festung auf einem Felsplateau versammelt, das den Krater eines alten, Vulkanes umsäumt. Die Geschenke wurden dargebracht, die Vorführungen dargeboten, und als die Nacht am schwärzesten war, erschien ein geflügeltes Wesen am Himmel nicht als Schatten vor dem Mond, sondern als Mond vor den Schatten der Welt. Ein Vogel von gewaltiger Größe doch Anmut. Der Lange Schwanz zu den Enden gegabelt, die Schwingen mächtig wie die eines Greifen, die Federn nicht etwa in güldenem Glanz sondern wie vom Lichte selber umhüllt. Freude herrschte unter den Anwesenden doch der Vogel schien keine Notiz davon zu nehmen. Dann zu meinem Erschrecken stürzte er in die Mitte des Vulkans und verging dicht über den Gluten umhüllt von Feuer so alt wie die Zeit. Noch immer jubelten die Anwesenden, denn sie glaubten zu wissen, dass sich der Vogel bald wieder erheben würde. Quälend langsam schlich die Zeit dahin. Nur die Magie der Geschenke und das Glühen des Vulkanes erhellten matt die Dunkelheit. Ich begann mich längst ernsthaft zu sorgen und auch unter den Anderen machte sich die Spannung der Erwartung breit. Und tatsächlich, funken stiben aus dem Krater empor und von einem Funkenregen umhüllt und schimmernd wie die Glut, stieg aus dem Feuer des Berges ein Ei empor, so klein wie ein Geborenes im ersten Winter. Es verharrte auf Höhe unserer Köpfe, gehalten von den Funken, mitten über dem Krater und wir warteten darauf, dass Prajos sein Antlitz über die Gipfel des Kandscharot erhebt. Gleich einem Gebet rezitierten die Druiden alte Verse, die von der Geburtsstunde des Lichtvogels künden:

``Und sieh die Sonne erhob sich über die Spitzen der Berge und ihre Strahlen füllten das Tal, langsam kroch das Licht die Hänge hinab und hüllte Pflanzen und Steine in güldenes Licht. Und sieh das Licht erfüllte alles was da ist, und das Ei bekam Risse und strebte zum Licht. Und sieh, schimmernd wie der Regenbogen, brachen sich die ersten Strahlen der güldenen Scheibe, an der feurigen Ummantelung des Eis und da\dots''

Da zerriss ein Kreischen die kühle Luft des Morgens. Der Himmel verfinsterte sich bis kein Licht der Götter mehr das Antlitz der Erde erreichte. Wir gerieten alle in Panik, keiner wusste was los ist. Kluge Köpfe eilten umher unschlüssig was sie tun sollten. Mächtige Helden, die der Gefahr sooft gespottet hatten sahen sich verwirrt um und bangt dessem das da kam. Ein Vogel stieg aus den Schwärzen des Himmels herab. Und ihm folgten Herrscharen der siebten Sphäre. Karakilim stürzten sich auf die Anwesenden Magier, einige Zants erschienen mitten unter uns und bagannen ihr grausiges Werk. Mehrköpfige Wölfe der Höllenpforte, rannten durch die Luft und stürzten sich in einen wilden Kampf mit den Elementaren Wesenheiten, die die Magier des Konzils verzweifelt zur Hilfe riefen. Ich jedoch sorgte mich nicht um all dies. Denn ich habe mich viel mit dem Gezücht der Niederhöllen beschäftigt, und wusste um das Wesen, das viel schrecklicher ist, als all das andere Gezücht. Der riesenhafte Vogel, umhüllt von Dunkelheit und selber Dunkler als die Schwärze der Nacht: Man nennt ihn den Nachtschattendämon. Und es heißt Kein Zauber, und keine Waffe, können ihn Verletzen, kein Göttlicher und kein Sterblicher könne ihn vernichten, solange die Nacht ihn vor den Augen der Götter verbirgt. Dies war kein Geschöpf der Höllen, sondern ein Teil von ihnen. Und es zerriss erneut die Zeit mit seinem Schrei, der jedes Blut in den Adern gefrieren lässt, bei dem die Herzen der Sterblichen, für den Moment aufhören zu schlagen. Und der Dämon stieß herab. Kein Zauber fand sein Ziel, kein elementares Wesen konnte in seiner Nähe bestehen. Nichts konnte ihm Schaden zufügen und als er sich wieder in die Lüfte erhob, trug er das Ei des Lichtvogels in seinen Klauen. Mit einem letzten Schrei verschwand er über den Bergen im Nordosten und so schnell wie der Spuk über uns herein gebrochen, so schnell war er auch vorbei. Die Kreaturen der Niederhöllen waren ihrem Anführer gefolgt oder längst in andere Sphären verschwunden. Nur die Verletzten zeugten von dem kurzen aber heftigen Kampf. Es war als wäre sonst nichts geschehen. Und ohne das Ei des Lichtvogels in unserer Mitte, würde vielleicht auch nichts mehr geschehen.

Die Stimmung grenzte an bodenlose Verzweiflung. Schließlich traten wir zusammen, mit den obersten Magiern und größten Helden unserer Zeit berieten wir und fassten den Beschluss, den Nachtschattendämon zu verfolgen. Wir würden das Ei des Lichtvogels zurück bringen. Die Konzilsmagier gaben uns je ein Artefakt mit, einen Dschinnenring zu jedem Element. Außerdem beschworen sie uns Luftdschinne, die uns ermöglichen sollten den Dämon zu verfolgen und den Kontinent in kurzer Zeit zu überqueren. Und da sind wir nun. Toran, Temyr, Ressanjinn, Ragnos, Irian und Ich. Und uns begleiten dieses Mal Raidri Conchobaire, sowie Sintbart der Entdecker (der auf die Fähigkeiten seines eigenen fliegenden Teppichs zurückgreifen kann), und außerdem Luzellin vom Blautann (die natürlich ihren Hexenbesen zur Verfügunghat und ebenfalls nicht auf die elementarenWesen angewiesen ist).

Ich hätte mir schönere Umstände einer solch beeindruckenden Reise wünschen können. Aber wer weiß, was die Zukunft bringen wird. Und so die Götter wollen, werden wir den Lichtvogel zurückbringen und es wird eine Zukunft geben.

\todo[inline]{Fragment}