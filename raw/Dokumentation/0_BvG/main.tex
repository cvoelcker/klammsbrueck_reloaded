\section{Heldentrutz und Purpurblitz}

Im Jahr 1014 Bosparans Fall veranlasst der Reichsbehüter Brin auf dem Reichstag zu Gareth die Gründung der Markgrafschaft Heldentrutz, um die Weidener Lande besser vor den verbleibenden Horden der Orks zu schützen, die nach wie vor das Svellttal besetzen. Als neue Markgräfin wird die junge Walpurga von Löwenhaupt eingesetzt, die Tochter des Herzogs von Weiden.  Um die Erhebung seiner Tochter gebührend zu feiern, richtet der alte Herzog Waldemar ein großes Turnier zu Trallop aus, das Glücksritter und Wagemutige aus dem ganzen Herzogtum anzieht. Als Ehrengäste werden auch die Herzogensöhne Tobriens erwartet, wobei der Jüngere der neuen Markgräfin als Mnn versprochen wurde. Doch Walpurga schätzt den Schönling keineswegs, ist er doch eher der Rahja als der Rondra zugetan.

Zu allem Überfluss planen aber auch die Feinde der Familien Löwenhaupt und Ehrenstein kalte Rache. Die durch den Reichstag von Gareth fast in die Versenkung getriebenen Anhänger des Reichsverräters Answin können es nicht ruhen lassen, dass der Herzog von Tobrien das Bündnis mit ihrem Anführer brach und Brin von Gareth so auf den Thron verhalf.

\subsection{Verbindungen zur Kampagne}
Einige wichtige NPCs werden eingeführt, vor allem die Familien Löwenhaupt und Ehrenstein. Die Beziehung Walpurgs zu den Ehrenstens, die eine Grundlage des Wolfswinters wird wird thematisiert. Innere Spannungen in der Reichspolitik werden betrachtet, Answinisten spielen auch während Borbarads Angriff auf Tobrien eine große Rolle.

Thematisch steht hier vor allem die Beziehung der Familien Löwenhaupt und Ehrenstein im Vordergrund, mythische Hintergründe werden eher ausgeklammert.

\subsection{Das Turnier zu Trallop}

Das Turnier findet auf einem großen Feld vor den Mauern Trallops statt. Da die Herzogenburg durch die Bewirtung von so vielen Gästen aus allen Nähten platzt, ist ein großes Zeltlager vor den Stadttoren errichtet worden, das die vielen Kampfesmutigen und Schaulustigen beherbergt. Auch die meisten Hotels und Gasthäuser in der Stadt sind völlig überlaufen.

Die Turnierwiese befindet sich nahe dem Neunaugensee, direkt an einem kleinen Wäldchen gelegen, in dem die Tiere vor der Sommerhitze geschützt untergebracht werden.

Turnierbesucher: 

\begin{itemize}
	\item Walpurga von Löwenhaupt, neu gekührte Markgräfin zu Heldentrutz
	\item Dietrad von Ehrenstein, ein Jüngling der Minne zugeneigt
	\item Bernfried von Ehrenstein, der Herzogensohn zu Tobrien
	\item Waldemar der Bär, der Herzog von Weiden
	\item Lokaladlige (nachschauen/erfinden)
	\item Ayla von Schattengrund, eine Rondrageweihte (noch einbauen)
	\item Brin vom Rhodenstein, ein Knappe
	\item Yolina, Herzogenmutter von Weiden
\end{itemize}

Backdrop: Turnier \& große Jagd (5 Tage)

\begin{itemize}
	\item 1. Tag: Buhurt
	\item 2. Tag: Lanzengänge
	\item 3. Tag: Zweikampf
	\item 4. Tag: Jagd
	\item 5. Tag: Abschluss
	\item parallel: Festspiele für einfache Leute: Bogenschießen und allerlei andere sportliche Disziplinen
\end{itemize}

\subsection{Gefährliche Intrigen}

Antagonisten: miese fiese Answinistenverräter, Motiv: Rache an den Häusern Ehrenstein und Löwenhaupt wegen des wahrgenommenen Verrates an Answin (der tobrische Herzog war einst Parteigänger Answins)

Was passiert:

\begin{itemize}
	\item Medici bestechen
	\item Giftanschlag
	\item Drogen an Kämpfer
	\item Tiere vergiften
	\item \$Zauberei (Raserei)
	\item Waffen manipulieren (Lanze splittert nicht)
	\item Unruhe im Trosslager stiften
\end{itemize}

\subsection{Szenen}

Einstieg: Tobi $\to$ Rondragefälliges Turnier, Dini $\to$ Jagdmeisterin

\begin{itemize}
	\item Dietrad von Ehrenstein versucht Walpurga mit Blumen und Gesang zu beeindrucken, diese wehrt ihn brüsk ab
	\item Dietrand versucht als Kämpfer am Turnier teilzunehmen und wird schwer verwundet, da er absolut keine Ahnung von dem hat, was er da tut
	\item Anschlag auf Dietrand im Lazarett (Dolch, Gift?) durch einen bestochenen Medicus
	\item Bernfried und Waldemar weisen einen hohen Preis für die Ergreifung der Täter aus
	\item weitere Anschläge, vor allem auf Gefolgsleute der Ehrensteins
\end{itemize}

\subsection{Finale}
Irgendwie versuchen die Verschwörer die Gruppe zu trennen, so dass Dietrand und wenige Gefährten alleine reiten.

Im Wald bei der Jagd lauert Ardarians Vater der Truppe mit einigen Spießgesellen auf, da seine Verschwörung aufgeflogen ist.

Es kommt zum Kampf, Ardarians Vater stirbt entweder, oder wird gefangen genommen (zweites wäre von Vorteil).
Der Baron von Sinopje erhebt Anklage gegen seinen Knappen und versucht ihn als Mitverschwörer von Wallenstein darzustellen (die Helden können Beweise in Sinopjes Zelt finden, aber dazu müssten sie einbrechen). 
Sollte Ardarians Vater leben, wird er zum Tode verurteilt, allerdings gesteht er, dass er auf Anweisung gehandelt hat und Informationen über seinen toten Sohn genutzt wurden, um ihn zu erpressen. Sollten sich die Helden für ihn stark machen, wird er "nur" eingekerkert, und Walpurga und Dietrad unterstützen die Suche nach der Wahrheit. Mögliche Argumente sind, dass Wallenstein noch zu vile weiß, um über die Klinge zu springen.

Sollte Sinopje überführt werden, kommt es zum Endkampf gegen ihn in Gestalt eines Werwolfes. Ansonsten wird er seine verräterischen Fäden weiter spinnen und ein dauerhafter Antagonist für Ardarian werden.


\subsection{Spielmitschrift}

Freiherr von Sturmfels beleidigt Ardarian hart, ist reicher Rinderbaron. Dietrad im Buhurt getroffen und geholfen, manipulierte Rüstung erkannt. Mit Dietrad angefreundet.

Wolfhardt, Knappe von Dietrad, wird beinahe verprügelt, Fenia schreitet ein. Baron von Sinopyje ist der Answinist, sein Knappe ist Schuld an der Lanze und Wolfhardt, ein weiterer Verschwörer hat die anderen Dinge zu verantworten (ist Tobias Vater!). Beide sind zu arm, um sich gorßartige Verschwörungen zu leisten, Arngrimm von Ehrenstein finanziert sie und nutzt ihre Wut auf Kunibrand aus, um sie zu motivieren. Möglicherweise hat er auch Informationen über den Mord an Adarians Bruder (oder tut zumindest so).

Stallbursche von Wallenstein Beihelfer bei der Verschwörung, nun nach Drohung Doppelagent (hat immer unterschiedlich viele Kinder).

Pferd von Dietrad wurde vergiftet, Fenia hat das erkannt, Lanze wurde alchemisch manipuliert um Dietrand zu ermorden. Walpurga taut durch Gesang am Abend gegenüber Dietrad auf.

Uralte Hüterin der Eisigen Stelen, Vision lahmendes Pferd, zwei steigende Einhörner für Fenia.

Namen:

Yolana aus Ysilia: Leibmagierin der Prinzen
Sal: Knappe von Sinopje
Baron von Sinopje: Verschwörer

"Wir wurden als unabhängige... investigativ... Ermittler angeheuert"

\paragraph{Zweiter Abend} 
Am letzten Tag des Turniers: 

Erste Reckin: AT 17 PA 15 Finte, Wuchtschlag, Gegenhalten und RS 4

Gisella das Kräuterweib versucht Rahjalieb zu verkaufen, weil sie Ardarian falsch vrestanden hat.





