\section{Hexennacht}

\subsection{Wichtige Ereignisse und Personen}

Während des Abenteuers sind folgende Kernereignisse wichtig:
\begin{itemize}
	\item der Rubinsplitter von Lin zerfällt zu Beginn des Abenteuers
	\item Luzelin erhält das Alicorn des schwarzen Einhorns
	\item die Nadel des zweiten Zeichens wird geschnitzt
	\item Luzelin, Morena, Achaz und Savolina einführen
\end{itemize}

Die wichtigen Personen für die weitere Kampagne sind:

\begin{itemize}
	\item Luzelin vom Blautann, Hexenoberhaupt Weidens und Erschafferin des zweiten Zeichens
	\item Morena vom Blautann, ihre Tochter
	\item Achaz die Hexe, Gegenspielerin Luzelins
	\item Savolina Nagrachsdottir, eine Hexe die den Wettkampf etwas zu Ernst nimmt
	\item Gwynna, die Prinzessin von Weiden, die als Vermittlerin auftritt
\end{itemize}

\subsection{Einführung}

Fenia kann nach dem Tralloper Turnier entweder aufbrechen, oder eine Weile in Trallop bleiben, ob als Gefolgsfrau derer von Löwenhaupt, oder als Geweihte im Tempel. So oder so könnte Gwynna sie als Jägerin anwerben, denn sie hat sowohl Talent mit Menschen als auch mit Tieren gezeigt.

Lin wird zur Hexennacht geladen (via magischem Boten, denn alle Hexen der Weidener Lande sollen bei diesem Ereigniss anwesend sein) und läuft dabei Fenia über den Weg. Diese wird entweder durch die Vermittlung Gwynnas, oder durch eine Rettung Lins vor einem wilden Tier, mit zum großen Hexentreffen im Blautann geschleift.

Die Sommersonnwendfeier der Hexen ist ein beeindruckendes Spektakel, welches ruhig eine Nacht dauern kann. Dabei herrschen wahrhaft rahjanische und levtanische Ausschweifungen, auch, weil ein paar junge Burschen und Mägde Weidens immer wieder keck genug sind, nach den legendären Hexen zu suchen. Diese lassen die Bauernsleute oft erschöpft und mit fetzenhaften Erinnerungen zurück. Die gesammte Feier schrabt immer wieder knapp an belkehlelschem Treiben vorbei.

Mögliche Szenen des Einstieges:

\begin{itemize}
	\item Ankunft bei der Hexennacht, großes Aufheben um den Hexer und die fremde Geweihte
	\item Fenia wird ein ``Begrüßungstrunk'' gereicht, eine heftige Droge. Lin bemerkt das
	\item folgende Hexen sollten eingeführt werden: Luzelin (eigeborene Mutter der Hexen Weidens und eine der angesehensten in ganz Aventurien), Morena (ihre Tochter, eine Stadthexe), Gwynna (die weise, alterslose Beschützerin des Herzogtums), Savolina (eine wunderschöne, ehrgeizgie Jägerin), Glorana (die Eishexe), Achaz (eine alte Vetel, die von ihrer alten Jugend träumt)
	\item feiernde Hexen, die lachen, tanzen und Sex haben, aber auch ruhig beeinander sitzen, plaudern oder sich sogar streiten. Insgesamt scheinen sich viele lose Grüppchen zu bilden
	\item lokale Bauernjungen und -mädchen, die das Fest ausspionieren wollen, werden verhext
\end{itemize}


\subsection{Die Jagd}

Luzelin bittet Fenia um einen wichtigen Dienst: im Blautann wurde ein schwarzes Einhorn gesichtet, eine überaus seltene Kreatur. Im Vertrauen verrät die Hexe, dass sie dies vorausgesehen hat und das sie das Horn des magischen Tieres für ein mächtiges Ritual benötigt. Doch auch die Hexen Achaz und Savolina werden darauf aufmerksam und beschließen einen Wettstreit aus der Jagd nach dem Tier zu machen.

Die Jagd wird eine kurze erzählerische Spurensuche im Blautann sein. Das Einhorn ist auf magische oder weltliche Weise zu finden, die Charaktere müssen allerdings vorher herausfinden, wie man ein Einhorn überhaupt findet. Außerdem trägt Luzelin den Charakteren auf, das Einhorn nach Möglichkeit nicht zu töten, da dies die Magie des Alicorns schwächt. Trotzdem ist das wichtigste, das Horn zu erhalten.

Ein kranker Elf, der das Salamandra nicht mehr hört überfällt die Helden (Hinweis auf UG)

Szenen der Jagd

\begin{itemize}
	\item ein wahnsinniger Elf überfällt die Helden
\end{itemize}

Mögliche Wege an ein Horn zu kommen sind:
\begin{itemize}
	\item Spuren des Einhorns führen an einen Bachlauf
	\item ein alter Jäger kann den Helden Tipps geben, wo besonders scheues Wild sich aufhält
	\item magische Spuren, die Lin finden kann, geben Hinweise auf den Aufenthaltsort
	\item Legenden besagen, dass am Weideort eines Einhorns ewiger Frühling herrscht
	\item Das Einhorn zu erlegen, was schwierig ist, da es fast freizauberische Kräfte hat
	\item Ein abgelegtes Horn zu finden
	\item Das Einhorn davon zu überzeugen, dass es sein Horn aufgibt
\end{itemize}

\subsection{Das Ritual}
Atavar Friedenslied

\subsection{Notizen aus dem Spiel}
\begin{itemize}
	\item Lin erdolcht Achaz mit dem Horn des Einhorns, das ist darauf hin etwas wütend
	\item Lin übernimmt die Verantwortung für seine Tat, um Luzelins Ritual zu retten
	\item Fenia wurde von Atavar Friedenslied begutachtet und als Trägerin für gut befunden
	\item die beiden haben Morena vom Blautann auf Walpurgas Hochzeit ``eingeladen''
	\item 
\end{itemize}


