\section{Albtraum ohne Ende}

Albtraum ohne Ende beschreibt die Rückkehr des Geistes Borbarad in die dritte Sphäre. Um seinen Meister außerhalb des Todes zu rufen, entreißt Liscom von Fasar einem ganzen Landstrich die Zeit. Dafür verführt er Laniare, eine Geweihte der Tsa, und lässt sie ein uraltes Ritual durchführen. Gleichzeitig lässt er durch den Nekromanten Korobar Dorfbewohner entführen und diese in die Burg Dragenfeldt bringen, wo er das Kernstück des Rituals vorbereitet.

Thematisch handelt das Abenteuer vom widernatürlichen Versuch Liscoms, Borbarads Seele in die heutige Zeit zu reißen. Da Borbarad außerhalb des Zeitstroms geworfen wurde, muss ihm die Zeit ''zurückgegeben'' werden, die er verloren hat. Das Ritual ist ein gewaltsames Auflehnen gegen den normalen Lauf der Natur und der Zeit.

\paragraph{Anfängliche Motivationen}

\begin{itemize}
	\item Ardarian: Erhält eine Einladung von Walpurga von Weiden, die ihre Trauung feiern will, auf dem Weg überraschen ihn die Ereignisse in Baliho
	\item Dschafar: wurde durch einen mysteriösen Unfall nach Weiden geworfen
\end{itemize}

\subsection{Dunkle Schatten über Baliho - Einstieg}
Einige kurze Szenen in Baliho, die darin münden, dass sich Dschafar und Tobias treffen. Vor allem die scheiternde Anwerbung durch Delian, ein, zwei ''fish out of water'' Szenen für Dschafar.

\paragraph{Dschafar} trifft auf die Wahrsagerin aus Khunchom, die Albträumen nach Baliho gefolgt ist. Sie muss ihn vor der Zukunft warnen, hat aber nur fetzenhafte Prophezeiungen. Sie ist relativ verzweifelt, da sie das Schicksal der ganzen Welt auf Dschafars Schultern lasten sieht, aber ihm das nicht vermitteln kann.

\paragraph{Ardarian} wird auf Delian von Wiedbrücks Anwerbung aufmerksam, und Delian wird sich nur sehr schwer überzeugen lassen, ihn mit zu nehmen. Sollte er es dennoch schaffen (z.B. mit inem sehr guten Überreden-Wurf, werden die Anwerbungen durch die Kirche und Delian am nächsten morgen konkurieren).

Sollte er abgelehnt werden, wird Mutter Linai ihn aufsammeln und ihn zum Trösten in die Schenke einladen, ansonsten spricht Delian diese Einladung aus.

\paragraph{Im Gasthaus} tanzt die Sharizad, Mutter Linai hat eine Vision und alles wird chaotisch. 

\paragraph{Mutter Linai} will gehen, die Helden müssen sie davon überzeugen, mitzugehen, oder an ihrer Statten gehen zu dürfen. 

\subsection{Der Weg nach Dragenfeldt - Rising Action}

\paragraph{Die Praioten von Anderath} wollen von der Gruppe wissen, was vor sich geht.

\paragraph{Ein kleiner Junge} ''bedroht'' die Helden auf dem Weg nach Dragenfeldt. Die Szene ist eigentlich sehr einfach zu umgehen, wirft aber ein Licht auf die Gruppe und ihr Sozialverhalten. Im Bauernhaus des Jungen treffen die beiden Reisenden auf Ayla Eiridias von Schattengrund, die dort Untershlupf vor einem Gewitter gesucht hat.

\paragraph{Aylas Traum}

\paragraph{Traum vom Gott} 










\paragraph{In Braunklamm} blockiert ein Toter der Formidablen Sechs das Mühlrad. Die Helden können dabei helfen, die Mühle wieder in Gang zu bringen und dabei die Identität des Toten erfahren. Der Tote hat einen Nagel mit der Aroqa Rune im Kopf.

Ilme ist die Stadtvordere, Müllerin und bezahlt den Helden einige Silberlinge. Dschafar steckt den Nagel ein. Die Stadtvordere erzählt mehr über Brückenschäden, Runhag, die Gerüchte um Korobar und Delians Söldnertrupp.

\paragraph{Traum vom Zweikampf} Ardarian blutet fast aus, Dschafar rettet ihn. Ardarian hält Dschafar für den Widersacher.

\paragraph{Runhag und der Kampfplatz} Dschafar und Ardarian verbringen viel Zeit mit diskutieren. Sie entdecken den Kampfplatz nicht. Der Baron der Sichelwacht empfängt die Helden kurz vor Sonnenuntergang. Er hat keine Praiospriester gesehen. In Runhag gibt es noch deutlich mehr Alpträume.










\paragraph{Traum von der Leere}

\paragraph{Ein Räuberbaron} fordert tatsächlich Wegzoll von der Gruppe. Der Räuberbaron will vor allem Dschafar festsetzen, da dieser ihn an ''diesen hinterhältigen Magus'' erinnert, der am Sichelsteig sein Unwesen treibt. (Gestrichen)


\paragraph{Druiden in einem kleinen Dorf} erforschen die schleichenden Veränderungen entlang einer Ader Sumus. Um die Menhire der Druiden fängt das Graß an zu sterben, und kleinere Lebewesen verenden. Die Kraftlinie leidet hier unter dem Ritual Liscoms und liegt im Sterben.

Dschafar überzeugt die Druiden und eine Hexe zu helfen.

\paragraph{Traum vom Tod}

Delians Lager finden die Helden und dabei das KGIA Logbuch. In der Nacht werden sie von Korobar und Untoten überfallen. Dschafar erbeutet ein Fesselfeld-Amulett (1 Mal verwendbar, wiederaufladbar).

\paragraph{Korobar, der Schrecken der Tobimora} überfällt die Helden kurz vor der Wüstenei. Er hatte Delian von Wiedbrück verfolgt, der in der Wüstenei fliehen konnte. Beim nachtlager entkommt er, doch seine Untoten sind alle vernichtet.

Finden Delian bei der Verfolgung Korobars hinter dem Fluss.

Die Hexe heißt Arla.

\paragraph{Traum vom Fallen}

\subsection{Die Wüstenei - Klimax}

\paragraph{In der Wüstenei} zeichnet sich schnell ab, das mehr als nur ein einsamer Schwarzmagus die hiesig Gegend bedroht. Die Helden können beobachten, wie die Umwelt innerhalb kürzester Zeit um sie herum zu verfallen beginnt, bis in Dragenfeldt nur noch Staub herumliegt.

\paragraph{Die Bewohner Dragenfeldts} begegnen den Helden einige Wegstunden außerhalb ihres Dorfes. Sie suchen nach Schutz und wollen vor dem Untergang des Ortes fliehen. Sie werden die Helden anbetteln ihnen zu helfen.

\paragraph{Traum vom Hass}

\paragraph{In Dragenfeldt} finden die Helden nur noch verfallene Häuser, die aussehen, als wären sie vor Jahrhunderten verlassen worden. Nur der Tsatempel steht in der Mitte allen Übels und bleibt dank des Schutzes der jungen Göttin völlig unberührt von allen Schrecken.

Im Tsatempel finden die Helden einen Großteil der Informationen, die sie benötigen um das Geschehene zu verstehen. Unter anderem ist dort Laniares Tagebuch und das Liber Zhamoricam zu finden.

\paragraph{Burg Dragentodt} thront über dem Dorf und ist offensichtlich der ursprung allen Übels. Die Helden müssen die Burg stürmen, doch zuerst stellt sich ihnen noch einmal Korobar mit den Überresten der formidablen Sechs entgegen.

\paragraph{Das Ritual jenseits der Zeit}

\subsection{Epilog - Dschafar}

Das erste Erscheinen des Auges

\subsection{Epilog - Tobi}

Hier heiraten Walpurga und Dietrad

\subsection{Scheitern}
Das Auge geht an Delian, oder der Magiermogul übernimmt Dschafars Körper.

\subsection{Kampfwerte}

\begin{multicols}{3}
\end{multicols}

