Die Beschreibungen sind aus der Wiki Aventurica entnommen und mit weiteren Kampagnen-relevanten Informationen angereichert.

\section{Magiebegabte}
\subsection{Rohal}
Der berühmte Rohal der Weise, einer der größten Magier aller Zeiten, beendete als Anführer eines Volksaufstands die grausame Zeit der Priesterkaiser und verbannte Gurvan Praiobur II. nach Jilaskan. Als Reichsbehüter brachte er dem Reich in seiner Regierungszeit von 466 BF bis 589 BF, auch als Rohalszeit bezeichnet, Frieden und Stabilität.

Niobara von Anchopal war seine Beraterin in astrologischen Fragen. Nach seiner Amtsniederlegung aufgrund von zunehmendem Misstrauen der Bevölkerung wegen seiner langen Amtszeit zog er mit einem Heer von Freiwilligen in die Gorische Wüste, um Borbarad zu bekämpfen.

Rohal war die letzte Inkarnation des Alveraniars des verborgenen Wissens.

Errungenschaften

Die Ius Concordia, der Codex Pax Aventuriana oder der Codex Albyricus sind einige seiner wichtigen Gesetzeswerke.
Er hob das Verbot des Rur-Gror-Glaubens auf Alabastrien auf und setzte Fürst Curfan von Sinoda ab.
Er verbot Hexenverbrennungen.
Der Weise soll alle Exemplare des Naranda Ulthagi in Sicherheit gebracht haben.
Vinsalt trägt Rohal im Wappen, da er den Bau der Stadt erlaubte.
Die Offenbarung des Nayrakis, die Sphärentheorie, Systemata Magia und die Rohalsche Astralschwingungstheorie werden ihm zugeschrieben.

\subsection{Prishya Garlischgrötz}
Prishya Garlischgrötz lernte an der Akademie der Erscheinungen und war lange Zeit Spektabilität der Akademie der hohen Magie zu Punin und Convocata prima der Großen Grauen Gilde des Geistes. Zu ihrer Zeit galt sie als strenge, niemals lachende Perfektionistin, die einem Wehrheimer Weibel alle Ehre gemacht hätte. 

\subsection{Saldor Foslarin}
Saldor Foslarin (eigentlich Saldor Sohn des Sablon aus der Sippe Foslarin) ist der Convocatus primus der Weißen Gilde. Er ist zudem die Spektabilität der Akademie Schwert und Stab zu Beilunk und gilt als sehr konservativ und jähzornig. Saldor Foslarin ist ein brillanter Kampfmagier und einer der wenigen zwergischen Magier. Sein Zwillingsbruder wurde von dem Drachen Naugadar getötet. 

\subsection{Salpikon Savertin}
Salpikon Savertin, Spektabilität der Schule der variablen Form zu Mirham, hat es mit diplomatischem Geschick und scharfem Geist geschafft, die individualistischen Magier des linken Weges zu bündeln und zu führen. Er ist stets als Botschafter der Gilde unterwegs und versucht das Bild des bösen Schwarzmagiers etwas grau einzufärben.

Salpikon arbeitet daran, die Schwarze Gilde zu einen. Er ist bekennender Feind des Borbaradianismus und hat aus dieser Motivation heraus den Bund der Schatten gegründet. 

\subsection{Tarlisin von Borbra}
Tarlisin von Borbra ist der Hochmeister der Grauen Stäbe. Er lebte bis zu seinem fünften Tsatag in Brabak, wo ihn Demelioë Nandoniella Terbysios unterrichtete. Nach seinem Studium in Andergast wandte er sich der Dämonologie und dem Borbaradianismus zu. Durch ein Tsa-Wunder wurde er von seinem Pakt befreit und hat sich seitdem der Bekämpfung der Borbaradianer verschrieben und ließ in Borbra einen Tsa-Tempel errichten. Tarlisin verfügt über keine Astralenergie und keinen Schatten mehr.

Er ist Finder und Träger des Ordenschwertes Famerlîn und Verfasser des Tractatus contra Daemones. 

\subsection{Khadil Okharim}
Khadil Okharim al'Sheik Tabilithash ibn Tarsaf Okharim al'Kunvuqatush ist Vorsteher der Khunchomer Magierakademie und ist einer der fähigsten Artefaktmagier Aventuriens. Der äußerst korpulente Lebemann ist aufgrund seiner freundlichen Art bei seinen Schülern sehr beliebt.

Da er Feqz noch vor Hesinde oder Mada als Gott der Magie ansieht, ist er innerhalb der Grauen Gilde nicht unumstritten, allerdings sorgt seine kaufmännische Ader auch dafür, dass sich seine Akademie keine Finanzierungssorgen zu machen braucht. 

\subsection{Rakorium Muntagonus}
Rakorium Muntagonus ist ein brillanter, aber leider mittlerweile dem Verfolgungswahn und Wahnsinn verfallener, Meister der Magica Transformatorica. Er ist nach langjähriger Beschäftigung mit echsischen Umtrieben – unter anderem leitete er mehrere Expeditionen nach Maraskan, wo er Brutstätten der Skrechu aufspürte – und aufgrund seiner Tätigkeit als Hüter des Codex Sauris inzwischen der Auffassung, hinter (fast) allem und jedem stecke eine echsische Verschwörung. Sie soll unter anderem das Ziel haben, Zze Tha wieder auferstehen zu lassen. 

\subsection{Dschelef ibn Jassafer}
Der Erzmagier Dschelef ibn Jassafer ist ein begnadeter Elementarist, Kenner der Saurologie und Experte für Bastrabuns Bann. Der Preisträger des Schwarzen Auges von Khunchom war lange Jahre Leiter der Pentagramm-Akademie zu Rashdul, bis ihn seine eigene Tochter Belizeth Dschelefsunni im traditionellen magischen Duell besiegte und vertrieb. 

\subsection{Luzelin vom Blautann}
Luzelin Silberhaar war das Oberhaupt des Blautannzirkels. Sie bewohnte zusammen mit ihrem Vertrauten Pallikratz am Finsterbach im Blautann in Weiden eine Grotte.

Die mächtige, nicht eigeborene Hexe war eine begabte Alchimistin, Artefaktmagierin und Herrschaftszauberin. Dreißig Jahre lang hatte sie Vorahnungen von Borbarads Rückkehr, weswegen sie über Jahrzehnte Zutaten sammelte, um das zweite Zeichen zu schaffen. Außerdem schenkte sie Raidri Conchobair das berühmte pailische Axtgehänge.

Von ihren Kolleginnen Glorana und Yolana wurde Luzelin gehasst. 

\subsection{Morena vom Blautann}
Die Viertelnivesin Morena vom Blautann besitzt die für Hexen so typischen roten Haare und gleichzeitig die schräg gestellten grünen Augen ihrer nivesischen Verwandten. Sie gehört der Schwesternschaft der Schönen der Nacht an und ist die Tochter Luzelin Silberhaars. Sie weiß nicht, dass die albernische Hexe Reisa Armehra ihre Schwester ist. Morena nahm an der Lichtvogel-Expedition und der Schlüssel-Expedition teil. 

\subsection{Kailäkinnen}
Kailäkinnen ist ein gebeugter alter Mann mit runzligem Gesicht. Seine Ziele als Schamane sind die Aufrechterhaltung der nivesischen Tradition, die Aussöhnung mit den Himmelswölfen und Friede unter den Nivesenstämmen.
Er warnte die Menschen vor der Rückkehr Borbarads, da er frühzeitig Visionen hatte. Jedoch wurde ihm vorher nur wenig Gehör geschenkt.

1039 BF ließ er bestimmten Leuten die Himmelslichter zukommen (welche seine Helfer bereits 1037 BF fanden), da er in seinen Visionen bereits von der kommenden Zeitenwende und die Bedrohung durch die die Nai Ashyyr, die neueste Generation der Shakagra erfahren hat. Dies tut er in der Hoffnung, dass die Träger der Himmelslichter zusammenfinden, wenn die Zeit gekommen ist. 

\subsection{Yolana von Ysilia}
Leibmagierin der Herzogensöhne von Tobrien.

\subsection{Krallerwatsch}
Krallerwatsch Sohn des Krallulatsch, der älteste Schamane vom Trollstamm der Tolpatatsch, nahm an der Schlüssel-Expedition teil. 

\section{Geweihte}

\subsection{Die Hüterin der Stelen}
Die Hüterin der Stelen ist eine uralte Firungeweihte, die als Prophetin die Eisigen Stelen beschützt. Sie gilt als Expertin für die Interpretation der Bilder auf den Stelen.

\subsection{Amando Laconda da Vanya}
Amando Laconda da Vanya ist ein Praios-Geweihter und derzeit amtierender Großinquisitor. Er war früher Praetor des Ragather Sonnentempels und der Puniner Gilbornshalle. Während der Zweiten Kirchenspaltung stand er auf der Seite des amtierenden Boten des Lichts Jariel Praiotin XII.. Er spielte während der Borbaradkrise eine herausragende Rolle und war Verbündeter der Sieben Gezeichneten. Auch aufgrund der Visionen von der Vierten Säule der Praios-Kirche während seiner Gefangenschaft spielte er eine wichtige Rolle während der Quanionsqueste. 

\subsection{Ucurian Jago}
Ucurian Jago, geborener Jago Rallersfurter, war von 1001 BF bis 1026 BF Hochmeister der Bannstrahler. Er war als starrsinniger Dogmatiker bekannt, aber auch als unerbittlicher Streiter gegen die dämonischen Invasoren während und nach der Borbarad-Krise.

Während des Schismas der Praios-Kirche zwischen 1013 und 1018 BF stand er auf der Seite des amtierenden Boten des Lichts, untersuchte die Weidener Wüstenei, Borbarads Umtriebe im Kloster Arras de Mott und vertrieb Borbarad 1019 BF, der die Gestalt des Delian von Wiedbrück angenommen hatte, aus dem Tuzaker Fürstenpalast, kämpfte später in der Schlacht an der Trollpforte und fiel schließlich 1026 BF bei den Kämpfen um Beilunk dem Schwarzen Drachen Rhazzazor zum Opfer. 

\subsection{Ayla von Schattengrund}
Ayla von Schattengrund stammte aus einer alten Kämpfertradition, hatte thorwalsche und tulamidische Vorfahren und zog einige Jahre als Kämpferin durch die Lande, ehe sie der Ruf der Göttin erreichte und sie sich in Arivor (historisch) den Ardariten anschloss und zur Dienerin Rondras geweiht wurde.

Viburn von Hengisfort, das alte Schwert der Schwerter, nannte 1014 BF am Sterbebett ihren Namen als Nachfolgerin, Dragosch von Sichelhofen verfälschte aber die Verkündigung und rief sich selbst zum höchsten Rondra-Geweihten aus und ließ Ayla in der Folge nach dem Karfunkel Pyrdacors suchen bis im Winter 1016 BF auf Burg Rhodenstein die Wahrheit ans Licht kam und Dragosch von Sichelhofen von Ayla im Duell tödlich verwundet wurde, die daraufhin ihren Platz als neues Schwert der Schwerter einnahm.

Ayla von Schattengrund nahm eine zentrale Rolle während des Borbaradkrieges und den nachfolgenden Rückeroberungsschlachten ein, nicht nur weil sie an vielen Schlachten teilnahm, sondern weil sie auch an divesen begleitenden Aspekten wie bspw. die Widerentstehung des Schwertes Siebenstreich wesentlich beteiligt war.

Ayla von Schattengrund war aber auch eine hervorragende Diplomatin und verstand es immer wieder die Rondra-Kirche aus vielen drohenden oder tatsächlichen bewaffneten Konflikten herauszuhalten, in dem sie ihren Geweihten eine Teilnahme auf der einen oder anderen Seite verbat - und trug auch die persönliche Konsequenz daraus, indem sie sich selbst von ihrer Baronie Schattengrund trennte und sich entlehnen ließ.

\subsection{Alter weiser Mentor FTW}
für Tobi meine geile\textsuperscript{TM} Rondrageweihten Idee.

\section{Adlige}
\subsection{Das Haus Gareth}
\subsubsection{Brin von Gareth}
Brin von Gareth war Hals Erbe und führte das Mittelreich als Reichsbehüter und aus Pietät seinem offiziellen Vater gegenüber nicht als Kaiser. Er kämpfte im Dritten Orkensturm gegen die Schwarzpelze, gegen den Usurpator Answin von Rabenmund und gegen Borbarad. 

\subsubsection{Emer von Gareth}
Emer ni Bennain war die Tochter König Cuanus von Albernia. Sie wurde im Alter von 13 Jahren mit Prinz Brin verlobt und mit 19 mit ihm verheiratet. Sie war die erste Prinzgemahlin seit langem, die aus dem Mittelreich stammte und insbesondere im Volk, aber auch im Adel sehr beliebt.

Nach dem Tode Brins am Vorabend der Schlacht an der Trollpforte führte Emer das Neue Reich als Regentin und bereitete es auf die Herrschaft ihrer Tochter Rohaja vor. In dieser Zeit hielten viele Mächtige um ihre Hand an, doch sie heiratete nicht erneut.

Am Tage der Schlacht in den Wolken wurde Emer vom untoten Kaiserdrachen Rhazzazor entführt, an die Goldene Pyramide in Warunk gekettet und mit der dämonischen Gabe des Großen Alptraumauges belegt. So musste sie immerzu verzerrte Horrorvisionen der Zukunft ihrer Kinder, des Reichs und des Kontinents ertragen, ohne die Augen davor verschließen zu können. Manchmal wurde sie von Mitgliedern des Nekromantenrats befragt.

In Folge der Ereignisse um den Donnersturm 1031 BF konnte sie erlöst werden und starb. 

\subsection{Das Haus Löwenstein}
\subsubsection{Walpurga von Löwenhaupt}
Walpurga gilt – ganz im Erbe ihres Vaters – als aufrechte und starke Kriegerin. Ihre Verkündigung der erschreckenden Nachricht von Borbarads Rückkehr brachte ihr die zweifelhafte Ehre der stehenden Redewendung "Walpurgensbotschaft" ein. Erst ihre zweite Ehe mit Herzog Bernfried von Tobrien gab ihr neuen Lebensmut. 

\subsubsection{Waldemar von Löwenhaupt}
Waldemar von Löwenhaupt bzw. Waldemar von Weiden, auch bekannt als Waldemar „der Bär“, gilt als einer der größten Recken der Rondrakirche und tapferer Streiter für das Mittelreich. An den Kriegerschulen von Trallop und Baliho erhielt er seine Ausbildung. Bekannt wurde er als gerechter, aber harter Herzog von Weiden, dem höfisches Gehabe immer zuwider war. Seinen Zweihänder „Windsturm“ legte er kaum einmal außer Reichweite. Er war einer der wenigen, die die Ochsenherde meisterlich beherrschten.

Waldemar fiel in der Schlacht auf den Vallusanischen Weiden im Kampf gegen Karmoth. Bald danach gilt er als minderer Heiliger der Rondra mit einer eigenen Säule in Arivor. 

\subsection{Das Haus Ehrenstein}
\subsubsection{Dietrand von Ehrenstein}
Dietrad von Ehrenstein war der jüngere Bruder von Bernfried von Ehrenstein. Er war aber nie zum Krieger bestimmt wie sein Bruder und widmete sich den schönen Künsten. Seine Sangeskunst war sehr bekannt. Verheiratet war er mit Walpurga von Weiden, mit der er die Kinder Arlan und Walbirg bekam. Durch seine adlige Abstammung wurde er zum Befehlshaber im Kampf um Eslamsbrück. Dort starb er im Kampf am 30. Rondra 1020 BF. 

\subsubsection{Raidri Conchobair}
Raidri Conchobair, der Schwertkönig, war Träger der Zwillingschwerter Antworter und Vergelter sowie des legendären Schwertes Siebenstreich. Während des Tuzakaufstands bezwang er die Blutzwillinge, wofür er den Jergan-Orden bekam. Er war einer der besten Schwertkämpfer Aventuriens, fiel in der Dritten Dämonenschlacht und ist mittlerweile Held der Rondra-Kirche. Er war ein Freund Cuanu ui Bennains.

Der ursprüngliche Blason seines Wappens war Gold auf Schwarz, nach seiner Ernennung als Markgraf von Winhall änderte sich dies zu Weiß auf Rot, was dem alten Wappen der Grafen von Winhall entspricht. 

\subsubsection{Bernfried von Ehrenstein}
Bernfried von Ehrenstein j. H., Herzog des freien und zwölfgöttlichen Tobrien, hat in seinem Leben viel Leid ertragen müssen. Er verlor Eltern, erste Gemahlin Efferdane von Eberstamm-Mersingen, seinen jüngeren Bruder und einen Großteil seines Landes in der Invasion Borbarads. Die zweite Ehe mit Herzogin Walpurga von Weiden gab ihm neuen Halt. 

\subsection{Der Familie von Wallenstein}

\subsection{Die Grafen von Ask}
\subsubsection{Irinja von Ask}
\subsubsection{Tirulf von Ask}
\subsubsection{Wahnfried von Ask}

\subsection{Strutzz}
Strutzz von Trollnase ist ein Troll, der als einziger seiner Art jemals in der Geschichte des Reiches mit einem Freiherrentitel bedacht wurde. Er gilt innerhalb seiner Spezies als großer Menschenkenner, und gebietet über die Baronie Trollnase in den Trollzacken.

Strutzz spielte angeblich im Zug der Tausend Oger im Jahre 1003 BF eine zentrale Rolle, als mit seiner Hilfe einige Trolle für den Kampf gegen die Oger begeistert werden konnten. Er überzeugte mit Hilfe der Sieben Gezeichneten im Jahre 1021 BF die Trolle im Kampf gegen Borbarad in der dritten Dämonenschlacht einzugreifen. 

\subsection{Delian Wiedbrück}
Delian von Wiedbrück fand relativ früh zur KGIA, wo er persönlicher Schützling Dexter Nemrods wurde. Zuerst in der Maraskanabteilung tätig, wechselte er bald zur Magieabteilung und übernahm deren Leitung. Aufgrund seiner Herkunft und seiner untersetzten Figur wurde er in KGIA-Kreisen oft als „Darpatbulle“ bezeichnet. 

\section{Einfaches Volk}
\subsection{Sefira von Rashdul}
Sefira von Rashdul ist eine Wahrsagerin der Tulamidenlande, die tatsächlich über magische Kräfte verfügt, sich dessen aber nicht bewusst ist. Sie deutet die Zukunft aus dem Legen der Inrahkarten und wird von einer halbblinde Krähe begleitet. In Khunchom war sie eine Zeit lang die Gefährtin Elizaveta Walsjakow

\subsection{Elizaveta Walsjakow}
Lins Mutter

\subsection{Familie da Merinal}
Die da Merinals sind eine adlige Artistenfamilie aus dem Lieblichen Feld.

Abbadi da Merinal (entführt in der Khom), Colon da Merinal (enführt während der Suche nach Bastrabun Bann)

\section{Übernatürliche und Heilige}
\subsection{Teclador/Bukhar}
Teclador, der angeblich jüngste der sechs Alten Drachen, wurde von Nandus geschaffen. Er wacht über das Gleichgewicht der Kräfte zwischen Göttern, Geisterwesen, Menschen, Tieren und Dere. Satinav soll ihm die Gabe der Prophezeiung verliehen haben, weswegen er bereits vor schwerwiegenden Ereignissen einzugreifen imstande wäre. Zum Gefolge Tecladors gehören Westwinddrachen.

Der Al'Anfaner Ritus der Boron-Kirche setzt Teclador mit einem Bewusstseinssplitter Bishdariels gleich. 

\subsection{Der heilige Jarlak}
Jarlak der Waidmann auch Jarlak von Ehrenstein war einst Baron von Ehrenstein, bis er bei einer firungefällen Jagd auf eine Horde Goblins traf, sie besiegte und von Firun die Alhanier-Klinge Schalljarß erhielt. Als dies als Erfüllung einer schicksalträchtigen Prophezeiung erkannt wurde, ernannte Rohal der Weise ihn zum Herzog von Tobimorien und beendete die Herrschaft der Herzogs-Protektoren in Warunk.

Inzwischen wird er in Tobrien als Heiliger des Wintergottes verehrt.

\section{Feinde}
\subsection{Borbarad}
Borbarad (Tulamidya: Bel'Ghor, Ur-Tulamidya: Bor Barrad, Maraskani: Dharzjinion) war der 1015 BF durch einen Zeitfrevel des Liscom von Fasar zurückgekehrte Tharsonius von Bethana. Zuerst noch körperlos gelang es ihm 1016 BF sich mit Hilfe von Pardona vollständig zu inkarnieren. Ab diesem Zeitpunkt setzte er in ganz Aventurien Ereignisse in Gang, die schlussendlich in den Borbaradkrieg und seine Verbannung in den Rausch der Ewigkeit führten. Dabei trat er auch unter diversen Tarnidentitäten auf, unter anderem als der Selemit Nammastar oder der Abenteurer Borotin Almachios oder übernahm gar die Identitäten anderer Personen.

Borbarad besaß zahlreiche legendäre Artefakte wie die Dämonenkrone, den fliegenden Schwarzen Wagen und den Sphärenschlüssel, der aus der Zeit stammen soll, als die Zitadelle der Kraft zerstört wurde. Sein Zauberstab, aus der Blutulme Seelenstamm gefertigt, trug den Namen Sharralmor. Seit seiner Wiederkehr ist er ferner im Besitz der Porträts der Mächtigen gewesen.

Borbarad war die letzte Inkarnation des Alveraniars des verbotenen Wissens.

Der Alveraniar des verbotenen Wissens – der als ein Sohn des Nandus gilt – entstand zusammen mit dem Alveraniar des verborgenen Wissens aus der Teilung des Alveraniars des Wissens im Fünften Zeitalter (in Folge der Spaltung Nandus' in Amazeran und Hesinde). Er folgte Amazeran nicht bei dessen Fall im Sechsten Zeitalter in die Niederhöllen

Der Alveriar des verbotenen Wissens wandelte ebenso wie sein Zwillingsbruder, der Alveraniar des verborgenen Wissens, in zahlreichen Inkarnationen auf Dere, wobei er stets im Kampf mit seinem Bruder lag. Nachdem einer von ihnen unterlag, verschwand bald darauf auch der andere. Der Alveraniar des verbotenen Wissens hinterließ dabei mehrere sogenannte Schwarze Festen (unter anderem in der Gor).

In seiner letzten Inkarnation wurde der Alveraniar des verbotenen Wissens während der Dritten Dämonenschlacht in den Rausch der Ewigkeit verbannt und kreist seitdem als Borbarad-Stern nahe dem Stern Satinav am Südhimmel. 

\subsection{Borbaradianische Verschwörung}
\subsubsection{Liscom von Fasar}
Liscom von Fasar, ursprünglich Liscom Ghosipar, war Absolvent, Lehrmeister und Spectabilitas Minor der Akademie der Geistigen Kraft zu Fasar. Schon früh wurde er durch seinen Lehrmeister, den sogenannten Bettelmönch, zum überzeugten Borbaradianer. 

\subsubsection{Der Bettelmönch}
Bei der als Wanderprediger bekannt gewordene Person unter dem Namen Bettelmönch handelt es sich um einen führenden Borbaradianer. Er war der Lehrmeister von Liscom von Fasar und leitet den Borbaradianerzirkel Salân al'ashtranim, der in einer schwer einzunehmenden Abtei im Khoramgebirge residiert. 

\subsubsection{Korobar}
Korobar der Norbarde lernte in jungen Jahren von einem privaten Lehrmeister die Kunst der schwarzen Magie. Nachdem er seinen Mentor erschlagen hatte, hinterließ er in Tobrien eine Spur nekromantischer Gräueltaten und erhielt den Beinamen „Schrecken der Tobimora“.

Jahre nach seinem Tod erhob er sich aufgrund der stärker werdenden Thargunitoth-Einflüsse aus Schwarztobrien als Untoter, wurde zuerst Teil des Endlosen Heerwurms und herrschte bis zu seiner Vernichtung durch die Golgariten über die Stadt Altzoll in der Warunkei. 

\subsubsection{Urdo von Gisholm}
Urdo von Gisholm war der illegitime Sohn eines darpatischen Adligen und bestand immer auf sein Erbe, weshalb er bereits in jungen Jahren seinen Halbbruder ermordete. Er war ein Borbaradianer, der sich vor allem auf das Jagen und Kundschaften verstand, zu seinen Meistern zählten Liscom von Fasar und der Bettelmönch. Er schloss später einen Pakt mit Lolgramoth. 

\subsubsection{Azaril Scharlachkraut}
Azaril Scharlachkraut, früher Vertraute von Rohezal vom Amboss, ist heute Anhängerin Borbarads. Sie wirbt Mitglieder für ihren Meister und leitet auch einen Borbaradianerzirkel. 

ie ist unter verschiedenen Decknamen bekannt und gilt als glühende Verfechterin eines philosophischen Borbaradianismus.

\subsubsection{Sulman al Venish}
Sulman al'Venish war ein Nekromant und Schüler von Liscom von Fasar. In seinen Körper fuhr bei der Schlacht um Kurkum der Nirraven, der so eine Verankerung in Aventurien gewann. Seit dieser Zeit wird der in Lumpen gewandete, auf einer Skelettmähre reitende oder einem untoten Perldrachen fliegende Sendbote Thargunitoths auch Seelensammler genannt. 

\subsubsection{Rayo Brabaker}
Der Meisterdieb und Rayo Brabaker wurde im Kloster unseres Herrn Borbarad von Gorien zum Borbaradianer und Tasfarelel-Paktierer. Borbarad erschien ihm persönlich und beauftragte ihn mit dem Diebstahl von Tausenden von Dukaten und wurde zu seinem Schatzmeister und einer der führenden Borbaradianer auf Maraskan. Nach Borbarads Tod wurde er zum Komtur von Tuzak und strebte als solcher wohl das Kommando über die größte Piratenflotte Aventuriens an. Nach dem Verlust seiner Herrschaft über Tuzak durch den Zwangsarbeiteraufstand 1032 BF gilt er als tot, hält aber tatsächlich als Schmuggler Grinjian weiterhin wichtige Fäden der Macht über Maraskan in den Händen. 

\subsection{Heptarchen und Heerführer}
\subsubsection{Lutisana von Perricum}
Lutisana von Perricum schloss sich in der Answinkrise dem Verräter an und wurde nach dessen Gefangennahme in Rulad eingesperrt. Dort entkam sie, als die Borbardianer die Gefängnisinsel überfielen. Sie schloss sich Borbarad an und wurde von ihm zur Heerführerin ernannt. 

\subsubsection{Arngrimm von Ehrenstein}
Der Dunkle Herzog Arngrimm von Ehrenstein ist ein werwölfisches Mitglied der tobrischen Herzogsfamilie, der erst für Borbarad, dann für den Dämonenkaiser Galotta als Herzog Tobrien regierte. Nach dessen Tod in der Schlacht in den Wolken ist Arngrimm neben Leonardo von Havena und Balphemor von Punin Mitglied des regierenden Triumvirates von Transysilien. 

\subsubsection{Galotta}
Der Beherrschungs-Magier Gaius Cordovan Eslam Galotta war Abgänger und Spektabilität der Akademie der Herrschaft in Elenvina und wurde später unter Kaiser Reto erster Hofmagier und blieb dieses auch nach dessen Tod unter Kaiser Hal.

Die Feindschaft des als brillanten Beherrscher, Beschwörer, Artefaktzauberer und Arkanologe gefeierten und unter anderem mit dem Schwarzes Auge von Khunchom ausgezeichneten Galotta mit Nahema führte 1002 BF zum Scharlachkappentanz und seiner Verbannung vom Hof, woraufhin er den Ogerzug auslöste und nach dessen Niederschlagung als Staatsfeind Nr.1 galt.

Galotta setzte sich erfolgreich nach Brabak und Al'Anfa ab, unterrichtete und forschte dort lange Zeit teilweise auch mit zweifelhaftem Erfolg und wurde nach Borbarads Rückkehr dessen wichtigster Anhänger – unter anderem, indem er die dämonische Feuerchimäre Amargant schuf, die Altaïa zerstörte, einen erneuten Versuch einer Ogerbeschwörung unternahm, aber auch durch seine Erfolge im Bereich der Antimagie und der Beherrschung.

Der Zwischenzeitlich mit Blakharaz paktierende Galotta rief nach der Dritten Dämonenschlacht das Dämonenkaiserreich Transysilien aus, ernannte sich zum Dämonenkaiser und griff schließlich im Jahr des Feuers zusammen mit Rhazzazor das Mittelreich mit der Fliegenden Festung und dem Endlosen Heerwurm an, wobei er in der Schlacht in den Wolken ums Leben kam. Einigen seiner letzten Worte kann man entnehmen, dass eine Hauptmotivation für all seine Taten das unerwiderte Begehren zu Nahema ai Tamerlein war. 

\subsubsection{Xeraan}
Xeraan war ein brillanter und skrupelloser Illusionsmagier mit enormer Goldgier. Seinen markanten Buckel verdankte er einer missglückten Beschwörung seines ersten Lehrmeisters Precolto.
Nachdem er von der Illusions-Akademie geworfen wurde, verübte er zahlreiche Verbrechen, unter anderem einen Raub im Hesindetempel von Elburum. Weiterhin versuchte er den Hort des Riesenlindwurms Ykkandil im Regengebirge zu plündern. Nach Borbarads Rückkehr 1016 BF schloss er sich diesem bereitwillig an und wurde von ihm in die Dämonenbeschwörung eingewiesen, woraufhin er die Unbesiegbare Legion von Yaq-Monnith schuf.

Xeraan war ein gerissener Feilscher, der Pakte mit mehreren Erzdämonen - unter anderem Belhalhar und Tasfarelel - einging, sich aus einigen aber auch wieder löste. Nach dem Zerbersten der Dämonenkrone errang er den Charyptoroth-Splitter, gründete das Piratenreich Xeraanien und die Borbaradkirche und erklärte sich zum Portifex Maximus. In der Unermesslichen Schatzkammer in Mendena hortete er seine immensen Schätze. 

\subsubsection{Rhazzazzor}
Rhazzazor war ein schwarzer, 20 Schritt langer, zu großen Teilen skelettierter Drache und Herrscher der Warunkei. Er ist auch als Schwarzer Drache, Xyxyx und (Schwarzer) Kurungur bekannt oder in früheren Zeiten bekannt gewesen. Sein Auftreten kann sogar ins Diamantene Sultanat zurück verfolgt werden.

Rhazzazor war ein Paktierer im Sechsten Kreis der Verdammnis und Träger des Thargunitoth-Splitters sowie des Untragbaren Halsbands der Thargunitoth und stets von untoten Krähen umgeben.

Rhazzazors Karfunkel befindet sich derzeit in den Gewölben unter Okdrâgosch. Da die Seele und Astralmacht eines Drachen im Karfunkel sitzen, ist Rhazzazors Seele vor dem direkten Zugriff Thargunitoths geschützt – ein Umstand, der ihn trotz seines Paktes die Jahrtausende überdauern ließ. 

\subsubsection{Helme Haffax}
Helme Haffax galt als genialster Heerführer und Stratege Aventuriens. Der aus einfachen Verhältnissen stammende Haffax leistete seine Knappenzeit am darpatischen Fürstenhof, bevor er eine äußerst erfolgreiche militärische Laufbahn einschlug. Er diente unter den mittelreichischen Herrschern Reto, Hal und Brin und wurde schließlich Reichserzmarschall und später Fürst-Marschall von Maraskan.

Als Borbarad zurückkehrte, verriet er das Mittelreich und wurde zum Befehlshaber der Armee des Dämonenmeisters. Durch einen Pakt mit Asfaloth konnte er den Alterungsprozess aufhalten. Nach der Dritten Dämonenschlacht wurde er einer der Heptarchen und herrschte über die Fürstkomturei, zu der seit 1029 BF auch die Piratenküste gehörte.

Er fingierte seinen Tod im Erntemond 1037 BF auf Burg Talbruck, führte dann aber 1039 BF einen Einfall seines Heeres in Perricum und Gareth an. Nachdem er den Belhalhar-Splitter in Perricum der Rondra-Kirche übergab und seinen Asfaloth-Pakt brach, tauschte er seine Seele mit dem Drachen Malgorrzata um in dessen Karfunkel zu schlafen.

\subsubsection{Dimiona von Zorgan}
Dimiona von Zorgan war das zweite Kind Sybia al'Nababs. Sie schloss sich Borbarad an und gründete als Dimiona von Oron das Moghulat Oron. Kurz bevor ihr Körper am Ende des 35-Tage-Krieges getötet wurde, tauschte sie ihre Seele mittels Seelenwanderung mit der der von Eleonora von Aranien. 1037 BF wird dies schließlich bemerkt und rückgängig gemacht. 

\subsubsection{Torxes von Freigeist}
Torxes von Freigeist, der Ewige Wanderer, ist ein Schelm, der am Tod seiner Gefährtin Aske während der Ogerschlacht zerbrach. Er zog ziellos umher, wurde kurz von Borbarad unter seinen Willen gezwungen und diente ihm als Herold. Er paktierte mit Lolgramoth und ging nach dem Fall des Dämonenmeisters an den Hof Galottas, obwohl – oder gerade weil – dieser für ihn der Hauptverantwortliche seines Leidens war. Der Schwarzschelm zieht derzeit ziellos durch Aventurien und stand zuletzt kurz davor, die Stadt Mengbilla ins Chaos zu stürzen. 

\subsection{Weitere Gegner}
\subsubsection{Pardona}
Pardona (Aventurien), Pyrdona (Pyrdakor), Bhardona (Hochelfen, Myranor), Brthona (Schurachai), Beinamen bei den heutigen Elfen sind Begierdenbringerin/-auslöserin und Die-alles-vergiftet.

Mit ihren mehr als 5000 Jahren ist Pardona eine der ältesten unter den verhüllten Meistern, ihre Kenntnisse in der Dämonologie und Chimärologie sind nach dem Tod Borbarads und Abu Terfas' unübertroffen. Sie ist im Besitz des „echsischen Originals“ des Daimonicons und der Schwarzklinge. Aufgrund ihrer Verbindung zum Namenlosen wurde sie von den Hochelfen bhardona (Isdira: die Begehrensauslöserin) genannt. 

\subsubsection{Achaz die Hexe}
Die aranische Hexe Achaz saba Arataz war eine langjährige Konkurrentin der Weidener Hexe Luzelin vom Blauen Tann. Ihre Gier nach ewiger Jugend veranlasste sie zu einem Pakt mit der Erzdämonin Asfaloth, die sie schließlich in die Niederhöllen riss und als ihre Legatin nach Dere zurückschickte.

\subsubsection{Abu Therfas}
Abu Terfas Ysasser Shenesach war vor Borbarads Rückkehr neben Zurbaran von Frigorn einer der letzten großen Chimärologen. Er lebte und forschte lange Zeit in seiner Festung Al'Churâm im Khoramgebirge und war im Besitz der Mondsilberhand, die er als Ersatz für seine fehlende linke Hand benutzte.

Nach seinem Pakt mit Asfaloth öffnete er ein altes Refugium Borbarads und versuchte dort, eine Pforte in Asfaloths Domäne wieder zu öffnen und den Großen Schwarm neu zu erschaffen. 