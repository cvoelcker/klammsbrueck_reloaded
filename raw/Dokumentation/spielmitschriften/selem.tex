\hypertarget{in-selem-1011-bf---1012-bf}{%
\section{In Selem (1011 BF - 1012
BF)}\label{in-selem-1011-bf---1012-bf}}

Dschafar findet sich nach der Begegnung mit Liscom von Fasar in dessen
alter Villa in Selem wieder. Der geflohene Diener Liscoms hatte noch
versucht, alle gefährlichen Briefe und Schriftstücke verschwinden zu
lassen, aber in der Hast der Flucht blieben viele kryptische Hinweise
auf kontinentumspannende Verschwörungen zurück.

\hypertarget{erste-analyse}{%
\subsection{Erste Analyse}\label{erste-analyse}}

Liscoms Villa enthält eine magische Bibliothek von beachtlichem Ausmaß,
allerdings sind die meisten Schriften Standardwerke. Insgesamt ist dies
für einen jungen Adepten dennoch ein glücklicher Fund, denn viele
Folianten enthalten Hinweise auf die Magie der alten Echsen, welche nur
hier in Selem zu finden sind. Insgesamt sind die Bücher wahrscheinlich
in Summe 100 Dukaten wert, doch die meisten eignen sich eher zum
Selbststudium.

Zusätzlich enthält Liscoms Villa noch einige Artefakte:

\begin{itemize}
\tightlist
\item
  ein Siegelring, der Nachrichten verschlüsselt und eine Brille, die das
  Lesen dieser Nachrichten ermöglicht
\item
  mehrere selbstleuchtende Laternen
\item
  Glyphen vor den Eingängen und Fenstern der Villa, welche ein
  schmerzhaftes Brennen bei gewaltsamen Eintritt auslösen
\end{itemize}

Das alte Studierzimmer des Meisters enthält viele verschlüsselte
Korrespondenzen und Briefe, zu großen Teilen philosophische Dispute. Mit
einigen anderen hatte Liscom regelmäßig Kontakt. Die wichtigsten von
ihnen sind:

\begin{itemize}
\tightlist
\item
  die Elfe Azaril, offensichtlich eine Adlige im Horasreich
\item
  Sulman al'Venish, ein Nekromant im Studium in Brabak
\item
  der Bettelmönch, der Leiter eines Klosters in der Nähe von Fasar, dem
  Liscom offensichtlich großen Respekt zollte
\end{itemize}

\hypertarget{wichtige-buxfccher}{%
\subsection{Wichtige Bücher}\label{wichtige-buxfccher}}

\begin{itemize}
\tightlist
\item
  Bobarads Testament
\item
  Niobaras sternkundliche Tafeln
\end{itemize}

\hypertarget{die-studienreise-dschafars-1014-bf---1015-bf}{%
\subsection{Die Studienreise Dschafars (1014 BF - 1015
BF)}\label{die-studienreise-dschafars-1014-bf---1015-bf}}

Dschafar verbleibt ein Jahr in Selem, um die Hinterlassenschaften
Liscoms zu studieren und die Funktionsweise der dunklen Pforte zu
erforschen. Leider stellt sich heraus, dass die Pforte durch die Flucht
und Verfolgung Urdos schwer beschädigt wurde. In der Bibliothek des
Magiers findet Dschafar dafür Hinweise auf weitere Pforten, z.B. in den
Akademien von Khunchom und Fasar, und Spekulationen über die
Funktionsweise der Pforten. Zudem pflegt er regen Briefwechsel mit Lin
und der Universitätsleitung und nötigt den alten Gefährten den
Rubinsplitter zur Universität zu bringen. Mit diesen Informationen und
einer nicht unbeträchtlichen Menge an Wissen und Geld macht sich
Dschafar auf den beschwerlichen Rückweg zurück in seine Heimat- und
Universitätsstadt. Dort bezahlt er erst mal einen Teil seines neuen
Vermögens an seine Familie, welche diese ``edle'' ``Spende'' als
vorläufige Entschuldigung akzeptiert, jedoch die Auflage stellt, dass
der Handel mit Artefakten und Universalschlüsseln fortan mit der
Familienabgabe belegt ist.

Als nächstes begibt sich Dschafar zu Khadil Okharim, welcher ihn zurück
in die Universität aufnimmt. Den Adaptus Major mit Schwerpunkt
``Fortgeschrittener Materietransport auf Basis von Matrixmanipulation
unter der Verwendung von stationären Artefakten'' im Blick, stürzt sich
Dschafar in das Nachtleben von Khunchom. Während seiner Studien stellt
Dschafar fest, dass ein wichtiger Schlüssel zum Verständnis der Dunklen
Pforten ihm jedoch fehlt. Offensichtlich bestimmen unbekannte Variablen,
wann und welche Pforten verbunden werden können, doch er findet nur vage
Hinweise auf magische Verknüpfungen zwischen den Pforten.

Im Rahmen seiner Studien erhält Dschafar die Erlaubnis und finanzielle
Unterstützung, eine Studienreise durchzuführen, die ihn über Fasar nach
Punin führt. In Fasar weißt ihn die Akademie ab, da sie ihre Geheimnisse
nicht teilen möchte.

Aus den Korrespondenzen von Liscom stechen vor allem die Briefe hervor,
die an einen Bettelmönch in einem Kloster im Rashdulswall gerichtet
sind. Dieser scheint ein Lehrmeister des Schwarzmagiers zu sein.
Dschafar entscheidet sich, dem Kloster einen Besuch abzustatten um den
Mönch über den kürzlich verblichenen Schwarzmagier auszufragen.

Das Kloster ist ein alter, verfallener Bau, in dem einige Personen unter
der strengen Aufsicht des Meisters leben. Der alte Mann ist eine
erschreckende Gestalt, seine fahle Haut lässt ihn skelettartig
erscheinen. Doch sein Verstand ist scharf wie ein Schwert. Er berichtet
Dschafar von der Magierphilosophie und dem Borbaradianismus, den Lehren,
dass Menschen frei vom Joch der Götter sein sollten, um selbst zu
mächtigen, überirdischen Wesen zu werden. Gleichzeitig verfolgt er eine
Philosophie der Askese, um sich von allen weltlichen Zwängen zu
befreien. Dschafar findet heraus, dass Liscoms Plan war, Borbarad zu
rufen. Wie andere Borbaradianer auch glaubte er daran, dass der Magier
ihm Macht schenken würde, wenn er genügend Anbetung erhielte, um
wiederzukehren. Doch er wollte nicht warten, sondern entschied sich,
magische Macht zu sammeln, um mit dieser Borbarad zu rufen.
Offensichtlich wollte er dazu auch den Karfunkel Tecladors verwenden.
Nach einigen Wochen bei den Bettelmönchen reist Dschafar weiter nach
Punin, um dort die dunklen Pforten weiter zu erforschen.

Unter Salandrion Finkenfarn lernt er mehr über die Magierphilosophie (in
der akademisch akzeptablen Form), insbesondere die Theorie, dass
göttliche Macht von der Menge an Gläubigen abhängig ist. Zusätzlich
erforscht er in der Bibliothek von Punin die Theorie der magischen
Pforten und der Limbologie. Gerade letzteres zieht Dschafar in den Bann
und schon bald experimentiert er mit der magischen Pforte und
Limbusmagie.

Bei einem Experiment wird er in den Limbus gerissen und findet sich
plötzlich unter einer großen Eiche in der Weidenschen Stadt Baliho
wieder. Verwirrt und magisch stark überladen (was diverse Verzerrungen
in der umgebenden Realität erzeugt), begibt er sich zunächst in ein
nahegelegenes Wirtshaus. Eine Frage jedoch lässt ihn nicht los: Welche
Erschütterung des magischen Gefüges hat diese ungeplante Reise
ausgelöst?
