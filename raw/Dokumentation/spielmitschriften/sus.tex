\hypertarget{staub-und-sterne-19.05.2018}{%
\section{Staub und Sterne
(19.05.2018)}\label{staub-und-sterne-19.05.2018}}

\hypertarget{ankunft}{%
\subsection{Ankunft}\label{ankunft}}

Lin kommt nach seiner ``Flucht'' per Schiff in Khunchom an. Nach einer
längeren Odyssee über Tavernen, Badehäuser begibt er sich auf den
Gauklerfest. Aus dem gesamten Umland sind die besten (und
mittelmäßigsten) Gaukler angereist, um sich zu treffen, zu feieren und
natürlich aufzutreten. Das Herz des Festes ist die gigantische
Zeltstadt, in welcher ein kunterbunter Haufen von Schauspielen,
Auftritten und Geschichtenerzählern versucht, möglichst viel zahlende
Kundschaft anzulocken.

Die Mitte der Zeltstadt ist ein freier Platz, in welchem die
Hauptattraktionen zu sehen sind. Dort hält sich auch Dschafar abul'l
yinnahim ibn Hahmud auf.

\hypertarget{der-altejunge-buka}{%
\subsubsection{Der alte/junge Buka}\label{der-altejunge-buka}}

Lin und Dschafar werden einem von mysteriösen Fremden beiseite genommen.
Dieser erzählt von seinen Visionen, in welcher wir gegen einen Meister
kämpfen. Dafür muss ``etwas'' gefunden werden, was gestohlen wurde/wird.
Als nächstes sollen jedoch ``die gefunden werden, die verloren sind''.

Vor dem Zelt treffen wir auf eine Freundin des Fremden, Sefaria, eine
Wahrsagerin. Von ihr erfahren wir den Namen des Fremden: Buka.

\hypertarget{bukas-hintergrundgeschichte}{%
\subsubsection{Bukas
Hintergrundgeschichte}\label{bukas-hintergrundgeschichte}}

Sefaria hat Buka in einem kleinen Dorf nahe Khunchom gefunden und
aufgenommen. Jeden Abend fängt Buka an, Geschichten mit einern enormen
Inbrunst zu erzählen. Obwohl er keinerlei Erinnerungen an sein frühreres
Leben hat, reist er seither mit Sefaria mit umher und erzählt einer
wachsenden Menge an Fans seine Geschichten.

\hypertarget{diebstahl-in-der-universituxe4t}{%
\subsubsection{Diebstahl in der
Universität}\label{diebstahl-in-der-universituxe4t}}

Aus den Bleitresoren der Universität wurde der Splitter eines einstmals
verfluchten Rubins gestohlen. Der Akademieleiter bittet Dschafar darum,
dass Gestohlene wiederzubringen. Dafür erhält er eine Flasche einer
Detektorflüssigkeit, welche aufleuchtet sollte sie in der Nähe sein.

\hypertarget{uxfcberfall-auf-die-gauklerfamilie}{%
\subsubsection{Überfall auf die
Gauklerfamilie}\label{uxfcberfall-auf-die-gauklerfamilie}}

Die völlig fertige Gauklerfamilie Damerinal kommt angereist und erzählt
von einem Angriff in der Wüste Gor. Beschrieben wurden dämonische bzw.
dschinnähnliche Vorkommnisse, während dessen der Sohn der Gauklerfamilie
verschwunden ist. In Absprache mit Buka (der uns übrigens begleiten
möchte) und Sefaria kaufen wir uns daher:

\begin{itemize}
\tightlist
\item
  dicke Lederkleidung gegen den Sand
\item
  gute Lederschläuche
\item
  Lederrucksäcke
\end{itemize}

\hypertarget{der-tafelberg}{%
\subsection{Der Tafelberg}\label{der-tafelberg}}

Nach einigem Umhören erfahren wir von der Scharte des Golgari, einem
Zugang zum Hochplateau des Tafelberges. Auf dem Weg sieht man häufig
roten Sand, wobei wohl ein starker Wind vor kurzem alle Spuren verwischt
hat. Vor der Scharte liegt ein Gauklerwagen zerschlagen, welche
anscheinend aus großer Höhe fallengelassen wurde. Nach näherer
Untersuchung erweist sich der Wagen als der verschwundene Wagen der
Familie Damerinal. Dieser weist leichte Spuren von Magie auf. Dschafar
blickt sich um und schaut direkt in eine sehr starke magische Quelle und
erblindet temporär. Beim Klettern hören wir Stimmen, die uns vor dem
Aufstieg warnen. Nach einigen Mühen erreichen wir das Plateau und nutzen
den Odem, womit wir die Arkane Struktur eines Dämons im Nordenosten.
Dabei fällt auch auf, dass Buka eine enorme magische Aura hat. Um seinen
Hals trägt er ein Halsband aus Mindorium. Wir finden einen
Tunneleingang, welcher von Buka nicht betreten werden kann ohne
ohnmächtig zu werden.

\hypertarget{die-miene}{%
\subsubsection{Die Miene}\label{die-miene}}

In der Miene arbeiten diverse Sklaven im Abbau, welche alle das selbe
Halsband tragen wie Buka. Regelmäßig wird das abgebaute Erz von einem
Dämon abgeholt. Der Eingang ist mit einem Artefakt versehen, welches die
Sklaven daran hindert die Mine zu verlassen. Da der Dämon sehr bald
wieder kommen wird, legen wir uns auf die Lauer. Noch am selben Tag
(bzw. Abend) kommt der Dämon und zieht den Karren. Da er Schienen im
Boden dafür benutzt, folgen wir erst am nächsten Tag.

\hypertarget{das-geheime-tal}{%
\subsubsection{Das geheime Tal}\label{das-geheime-tal}}

Nach einiger Zeit stoßen wir auf einen großen Artefaktkreis, welcher den
selben Effekt hat wie der in der Miene. Wir springen drüber und rollen
in eine versteckte Höhle. Nach kurzer Erkundung kommen wir in ein großes
Tal mit einem hohen weißem Turm in der Mitte und einer Brücke am
Eingang. Überall lässt sich eine gewisse Menge Magie erkennen. Als
Dschafar eine Gestalt in der Ferne sieht, entscheidet sich die Gruppe
für den spontanen Sprung drüber. Unten befindet sich ein kleiner Urwald,
durch welchen wir uns dem Turm von unten nähern. Während wir laufen,
hören wir eine Stimme vom Turm, welche uns zum Tee mit dem Meister
einlädt.

\hypertarget{der-turm}{%
\subsection{Der Turm}\label{der-turm}}

Am Turm angekommen finden wir unten eine kleine Tür und eine
Schmelzhütte. Nachdem wir die Tür geöffnet haben, finden wir uns in
einem Keller mit einer Menge Mindorium, welches wir sofort einsacken.
Lin trifft sich mit dem Meister Liscom von Fasar zum Tee und wird
gefangengenommen. Dabei wird er kurzzeitig gefangengenommen und muss von
Dschafar gerettet werden. Da Buka hat durscheinen lassen, dass er a) ein
Drache ist und b) sein Funkelstein hier im Turm ist, gehen wir weiter
hoch, aber nicht ohne das wir den Diener des Meisters versteinern
lassen.

In der Bibliothek erhalten wir einen Einblick in den Charakter Liscom:
Diverse Bücher über Beschwörungen und Dämonen (einige von Borbarad
persönlich) sowie versteckte Korrespondenz. Ein Stockwerk weiter
befindet sich der Beschwörungsraum des Meisters, wo dieser eine
Beschwörung vorbereitet. Im 3. OG ist das Schlafzimmer von Liscom. Dort
befindet sich der Rubinsplitter der Universität, den Lin einsteckt.

In einem unbeobachteten Moment stiehlt Lin den Tresor, in welchem sich
der Karfunkelstein von Bukahr befindet. Es kommt zu einem Schlagabtausch
mit Liscom, es gelingt jedoch, dass Buka seinen Stein vorher erhält.
Dieser verwandelt sich in seine Drachenform zurück und vernichtet
Liscom. Als Dank für unsere Hilfe erhalten wir eine Schuppe.

Wir folgen dem flüchtenden Diener zu einer dunkle Pforte. Nach
Aktivierung wird Dschafar in eine unbekannte Stadt teleportiert. Lin
reist mit den Sklaven aus der Mine zum Wirtshaus nahe der Scharte am
Golgari.

\hypertarget{liscoms-villa}{%
\subsection{Liscoms Villa}\label{liscoms-villa}}

Dschafar landet in Liscoms Stadtvilla in der Stadt Selem tief in den
südlichen Sümpfen Aventuriens. Dort findet er viele Hinweise auf ein den
Kontinent umspannendes Netzwerk von borbaradianischen Zirkeln und
Gruppen, die lose miteinander im Kontakt zu stehen scheinen.
