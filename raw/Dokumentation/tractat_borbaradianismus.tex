\documentclass[11pt]{article}
\usepackage[german]{babel}
\usepackage[utf8]{inputenc}
\usepackage{multicol}
\usepackage[top=2cm,bottom=3cm,left=2cm,right=2cm]{geometry}
\setlength{\parindent}{0pt}

\title{Traktatus betreffend Tharsonius von Bethana, seine philosophischen Lehren und Thesen}

\author{Von Magister Karjunon Silberbraue, Schule der Vier Türme zu Mirham}
\date{Gegeben zu Mirham im 1004. Jahre nach dem Falle Bosparans\footnote{Text ist dem Borbaradprojekt entnommen}}


\begin{document}
\maketitle

\section{Vorwort}
Mit diesem Skriptum möchte ich eine Lücke füllen, die seit langem in vielen Lehrplänen klafft und auch
bei Gelehrten der Philosophia und der Arkanen Wissenschaften gelegentlich anzutreffen ist.
Die historische Gestalt des Tharsonius von Bethana, besser bekannt als Borbarad, wird auf der einen
Seite häufig verdammt (eine hauptsächlich unter den werten Collegae vom Bund des Weißen Pentagramms, aber auch unter einigen Collegae meiner eigenen Gilde, der Bruderschaft der Wissenden, verbreitete Einstellung) oder aber auf der anderen Seite mystifiziert und als religiöses Symbol glorifiziert,
von Anhängern jener Geisteshaltung, die man gemeinhin mit dem Begriff "Borbaradianismus" bezeichnet.

Bei sachkundiger intensiverer Auseinandersetzung mit Person und Lehre des Tharsonius von Bethana
scheinen beide Richtungen wenig berechtigt.
Grundsätzlich ist die von Borbarad überlieferte Schriftensammlung natürlich deutlich zu umfangreich um
in einem kurzen Traktatus vollständig abgebildet zu werden, doch will ich in dieser meiner Arbeit versuchen, einen Einblick zu geben und ein allgemeines und objektives Bild zu zeichnen, wie es sich mir nach
langjährigen Studien und erbaulichen Diskussionen mit Freunden und Collegae darbietet. Mein besonderer Dank gilt den Magistern meiner Heimatakademie, der Schule der Vier Türme, insbesondere Magistra
Eleonora von Elburum für ihre Beiträge zur arkanen Seite des Werkes Borbarads, aber auch den zahlreichen Scholaren und Collegae, die mir durch Wort und Widerwort in regen Disputationes manchen Einblick verschafft haben, namentlich möchte ich hier stellvertretend für viele andere Morwen Jin-Rah und
Liscom ay Fasar nennen.

\section{Introduktion}
Viel wurde bereits über die historische Person des Tharsonius von Bethana geschrieben, daher will ich an
dieser Stelle nur noch einmal kurz das Wesentlichste zusammenfassen. Tharsonius von Bethana wurde –
so weit wir wissen – kurz nach dem Ende der Priesterkaiserzeit in ärmlichsten Verhältnissen in Bethana
geboren. Bereits in früher Kindheit machte er durch großes Talent auf sich aufmerksam, was ihm, ob-
wohl mittellos, die Türen der Kusliker und später der Puniner Akademie öffnete. Auf der einen Seite
machte er durch ein häufig unbeherrschtes und sehr selbstbewusstes Auftreten von sich reden, auf der
anderen Seite durch sein außergewöhnliches Talent. So soll er noch während seiner Elevenzeit in den
Ruinen Bosparans einen Fluch gebannt haben, welcher dort seit dem Fall der Vieltürmigen die Ruinenstadt dem Wiederaufbau verschlossen hielt. Sein Talent lies wiederum den Weisen Rohal aufmerksam
werden, welcher zu dieser Zeit bekanntlich die Geschicke des Reiches, das zu der damaligen Zeit den
größten Teil der bekannten Welt umfasste, lenkte und Tharsonius nach mit Bravour bestandener Prüfung
in Punin als Schüler zu sich holte.

Da Tharsonius und Rohal physisch recht ähnlich anzuschauen waren, gingen einige Zeitgenossen und,
wenn man den spärlichen Quellen glauben kann, wohl auch Tharsonius selbst einige Zeit lang davon aus,
dass es sich bei ihm um den leiblichen Sohn des Weisen Rohal handeln müsse. Nichtsdestotrotz kam es
zwischen Rohal und Borbarad nach einiger Zeit zu Zwistigkeiten, die Tharsonius dazu brachten, Rohal zu
verlassen und seine eigenen Studien zu betreiben. Zunächst forschte und lehrte er viele Jahre an der
Al'Achami zu Fasar, wo er auch den Rang der Spektabilität bekleidete. Seine Forschungen befassten sich
im Wesentlichen mit dem menschlichen Geist und der Möglichkeit desselbigen, den Einwirkungen mittels
magischer Kraft zu entgehen. Im historischen Kontext befasste er sich mit der Zeit der Magiermogule, er
stellte umfangreiche Ausgrabungen in Zhammorrah, einer in den Skorpionkriegen (etwa 1300 vor
Bosparans Fall) zerstörten Stadt an (die Ruinen finden sich bei Samra, zwischen Punin und Khunchom)
und führte auch Grabungen an einigen anderen mit diesem mächtigen Magierreich assoziierten Stätten
(häufig sogenannte "Magiergräber") durch. Später errichtete er sein Domizil in der Gorischen Wüste, die
Schwarze Festung – in diese Zeit auch fällt wohl die Annahme des Kriegsnamens "Borbarad". Im Jahre
595 BF, dem 129. Jahr seiner Herrschaft, sammelte Rohal ein Heer und führte es in die Gorische Wüste
gegen die Feste seines einstigen Schülers. Dieses Ereignis ging als Höhepunkt des "Krieges der Magier"
in die Geschichte ein. Selbiger Krieg der Magier hatte 6 Jahre zuvor begonnen, nachdem Rohal ob der
Kritik an seiner bis dato bereits mehr als hundert Jahre währenden Regierungszeit die Krone des Reiches
niedergelegt hatte und sich die Magiergilden um seine Nachfolge auf dem Thron des Mittelreiches stritten. Dieser Krieg fand ein Jahr später ein Ende, als die Bürger Gareths das alte, von den Priesterkaisern
abgelöste Kaisergeschlecht wieder an die Spitze des Reiches brachten und im sogenannten Garether
Pamphlet fürderhin Magiebegabte von der Herrscherwürde ausschlossen.

In jener bedeutendsten Schlacht der Magierkriege in der Gorischen Wüste wurden beide Armeen größtenteils vernichtet und sowohl Rohal als auch Tharsonius verschwanden. Es liegen leider keine Aufzeichnungen über den Verlauf der Schlacht vor, da die wenigen Überlebenden nicht mehr klaren Verstandes
waren, beziehungsweise sich ihre Aussagen eklatant widersprechen – so dass sich kein klares Bild dieser
letzten Stunden des Tharsonius zeichnen lässt. Es ist aber anzunehmen, dass die beiden größten Magier
ihres Zeitalters das Schicksal ihrer Armeen teilten und an diesem Tag ebenfalls in Borons Hallen eingingen. Der Ort dieser Schlacht, die Gorische Wüste gilt bis in die heutige Zeit als verflucht und wurde auch
erst von wenigen Forschern aufgesucht. Die Unsicherheit über ihr Ableben führte allerdings dazu, dass
sich bis in die heutige Zeit Anhänger der beiden Magier finden, die eine Wiederkehr Rohals (sogenannte
Rohalisten) oder Borbarads (sogenannte Borbaradianer) prophezeien oder gar auf magische Weise herbeizuführen suchen.
In diesem Traktatum soll nur auf den Borbaradianismus eingegangen werden, die Kernaussagen lassen
sich jedoch ebenso auf die Rohalisten anwenden, obwohl deren Methoden etwas weniger radikal und
daher auch etwas weniger kritisch zu bewerten sind.

\section{Die historische Lehre Borbarads}
Betrachtet man die Schriften des Tharsonius von Bethana genauer, stellt man zwei wesentliche Brüche
fest, zum einen zwischen der historischen Lehre Borbarads auf der einen und dem, was heutzutage darunter verstanden wird auf der anderen Seite als auch innerhalb des Werkes selbst.
Zwei Grundideen durchziehen die Schriften und Berichte, die uns von beziehungsweise über Borbarad
überliefert sind: Der Gedanke der Freiheit und die Frage nach der Zugänglichkeit von Wissen.
Unter Freiheit versteht Borbarad im Wesentlichen die persönliche Freiheit des Individuums. Jedwede
Einengung derselben behindere die Entwicklung des Geistes und sei so eklatant gegen Grundbestrebungen und damit gegen die Natur des Menschen gerichtet. Einengungen können dabei von verschiedenen
Seiten kommen.

Ad primum ist natürlich die offensichtlichste Einengung diejenige, welche von Gesetzen, Vorschriften
und Regelungen der weltlichen Ordnung, der Herrscher, der Gerichtsbarkeit verursacht wird, zu nennen.
Borbarads Bestrebungen, gegen diese Art der Ordnung vorzugehen, waren jedoch nicht sonderlich ausgeprägt und er geht auch in seinen Schriften kaum darauf ein. Er hielt diesen Aspekt wohl für derart
offensichtlich, dass eine weitere Auseinandersetzung sich nicht lohnen würde. Und in der Tat ist eine
Einengung aus diesen Fällen heraus ja auch evident – inwieweit diese für ein Zusammenleben vernunftbegabter Wesen jedoch unerlässlich sind, mag dahingestellt bleiben. Dies wirft die Frage auf: Wandte
sich Borbarad ab von dem, was mein geschätzter Collega Atherion in Fasar einmal als "grundlegend
vertretbare Einschränkung jedweder Freiheit aus Gründen konfliktminimierten Zusammenlebens" bezeichnete? Ergo, wandte er sich gegen die Ordnung der Welt wie wir sie kennen, eine Ordnung die auf
Herrschen und Beherrschtwerden basiert? Ich glaube, dass dies nicht der Fall ist. Borbarad zielt hier
vielmehr auf die Möglichkeit ab, durch Überschreitung von Freiheitsbegrenzungen anderer Art, auf die
ich sogleich kommen werde, die eigene Position in diesem Kontinuum neu zu definieren. Damit steht er
in der klassischen Tradition der Tulamidenlande, die ja seit jeher keinen Geburtsadel kennen und Herrscher nach deren persönlicher Macht und persönlichem Einfluss bestimmen.

Ad secundum, der Beschränkung der Freiheit aus soziokulturellen Gründen. Hierunter versteht Borbarad
die Beschränkungen, die das menschliche Zusammenleben, unabhängig von konkreten Gesetzen, dem
Menschen auferlegt. Hierbei sind Themen wie die Fürsorge im Familien- oder Sippenverbund, Verpflichtungen die sich aus Verwandtschaftsbeziehungen jedweder Art ergeben, aus Tradition und Überlieferung
tradierte Vorgehensweisen und ähnlich gelagerte Fälle von Bedeutung. Hier spricht sich Borbarad klar
dafür aus, seine eigene Freiheit über die der anderen und der Überlieferung zu stellen. Er plädiert dafür,
Überliefertes fortwährend auf seine Sinnhaftigkeit für den konkreten Fall zu überprüfen und sich von
alten Vorstellungen zu lösen, wenn sie den konkreten Anforderungen der Gegenwart nicht gerecht werden. Gleichzeitig wendet er sich gegen Gefühle wie Verantwortungsbewusstsein, Götter- und Obrigkeitengehorsam. Er weist darauf hin, dass man sich, folgt man derartigem, den Weg zum eigenen Selbst
und zur größeren Erkenntnis verbaut, da diese nur durch einen wirklich freien und ungebundenen Geist
erreicht werden kann.

Ad tertium ist der strittigste Punkt zu nennen, die Grenzen des eigenen Selbst beziehungsweise die von
den Göttern gezogenen Grenzen. Ich gebe zu, dass die Abgrenzung der Punkte nicht völlig eindeutig ist,
da in obigen ja in einem gewissen Sinne bereits ein Verstoß gegen Praios' Gesetz impliziert wird – zumindest ein Verstoß gegen die Auslegung von Praios' Willen durch seine derischen Stellvertreter. Mit
diesem Gedanken ist aber mehr das Überschreiten des philosophisch-ethisch begründeten Selbstbeschränkungsprozesses gemeint, unabhängig davon, ob aus theologischen oder moralphilosophischen
Überlegungen heraus abgeleitet. Hier sind Begriffe wie Moral, Gewissen, göttliche Ordnung und ähnliches
relevant. Borbarad wendet sich in einem recht radikalen Schritt gegen diese zweifellos einengenden
Paradigmen und spricht sich explizit dafür aus, diese Grenzen bewusst zu überschreiten. Bekanntester
und strittigster Fall in diesem Zusammenhang ist Borbarads Einstellung zur Magie. Während die allgemeine Lehre davon ausgeht, dass die Magie grundsätzlich nur wenigen zur Verfügung steht, unabhängig
davon, ob man ihr Entstehen auf einen Frevel zurückführt und somit für per se negativ erklärt oder als
natürlichen Teil von Los' Schöpfung begreift, spricht Borbarad davon, dass lediglich die menschliche
Selbstbegrenzung jeden einzelnen daran hindert, die Gabe der Magie für sich zu nutzen. Borbarad wen-
det dabei Analogien an, die vereinfacht gesagt darlegen, dass es wie in jeder anderen Begabung auch
mehr oder weniger Begabte in der Kunst der Magie gibt. Aber so wie auch ein wenig begabter Kämpfer
durch ausreichend Übung und Selbstdisziplin zu einem passablen Streiter werden kann, so kann auch
jemand, der in der Magie keine natürliche Begabung aufweist, durch ausreichende, unter Umständen
unter Anleitung ausgeführte Übungen und Praktiken einen Zugang zur Magie finden. Dieser Schritt, als
Überschreiten der göttlichen Ordnung interpretierbar, ist hierbei ein Schritt auf dem Weg zur persönlichen Freiheit des Einzelnen. Die Magie an sich erkennt Borbarad hier, für die Magietheoretiker neuerer
Zeit recht ungewöhnlich, nicht als eigenes Ziel sondern lediglich als Werkzeug an, wenn auch als das
edelste und mächtigste Werkzeug, das dem Menschen zur Verfügung steht.

Der zweite Punkt in Borbarads Werk besagt, dass jedes Wissen, egal welcher Art, dem Suchenden offen-
stehen, und dass – im Gegenschluss – der Suchende auf der Suche nach Wissen keine Grenzen aner-
kennen soll. Tharsonius von Bethana argumentiert folgendermaßen: Hesinde gab den Menschen einen
freien Geist, Forscherdrang und ein beständiges Verlangen nach der Mehrung von Wissen. Gleichzeitig
steht das Wesen der Götter derart hoch über der Schöpfung, dass sie die Möglichkeiten und Eventualitäten, die aus ihren Gaben erwachsen, abschätzen können. Daher kann Hesindes Gabe, der forschende
Geist nur als Aufforderung verstanden werden, diesen auch einzusetzen. Würde sich daraus grundlegend
schlechtes ergeben, hätten die Götter den Menschen dann diesen forschenden Geist gegeben? Auf der
anderen Seite: Wenn der Geist nichts grundsätzlich schlechtes hervorbringen kann, bzw. das Gute immer zumindest das Schlechte aufwiegt, wie kann dann eine Einschränkung des Strebens nach Wissen
göttergefällig und vertretbar sein? Diese Argumentation zielt natürlich auf Praktiken und Wissensgebiete
ab, die heute als "Verbotenes Wissen" gelten: Die Nekromantie, die Beschwörung siebtspährischer Wesenheiten, vulgo Dämonologie, usw. Wenn derlei gegen die Gebote der Götter wäre, hätten diese dann
nicht von vorneherein verhindert, dass der menschliche Geist sich in diese Richtung entwickeln und
derlei verstehen und fassen kann?

Gleichzeitig hält Borbarad den menschlichen Geist für ein großartiges und mächtiges Werkzeug mit dem
sich nahezu jedes Ziel erreichen lassen sollte. Auf den Punkt gebracht: Alles was denkbar ist, ist möglich
und alles was möglich ist, ist denkbar. Obwohl diese Ansicht in manchen Kreisen sicher auf heftigsten
Widerspruch stoßen wird, ist sie doch in unseren Kreisen weithin akzeptiert und allenfalls theologisch zu
hinterfragen und soll daher nicht weiter Gegenstand dieses Skriptums sein.
Als wahrlich großer Schritt im Gedankengebäude des Tharsonius scheint vor allem, dass Borbarad dafür
plädiert, die Sphäre des üblichen menschlichen Seins geistig zu verlassen, sich somit über das zu erheben, was im normalen Sprachgebrauch als "menschlich" gilt. Dadurch versucht er gleichsam einen Perspektivenwechsel herbeizuführen. Um diesen Gedanken zu veranschaulichen soll hier die schöne Metapher von der Ameise auf der Melone angeführt werden. Stellen wir uns eine Ameise vor, die sich auf
einer Melone befindet. Wenn sich diese Ameise nun fragt, wie ihre Welt, die Melone, beschaffen ist, ob
sie endlich oder unendlich sich erstreckt, kann sie zu dem Schluss kommen, diese Frage in Empirie zu
entscheiden, vulgo versuchen, ein Ende der Welt zu finden. Sie wird also in eine Richtung gehen und,
aufgrund der Gestalt der Melone niemals an ein Ende gelangen. Ergo ist die logische conclusio für die
Ameise die Unendlichkeit der Welt. Der Betrachter von außerhalb jedoch erkennt durch den Wechsel der
Perspektive den Fehler in der für sich logischen Schlussfolgerung der Ameise, da es ihm sein Blickwinkel
erlaubt, das größere Muster zu begreifen. Borbarad regt nun an, sich des Blickwinkels der Ameise zu
entledigen und sich durch die Abkehr von menschlicher Norm, Moral und Selbstbindung das größere
Ganze vor Augen zu führen. Zweifellos geht Borbarad in seiner Gedankenwelt davon aus, dass es sich,
um im Bild zu bleiben, beim Menschen nicht um eine Ameise sondern um ein flugfähiges Insekt handelt,
das mit der Benutzung der Flügel, des menschlichen Geistes, eine höhere Dimension erreichen könnte
und sollte und nur durch die oben beschriebenen Vorschriften und religiösen und moralischen Normen
am Gebrauch der Flügel gehindert wird, beziehungsweise sich diesen Gebrauch selbst versagt.
Ein weiterer Punkt, der sich in den Schriften des Tharsonius von Bethana immer wieder finden lässt und
der den eingangs erwähnten Bruch in seiner Lehre darstellt, ist der pseudoreligiöse Aspekt seines Werkes. Borbarad hinterließ in seinem Testament eine Anweisung, wie er nach einem etwaigen Verschwinden nicht näher definierter Art wieder nach Dere zurückgerufen werden könnte. Er beschreibt ein Ritual,
mit dessen Hilfe durch einen sogenannten Seelengötzen, eine Statue aus Glas zu deren Anfertigung
ebenfalls Anweisungen gegen werden, zum einen dem Opfernden die Gabe der Magie zuteil werden
kann, zum anderen die Seele des Opfernden zu Boron gelangt, der im Tausch dafür, falls genug Seelen
geopfert werden, die Seele des Borbarad wieder freilassen würde. Auf die Implikationen dieser Aussage
soll im nächsten großen Kapitel noch näher eingegangen werden.

Diese Aussage und die Tatsache, dass sich Borbarad somit als Objekt religiöser Aufmerksamkeit zu verstehen scheint (evident der Denkschule der Magiertheorie des Ometheon folgend) steht, wie leicht ersichtlich, in eklatantem Widerspruch zu seinen sonstigen, oben dargelegten Thesen. Er gibt jedoch eine
sehr detaillierte Beschreibung entsprechender pseudoreligiöser Praktiken und scheint selbige also durchaus ernst gemeint zu haben. Auch verspricht er in diesem Kontext demjenigen, der ihn zurück in die 3.
Sphäre beziehungsweise ins Leben ruft 77 Zauberformeln. Unter Gelehrten wurde einige Zeit die Mei-
nung vertreten, dass dieser Teil von Tharsonius' Schriften, ausschließlich in "Borbarads Testament" niedergelegt und zeitlich gesehen das letzte von Borbarad verfasste Werk, nicht von ihm selbst sondern von
einem seiner Schüler nach seinem Dahinscheiden verfasst wurde. Diese These konnte bisher weder
widerlegt noch bestätigt werden, daher muss bis zum Beweis des Gegenteils davon ausgegangen werden, dass auch dieser Part originär auf Tharsonius von Bethana zurückzuführen ist.

\section{Borbaradianismus heute}
Gerade der zuletzt angesprochene Punkt, die pseudoreligiösen Aspekte seines Werkes (zur Frage nach
der grundlegenden religiösen Wertigkeit sei auf die Magiertheorie Ometheons und die kritischen Abhandlungen zu selbiger – zum Beispiel von Spektabilitas Thomeg Atherion – verwiesen) dominieren heute das
Bild, das man sich in Gelehrtenkreisen vom Borbaradianismus macht. Daher soll auch dieser Aspekt hier
primär beleuchtet werden. Immer wieder hört man von Versuchen, die Praktiken aus Borbarads Testament, die Opferung der eigenen Seele an den Seelengötzen durchzuführen. Dabei scheinen diese Rituale
häufig tatsächliche Resultate hervorzubringen, so ist den meisten Ausführenden danach ein intuitives
Verständnis für die magischen Formeln, die gemeinhin zum Kanon der Borbaradianerzauber gezählt und
auch in Borbarads Testament als Thesis erwähnt werden gegeben. Auch scheinen sich zuvor nicht magisch begabte Personen nach der Durchführung dieser Rituale in begrenztem Maße magischer Kräfte
bedienen zu können. Nach einschlägigen Untersuchungen solcher Fälle, die ich selbst vornehmen konnte,
lässt sich mit an Sicherheit grenzender Wahrscheinlichkeit sagen, dass diesen Personen das Öffnen der
Verbotenen Pforten, also die bewusste Umsetzung von Lebenskraft in astrale Energie – wie sie ja auch in
Fachkreisen durchaus bekannt ist und gelehrt wird – auf eine intuitive Art möglich zu sein scheint und
sie somit ausschließlich mit eigener Lebenskraft ihre Effekte erzielen. Es handelt sich hierbei also keineswegs um einen Zugang zur Magie im engeren Sinne, der den Opfernden gewährt wird, wie es in
Borbarads Testament behauptet wird. Die vis astralae in verwertbarer Form bleibt den Subjekten weiterhin verwehrt. Zwar ist es im Wesentlichen eine Frage der Definition und Abgrenzung, ob eine solche
Person als der astralen Künste mächtig einzustufen ist, ich würde in Ahnlehnung an die gängigeren Definitionen dies jedoch verneinen.

Allerdings scheinen die fraglichen Personen bedeutend geringeren Widerwillen und weniger Schmerzen
beim Einsatz der Verbotenen Pforten zu empfinden, als dies normalerweise der Fall ist. Der Grad dieses
Unterschiedes ist von Subjekt zu Subjekt verschieden, doch geschah es mehr als einmal, dass ein Indi-
viduum nach entsprechender Stimulatio seine gesamte Lebenskraft in einen Effekt borbaradianischer
"Zauberei" legte und somit sein Leben beendete. Gelegentlich scheint sogar eine gewisse Tendenz der
Subjekte in diese Richtung zu bestehen obwohl nur wenige der von mir untersuchten diesen letzten
Schritt ohne explizite Lebensgefahr gingen.

Dies wirft die Frage auf: Worum handelt es sich wirklich bei diesem Ritual? Da wir bekanntermaßen
keine Möglichkeit haben, den Verbleib der sogenannte Geistseele, um die es in diesem Zusammenhang
ja geht (in Fachkreisen auch als Nayrakis bezeichnet) zu verfolgen – und deren schiere Existenz ja von
vielen Collegae bestritten wird –, kann auch keine eindeutige Antwort auf die Frage gegeben werden, ob
selbige wirklich vom Körper gelöst wird. Die Andeutung, dass die Seele zu Boron geht und dieser dann
einst Borbarads Seele freilassen würde, erscheint mehr als fraglich, da dies zumindest jeder Erfahrung
mit dem Wesen des Totengottes widersprechen würde. Da die Frage nach der Seele also nicht geklärt
werden kann und ansonsten das Ritual auch höchst dubios und fern fachkundiger Wissenschaften erscheint, muss die Frage nach dem Grund für die regelmäßig zu beobachtenden Effekte zunächst dahingestellt bleiben. Hier böten sich weitere Forschungsansätze.

Der meiner Meinung nach wesentlich bedeutendere Aspekt von Borbarads Werk, der höchst interessante
philosophische Ansatz, geriet in der Vergangenheit häufig in den Schatten dieser obigen Fragestellung.
Heutzutage wird die Lehre des Tharsonius von Bethana zwar an vielen Schulen gelehrt, allerdings nicht
in dem Maße, wie sie es sicher verdient hätte. Häufig wird der Ansatz ob seines revolutionären Charakters auch von eigentlich weltoffenen Institutionen wie der Al'Achami in Fasar rundheraus abgelehnt. Ich
kann hier in diesem Skriptum nur dafür plädieren, den Lehren die Bedeutung beizumessen, die ihnen
zusteht.

\section{Conclusio}
Zweifellos handelte es sich bei Tharsonius von Bethana um einen der größten Gelehrten unseres Zeitalters, einen würdigen Zeitgenossen Rohals und brillanten Philosophen und Magietheoretiker. Seine Schriften sind in der Tat bahnbrechend, weisen sie doch darauf hin, was dem Menschen möglich sein könnte
wenn er begönne, Grenzen zu überschreiten und ausgetretene Bahnen zu verlassen, seinem Geist die
Entfaltung, die ihm möglich ist, zugestehen würde. Selbstverständlich bietet diese Lehre auch viel Konfliktpotential, steht sie doch diametral gegen die Lehre vieler Religionen, namentlich gegen die Doktrin
der Praioskirche. Sicher war Tharsonius in seiner Lehre sehr radikal und man wird sie nicht ohne Hinterfragung annehmen und verbreiten können, doch bietet sie neue Ansatzpunkte und es ist die Aufgabe
jedes einzelnen, für sich die Thesen zu interpretieren und in den Kontext der derischen und alveranischen Ordnung zu stellen.

Zu einem anderen Schluss muss man kommen, wenn man die pseudoreligiösen Aspekte seines Werkes
betrachtet. Nicht nur der Widerspruch zu seiner sonstigen Lehre ist ein Warnzeichen, auch die etwas
absonderlichen Aussagen wie der Seelentausch mit Boron sollten dem Gelehrten Warnung sein. Entweder wurde Tharsonius in seinen späten Jahren tatsächlich von seiner eigenen Lehre benebelt und begann
sich als transderische Wesenheit zu sehen – oder die Schrift stammt nicht von ihm selbst. Auf jeden Fall
muss ein Wort der Warnung ausgesprochen werden: Die beschriebenen Rituale zeigen ungeklärte Resultate und niemand vermag heute zu sagen was im Detail wirklich bewirkt wird. Und dass im Bereich
transsphärischer Manipulationen Halbwissen dieser Art mehr als gefährlich ist, mussten ja bereits viele
Scholaren schmerzlich und zumeist sehr endgültig erfahren.

Daher möchte ich mit der Aufforderung schließen: Sehen wir Borbarad als das, was er wirklich war:
Einen großen Denker, hervorragenden Philosophen und hochgebildeten Wissenschaftler sowie höchst
fähigen Magier. Aber sehen wir auch die Schwächen seines Werkes und ordnen wir Tharsonius von Bethana da ein wo er hingehört: Unter die führenden Philosophen der jüngeren Geschichte und unter die
größten Magier der letzten tausend Jahre. Aber hüten wir uns auf der anderen Seite davor, mehr in
diesen Mann hineinzuinterpretieren. Borbarad war nicht unfehlbar und kein Halbgott, folglich sollte er
auch kein Objekt religiöser Verehrung sein. Lasst uns daher seine Schriften mit wachen Sinnen und
geschärftem Geist studieren, auf dass wir übernehmen, was des Übernehmens wert erscheint und kritisch beleuchten, was der kritischen Hinterfragung bedarf.

Scholaren, die sich näher mit dem Werke des Tharsonius von Bethana befassen wollen, sei die Lektüre
von "Borbarads Testament" und seinen weiteren Schriften empfohlen, die gegen einen geringen Obolus
in der Schule der Vier Türme eingesehen werden können.
\end{document}